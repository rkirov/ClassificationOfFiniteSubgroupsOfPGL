% This file makes a web version of the blueprint
% It should include all the \usepackage needed for this version.
% The template includes standard AMS packages.
% It is otherwise a very minimal preamble (you should probably at least
% add cleveref and tikz-cd).

\documentclass{report}

% \usepackage[utf8]{inputenc}
\usepackage{amssymb, amsthm, amsmath, amsfonts}
\usepackage{hyperref}
\usepackage[showmore, dep_graph]{blueprint}

\usepackage{graphicx}
\DeclareGraphicsExtensions{.svg,.png,.jpg}
\usepackage[capitalize]{cleveref}
% \usepackage[showmore,dep_graph, project=../../, dep_by=chapter]{blueprint}
%%\usepackage[dep_graph, coverage, project=../project, dep_by=section]{blueprint}

\usepackage{tikz}
\usepackage{tikz-cd}
% For the website minted is not available yet, so we use listings - we define code environment which uses listings
% \usepackage{listings}

\usepackage{fancyvrb}
\usepackage{fontspec}
\setmainfont{DejaVu Serif}

% Define a "code" environment using fancyvrb
% \newenvironment{code}{%
%   \VerbatimEnvironment
%   \begin{Verbatim}
% }{%
%   \end{Verbatim}
% }


% In this file you should put all LaTeX macros and settings to be used both by
% the pdf version and the web version.
% This should be most of your macros.

% The theorem-like environments defined below are those that appear by default
% in the dependency graph. See the README of leanblueprint if you need help to
% customize this. 
% The configuration below use the theorem counter for all those environments
% (this is what the [theorem] arguments mean) and never resets it.
% If you want for instance to number them within chapters then you can add
% [chapter] at the end of the next line.

% \newtheorem{theorem}{Theorem}
% \newtheorem{proposition}[theorem]{Proposition}
% \newtheorem{lemma}[theorem]{Lemma}
% \newtheorem{corollary}[theorem]{Corollary}
% \newtheorem{definition}[theorem]{Definition}


\newcommand{\Z}{\mathbb{Z}}
\newcommand{\N}{\mathbb{N}}
\newcommand{\A}{\mathbb{A}}
\newcommand{\Q}{\mathbb{Q}}
\newcommand{\R}{\mathbb{R}}
\newcommand{\F}{\mathbb{F}}
\newcommand{\Fbar}{\bar{\F}}
% \newcommand{\Qp}{\mathbb{Q}_p}
% \newcommand{\Ql}{\mathbb{Q}_\ell}
% \newcommand{\Qbar}{\overline{\Q}}
% \newcommand{\Qpbar}{\overline{\Q}_p}
% \newcommand{\Qlbar}{\overline{\Q}_\ell}
\newcommand{\C}{\mathbb{C}}
\newcommand{\bigslant}[2]{{\raisebox{.2em}{$#1$}\left/\raisebox{-.2em}}}
% \newcommand{\GQ}{\Gal(\Qbar/\Q)}
% \newcommand{\GQp}{\Gal(\Qpbar/\Qp)}
% \newcommand{\GQl}{\Gal(\Qlbar/\Ql)}
% \newcommand{\m}{\mathfrak{m}}
% \newcommand{\GK}{\Gal(K^{\sep}/K)}
% \newcommand{\GN}{\Gal(\overline{N}/N)}
% \newcommand{\Kbar}{\overline{K}}
% \newcommand{\Qhat}{\widehat{\Q}}
% \newcommand{\calO}{\mathcal{O}}
% \newcommand{\calOhat}{\widehat{\calO}}
% \newcommand{\bbH}{\mathbb{H}}
% \newcommand{\p}{{\mathfrak{p}}}
% \newcommand{\rhobar}{\overline{\rho}}
% \newcommand{\Zhat}{\widehat{\Z}}
\DeclareMathOperator{\Gal}{Gal}
\DeclareMathOperator{\avoid}{avoid}
\DeclareMathOperator{\Aut}{Aut}
\DeclareMathOperator{\GL}{GL}
\DeclareMathOperator{\PGL}{PGL}
\DeclareMathOperator{\PSL}{PSL}
\DeclareMathOperator{\SL}{SL}
\DeclareMathOperator{\Spec}{Spec}
\DeclareMathOperator{\sep}{sep}
\DeclareMathOperator{\ab}{ab}
\DeclareMathOperator{\tr}{tr}
\DeclareMathOperator{\Hom}{Hom}
\DeclareMathOperator{\Frob}{Frob}
% This file makes a web version of the blueprint
% It should include all the \usepackage needed for this version.
% The template includes standard AMS packages.
% It is otherwise a very minimal preamble (you should probably at least
% add cleveref and tikz-cd).

\documentclass{report}

% \usepackage[utf8]{inputenc}
\usepackage{amssymb, amsthm, amsmath, amsfonts}
\usepackage{hyperref}
\usepackage[showmore, dep_graph]{blueprint}

\usepackage{graphicx}
\DeclareGraphicsExtensions{.svg,.png,.jpg}
\usepackage[capitalize]{cleveref}
% \usepackage[showmore,dep_graph, project=../../, dep_by=chapter]{blueprint}
%%\usepackage[dep_graph, coverage, project=../project, dep_by=section]{blueprint}

\usepackage{tikz}
\usepackage{tikz-cd}
% For the website minted is not available yet, so we use listings - we define code environment which uses listings
% \usepackage{listings}

\usepackage{fancyvrb}
\usepackage{fontspec}
\setmainfont{DejaVu Serif}

% Define a "code" environment using fancyvrb
% \newenvironment{code}{%
%   \VerbatimEnvironment
%   \begin{Verbatim}
% }{%
%   \end{Verbatim}
% }


% In this file you should put all LaTeX macros and settings to be used both by
% the pdf version and the web version.
% This should be most of your macros.

% The theorem-like environments defined below are those that appear by default
% in the dependency graph. See the README of leanblueprint if you need help to
% customize this. 
% The configuration below use the theorem counter for all those environments
% (this is what the [theorem] arguments mean) and never resets it.
% If you want for instance to number them within chapters then you can add
% [chapter] at the end of the next line.

% \newtheorem{theorem}{Theorem}
% \newtheorem{proposition}[theorem]{Proposition}
% \newtheorem{lemma}[theorem]{Lemma}
% \newtheorem{corollary}[theorem]{Corollary}
% \newtheorem{definition}[theorem]{Definition}


\newcommand{\Z}{\mathbb{Z}}
\newcommand{\N}{\mathbb{N}}
\newcommand{\A}{\mathbb{A}}
\newcommand{\Q}{\mathbb{Q}}
\newcommand{\R}{\mathbb{R}}
\newcommand{\F}{\mathbb{F}}
\newcommand{\Fbar}{\bar{\F}}
% \newcommand{\Qp}{\mathbb{Q}_p}
% \newcommand{\Ql}{\mathbb{Q}_\ell}
% \newcommand{\Qbar}{\overline{\Q}}
% \newcommand{\Qpbar}{\overline{\Q}_p}
% \newcommand{\Qlbar}{\overline{\Q}_\ell}
\newcommand{\C}{\mathbb{C}}
\newcommand{\bigslant}[2]{{\raisebox{.2em}{$#1$}\left/\raisebox{-.2em}}}
% \newcommand{\GQ}{\Gal(\Qbar/\Q)}
% \newcommand{\GQp}{\Gal(\Qpbar/\Qp)}
% \newcommand{\GQl}{\Gal(\Qlbar/\Ql)}
% \newcommand{\m}{\mathfrak{m}}
% \newcommand{\GK}{\Gal(K^{\sep}/K)}
% \newcommand{\GN}{\Gal(\overline{N}/N)}
% \newcommand{\Kbar}{\overline{K}}
% \newcommand{\Qhat}{\widehat{\Q}}
% \newcommand{\calO}{\mathcal{O}}
% \newcommand{\calOhat}{\widehat{\calO}}
% \newcommand{\bbH}{\mathbb{H}}
% \newcommand{\p}{{\mathfrak{p}}}
% \newcommand{\rhobar}{\overline{\rho}}
% \newcommand{\Zhat}{\widehat{\Z}}
\DeclareMathOperator{\Gal}{Gal}
\DeclareMathOperator{\avoid}{avoid}
\DeclareMathOperator{\Aut}{Aut}
\DeclareMathOperator{\GL}{GL}
\DeclareMathOperator{\PGL}{PGL}
\DeclareMathOperator{\PSL}{PSL}
\DeclareMathOperator{\SL}{SL}
\DeclareMathOperator{\Spec}{Spec}
\DeclareMathOperator{\sep}{sep}
\DeclareMathOperator{\ab}{ab}
\DeclareMathOperator{\tr}{tr}
\DeclareMathOperator{\Hom}{Hom}
\DeclareMathOperator{\Frob}{Frob}
% This file makes a web version of the blueprint
% It should include all the \usepackage needed for this version.
% The template includes standard AMS packages.
% It is otherwise a very minimal preamble (you should probably at least
% add cleveref and tikz-cd).

\documentclass{report}

% \usepackage[utf8]{inputenc}
\usepackage{amssymb, amsthm, amsmath, amsfonts}
\usepackage{hyperref}
\usepackage[showmore, dep_graph]{blueprint}

\usepackage{graphicx}
\DeclareGraphicsExtensions{.svg,.png,.jpg}
\usepackage[capitalize]{cleveref}
% \usepackage[showmore,dep_graph, project=../../, dep_by=chapter]{blueprint}
%%\usepackage[dep_graph, coverage, project=../project, dep_by=section]{blueprint}

\usepackage{tikz}
\usepackage{tikz-cd}
% For the website minted is not available yet, so we use listings - we define code environment which uses listings
% \usepackage{listings}

\usepackage{fancyvrb}
\usepackage{fontspec}
\setmainfont{DejaVu Serif}

% Define a "code" environment using fancyvrb
% \newenvironment{code}{%
%   \VerbatimEnvironment
%   \begin{Verbatim}
% }{%
%   \end{Verbatim}
% }


% In this file you should put all LaTeX macros and settings to be used both by
% the pdf version and the web version.
% This should be most of your macros.

% The theorem-like environments defined below are those that appear by default
% in the dependency graph. See the README of leanblueprint if you need help to
% customize this. 
% The configuration below use the theorem counter for all those environments
% (this is what the [theorem] arguments mean) and never resets it.
% If you want for instance to number them within chapters then you can add
% [chapter] at the end of the next line.

% \newtheorem{theorem}{Theorem}
% \newtheorem{proposition}[theorem]{Proposition}
% \newtheorem{lemma}[theorem]{Lemma}
% \newtheorem{corollary}[theorem]{Corollary}
% \newtheorem{definition}[theorem]{Definition}


\newcommand{\Z}{\mathbb{Z}}
\newcommand{\N}{\mathbb{N}}
\newcommand{\A}{\mathbb{A}}
\newcommand{\Q}{\mathbb{Q}}
\newcommand{\R}{\mathbb{R}}
\newcommand{\F}{\mathbb{F}}
\newcommand{\Fbar}{\bar{\F}}
% \newcommand{\Qp}{\mathbb{Q}_p}
% \newcommand{\Ql}{\mathbb{Q}_\ell}
% \newcommand{\Qbar}{\overline{\Q}}
% \newcommand{\Qpbar}{\overline{\Q}_p}
% \newcommand{\Qlbar}{\overline{\Q}_\ell}
\newcommand{\C}{\mathbb{C}}
\newcommand{\bigslant}[2]{{\raisebox{.2em}{$#1$}\left/\raisebox{-.2em}}}
% \newcommand{\GQ}{\Gal(\Qbar/\Q)}
% \newcommand{\GQp}{\Gal(\Qpbar/\Qp)}
% \newcommand{\GQl}{\Gal(\Qlbar/\Ql)}
% \newcommand{\m}{\mathfrak{m}}
% \newcommand{\GK}{\Gal(K^{\sep}/K)}
% \newcommand{\GN}{\Gal(\overline{N}/N)}
% \newcommand{\Kbar}{\overline{K}}
% \newcommand{\Qhat}{\widehat{\Q}}
% \newcommand{\calO}{\mathcal{O}}
% \newcommand{\calOhat}{\widehat{\calO}}
% \newcommand{\bbH}{\mathbb{H}}
% \newcommand{\p}{{\mathfrak{p}}}
% \newcommand{\rhobar}{\overline{\rho}}
% \newcommand{\Zhat}{\widehat{\Z}}
\DeclareMathOperator{\Gal}{Gal}
\DeclareMathOperator{\avoid}{avoid}
\DeclareMathOperator{\Aut}{Aut}
\DeclareMathOperator{\GL}{GL}
\DeclareMathOperator{\PGL}{PGL}
\DeclareMathOperator{\PSL}{PSL}
\DeclareMathOperator{\SL}{SL}
\DeclareMathOperator{\Spec}{Spec}
\DeclareMathOperator{\sep}{sep}
\DeclareMathOperator{\ab}{ab}
\DeclareMathOperator{\tr}{tr}
\DeclareMathOperator{\Hom}{Hom}
\DeclareMathOperator{\Frob}{Frob}
% This file makes a web version of the blueprint
% It should include all the \usepackage needed for this version.
% The template includes standard AMS packages.
% It is otherwise a very minimal preamble (you should probably at least
% add cleveref and tikz-cd).

\documentclass{report}

% \usepackage[utf8]{inputenc}
\usepackage{amssymb, amsthm, amsmath, amsfonts}
\usepackage{hyperref}
\usepackage[showmore, dep_graph]{blueprint}

\usepackage{graphicx}
\DeclareGraphicsExtensions{.svg,.png,.jpg}
\usepackage[capitalize]{cleveref}
% \usepackage[showmore,dep_graph, project=../../, dep_by=chapter]{blueprint}
%%\usepackage[dep_graph, coverage, project=../project, dep_by=section]{blueprint}

\usepackage{tikz}
\usepackage{tikz-cd}
% For the website minted is not available yet, so we use listings - we define code environment which uses listings
% \usepackage{listings}

\usepackage{fancyvrb}
\usepackage{fontspec}
\setmainfont{DejaVu Serif}

% Define a "code" environment using fancyvrb
% \newenvironment{code}{%
%   \VerbatimEnvironment
%   \begin{Verbatim}
% }{%
%   \end{Verbatim}
% }


\input{preamble/common}
\input{preamble/web}


\usepackage{xcolor}     % For coloring code elements

\home{https://AlexBrodbelt.github.io/ClassificationOfFiniteSubgroupsOfPGL}
\github{https://github.com/AlexBrodbelt/ClassificationOfFiniteSubgroupsOfPGL}
\dochome{https://AlexBrodbelt.github.io/ClassificationOfFiniteSubgroupsOfPGL/docs}

\title{Classification of finite subgroups of PGL}
\author{AlexBrodbelt}

\begin{document}
\maketitle
\input{content}
\end{document}


\usepackage{xcolor}     % For coloring code elements

\home{https://AlexBrodbelt.github.io/ClassificationOfFiniteSubgroupsOfPGL}
\github{https://github.com/AlexBrodbelt/ClassificationOfFiniteSubgroupsOfPGL}
\dochome{https://AlexBrodbelt.github.io/ClassificationOfFiniteSubgroupsOfPGL/docs}

\title{Classification of finite subgroups of PGL}
\author{AlexBrodbelt}

\begin{document}
\maketitle
% In this file you should put the actual content of the blueprint.
% It will be used both by the web and the print version.
% It should *not* include the \begin{document}
%
% If you want to split the blueprint content into several files then
% the current file can be a simple sequence of \input. Otherwise It
% can start with a \section or \chapter for instance.
\input{chapter/Ch1_AbstractAndAcknowledgements}
\input{chapter/Ch2_Introduction}
\input{chapter/Ch3_Preliminaries}
\input{chapter/Ch4_ReductionOfProblem}
\input{chapter/Ch5_PropertiesOfSLOverAlgClosedField}
\input{chapter/Ch6_MaximalAbelianSubgroupClassEquation}
\input{chapter/Ch7_DicksonsClassificationTheorem}
\input{chapter/bibliography}
\end{document}


\usepackage{xcolor}     % For coloring code elements

\home{https://AlexBrodbelt.github.io/ClassificationOfFiniteSubgroupsOfPGL}
\github{https://github.com/AlexBrodbelt/ClassificationOfFiniteSubgroupsOfPGL}
\dochome{https://AlexBrodbelt.github.io/ClassificationOfFiniteSubgroupsOfPGL/docs}

\title{Classification of finite subgroups of PGL}
\author{AlexBrodbelt}

\begin{document}
\maketitle
% In this file you should put the actual content of the blueprint.
% It will be used both by the web and the print version.
% It should *not* include the \begin{document}
%
% If you want to split the blueprint content into several files then
% the current file can be a simple sequence of \input. Otherwise It
% can start with a \section or \chapter for instance.
\chapter{Abstract and Acknowledgements}\label{Ch1_AbstractAndAcknowledgements}

\section{Acknowledgements}

I thank my supervisor Prof. David Jordan for his invaluable support and guidance throughout the project,

I would also like to thank Christopher Butler for providing the TeX code so I could easily set up the blueprint, and hopefully, improve and add to his amazing exposition of
\textbf{Dickson's Classification Theorem}.

I would also like to thank Prof. Kevin Buzzard for his support, patience and guidance throughout the project. His advice and comments on how I should go about formalising mathematics have been of utmost value. 

Finally, I would like to thank the many members of the Lean Zulip community who have provided insightful ideas and comments that have helped me progress much faster than otherwise, 
this also includes assistance with technical issues with setting up the blueprint and so forth. I am grateful to:

\begin{itemize}
    \item Artie Khovanov
    \item David Loeffler
    \item Mitchell Lee
    \item Yakov Pechersky
    \item Edward van de Meent
    \item Ruben Van de Velde
    \item Andrew Yang
    \item Johan Commelin
    \item Scott Carnahan
    \item Damiano Testa
    \item Aron Liu
\end{itemize}

\section{Abstract}


\section{Popular science summary}

In order to explain what this paper is about, it is necessary to first define a few of the mathematical concepts which it concerns. A \textit{group} is a set of objects, called \textit{elements}, together with a rule, called an \textit{operation}, which tells us how two elements combine with each other to make a third. Furthermore, to be considered a group it must also satisfy 4 conditions, called \textit{axioms}. One of which is that the group must be \textit{closed} under it's operation. This means that whenever any two elements in the group are combined, the resulting element is also part of the group. The remaining axioms require that the group must also be \textit{associative}, have an \textit{identity} element and each element must have an \textit{inverse}. The way in which the elements in a group act with each other is called the group's \textit{structure}. If 2 groups have the same number of elements and share the same structure, then they are regarded as being \textit{isomorphic} to each other, which essentially means that they equivalent. Many everyday things can be regarded as groups, such as the symmetries of geometrical objects, or the number systems we use. \\
\\
The set of 2 x 2 matrices whose \textit{determinant} is equal to 1, together with the operation of ordinary matrix multiplication, forms a group called the \textit{special linear group}. This is a group because the product of 2 matrices has a determinant equal to the product of the determinants of the 2 matrices, so since 1 x 1 = 1, this new element also belongs to the group, hence the axiom of being closed is satisfied. Furthermore, it is crucial that the entries in the matrices are taken from a specified \textit{ring} or \textit{field}. Rings and fields are, like groups, abstract mathematical objects, albeit they satisy even more axioms than groups do. Crucially, rings and fields have both an additive and a multiplicative identity. \\
\\
This paper focuses on $\SL_2(F)$, which is the two-dimensional special linear group whose entries are taken from an \textit{algebraically closed} field. Algebraically closed fields are infinite in size, which means that the resulting special linear group is also infinite. A \textit{subgroup} of a group is simply a group with the added requirement that each of it's elements must also belong to the original group. Thus a finite subgroup of $\SL_2(F)$ is any finite set of elements belonging to this infinite group $\SL_2(F)$, which satisfy the 4 axioms of being a group. \\
\\
This paper classifies all the possible structures which a finite subgroup of $\SL_2(F)$ could have. The result has implications within the study of finite \textit{simple} groups. This classification was first done by American mathematician Leonard Eugene Dickson in 1901. The purpose of this reformulation is to make it accessible to a wider audience by providing a more detailed explanation at the various stages of the proof.

\section{Abstract}

This paper is a reformulation of Leonard Dickson's complete classification of the finite subgroups of the two-dimensional special linear group over an arbitrary algebraically closed field, $\SL_2(F)$. The approach is to construct a class equation of the conjugacy classes of maximal abelian subgroups of an arbitrary finite subgroup of $\SL_2(F)$. In turn, this leads to only 10 possible classes of structures of this subgroup up to isomorphism.

\section{Acknowledgements from Christopher Butler}

I would like to take this opportunity to thank my advisor Arne Meurman. This paper would not have been possible without the guidance and insight he gave during our weekly discussions.

\cleardoublepage



\chapter{Introduction}\label{Ch2_Introduction}

\section{What is the formalization of mathematics?}

Formalization of mathematics is the art of teaching a computer what a piece of mathematics means.

That is, it is the process of carefully writing down a mathematical statement typically in first order logic or higher order logic and then scrutinously justifying each step of the proof to a computer program that checks the validity of every step of the reasoning. 

Typically one formalizes mathematics with the help of a proof assistant or interactive theorem prover, a piece of software which enables a human to write down mathematics and have the software verify the claims.

There exist many proof assistants, such examples are Lean, Isabelle, Coq, Metamath, etc.

For this project I have opted to use Lean due to its rapid growing mathematics library and its dependent type theory. I shall explain in more detail these last two reasons, but first I will comment on what Lean is.

\subsubsection{What is Lean?}

Lean is both a functional programming language and an interactive theorem prover that is being developed at Microsoft research and AWS by Leonardo de Moura and his team. It has been designed for both use in cutting-edge mathematics and the verification of software which is often essential to safety critical systems where correctness is of extreme
TODO:

- Brief explanation of type theory and curry-howard isomorphism.

- Example of formal proof and comparison with informal proof.


\begin{verbatim}
theorem add\textunderscore comm (a b : Nat) : a + b = b + a :=
  Nat.add\textunderscore comm a b
\end{verbatim}


\section{Fermat's Last Theorem}

TODO:

-History of Fermat's Equation

-Problem statements

-Developments in number theory that lead to the resolution of the conjecture.


\section{Formalizing Fermat's Last Theorem}

Following the sequence of success stories ranging from the Liquid Tensor Experiment to the formalization of the Polynomial Freiman-Rusza conjecture. 

Prof. Kevin Buzzard from Imperial College London has received a five-year grant that will allow him to lead the formalization of Fermat's Last Theorem. This grant kicked in in October of 2024. 

At the time of writing, since October of 2024, a digital blueprint has been set up to manage the project.

Alongside other infrastructure like the project dashboard, mathematicians around the world can claim tasks that are set by Prof. Kevin Buzzard and if in return a task is returned with a "sorry" free proof then one can claim the glory of having completed the task.

TODO:

- Current goal of the formalization

- Explain somewhat the modern approach and the highly sought after Modularity Lifting Theorem.

- My task: Classification of finite subgroups of $\PGL_2(\bar{\F}_p)$



\section{Classification of finite subgroups of the $\PGL_2(\Fbar_p)$}

TODO:

-Why are the finite subgroups of  $\PGL_2(\bar{\F}_p)$ relevant to number theory: i.e: Automorphic forms, Galois representations, etc.


The primary concern of this project is to formalise Theorem 2.47 of [DTT] which states:

\begin{enumerate}
    \item If $H$ is finite subgroup of $\PGL_2(\C)$ then $H$ is isomorphic to one of the following groups: the cyclic group $C_n$ of order $n$ ($n \in \Z_{>0}$), the dihedral group $D_{2n}$ of order $2n$ ($n \in \Z_{>1}$), $A_4$, $S_4$ or $A_5$.
\item If $H$ is a finite subgroup of $\PGL_2(\Fbar_p)$ then one of the following holds:
\begin{enumerate}
    \item $H$ is conjugate to a subgroup of the upper triangular matrices;
    \item $H$ is conjugate to $\PGL_2(\F_{\ell^r})$ and $\PSL_2(\F_{\ell^{r}})$ for some $r \in \Z_{>0}$;
    \item $H$ is isomorphic to $A_4$, $S_4$, $A_5$ or the dihedral group $D_{2r}$ of order $2r$ for some $r \in \Z_{>1}$ not divisible by $\ell$

\end{enumerate}
    Where $\ell$ is assumed to be an odd prime.
\end{enumerate}


By definition the Projective General Linear Group is:

\begin{equation}
    \PGL_n(F) = \GL_n(F) / (Z(\GL_n(F))) = \GL_n(F) / (F^\times I) 
\end{equation}

Similarly, the Projective Special Linear Group is:

\begin{equation}
    \PSL_n(F) = \SL_n(F) / (Z(\SL_n(F))) = \SL_n(F) / (\langle -I\rangle)
\end{equation}

Given we are working over an algebraically closed field $F$, it turns out that for any $n \in \N$, $\PGL_n(F)$ is isomorphic to $\PSL_n(F)$.

This isomorphism will be crucial as it will allow us to focus on classifying finite subgroups of $SL_2(F)$ to classify the finite subgroups of $\PGL_2(F)$.

The goal of the next chapter is to prove and formalize this result.
\chapter{Preliminaries}\label{Ch3_Preliminaries}

This section briefly outlines some standard group theory results which perhaps may not have been covered in a first course in Group Theory. Since they are not the main focus of this paper, most of the proofs have been omitted. A more
advanced reader may choose to skip this first chapter, using it only for reference purposes as and when the results are subsequently cited. 

\section{Some Elementary Theorems}

The following theorems are all well-known fundamental results in group theory. If the reader is interested in the proofs, they can be found in Hungerford \cite{hungerford}.

\begin{theorem}\label{lagrange} \textit{Let $G$ be a finite group. Then the order of any subgroup of $G$ divides the order of $G$.} \\
\end{theorem} 

\begin{theorem}\label{1stiso} \textit{Let $\phi  :G \rightarrow G'$ be a homomorphism of groups. Then, $$G/Ker \; \phi \cong Im \; \phi.$$ Hence, in particular, if $\phi$ is surjective then, $$G/Ker \; \phi \cong G'.$$} \\
\end{theorem} 

\vspace{-10mm}

\begin{theorem}\label{2ndiso} \textit{Let $H$ and $N$ be subgroups of $G$, and $N \vartriangleleft G$. Then, $$H/H \cap N \cong HN/N.$$} \\
\end{theorem} 

\vspace{-10mm}

\begin{theorem}\label{3rdiso} \textit{Let $H$ and $K$ be normal subgroups of $G$ and $K \subset H$. Then $H/K$ is a normal subgroup of $G/K$ and, $$(G/K)/(H/K) \cong G/H.$$} \\
\end{theorem} 

\vspace{-10mm}

\begin{theorem}\label{cauchy} \textit{If the order of a finite group $G$ is divisible by a prime number $p$, then $G$ has an element of order $p$.} \\
\end{theorem} 

\section{Sylow Theory}

In 1872, Norweigian mathematician Peter Ludwig Sylow published his theorems regarding the number of subgroups of a fixed order that a given finite group contains. Today these are collectively known as the Sylow Theorems and play a vital role in determining the structure of finite groups. I will use the results of these theorems several times throughout this paper and I state them here without proof. If the reader would like to read further, the proofs can be found in most introductory texts on group theory, such as Bhattacharya \cite{bhattacharya}, except Corollary \ref{5thsylow} which can be found in Alperin and Bell \cite[p.64]{alperin} . \\


\begin{definition}
\lean{Sylow}
\leanok 
Let $G$ be a finite group and $p$ a prime, a \textbf{Sylow $\pmb{p}$-subgroup} of $G$ is a subgroup of order $p^r$, where $p^{r+1}$ does not divide the order of $G$. \\
\\
Let $p$ be a prime. A group $G$ is called a \textbf{$\pmb{p}$-group} if the order of each of it's elements is a power of $p$. Similarly, a subgroup $H$ of $G$ is called a \textbf{$\pmb{p}$-subgroup} if the order of each of it's elements is a power of $p$.
\end{definition}

In each of the following results, $G$ is a finite group of order $p^r m$, where $p$ is a prime which does not divide $m$. \\
\\

\begin{theorem}[Sylow's first theorem]
\lean{Sylow.exists_subgroup_card_pow_prime}
\leanok
\textit{If $p^k$ divides $|G|$, then $G$ has a subgroup of order $p^k$.} \\

\end{theorem}

\begin{theorem}[Sylow's second theorem]
\lean{Sylow.equiv.proof_1}
\leanok
\textit{All Sylow $p$-subgroups of G are conjugate.} \\
\end{theorem}

\begin{theorem}[Sylow's third theorem]
\lean{card_sylow_modEq_one}
\leanok
\textit{The number of Sylow $p$-subgroups $n_p$ divides $m$ and satisfies $n_p \equiv 1 ($mod $p)$.} \\
\end{theorem}

\begin{corollary}[Sylow's fourth theorem]
\label{Sylow.unique_of_normal}
\lean{Sylow.unique_of_normal}
\leanok    
 \textit{A Sylow $p$-subgroup of $G$ is unique if and only if it is normal.} \\
\end{corollary}

\begin{corollary}[Sylow's fifth theorem]
\label{IsPGroup.exists_le_sylow}
\lean{IsPGroup.exists_le_sylow}
\leanok
\textit{Any $p$-subgroup of $G$ is contained in a Sylow $p$-subgroup.} \\
\end{corollary}

\section{Group Action}

\begin{definition} Let $G$ be a group and $X$ be a set. Then $G$ is said to \textbf{act} on $X$ if there is a map $\phi : G \times X \rightarrow X$, with $\phi(a,x)$ denoted by $a^*x$, such that for $a,b \in G$ and $x \in X$, the following 2 properties hold:
\begin{align*} &(i) \quad a\,^*(b\,^*x) = (ab)^*x,
\\  &(ii) \quad I_G\,^*x = x.
\end{align*}

The map $\phi$ is called the \textbf{group action} of $G$ on $X$.
\end{definition}

\begin{definition} Let $G$ be a group acting on a set $X$ and let $x \in X$. Then the set,
\begin{align*} Stab(x) = \{ g \in G  :  gx = x \},
\end{align*}
is called the \textbf{stabiliser} of $x$ in $G$. Each $g$ in $S_G(x)$ is said to \textbf{fix} $x$, whilst $x$ is said to be a \textbf{fixed point} of each $g$ in $S_G(x)$. Also, the set,
\begin{align*} \text{Orb}(x) = \{ gx : g \in G \},
\end{align*}
is called the \textbf{orbit} of $x$ in $G$.  
\end{definition} 

The orbit and the stabiliser of an element are closely related. The following theorem is a consequence of this relationship and it will be useful throughout this paper. \\

\begin{theorem} [Orbit-Stabilizer theorem]
    \textit{Let $G$ be a finite group acting on a set $X$. Then for each $x \in X$}, $$|G| = |\text{Orb}(x)| |\text{Stab}(x)|.$$ \\
\end{theorem}

The following standard theorem will all play a vital roll later on.

\begin{theorem}\label{symhomoker} Let $G$ be a group and $H$ a subgroup of $G$ of finite index $n$. Then there is a homomorphism $\phi : G \longrightarrow S_n$ such that,
\begin{align*} ker(\phi) = \bigcap\limits_{x \in G} x H x^{-1}.
\end{align*}
\end{theorem}

\begin{proof} See \cite[p.110]{bhattacharya} for proof.
\end{proof}

\section{Conjugation}

\begin{definition}[Conjugate elements]
\label{IsConj}
\lean{IsConj}
Let $G$ be a group and $a$ an element of $G$. An element $b \in G$ is said to be \textbf{conjugate} to $a$ if $b=xax^{-1}$ for some $x \in G$. \\
\end{definition}

\begin{remark}
\label{conj_elem}
In Lean, to state that two elements $g, h \in G$ where $G$ is a group, we use the slightly more general definition of conjugacy over monoids.

That is to say, given $g, h \in G$ where $G$ is a group (or more generally monoid) and impose that $g$ and $h$ are conjugate, instead of writing the equality which has type \texttt{Prop}:

\begin{verbatim}
∃ c : α, c * a * c⁻¹ = b
\end{verbatim}

We use the following statement of type \texttt{Prop} that has been defined in Mathlib under the name of \texttt{IsConj}.

The reason we would choose this over the naive statement is because Mathlib will contain a lot of very useful lemmas attached to this definition.

Saying two elements are conjugate is writing something like the following:

Assuming the terms \texttt{g : G} and \texttt{h : G} of the type \texttt{G} (which has the \texttt{Group} typeclass instance) are in scope.

\begin{verbatim}
IsConj g h
\end{verbatim}
\end{remark}


\begin{definition}[Conjugate subgroups]
Let $H_1$ be a proper subgroup of $G$ and fix $x \in G \setminus H_1$. The set $H_2 = \{g \in G : g= xh_1x^{-1}$, $\forall h_1 \in H_1\}$ is said to be a \textbf{conjugate subgroup} of $H_1$. We write $H_2 = xH_1x^{-1}$. It is trivial to show that $H_2$ is a subgroup of $G$.
\end{definition}

\begin{remark}
In Lean, to state that two subgroups $H, K$ of a group $G$ are conjugate subgroups similar to how is done in \ref{conj_elem} we can open the \texttt{MulAut} namespace to make use of the custom syntax:

\begin{verbatim}
conj c • H = K 
\end{verbatim}

This notation and API is useful because conjugation by a particular element is defined to be an element in the automorphism group of $G$, $\Aut(G)$. 

This becomes particularly crucial when formalizing the interactions of subgroups with the complete lattice structure on the set of subgroups of a group. 

These interactions and more discussion about this lattice structure will happen later on.
\end{remark}

Conjugation plays an important roll thoughout the paper, in particularly the following properties about conjugate elements and subgroups.

\begin{proposition}\label{conjugateprop} Let $a$, $b$ be conjugate elements of a group $G$ and $A$, $B$ be conjugate subgroups of $G$. Then the following properites hold: \vspace{3mm} \\
(i) If either $a$ or $b$ has finite order, then both $a$ and $b$ have the same order. \vspace{3mm} \\
\end{proposition}
\begin{proof}
    (i) Since $a$ and $b$ are conjugate elements in $G$, $b = xax^{-1}$ for some $x \in G$. Suppose that $b$ has finite order and $b^k = I_G$ for some $k \in \mathbb{Z}^+$,
    \begin{equation*} I_G = b^k = (xax^{-1})^k = xa^{k}x^{-1} \Rightarrow a^k = I_G.
    \end{equation*}
    Alternatively suppose that $a$ has finite order and $a^k = I_G$ for some $k \in \mathbb{Z}^+$,
    \begin{equation*} a^k = I_G \Rightarrow I_G = xa^{k}x^{-1} = (xax^{-1})^k = b^k.
    \end{equation*}
    Thus $a^k = I_G \iff b^k = I_G$. Thus $a$ and $b$ have the same order. \\
\end{proof}

\begin{proposition}
(ii) $A \cong B$. \\
\end{proposition}

\begin{proof}
\\
(ii) Since $A$ and $B$ are conjugate, there exists some $x \in G$ such that $B=xAx^{-1}$. Define the map $\phi$ by,
\begin{align*}
\phi:A &\longrightarrow xAx^{-1}, \\
a_1 &\longmapsto xa_1x^{-1} \tag{$\forall \; a_1 \in A$}. \end{align*}

We show that $\phi$ is a homomorphism between $A$ and $B=xAx^{-1}$.

\begin{equation*}
\phi(a_1a_2) = xa_1a_2x^{-1} = ( xa_1x^{-1})( xa_2x^{-1}) = \phi(a_1) \phi(a_2).
\end{equation*}
\\
Now consider an arbitrary $k \in ker(\phi)$.

\begin{equation*}
k \in ker(\phi) \iff \phi(k) = I_G \iff  xkx^{-1} = I_G \iff k = I_G.
\end{equation*}
\\
So $ker(\phi) = \{ I_G \}$ which means $\phi$ is injective. Now let $b_1 \in B = xAx^{-1}$. Thus $b_1 = xa_1x^{-1}$ for some $a_1 \in A$. Since $a_1 \in A$, $\phi(a_1) = xa_1x^{-1} = b_1$ and so $\phi$ is surjective. Thus $\phi$ is an isomorphism and $A$ and $B$ are isomorphic.

\end{proof}

The final part of this proposition is an important result which shows that since conjugate subgroups are isomorphic, conjugation preserves group structure and properties. In particular, conjugate subgroups have the same cardinality and if one is abelian or cyclic, then so is the other.

\section{Automorphism}

\begin{definition} An \textbf{automorphism} of a group $G$ is a isomorphism from $G$ onto itself. The set of all automorphisms of $G$ forms a group under composition and is denoted by $Aut(G)$.\\
\\
An \textbf{inner automorphism} is an automorphism whereby $G$ acts on itself by conjugation. That is, each $g \in G$ induces a map, $i_g : G \rightarrow G$, where $i_g(x) = g x g^{-1}$ for each $x \in G$. The set of all inner automorphisms is denoted by $Inn(G)$ and is a normal subgroup of $Aut(G)$ (For proof of this see \cite[p.104]{bhattacharya}.
\end{definition}

\section{Direct Product}

\begin{definition} If $G_1, G_2,...,G_n$ are groups, we define a coordinate operation on the Cartesian product $G_1 \times G_2 \times...\times G_n$ as follows:
\begin{align*} (a_1, a_2, ..., a_n) (b_1, b_2, ..., b_n) = (a_1 b_1, a_2 b_2, ..., a_n b_n),
\end{align*}
where $a_i, b_i \in G_i$. It is easy to verify that $G_1 \times G_2 \times...\times G_n$ is a group under this operation. This group is called the \textbf{direct product} of $G_1, G_2,...,G_n$.
\end{definition}

\begin{lemma} \label{directproductN} Let $A$ and $B$ be normal subgroups of $G$ with $A \cap B = \{ I_G \}$. Then $AB \cong A \times B$.
\end{lemma}

\begin{proof}

First note that the elements of $A$ commute with the elements of $B$, since $\forall \; a \in A$ and $b \in B$,
\begin{align*} aba^{-1}b^{-1} &=  a(ba^{-1}b^{-1}) \in A, \tag{since $A \vartriangleleft G$}
\\ aba^{-1}b^{-1} &=  (aba^{-1})b^{-1} \in B. \tag{since $B \vartriangleleft G$}
\end{align*}

Therefore $aba^{-1}b^{-1} \in A \cap B = \{ I_G \}$, and $ab = ba$. \\
\\
Define the operation $*$ on $A \times B$ by $(a_1 , b_1)*(a_2 , b_2) = (a_1 a_2 , b_1 b_2)$. Now define the map $\phi$ by,
\begin{align*}
\phi:A \times B &\longrightarrow AB, \\
(a,b) &\longmapsto ab \tag{$\forall \; a \in A, \; b\in B$}. \end{align*}

We show that $\phi$ is a homomorphism between $A \times B$ and $AB$.
\vspace{-0.5mm}
\begin{align*}
\phi((a_1,  b_1)*(a_2, b_2)) &= \phi (a_1 a_2 , b_1 b_2) \\
&=  a_1 a_2  b_1 b_2 \\
&=  a_1 b_1 a_2 b_2  \\
&= \phi(a_1 , b_1) \phi(a_2 , b_2). \end{align*}

Thus $\phi$ is a homomorphism and clearly surjective. It remains to show that it is injective. 
\vspace{-0.5mm}
\begin{align*} \phi(a_1 , b_1) &= \phi(a_2 , b_2), \\
a_1 b_1 &= a_2 b_2, \\
a_1 b_1 b_2^{-1} &= a_2, \\
b_1 b_2^{-1} &= a_1^{-1} a_2 \in A \cap B.
\end{align*}

Since $A \cap B = \{ I_G \}$, we have $b_1 b_2^{-1} = I_G = a_1^{-1} a_2$ and so $b_1 = b_2$, $a_1 = a_2$ and $\phi$ is injective. So $\phi$ is an isomorphism and $AB \cong A \times B$.
\\
\end{proof}

\begin{lemma}\label{directproductZ}
Let $A$ and $B$ be subgroups of $G$. If $A \cap B = \{ I_G \}$ and $ab = ba$ $\forall a \in A$, $b \in B$. Then $AB \cong A \times B$.
\end{lemma}

\begin{proof} Since $A$ and $B$ commute, the argument outlined in Lemma \ref{directproductN} also holds here.
\end{proof}

% \newpage



\chapter{Reduction of classification of finite subgroups of $\PGL_2(\Fbar_p)$ to classification of finite subgroups of $\PSL_2(\Fbar_p)$}\label{Ch4_ReductionOfProblem}

\section{Over an algebraically closed field $\PSL_n(F)$ is isomorphic to the projective $\PGL_n(F)$}


When $F$ is algebraically closed and $\textrm{char}(F) \neq 2$ it one can construct an isomorphism between 
the projective special linear group and the projective general linear group.

\begin{definition}
\label{SL_monoidHom_PGL}
\lean{SL_monoidHom_PGL}
\leanok
    Let $\varphi : \SL_n(R) \rightarrow \PGL_n(R)$ be the injection of $\PSL_n(R)$ into $\PGL_n(R)$ defined by
    \[
     S \mapsto i(S) \;  (R^\times I) 
    \]

    where $i : \SL_n(F) \hookrightarrow \GL_n(F)$ is the natural injection of the special linear group into the general linear group.
\end{definition}



We now prove a useful fact about elements that belong to the center of $\GL_n(R)$.

\begin{lemma}
    \label{GeneralLinearGroup.mem_center_general_linear_group_iff}
    \lean{GeneralLinearGroup.mem_center_general_linear_group_iff}
    \leanok
     Let $R$ be a commutative ring, then $G \in GL_n(F)$ belongs to center of $\GL_n(R)$, $Z(\GL_n(R))$ if and only if $G = r \cdot I$ where $r \in R^\times$.
    \end{lemma}
    
    \begin{proof}
        \leanok
        \begin{itemize}
        \item Suppose $G \in \GL_n(F)$ belongs to $Z(\GL_n(F))$ then for all $H \in \GL_n(F)$ we have that $G H = H G$. We will find it sufficient to only consider the case where $H$ is a transvection matrices.
        Let $1 \leq i < j \leq n$, then the transvection matrices are of the form $T_{ij} = I + E_{ij}$ where $E_{ij}$ is the standard basis matrix given by
        \[
        E_{{ij}_{kl}} = \begin{cases}
        1 & \text{if $i = k$ and $l = j$}\\
        0 & \text{otherwise}
        \end{cases}
        \] 
    
        Given $T_{ij} G = (I + E_{ij}) G = G T_{ij} (I + E_{ij})$, and addition is commutative we can use the cancellation law to yield that
        
        \[
        E_{ij} G = G E_{ij}
        \]
    
        But $G$ only commutes with $E_{ij}$ for all $i \neq j$ if $G = r \cdot I$ for some $r \in R^\times$.
        
        \item Suppose $G = r \cdot I$ for some $r \in R^\times$ then it is clear that for all $H \in \GL_n(F)$ that $r \cdot I  H = r \cdot H = H \cdot r = H (r \cdot I)$
        \end{itemize}
    \end{proof}


\begin{lemma}
\label{center_SL_le_ker}
\uses{SL_monoidHom_PGL}
\lean{center_SL_le_ker}
\leanok
Let $R$ be a non-trivial commutative ring, then $Z(\SL_n(R)) \le \ker (\varphi)$.
\end{lemma}
\begin{proof}
\uses{GeneralLinearGroup.mem_center_general_linear_group_iff}
\leanok
If $S \in Z(\SL_n(R)) \leq \SL_n(F)$ then $S = \omega I$ where $\omega$ is a primitive root of unity.

Because $\varphi = \pi_{Z(\GL_n(F))} \circ i$, the kernel of $\varphi$ is $i^{-1}(Z(\GL_n(F)))$, where we recall that $i : \SL_n(R) \hookrightarrow \GL_n(F)$ is the injection of $SL_n(F)$ into $\GL_n(F)$.

But given $i(S) = i(\omega \cdot I) = \omega \cdot I$ is of the form $r \cdot I$ where $r \in R^\times$ by \ref{GeneralLinearGroup.mem_center_general_linear_group_iff} it follows that $S \in \ker \varphi$, as desired.
\end{proof}




\begin{definition}
\label{PSL_monoidHom_PGL}
\uses{SL_monoidHom_PGL}
\lean{PSL_monoidHom_PGL}
\leanok
    Given $Z(\SL_n(F)) \leq \ker \varphi$ as shown in \ref{center_SL_le_ker}, by the universal property there exists a unique homomorphism $\bar{\varphi} : \PSL_n(F) \rightarrow \PGL_n(F)$ which is the lift of $\varphi$. 
    
    Where $\varphi = \bar{\varphi} \circ \pi_{Z(\SL_n(F))}$ and $\pi_{Z(\SL_n(F))} : \SL_n(F) \rightarrow \PSL_n(F)$ is the canonical homomorphism from the group into its quotient.
\end{definition}



\begin{lemma}
\label{Injective_PSL_monoidHom_PGL}
\lean{Injective_PSL_monoidHom_PGL}
\uses{PSL_monoidHom_PGL}
\leanok
    The homomorphism $\bar{\varphi}$ is injective.
\end{lemma}
\begin{proof}
\uses{GeneralLinearGroup.mem_center_general_linear_group_iff}
\leanok

To show $\bar{\varphi}$ is injective we must show that $\ker \bar{\varphi} \leq \bot_{\PSL_n(F)}$ where $\bot_{\PSL_n(F)}$ is the trivial subgroup of $\PSL_n(F)$.

Let $[S] \in \PSL_n(F)$ and suppose $[S] \in \ker \bar{\varphi}$. If $[S] \in \ker \bar{\varphi}$ then $\bar{\varphi} ([S]) = [1]_{\PGL_n(F)}$. But on the other hand, $\bar{\varphi} ([S]) = \varphi(s)$ and so $\varphi(S) = 1_{\PGL_n(F)}$, 
and thus $S \in Z(\GL_n(F))$, from \ref{GeneralLinearGroup.mem_center_general_linear_group_iff} it follows that $s = r \cdot I$ for some $r \in R^\times$. But given $S \in \SL_n(F)$ we know that 

\begin{equation*}
    \det(S) = \det(r \cdot I) = r^n \cdot 1 = 1 \implies \text{$r$ is a $n$th root of unity}
\end{equation*}

Therefore, given elements of $Z(\SL_n(F))$ are those matrices of the form $\omega \cdot I$ where $\omega$ is a $n$th root of unity, we can conclude that $[S] = [1]_{\PSL_n(F)}$ and thus $\ker \bar{\varphi} \leq \bot_{\PSL_n(F)}$ as required.

Which shows that the homomorphism $\bar{\varphi}$ is injective.
\end{proof}
    
    \begin{remark}[Quotients and their maps in Lean]
    When formalising results on quotient groups or for that matter any quotient type, it is valuable to appreciate which model Lean uses for quotients. 
    
    Typically, when one thinks of the elements of the quotient group say $\Z/2\Z$ there are two elements: 
    $[0]$ which represents the coset $\{\ldots, -2, 0, 2, \ldots\}$, and $[1]$ which represents the coset $\{\ldots, -3, -1, 1, 3, \ldots\}$
    since under the equivalence relation, $a \sim b$ if and only if $a - b \in 2\Z$. In this new setting all the elements belonging to the same coset, 
    or equivalence class, are now considered to be indistinguishable.

    Similarly, when defining a group homomorphism from $\theta : (\Z /2\Z, \dot{+}) \rightarrow G$, under this model one has to make sure that
    all the elements in $[0]$ are sent to the same element $g \in G$ in the target; and likewise, all the elements of $[1]$ are sent to the same element $h \in H$.

    Otherwise, should $\theta([0]) \ne \theta([2])$ then this would mean that $\theta$ would be treating what we thought were the indistinguishable elements $0$ and $2$, as different,
    This is the idea of showing the \textit{well-definedness} of a map on a quotient.

    In general, one of the ideas of quotients (not only group quotients) is to somehow eliminate redundant information.

    Let us run with the following amusing example in every day life:
    
    Suppose the lights in a room are on, and suppose Bob asks Alice what would happen should he press the light switch $n \in \N$ times. It then occurs to
    Alice that in this setting pressing the light switch on $12$ times or $1400$ times, or for that matter, any even number of times yields indistinguishable outcomes,
     the lights will be on; so in a sense the elements belonging to the set of even numbers are indistinguishable from each other, 
     what is more is that we are not interested in so much the number but the parity of the number.

    In this particular example, we defined a map $\psi : \N \rightarrow \{\text{On}, \; \text{Off}\}$ where we realised
    that both $12$ and $1400$ and all even natural numbers seem to behave equivalently under this map if and only if their difference is an even integer,
    that is, $a \sim b$ if and only if $a - b \in 2\Z$ where $a$ and $b$ are promoted to being elements of $\N \subset \Z$.    
    
    Given this map $\psi$ behaves the same on all elements which are indistinguishable, it seems natural to 
    define a map which now takes in the only relevant information which determines if the lights are on or not,
    the parity of the number of times the light switch has been pressed.
    
    \[
    \bar{\psi} : \N / \sim \rightarrow \{\text{On}, \; \text{Off}\}
    \]

    The quotient on $\N$ now allows us to say treat $[12] = [1400]$, as they are equal sets, and our new map $\bar{\psi}$ now recognises them to be the same
    under this new light. 

    However, we could have also phrased this observation as saying that $\psi$ respects the equivalence relation on $\sim$ and
    thus have defined a $\bar{\psi}$ to be the map which given the parity, an element of the new abstract object $\N / \sim = \{[0], [1]\}$,
    outputs whether the light is on or off. 

    The upshot of all of this is that when we define a quotient and a map from a quotient, we ultimately want such a function to respect the equivalence relation. Whether the elements of a quotient are
    modelled as a coset, a set of equivalent elements, or as an abstract object which satisfies our needs should not be the main concern. 

    This is what the definition of quotients in Lean recognises, but it also recognises that it would be rather strange to think of a term of a quotient type as a set, since it would be clunky to  constantly work with
    the type \texttt{Set (Set s)}, to define a quotient;  instead one simply modifies the definition of equality on terms, and in particular, when wanting to define in Lean the lift of an existing homomorphism $\gamma : G \rightarrow H$
    to $\bar{\gamma} : G/N \rightarrow H$, the most natural way to define/verify such a lift is sensible in Lean is to prove that equivalent elements map to the same output under $\gamma$.

    In fact, this is exactly what the general \texttt{Quot.lift} does:

 

    Similarly, \texttt{QuotientGroup.lift}, the universal property for factor groups, corresponds to:

    

    From this last definition one can see that there is no trace whatsoever to cosets. It is still possible to formalise a such a
    definition in a way that is akin to the notion of well-definedness, which is closer to the model of quotients as sets of subsets,
    since one can for example invoke \texttt{Quotient.exists_rep} which states:

   
    
    In fact, some of the theorems and definitions below heavily rely on this notion.
    Yet it becomes extremely useful later on to come to terms with this model of quotients and their maps which
    hinges on the universal property.
    \end{remark}

Before we can show that $\bar{\varphi}$ is surjective we need the following
lemma which allows us to find a suitable representative for an arbitrary element of $\PGL_n(F)$.

\begin{lemma}
\label{exists_SL_eq_scaled_GL_of_IsAlgClosed}
\lean{exists_SL_eq_scaled_GL_of_IsAlgClosed}
\leanok
If $F$ is an algebraically closed field then for every $G \in \GL_n(F)$ there exists a nonzero constant $\alpha \in F^\times$ and an element $S \in \SL_n(F)$ such that 
\begin{equation*}
    G = \alpha \cdot S
\end{equation*}
\end{lemma}

\begin{proof}
\leanok
Let $G \in \GL_n(R)$ then define
\begin{equation*}
    P(X) := X^n - \det(G)
\end{equation*}

By assumption $F$ is algebraically closed and $\det(G) \in F^\times$ thus there exists a root $\alpha \in F^\times$ such that 

\begin{equation*}
    \alpha^n - \det(G) = 0 \iff \alpha = \sqrt[n]{\det(G)} 
\end{equation*}

Let $S = \frac{1}{\alpha} \cdot G$, by construction $S \in \SL_n(F)$ as 

\begin{equation*}
    \det(S) = \left(\frac{1}{\alpha^n}\right) \cdot \det(G) = \frac{1}{\det(G)} \det(G) = 1
\end{equation*}
\end{proof}


\begin{lemma}
\label{Surjective_PSL_monoidHom_PGL}
\uses{PSL_monoidHom_PGL}
\lean{Surjective_PSL_monoidHom_PGL}
\leanok
    The map $\bar{\varphi}$ is surjective.
\end{lemma}
\begin{proof}
\uses{exists_SL_eq_scaled_GL_of_IsAlgClosed}
\leanok
    Let $G \; (F^\times I) = [G] \in \PGL_n(F)$, then $G \in \GL_n(F)$ we can find a representative of $[G']$ that lies within the special linear group.
    Given elements of the special linear group are matrices with determinant equal to one, we must scale $G$ to a suitable factor to yield a representative which lies within $\SL_n(F)$. Suppose $\det(G) \ne 1$ and let
    \[
    P(X) := X^n - \det(G) \in F[X]
    \]
    By assumption, $F$ is algebraically closed so there exists a root $\alpha \ne 0\in F$ such that 
    \[
    \alpha^n - \det(G) = 0 \iff \alpha^n = \det(G)
    \]
    We can define
    \[
    G' := \frac{1}{\alpha} \cdot G \quad \text{where} \quad \det(G') = \frac{1}{\alpha^n} \det(G) = 1.
    \]
    Thus $G' \in \SL_n(F) \leq \GL_n(F)$ and given $G' = \frac{1}{\alpha} G$ we have that $G'  \; (F^\times I) = G \; (F^\times I)$.
    
    Therefore, $\varphi(G') = i(G') (F^\times I) = G' (F^\times I) = G (F^\times I)$.
\end{proof}


\begin{lemma}
\label{Bijective_PSL_monoidHom_PGL}
\uses{PSL_monoidHom_PGL}
\lean{Bijective_PSL_monoidHom_PGL}
\leanok
    The map $\bar{\varphi}$ is bijective
\end{lemma}
\begin{proof}
\uses{Injective_PSL_monoidHom_PGL, Surjective_PSL_monoidHom_PGL}
\leanok
 We have shown that $\bar{\varphi}$ is injective in \ref{Injective_PSL_monoidHom_PGL} and have shown that $\bar{\varphi}$ is surjective in \ref{Surjective_PSL_monoidHom_PGL}. 
 Therefore, $\bar{\varphi}$ defines a bijection from $\PSL_n(F)$ to $\PGL_n(F)$.
\end{proof}


\begin{theorem}
\label{PGL_iso_PSL}
\uses{PSL_monoidHom_PGL}
\lean{PGL_iso_PSL}
\leanok
    If $F$ is an algebraically closed field, then the map $\bar{\varphi} : \PSL_n(F) \rightarrow \PGL_n(F)$ defines a group isomorphism between $\PSL_n(F)$ and $\PGL_n(F)$.
\end{theorem}

\begin{proof}
\uses{Bijective_PSL_monoidHom_PGL}
\leanok
    The map $\bar{\varphi}$ was shown to be a bijection in \ref{Bijective_PSL_monoidHom_PGL} and given $\bar{\varphi}$ is mulitplicative as it was defined to be the lift of the homomorphism $\varphi$, we can conclude that 
    $\bar{\varphi}$ defines a group isomorphism between $\PSL_n(F)$ and $ºPGL_n(F)$
\end{proof}


\begin{remark}[Noncomputable]
    Observe in the definition above it was necessary to add the \texttt{noncomputable} keyword before the definition, the reason for this is
    because the result \texttt{MulEquiv.ofBijective} implicitly uses the axiom of choice which means it is not possible for Lean to generate
    executable code.
\end{remark}



% \begin{center}
% \begin{tikzcd}
% 	{\SL_n(F)} && {\SL_n(F)} \\
% 	&& {} \\
% 	{\PSL_n(F)} && {\PGL_n(F)}
% 	\arrow["i", from=1-1, to=1-3]
% 	\arrow["{\textrm{can}_{\langle-I\rangle}}"', from=1-1, to=3-1]
% 	\arrow["{\textrm{can}_{F^\times I}}", from=1-3, to=3-3]
% 	\arrow[dotted, from=3-1, to=3-3]
% \end{tikzcd}
% \end{center}
% \end{proof}

This isomorphism will be essential to the classification of finite subgroups of $\PGL_2(\bar{\F}_p)$, as we only need understand a the classification of subgroups of $\PSL_2(\Fbar_p)$ to reach the desired result.


\section{Christopher Butler's exposition}

Following from the isomorphism defined in the previous section, we can now proceed to classify the finite subgroups of $\PGL_2(\bar{\F}_p)$ by classifying the finite subgroups of $\PSL_2(\bar{\F}_p)$. 
In turn, one can begin classifying the finite subgroups of $\PSL_2(\Fbar_p)$ by classifying the finite subgroups of $\SL_2(\Fbar_p)$ and then considering what happens after
quotienting by the center, $Z(\SL_2(F)) = \langle -I\rangle$.

We now turn our attention to the more general setting when $F$ is an arbitrary field that is algebraically closed, as this will turn out to be sufficient for our purposes.

Given $|\langle -I \rangle| = 2$ when $\textrm{char} F \ne 2$; and $\langle -I\rangle = \bot$ when $\textrm{char} F = 2$.
When a finite subgroup of $\SL_2(F)$ is sent through the canonical mapping $\pi_{Z(\SL_2(F))} : \SL_2(F) \rightarrow \PSL_2(F)$ 
the resulting subgroup will either shrink by a factor of two or it will remain intact should the center not be contained within the subgroup. 

We now proceed to classify all finite subgroups of $\SL_2(F)$ when $F$ is algebraically closed field. 
From now on, we follow Christopher Butler's exposition of Dickson's classification of finite subgroups of $\SL_2(F)$ over an algebraically closed field $F$. 

Christopher has been kind enough to provide the TeX code so I could prepare this blueprint which crucially hinges on the result which his exposition \cite{butler} covers.
\chapter{Properties of the two dimensional $\SL_2(F)$}\label{Ch5_PropertiesOfSLOverAlgClosedField}


\section{General Notation}

Throughout this paper, $F$ will denote an arbitrary algebraically closed field. 
The letter $p$ will be used to denote the characteristic of $F$. 
Recall that the definition of the characteristic of a field is:

\begin{definition}[Characteristic of a field]
    Let $F$ be a field, the characteristic of a field, denoted by $\textrm{char}(F) \in \N$, is the smallest natural number $p \in \N_0$ such that

    \[
    \underbrace{1 + \ldots + 1}_{p} = 0
    \]

    where in the case there is no such number then $p = 0$.
\end{definition}

\begin{example}
    $\Z /p\Z$ is a field of characteristic $\textrm{char}(\Z/p\Z) = p$ as $p \cdot 1 = 0$.
\end{example}

\begin{example}
    The field $\Q$ is a field with $\textrm{char}(\Q) = 0$ as $n \cdot 1 \ne 0$ for all $n \in \N \subset \Q$.
\end{example}

\begin{remark}[The characteristic is either prime or zero]
    The characteristic of a field is either a prime number or zero.
\end{remark}

Unless otherwise stated, the letters $\alpha, \beta, \gamma, \delta$ and $\sigma$ will denote elements of $F$; 
whereas $\delta$ and $\rho$ will denote elements of $F^\times$, where $F^\times$ are the invertible, or equivalently, non-zero elements of $F$.

\section{Subsets of $\SL_2(F)$}

In this chapter we make some useful observations about specific elements and subgroups of $\SL_2(F)$. 

First, we define the following elements of $\SL_2(F)$.

\subsubsection{Special matrices of $\SL_2(F)$}

\begin{definition}[The diagonal matrix of $SL_2(F)$]
\label{SpecialMatrices.d}
\lean{SpecialMatrices.d}
\leanok
    Given an element $\delta \in F^\times$ we define the diagonal matrix:
    \[
    d_\delta = \begin{bmatrix}
        \delta & 0\\
        0 & \delta^{-1}
    \end{bmatrix}
    \]
\end{definition}


\begin{remark}[Constructing a term of $\SL_2(F)$]
    To construct a term of $\SL_2(F)$ one has to bear in mind that the special linear group is defined to
    be a subtype of matrices with determinant one, thus, in the \textit{anonymous constructor}
    one has to provide:

    \begin{itemize}
        \item The term of type \texttt{Matrix (Fin 2) (Fin 2) F}.
        \item The proof term that proves that the matrix term of type \texttt{Matrix (Fin 2) (Fin 2) F} has determinant one.
    \end{itemize}
\end{remark}

\begin{definition}[The shear matrix of $SL_2(F)$]
\label{SpecialMatrices.s}
\lean{SpecialMatrices.s}
\leanok
    Given an element $\delta \in F$ we define the shear matrix:
    \[
    s_\sigma  = \begin{bmatrix}
    1 & 0\\
    \sigma & 1
    \end{bmatrix}
    \]
\end{definition}


\begin{definition}[Rotation by $\pi / 2$ radians matrix]
\label{SpecialMatrices.w}
\lean{SpecialMatrices.w}
\leanok
 We denote the matrix which corresponds to a rotation by $\pi / 2$ radians to be:
 \[
 w = \begin{bmatrix}
    0 & -1\\
    1 & 0
 \end{bmatrix}
 \]
\end{definition}


The matrices $d$, $s$ and $w$ satisfy the following relations:


\begin{lemma}[Closure of $D$ under multiplication]
\label{SpecialMatrices.d_mul_d_eq_d_mul}
\uses{SpecialMatrices.d}
\lean{SpecialMatrices.d_mul_d_eq_d_mul}
\leanok
For any $\delta, \rho \in F^\times$ we have that
\[
d_\delta d_\rho = d_{\delta\rho}
\]
\end{lemma}
\begin{proof}
\leanok
    We verify by matrix multiplication that indeed:

    \begin{equation*}
        d_\delta d_\rho = \begin{bmatrix} \delta & 0 \\ 0 & \delta^{-1} \end{bmatrix} \begin{bmatrix} \rho & 0 \\ 0 & \rho^{-1} \end{bmatrix} = 
        \begin{bmatrix} \delta \rho & 0 \\ 0 & \delta^{-1} \rho^{-1} \end{bmatrix} = d_{\delta \rho}.
    \end{equation*}
\end{proof}


\begin{lemma}[Closure of $S$ under multiplication]
\label{SpecialMatrices.s_mul_s_eq_s_add}
\uses{SpecialMatrices.s}
\lean{SpecialMatrices.s_mul_s_eq_s_add}
\leanok
    For any $\sigma, \gamma \in F$ we have that
    \[
    s_\sigma s_\gamma = s_{\sigma + \gamma}.
    \]
\end{lemma}
\begin{proof}
\leanok
    We verify by matrix multiplication that indeed:
\begin{equation*}
    s_\sigma s_\gamma = \begin{bmatrix} 1 & 0 \\ \sigma & 1 \end{bmatrix} \begin{bmatrix} 1 & 0 \\ \gamma & 1 \end{bmatrix} = \begin{bmatrix} 1 & 0 \\ \sigma + \gamma & 1 \end{bmatrix} = s_{\sigma + \gamma}.
\end{equation*}
\end{proof}


\begin{lemma}
    \label{SpecialMatrices.s_pow_eq_s_mul}
    \uses{SpecialMatrices.s}
    \lean{SpecialMatrices.s_pow_eq_s_mul}
    \leanok
    For any $\sigma \in F$ and for any $n \in \N$, we have that $s_\sigma^n = s_{n \cdot \sigma}$
\end{lemma}
\begin{proof}
\uses{SpecialMatrices.s_mul_s_eq_s_add}
\leanok
    We prove this by induction, indeed for $n= 0$ the identity holds trivially.

    Suppose $s_\sigma^n = \begin{bmatrix}
        1 & 0\\
        n \cdot \sigma & 0\end{bmatrix}$ then consider $s_\sigma^{(n + 1)}$. Since 

        \[
        s_\sigma^{(n + 1)} = s_\sigma^n s_\sigma = s_{n \cdot \sigma} s_\sigma = s_{(n + 1)\sigma}
        \]
\end{proof}


\begin{lemma}[Order of nontrivial $s_\sigma$ ]
    \label{SpecialMatrices.order_s_eq_char}
    \uses{SpecialMatrices.s}
    \lean{SpecialMatrices.order_s_eq_char}
    \leanok
    The order of $s_\sigma$ for any $\sigma \ne 0$ is $\textrm{char}(F)$
    \end{lemma}
    
    \begin{proof}
    \uses{SpecialMatrices.s_pow_eq_s_mul}
    \leanok
    Let $p$ denote the characteristic of the field, and let $\sigma \in F$, by \ref{SpecialMatrices.s_pow_eq_s_mul} we know that for any $s_\sigma^p = s_{p \cdot \sigma}$. 
    Since $p$ is the characteristic of the field, we have that $p \cdot \sigma = 0$, and so $s_{p \cdot \sigma} = s_0 = I$
    \end{proof}


\begin{lemma}
\label{SpecialMatrices.d_mul_s_mul_d_inv_eq_s}
\uses{SpecialMatrices.d, SpecialMatrices.s}
\lean{SpecialMatrices.d_mul_s_mul_d_inv_eq_s}
\leanok
    We have that for all $\delta \in F^\times$ and $\sigma \in F$
    \[
    d_\delta s_\sigma d^{-1}_\delta = s_{\sigma \delta^{-2}}.
    \]
\end{lemma}
\begin{proof}
\leanok
    We verify by matrix multiplication that indeed:

    \begin{equation*}
        d_\delta s_\sigma d^{-1}_\delta = \! \begin{bmatrix} \delta & 0 \\ 0 & \delta^{-1} \end{bmatrix} \begin{bmatrix} 1 & 0 \\ \sigma & 1 \end{bmatrix} \begin{bmatrix} \delta^{-1} & 0 \\ 0 & \delta \end{bmatrix} = \begin{bmatrix} \delta & 0 \\ 0 & \delta^{-1} \end{bmatrix} \! \begin{bmatrix} \delta^{-1} & 0 \\ \sigma \delta^{-1} & \delta \end{bmatrix} \! = \! \begin{bmatrix} 1 & 0 \\ \sigma \delta^{-2} & 1 \end{bmatrix} \! = s_{\sigma \delta^{-2}}.
    \end{equation*}
\end{proof}




\begin{lemma}
\label{SpecialMatrices.w_mul_d_eq_d_inv_w}
\uses{SpecialMatrices.d, SpecialMatrices.w}
\lean{SpecialMatrices.w_mul_d_eq_d_inv_w}
\leanok
For any $\delta \in F^\times$ we have:
\[ 
w d_\delta w^{-1} = d^{-1}_\delta.
\]
\end{lemma}
\begin{proof} 
\leanok
We verify by matrix multiplication that indeed
\begin{align*}
w d_\delta w^{-1} &= \begin{bmatrix} 0 & 1 \\ - 1 & 0 \end{bmatrix} \begin{bmatrix} \delta & 0 \\ 0 & \delta^{-1} \end{bmatrix} \begin{bmatrix} 0 & - 1 \\ 1 & 0 \end{bmatrix}\\
&=  \begin{bmatrix} 0 & 1 \\ - 1 & 0 \end{bmatrix} \begin{bmatrix} 0 & - \delta \\ \delta^{-1} & 0 \end{bmatrix}\\
&= \! \begin{bmatrix} \delta^{-1} & 0 \\ 0 & \delta \end{bmatrix} \!= d^{-1}_\delta. 
\end{align*}
\end{proof}


We can now express familiar kinds of matrices of $\SL_2(F)$ in terms of these three matrices:

First we note the following observations:
\begin{corollary}
    \label{det_eq_mul_diag_of_lower_triangular}
    \lean{det_eq_mul_diag_of_lower_triangular}
    \leanok
    The determinant of a $2 \times 2$ lower triangular matrix, $M$, is the product of the diagonal entries $\det(M) = M_{11} M_{22}$.
\end{corollary}
\begin{proof}
\leanok
We use the $2 \times 2$ determinant formula.
\end{proof}


\begin{corollary}
    \label{SpecialLinearGroup.fin_two_diagonal_iff}
    \lean{SpecialLinearGroup.fin_two_diagonal_iff}
    \leanok
    A $2 \times 2$ matrix of $\SL_2(F)$, $x$ is a diagonal matrix if and only if $x = d_\delta$ for some $\delta \in F^\times$.
\end{corollary}
\begin{proof}
\uses{det_eq_mul_diag_of_lower_triangular}
\leanok
    Since $x$ is diagonal and belongs to the special linear group, the determinant is $x_{11} x_{22} = 1$ which shows $x_{11} = x_{22}^{-1}$, as required.
\end{proof}


\begin{corollary}
    \label{SpecialLinearGroup.fin_two_shear_iff}
    \uses{SpecialMatrices.s, det_eq_mul_diag_of_lower_triangular}
    \lean{SpecialLinearGroup.fin_two_shear_iff}
    \leanok
    A matrix of $\SL_2(F)$, $x$ is a shear matrix, that is of the form $\begin{bmatrix}
        \alpha & 0\\
        \sigma & \alpha
    \end{bmatrix}$ if and only if either $x = s_\sigma$ or $x = - s_\sigma$ for some $\sigma \in F$.
\end{corollary}
\begin{proof}
\leanok
Again using the formula for the determinant of a $2 \times 2$ matrix to show that indeed if $x$ is a shear matrix in the special linear group 
then $\alpha^2 = 1$ which shows $\alpha = \pm 1$, as required.
\end{proof}


\begin{corollary}
    \label{SpecialLinearGroup.fin_two_antidiagonal_iff}
    \lean{SpecialLinearGroup.fin_two_antidiagonal_iff}
    \leanok
    A matrix $A \in \SL_2(F)$ is anti-diagonal, that is of the form $\begin{bmatrix}
        0 & \beta\\
        \gamma & 0
    \end{bmatrix}$ if and only if $A = d_\delta w$
\end{corollary}
\begin{proof}
\leanok
This is shown by direct computation, we observe that $w$ flips the rows and changes the sign of one the flipped rows to account for the determinant needing to be equal to one.
\end{proof}


From these relations we can now single out the following subgroups of $\SL_2(F)$.

\subsubsection{Special subgroups of $\SL_2(F)$}

\begin{definition}[The subgroup of diagonal matrices]
\label{SpecialSubgroups.D}
\lean{SpecialSubgroups.D}
\leanok
    The set of diagonal matrices with matrix multiplication is a subgroup of $\SL_2(F)$: 
    \[
    D = \{d_\delta \; | \; \delta \in F^\times \} = \left\{ \begin{bmatrix}\delta & 0\\ 0 & \delta^{-1}\end{bmatrix} \; | \; \delta \in F^\times \right\}
    \]
\end{definition}


\begin{definition}[The subgroup of shear matrices]
\label{SpecialSubgroups.S}
\lean{SpecialSubgroups.S}
\leanok
    The set of shear matrices with matrix multiplication is a subgroup of $\SL_2(F)$:
    \[
    S = \{s_\sigma \; | \sigma \in F\} = \left\{\begin{bmatrix}1 & 0\\ \sigma & 1\end{bmatrix} \; | \; \sigma \in F \right\}
    \]
\end{definition}


\begin{definition}[The subgroup of lower triangular matrices]
\label{SpecialSubgroups.L}
\lean{SpecialSubgroups.L}
\leanok
    The set of lower triangular matrices (see below) with matrix multiplication is a subgroup of $\SL_2(F)$
    \[
    L = DS
    \]
    where $DS = \{d_\delta s_\sigma \; | \; \delta \in F^\times \text{ and } \sigma \in F \}$ is the pointwise product of $D$ and $S$.
\end{definition}


\begin{definition}[The subgroup of containing diagonal and antidigonal matrices]
    \label{SpecialSubgroups.DW}
    \lean{SpecialSubgroups.DW}
    \leanok
    The set of all diagonal and anti-diagonal matrices with matrix multiplication is a subgroup of $\SL_2(F)$

    \begin{equation}\label{antidiag} DW = \langle D, w\rangle  = \{d_\delta \} \cup \{ d_\delta w \} 
        % =  \left\{  \begin{bmatrix} \delta & 0 \\ 0 & \delta^{-1} \end{bmatrix} \begin{bmatrix} 0 & 1 \\ -1 & 0 \end{bmatrix} \right\} 
        % = \left\{ \begin{bmatrix} 0 & \delta \\ -\delta^{-1} & 0 \end{bmatrix}  \right\}. 
    \end{equation}
\end{definition}


\begin{remark}
    It is possible to have specified the subgroup $DW$ in \ref{SpecialSubgroups.DW} as the supremum $D \sqcup \langle w \rangle$
    but then it would require some additional work to show that the underlying set is indeed $D \cup Dw$.
\end{remark}


\begin{corollary}
    \label{mem_L_iff_lower_triangular}
    \uses{SpecialMatrices.d, SpecialMatrices.s}
    \lean{mem_L_iff_lower_triangular}
    \leanok
    The subgroup $L \le \SL_2(F)$ is the subgroup of $2 \times 2$ lower triangular matrices with determinant one, $L =\left\{\begin{bmatrix}
    \alpha & 0\\
    \gamma & \delta
    \end{bmatrix} \; | \; \alpha, \gamma, \delta \in F \text{ and } \alpha \delta = 1 \right\}$.
\end{corollary}
\begin{proof}
\leanok
    Observe that for every $l \in L$ there is some $\delta \in F^\times$ and $\sigma \in F$ such that $l  = d_\delta s_\sigma = \begin{bmatrix}
        \delta & 0\\
        \sigma * \delta^{-1} & \delta^{-1}
    \end{bmatrix}$ which is lower triangular. 
    
    Furthermore, for every lower triangular matrix $L = \begin{bmatrix}
        a & 0\\
        b & c
    \end{bmatrix}$ 
    
    Setting $\delta = a \in F^\times$ as $a d = 1$ and setting $\sigma = a c$

    indeed yields the equality

    \[
    d_\delta s_\sigma = \begin{bmatrix}
        a & 0\\
        c & d
    \end{bmatrix}
    \]

    Thus $L = D S$ is the set of lower triangular matrices.
\end{proof}


\begin{remark}
    To define the subgroups $D$, $S$ and $L$ in Lean. 
    
    One has to:
    
    \begin{enumerate}
        \item Specify what the underlying set is, what is called the \texttt{carrier}.
        \item Prove that the set is closed under multiplication, that is, provide a proof term to the field \texttt{mul\textunderscore mem'}.
        \item Prove that the set contains the identity element of the group, that is, provide a proof term to the field \texttt{one\textunderscore mem'}.
        \item Show that the group is closed under the inversion operator $(-)^{-1}$, \texttt{inv\textunderscore mem'}.
    \end{enumerate}

    Once these four fields have been filled in, one has succesfully defined a subgroup in Lean.
\end{remark}


\begin{remark}
    Despite the definition of $L$ as being $D S$, some work has to be shown that indeed $DS = D \sqcup S$.
    
    If either $D$ or $S$ were normal in $\SL_2(F)$, this fact would be immediate as we would be able to use \texttt{mul\_normal} or \texttt{normal\_mul}:
    
   
    

    However, given neither $D$ or $S$ are normal in $\SL_2(F)$ slightly more work is needed to show this.
    
    It is interesting how Lean really forces either increased understanding or increased frustration.
\end{remark}

These elements and subgroups are fundamental to this paper and the notation will be used throughout.

\begin{definition}[$(D, \cdot) \cong (F^\times, \cdot)$]
\label{SpecialSubgroups.D_iso_units}
\uses{SpecialSubgroups.D, SpecialMatrices.d_mul_d_eq_d_mul}
\lean{SpecialSubgroups.D_iso_units}
\leanok
The map $\psi : F^\times \overset{\sim}{\rightarrow} D$ defined by $\delta \mapsto d_\delta$ defines a group isomorphism.
\end{definition}

\begin{proof}
    \leanok
    The function $\psi: F^\times \rightarrow D$ defined by $\psi(\delta) = d_\delta$ is a homomorphism between the group $F^\times$ under normal multiplication and $D$ under normal matrix multiplication:
\begin{align*} 
  \psi(\delta \rho) = d_{\delta \rho} =  d_\delta d_\rho = \psi(\delta) \psi(\rho). 
\end{align*}
Observe that $\psi$ is trivially injective and surjective and thus an isomorphism. So $D\cong F^\times$ and $D$ is a subgroup of $L$.\\
\end{proof}




\begin{definition}[ $(S, \cdot) \cong (F, +)$ ]
\label{SpecialSubgroups.S_iso_F}
\uses{SpecialSubgroups.S}
\lean{SpecialSubgroups.S_iso_F}
\leanok
    The map $\phi : F \overset{\sim}{\rightarrow} S$ defined by $\sigma \mapsto s_\sigma$ defines a group isomorphism.
\end{definition}

\begin{proof}
\uses{SpecialMatrices.s_mul_s_eq_s_add}
\leanok
     The function $\phi: F \rightarrow T$ defined by $\phi(\sigma) = s_\sigma$ is a homomorphism between the group $F$ under addition and $S$ under normal matrix multiplication:
\begin{align*} \phi(\sigma + \gamma) = s_{\sigma + \gamma} = s_\sigma s_\gamma = \phi(\sigma) \phi(\gamma).
\end{align*}
It is clear that $\phi$ is injective and surjective and thus an isomorphism. So $ S \cong F$ and $S$ is a subgroup of $L$. \\
\end{proof}


\begin{remark}[Multiplicative]
    Putting the keyword \texttt{Multiplicative} in front a structure which carries an additive structure
    it creates a copy of the additive structure and carries it over to be defined as a multiplicative structure
    on the type. 
\end{remark}


\begin{lemma}
\label{SpecialSubgroups.normal_S_subgroupOf_L}
\lean{SpecialSubgroups.normal_S_subgroupOf_L}
\leanok
$S$ is a normal subgroup of $L$
\end{lemma}
\begin{proof}
    \leanok
    Let $s_\gamma$ and $d_\delta s_\sigma$ be arbitrary elements of $S$ and $L$ respectively. Conjugating $s_\gamma$ by $d_\delta s_\sigma$ gives,
\begin{align*} (d_\delta s_\sigma) s_\gamma (d_\delta s_\sigma)^{-1} &= (d_\delta s_\sigma) s_\gamma (s^{-1}_\sigma d^{-1}_\delta) \\[1.5ex]
&=
d_\delta (s_\sigma s_\gamma s_{-\sigma}) d^{-1}_\delta \qquad \tag{since $s^{-1}_\sigma=s_{-\sigma}$} \\[1.5ex] 
&=
d_\delta s_\gamma d^{-1}_\delta \\[1.5ex] 
&= s_{\gamma \delta^{-2}} \in S. 
\end{align*}
Since $s_\gamma$ was chosen arbitrarily from $\SL_2(F)$we have ($d_\delta s_\sigma) S (d_\delta s_\sigma)^{-1} = S$ and since $d_\delta s_\sigma$ was chosen arbitrarily from $L$, we have that $S \vartriangleleft L$. \\
\end{proof}


\begin{remark}[Subgroups of subgroups in Lean]
    \label{lattice}
    In Lean, $S$ is considered to be a subgroup of $\SL_2(F)$, yet it it is also a subgroup of $L$. 
    
    It is fairly easy to see that $S \not\lhd \SL_2(F)$, so when we say that $S \lhd L$, 
    we are implicitly restricting $S$ to be a subset of $L$ and thus we are actually thinking about the subgroup $S \cap L$,
    but in fact this does not change anything because $S = S \sqcap L$ as $S \le \SL_2(F)$.

    Informally we do not think twice about this, but when formalising this we do need to be clear which is the ambient group for $S$
    to be normal and for $S$ to be normal in an ambient group, it must be considered to be a subgroup of $L$, rather than $\SL_2(F)$.
    
    So this is why the informal statement corresponds to the formal statement:

    

    This example highlights how as useful as it is that Lean keeps track of what the ambient groups are, it can be tedious to change the perspective from which we view the object, where in this case we restricted
    a subgroup to be a subgroup of another subgroup that contains it. One of the challenges of Lean is becoming comfortable with these \textit{coercion}.
    
    On the positive side, the automation Lean offers, that is, the tactics and the unification algorithm (the algorithm which allows you to substitute equal terms when say you use the \texttt{rw} tactic) are continually being refined, 
    and it is increasingly able to do a lot of this bookkeeping on without human aid.
\end{remark}


\begin{lemma}
\label{SpecialSubgroups.D_join_S_quot_S_subgroupOf_D_join_S_mulEquiv_D_subgroupOf_D_join_S}
\uses{SpecialSubgroups.D, SpecialSubgroups.S}
\lean{SpecialSubgroups.D_join_S_quot_S_subgroupOf_D_join_S_mulEquiv_D_subgroupOf_D_join_S}
\leanok
    $L / S \cong D$.
\end{lemma}
\begin{proof} 
\uses{SpecialSubgroups.normal_S_subgroupOf_L}
\leanok
The function $\pi: L \rightarrow D$ defined by $\pi(d_\delta s_\sigma) = d_\delta$ is a homomorphism between $L$ under normal matrix multiplication and $D$ under normal matrix multiplication:
\begin{align*} \pi(d_\delta s_\sigma d_\rho s_\gamma) &= \pi(d_\delta d_\rho s_\sigma s_\gamma) \tag{where $\sigma = \sigma \rho^{2}$}
\\ &= d_\delta d_\rho
\\ &= \pi(d_\delta s_\sigma)\pi(d_\rho s_\gamma).
\end{align*}

We see that $\pi$ is trivially surjective and has kernel
\begin{align*}  \ker(\pi) &= \{ d_\delta s_\sigma \in L : \pi(d_\delta s_\sigma) = I_{\SL_2(F)}\} = S.
\end{align*}
Thus by the First Isomorphism Theorem,
\begin{align*} L / \ker(\pi) &\cong \text{Im}(\pi), \\
L / &\cong D.
\end{align*}
\end{proof}



\begin{remark}
Interestingly, this proof was quite hard to formalise for reasons I will expand on below, but first let me introduce some ideas.

There are two complete lattice structures at play here:
\begin{enumerate}
  \item One where the top element is $\top = \SL_2(F)$
  \item Another, where the top element is $\top = D \sqcup S$, this lattice is a sublattice of the first one.
\end{enumerate}  

The second sublattice is crucial because we need $S$ to be normal in an ambient group, and clearly $S \not\lhd \SL_2(F)$; therefore when restricting $S$ to begin a
subgroup of $D \sqcup S = L$. 

Given $S$ is a subgroup of $D \sqcup S = L$ since $S \le S\sqcup D = L$ and by \ref{SpecialSubgroups.normal_S_subgroupOf_L}
we know $S$ is normal in $L$.

We can then use then define the desired isomorphism by theorem
\texttt{QuotientGroup.quotientInfEquivProdNormalQuotient} which corresponds to the statement:



Which is in fact the second isomorphism theorem! Not the first isomorphism theorem! 

Which contrasts to how the statement was proved informally, where in for this particular theorem,
\texttt{QuotientGroup.quotientInfEquivProdNormalQuotient}, \texttt{H} is specialized to:




And \texttt{N} is specialized to:



Recall that within Lean, \texttt{F} denotes the base field for $\SL_2(F)$, $D$ and $S$.

Written informally, it then corresponds to the desired statement

\[
D \cong \frac{D}{\bot} = \frac{D}{S \sqcap D} \cong \frac{D \sqcup S}{S} = \frac{L}{S}
\]
\end{remark}

\section{The Center of $\SL_2(F)$}

\begin{definition}
% \lean{Subgroup.center}
% \leanok
The \textbf{center} $Z(G)$ of a group $G$ is the set of elements of  $G$ that commute with every element of $G$.
\begin {equation*} Z(G) = \{ z \in G : \forall g \in G, \hspace{6pt} gz=zg \}. \end{equation*}
It is an immediate observation that $Z(G)$ is a normal subgroup of $G$, 
since for each $z \in Z$, $gzg^{-1} = gg^{-1}z = z$, $\forall g \in G$. It's also clear that a group is abelian if and only if $Z(G)=G$.
\end{definition}

\begin{definition}
\label{SpecialSubgroups.Z}
\lean{SpecialSubgroups.Z}
\leanok
    Let $R$ be a commutative ring and define $Z$ to be the subgroup generated by $- I \in \SL_2(R)$
\end{definition}

\begin{remark}[Z as the subgroup closure of $\{-I\}$]
    Observe that the subgroup generated by an element $g \in G$, $\langle g \rangle$, 
    can be thought of more generally within any lattice (such as the lattice on modules) as the closure of a singleton set ${g}$. 
    
    Therefore, the subgroup generated by $-I$ is equal to
    
    \[\langle -I \rangle = \overline{\{-1\}} = \inf \{ K \le G \; | \; \{-1\} \subseteq K \}.\]

    When taking the closure of a singleton within the subgroup lattice; 
    the closure corresponds to taking the powers of the element in the singleton $\{g\}$,
    which is what is typically understood as the subgroup generated by $g$.

    The way $Z$ is defined in Lean is thus:

   
\end{remark}

\begin{corollary}
\label{SpecialSubgroups.closure_neg_one_eq}
\lean{SpecialSubgroups.closure_neg_one_eq}
\leanok
The subgroup closure of the singleton $\{-I\}$, or equivalently, the subgroup generated by $-I$ equals $\overline{\{-I\}} = \{I, -I\}$
\end{corollary}
\begin{proof}
\leanok
Since $-1^2 = 1$, we have that $-I^2 = I$ and thus the result follows.
\end{proof}



\begin{lemma}
\label{SpecialSubgroups.center_SL2_eq_Z}
\uses{SpecialSubgroups.Z}% Matrix.SpecialLinearGroup.mem_center_iff}
\lean{SpecialSubgroups.center_SL2_eq_Z}
\leanok
The center $Z(\SL_2(F)) = \langle - I_{\SL_2(F)}\rangle = Z$.
\end{lemma}
\begin{proof} 
\leanok
    Take an arbitrary element $x=\begin{bmatrix} \alpha & \beta \\ \gamma & \delta \end{bmatrix} \in \SL_2(F)$and  an arbitrary element $z = \begin{bmatrix} z_1 & z_2 \\ z_3 & z_4 \end{bmatrix} \in Z$ and consider their product:

\begin{align}\label{myeq1} zx = \begin{bmatrix} z_1 & z_2 \\ z_3 & z_4 \end{bmatrix} \begin{bmatrix} \alpha & \beta \\ \gamma & \delta \end{bmatrix} &= \begin{bmatrix} \alpha & \beta \\ \gamma & \delta \end{bmatrix} \begin{bmatrix} z_1 & z_2 \\ z_3 & z_4 \end{bmatrix} = xz, \nonumber \\[1.5ex]
\begin{bmatrix} z_1 \alpha + z_2 \gamma & z_1 \beta + z_2 \delta \\ z_3 \alpha + z_4 \gamma & z_3 \beta + z_4 \delta \end{bmatrix} &= \begin{bmatrix} z_1 \alpha + z_3 \beta & z_2 \alpha + z_4 \beta \\ z_1 \gamma + z_3 \delta & z_2 \gamma + z_4 \delta \end{bmatrix}.
\end{align}

\noindent Equating either the top left or bottom right entries, we see that $z_2 \gamma = z_3 \beta$. Since $\beta$ and $\gamma$ can take any values in $F$, for equality to always hold we must have $z_2 = 0 = z_3$. Hence equation (\ref{myeq1}) simplifies to

\begin{equation*} \begin{bmatrix} z_1 \alpha & z_1 \beta \\ z_4 \gamma & z_4 \delta \end{bmatrix} = \begin{bmatrix} z_1 \alpha & z_4 \beta \\ z_1 \gamma & z_4 \delta \end{bmatrix}. \end{equation*}

Thus 
\begin{equation*} 
    z_1 = z_4 \qquad  \text{and} \qquad z =  
    \begin{bmatrix} z_1 & 0 \\ 0 & z_1 \end{bmatrix}. 
\end{equation*}
Since we are working in the special linear group, det$(z)=1$, thus $z_1 = \pm 1$ and $Z = \langle - I_{\SL_2(F)}\rangle$ as required. Observe that this is a cyclic group of order 2 except in the case of $p = 2$ where $- I_{\SL_2(F)} = I_{\SL_2(F)}$. \\
\end{proof}


Following this result, for ease of notation, $Z(\SL_2(F))$ will be denoted simply by $Z$ throughout the rest of this blueprint.

\begin{lemma}
\label{SpecialSubgroups.exists_unique_orderOf_eq_two}
\lean{SpecialSubgroups.exists_unique_orderOf_eq_two}
\leanok
    If $p\neq 2$, then $\SL_2(F)$ contains a unique element of order 2. \\
\end{lemma}
\begin{proof}
\leanok
Consider an arbitrary element $x \in \SL_2(F)$with order 2. That is $x^2 = I_{\SL_2(F)}$, $x \neq I_{\SL_2(F)}$and thus $x=x^{-1}$.
\begin{equation*} 
    x = \begin{bmatrix} \alpha & \beta \\ \gamma & \delta \end{bmatrix} = \begin{bmatrix} \alpha & \beta \\ \gamma & \delta \end{bmatrix}^{-1} = \begin{bmatrix} \delta & - \beta \\ - \gamma & \alpha \end{bmatrix}.
\end{equation*}
\noindent Thus $\alpha = \delta$, $\beta = - \beta \Rightarrow 2\beta = 0$ and $\gamma = - \gamma \Rightarrow 2\gamma = 0$. In the case of $p \neq 2$ this gives $\beta = 0 = \gamma$. So
\begin{equation*} 
    x = \begin{bmatrix} \alpha & 0 \\ 0 & \alpha \end{bmatrix}.
\end{equation*}
\noindent Also $\alpha^2 = 1$ since $x \in$ $\SL_2(F)$, so $\alpha = \pm 1$. For $x$ to have order 2, we must have $\alpha = - 1$. Hence there is a unique element of order 2, namely $- I_{\SL_2(F)}$.
\end{proof}


\begin{lemma}
    \label{SpecialSubgroups.card_Z_eq_two_of_two_ne_zero}
    \lean{SpecialSubgroups.card_Z_eq_two_of_two_ne_zero}
    \leanok
    If $\textrm{char}(F) \ne 2$ then $|Z| = 2$.
\end{lemma}
\begin{proof}
\leanok
    If $\textrm{char}(F) \ne 2$ then $1 \ne -1$ as $2 \ne 0$ therefore, $I \ne -I$ which shows that $Z = \{I , -I\}$ contains two distinct elements.
\end{proof}


\begin{lemma}
    \label{SpecialSubgroups.card_Z_eq_one_of_two_eq_zero}
    \lean{SpecialSubgroups.card_Z_eq_one_of_two_eq_zero}
    \leanok
    If $\textrm{char}(F) = 2$ then $|Z| = 1$. 
\end{lemma}
\begin{proof}
\leanok
    If $\textrm{char}(F) = 2$ then $1 = -1$ as $2 = 0$ therefore, $I = -I$ which shows that $Z = \{I , -I\} = \{I\}$ only contains one element.
\end{proof}


\begin{lemma}[$Z$ is cyclic]
    \label{SpecialSubgroups.IsCyclic_Z}
    \lean{SpecialSubgroups.IsCyclic_Z}
    \leanok
\end{lemma}
\begin{proof}
\leanok
    By construction, $Z = \overline{\{-I\}} = \{-I^k \; | \; k \in \Z \} = \langle -I \rangle$, therefore $Z$ is generated by a single element and is thus cyclic.
\end{proof}


\begin{remark}[Typeclass instances]
    Observe that instead of telling Lean that \texttt{Is\_Cyclic\_Z} is a \texttt{theorem} we declare it to be
    an \texttt{instance} since we would like Lean to look for this fact whenever it requires it for a
    theorem that may require the assumption that $Z$ is a commutative subgroup.
\end{remark}

In the next chapter it will be useful to record the interactions between $S$ and $Z$. 
For instance we define the following subgroup

\begin{definition}
    \label{SpecialSubgroups.SZ}
    \uses{SpecialSubgroups.S, SpecialSubgroups.Z, SpecialMatrices.s_mul_s_eq_s_add}
    \lean{SpecialSubgroups.SZ}
    \leanok
    We define the subgroup $SZ$ to be the subgroup with the underlying set $S \cup -S$, or equivalently the pointwise product $SZ$.
\end{definition}



\begin{corollary}
\label{SpecialSubgroups.S_mul_Z_subset_SZ}
\uses{SpecialSubgroups.SZ}
\lean{SpecialSubgroups.S_mul_Z_subset_SZ}
$SZ = S \cup -S$
\leanok
\end{corollary}
\begin{proof}
\leanok
    By construction an element of $SZ$ is either of the form $s_\sigma  I = s_\sigma \in S$ or $s_\sigma -I = -s_\sigma\in -S$. The reverse subset inclusion is very similar.
\end{proof}


\begin{lemma}
    \label{SpecialSubgroups.S_join_Z_eq_SZ}
    \uses{SpecialSubgroups.Z, SpecialSubgroups.S, SpecialSubgroups.SZ}
    \lean{SpecialSubgroups.S_join_Z_eq_SZ}
    \leanok
    The join of subgroups satisfies $S \sqcup Z = SZ$
\end{lemma}
\begin{proof}
\uses{SpecialSubgroups.closure_neg_one_eq, SpecialSubgroups.S_mul_Z_subset_SZ}
\leanok
We show that $S \sqcup Z = SZ$ by antisymmetry, that is, we show both that
\begin{itemize}
    
    \item $S \sqcup Z \subseteq SZ$
    
    Let $x \in S \sqcup Z$, then if $x$ is in the subgroup closure then if $K$ is a subgroup whose underlying set contains $SZ$ then $x$ is in $K$,
    but since $SZ$ was shown to be a subgroup in \ref{SpecialSubgroups.SZ} we can conclude that $x \in SZ$ and thus $x = s_\sigma z$ for some $\sigma \in F$.
    
    \item $SZ \subseteq S \sqcup Z$
    
    Let $s_\sigma z \in SZ$ then we must show that $s_\sigma z$ is the subgroup closure of $S$ and $Z$ but since the subgroup closure 
    must at least contain the pointwise product whose underlying set is equal to the union $S \cup -S$, we are done.
\end{itemize}
\end{proof}


\section{Conjugacy of the Elements of $\SL_2(F)$}

\subsubsection{Classification of elements of $\SL_2(F)$ up to conjugation}

\begin{lemma}[Upper triangularizability criteria]
\label{isConj_upper_triangular_iff}
\lean{isConj_upper_triagnular_iff}
\leanok
    A matrix $M \in\textrm{Mat}(2; F)$ is triangularizable if and only if there exists an invertible matrix $C \in \GL_2(F)$ such that the bottom left entry
    $C M C^{-1}_{21} = 0$.
\end{lemma}

\begin{proof}
\leanok
    Given a matrix $U$ is in upper triangular form if and only if
    \[
    U = \begin{bmatrix}
    a & b\\
    0 & d
    \end{bmatrix}
    \]
    
    that is, the bottom left entry is zero. It then follows that $M$ is triangularizable if and only if
    there exists a $C \in \GL_2(F)$ such that $C M C^{-1}$ is in upper triangular form, that is, the bottom left entry of $C M C^{-1}$ is zero. 
\end{proof}


\begin{lemma}[Upper triangularizability of a $2 \times 2$ matrix over an algebraically closed field]
\label{isTriangularizable_of_algClosed}
\lean{isTriangularizable_of_algClosed}
\leanok
    When $F$ is an algebraically closed field, 
    for any $M \in \textrm{Mat}(2; F)$ there exists an invertible matrix $C \in \SL_2(F) \le \GL_2(F)$ such that $C M C^{-1} = U$ where
    \[
    U = \begin{bmatrix}
        a & b\\
        0 & d
    \end{bmatrix}\] for some $a, b, d \in F$.
\end{lemma}
\begin{proof}
    \uses{isConj_upper_triangular_iff}
    \leanok
We prove this by direct computation. 

Let 

\[
M = \begin{bmatrix}
\alpha & \beta\\
\gamma & \delta
\end{bmatrix} \in\textrm{Mat}(2; F)
\]

By lemma \ref{isConj_upper_triangular_iff}, we only need to show that we can find a matrix $C \in \SL_2(F)$ such that when it acts on $M$ by conjugation, the bottom left entry is annihilated.

\begin{itemize}
    \item Suppose on the one hand that $\beta \ne 0$
    
    Observe that 
    
    \begin{equation}\label{triang}
        s_\sigma M s_\sigma^{-1} = \left(\begin{bmatrix}
            -\beta \sigma + \alpha & \beta \\
            -\beta \sigma^{2} + \alpha \sigma - \delta \sigma + \gamma & \beta \sigma + \delta
            \end{bmatrix}\right)
    \end{equation}

    Given $F$ is algebraically closed we can set $\sigma \in F$ to be a root of the polynomial

    \[
    P(X) := -\beta X^{2} + \alpha X - \delta X + \gamma 
    \]

    setting $C := s_\sigma$ yields the desired element which triangularises $M$.
    

    \item Suppose on the other hand that $\beta = 0$
    
    Given the top right entry is zero, we only need find a matrix in $\SL_2(F)$ which flips the anti-diagonal entries (modulo modifying the signs)
    it is thus sufficient to use 
        \[
        w = \begin{bmatrix}
        0 & -1\\
        1 & 0
        \end{bmatrix} \quad \text{as indeed} \quad w M w^{-1} = \begin{bmatrix}
            \delta & -\gamma\\
            0 & \alpha
        \end{bmatrix} \text{ is in triangular form} 
        \]
\end{itemize}

\end{proof}


\begin{corollary}[Upper triangular matrices are conjugate to lower triangular matrices]
    \label{lower_triangular_isConj_upper_triangular}
    \uses{SpecialMatrices.w}
    \lean{lower_triangular_isConj_upper_triangular}
    \leanok
    For every $U \in\textrm{Mat}(2; F)$ that is upper triangular the matrix $w U w^{-1}$ is a lower triangular matrix
\end{corollary}
\begin{proof}
    \leanok
    Direct computation shows this result, see the Lean code!
\end{proof}


\begin{lemma}
    \label{upper_triangular_isConj_diagonal_of_nonzero_det}
    \lean{upper_triangular_isConj_diagonal_of_nonzero_det}
    \leanok
    An upper triangular matrix $U = \begin{bmatrix}
        \alpha & \beta\\
        0 & \delta
    \end{bmatrix}$ is conjugate to a diagonal matrix if $\alpha - \delta \ne 0$
\end{lemma}
\begin{proof}
    \leanok
We show this by direct computation.

Conjugation of $M$ by the matrix 

\[
C := \begin{bmatrix}
    1 & \frac{\beta}{\alpha - \delta}\\
    0 & 1
\end{bmatrix}
\]

yields a diagonal matrix (see the Lean code for the computation!).
\end{proof}


% \begin{remark}[Automation in Lean]
%     Observe that the proof of the theorem above primarily uses tactics! Tactics l 
% \end{remark}


\begin{proposition}
\label{SL2_IsConj_d_or_IsConj_s_or_IsConj_neg_s_of_AlgClosed}
\uses{SpecialMatrices.s, SpecialMatrices.d}
\lean{SL2_IsConj_d_or_IsConj_s_or_IsConj_neg_s_of_AlgClosed}
\leanok
    Each element of $\SL_2(F)$ is conjugate to either $d_\delta$ for some $\delta \in F^\times$, or to $\pm s_\sigma$ for some $\sigma \in F$.
\end{proposition}

\begin{proof}
\uses{isTriangularizable_of_algClosed, lower_triangular_isConj_upper_triangular, upper_triangular_isConj_diagonal_of_nonzero_det}
\leanok Since $F$ is algebraically closed, any element $x \in \SL_2(F)$can be regarded as a linear transformation in the 2 dimensional vector space over $F$, with the eigenvalues $\pi_1$ and $\pi_2$. \\
\\
\space If $\pi_1$ and $\pi_2$ are distinct, then $x$ is thus diagonalisable. That is, there exists an invertible matrix $a \in GL(2, F)$ such that $y = axa^{-1}$ is a diagonal matrix. Furthermore, we can multiply $a$ by a suitable scalar to find an element in $\SL_2(F)$which conjugates $x$ and $y$:

\begin{align*}
    \text{Set} \; b = \frac{a}{\sqrt {\text{det}(a)}}, \quad \text{thus } \; bxb^{-1} =\frac{a}{\sqrt {\text{det}(a)}} \; x \; (\sqrt{\text{det}(a)} \; )\,a^{-1} = axa^{-1} = y.
\end{align*}

Observe that det$(b)=1$, hence $x$ and $y$ are conjugate in $L$. Furthermore, since $y$ is a diagonal matrix it must belong to the set $D$, showing that $x$ is conjugate to $d_\delta$ for some $\delta \in F^\times$. \\
\\
\space If $\pi_1 = \pi_2$ then $x$ has just one repeated eigenvalue. Suppose that $x$ is diagonalisable. Then there exists an element $c \in GL(2, F)$ and a diagonal matrix $\pi_1 I_G$ such that $x = c(\pi_1 I_G)c^{-1} = \pi_1 I_G$. Thus $x = \pm I_G$, which trivially belongs to both $D$ and $Z$. \\
\\
Now assume that $x$ is not diagonalisable. Chapter 7 of \cite{matrix} shows that there exists an element $d \in GL(2, F)$, such that $x= djd^{-1}$, where, $$j = \begin{bmatrix} \pi_1 & 1 \\ 0 & \pi_1 \end{bmatrix}$$ is the Jordan Normal Form of $x$. By the method described above, we can multiply $d$ by a suitable scalar to show that $x$ is conjugate to $j$ in $L$. Now we conjugate $j$ by an element of $\SL_2(F)$whose top left entry is 0.

\begin{align*}
    \begin{bmatrix} 0 & -\gamma^{-1} \\ \gamma & \delta \end{bmatrix} \begin{bmatrix} \pi_1 & 1 \\ 0 & \pi_1 \end{bmatrix} \begin{bmatrix} \delta & \gamma^{-1} \\ -\gamma & 0 \end{bmatrix} = \begin{bmatrix} 0 & -\gamma^{-1} \\ \gamma & \delta \end{bmatrix} \begin{bmatrix} \pi_1 \delta - \gamma & \pi_1 \gamma^{-1} \\ -\pi_1 \gamma & 0 \end{bmatrix} = \begin{bmatrix} \pi_1 & 0 \\ -\gamma^{2} & \pi_1 \end{bmatrix}
\end{align*}
\\
Now clearly the determinant of $x$ is equal to the determinant of $j$, namely 1, which means that $\pi_1 = \pm 1$. This shows that $j$ is conjugate in $\SL_2(F)$to some element in $\times Z$ as well as $x$. Furthermore, since conjugation is transitive, $x$ is conjugate to $\pm s_\sigma$ for some $\sigma \in F$.

\end{proof}


\begin{remark}
formalising the classification of elements of $\SL_2(F)$ up to conjugation in Lean was surprisingly difficult because the informal proof of proposition \ref{SL2_IsConj_d_or_IsConj_s_or_IsConj_neg_s_of_AlgClosed} extracted from Christopher Butler's exposition 
uses the Jordan Normal Form theorem, and at the time of writing, the Jordan Normal form theorem is still not yet in \texttt{mathlib}.

The original approach to formalise the Jordan Normal form theorem for $2 \times 2$ matrices involved studying the eigenspace and generalized eigenspaces of the endomorphism associated to a $2 \times 2$ matrix.
This is one the standard approached taught in an undergraduate curriculum, yet surprisingly, to formalise the $2 \times 2$ case with this approach was rather untractable.

The reason this approach, and often other standard techniques might not integrate well with \texttt{mathlib}, often happens for the following reasons I will now outline.

The crux of formalising a mathematical result always lies at finding the right abstraction, as illustrated in \ref{lattice}, understanding the lattice structure on the set of subgroups becomes an indispensable tool for formalising
results regarding subgroups and their properties. For this particular formalisation, the right abstraction was not entirely.

Is it best to prove the theorem for matrices or for endomoprhisms? Which will be the easiest approach? Which approach is most general? Which approach yields the most amount of useful lemmas?

The reason why the Jordan Normal Form theorem is not yet in \texttt{mathlib} is because it hinges on the following two results which have not been formalised yet:

\begin{enumerate}
    \item The classification of nilpotent endomorphisms.
    \item The classification of semisimple endomorphisms.
\end{enumerate}

Such formalisation would be an amazing project to undertake. But bear in mind, the theorem formalized is the more general Jordan-Chevallier theorem.

To the authors understanding, the general theorem will be formalised by studying the eigenspace and general eigenspace. This approach turned out to be essentially equivalent in 
difficulty to formalising the special case of $2 \times 2$ matrices over an algebraically closed field with the same approach since the argument is inductive on the dimension.

Therefore, after discussions with Prof. Kevin Buzzard's it turned out to be much more effective approach to classify matrices of the special linear group up to 
conjugation by splitting on a few different cases of what a $2 \times 2$ matrix might look like and finding the suitable matrices by which to conjugate to put them in either the form of $d_\delta$ or $\pm s_\sigma$.
\end{remark}

\section{Centralizers \& Normalizers}

Both the centralizer and normalizer of a subset $H$ are subgroups of $G$. Note also that the centralizer is a stronger condition than the 
normalizer and any element in the centralizer of $H$ is also in its normalizer. If $H$ is a singleton then it's clear that its centralizer and normalizer are equal.\\

\subsubsection{Normalizers}

\begin{definition}
The \textbf{normalizer} $N_G(H)$ of a subset $H$ of a group $G$ is the set of elements of $G$ which stabilise $H$ under conjugation.
\begin{equation*} N_G(H) = \{ g \in G : gHg^{-1}=H\}. \end{equation*}
\end{definition}


\begin{corollary}
    \label{lower_triangular_iff_top_right_entry_eq_zero}
    \lean{lower_triangular_iff_top_right_entry_eq_zero}
    \leanok
    A matrix $M \in \textrm{Mat}(2; F)$ is lower triangular if and only if the $M_{12} = 0$.
\end{corollary}
\begin{proof}
    \leanok
It is easy to see the top right entry must be zero for a matrix to be lower triangular.
\end{proof}


\begin{proposition}[Normalizer of subgroups of $S$ are contained in $L$]
\label{normalizer_subgroup_S_le_L}
\uses{SpecialSubgroups.S, SpecialSubgroups.L}
\lean{normalizer_subgroup_S_le_L}
\leanok
 For any subgroup $S_0 \leq S$ with order greater than 1, we have that the normalizer $N_{\SL_2(F)}(S_0) \subset L$.
\end{proposition}
\begin{proof}
    \uses{mem_L_iff_lower_triangular, lower_triangular_iff_top_right_entry_eq_zero}
    \leanok
Let $s_\sigma$ be an arbitary element of $S_0$ with $\sigma \neq 0$. To determine the normalizer of $S_0$ in $\SL_2(F)$we consider which $x \in \SL_2(F)$ satisfy $x s_\sigma x^{-1} \in S_0$.
\begin{align*} x s_\sigma x^{-1} &= \begin{bmatrix} \alpha & \beta \\ \gamma & \delta \end{bmatrix} \begin{bmatrix} 1 & 0 \\ \sigma & 1 \end{bmatrix} \begin{bmatrix} \delta & - \beta \\ - \gamma & \alpha \end{bmatrix}
\\[1.5ex] &= \begin{bmatrix} \alpha & \beta \\ \gamma & \delta \end{bmatrix} \begin{bmatrix} \delta & - \beta \\ \delta \sigma - \gamma & \alpha - \beta \sigma \end{bmatrix}
\\[1.5ex] &= \begin{bmatrix} \alpha \delta - \beta \gamma + \beta \delta \sigma & - \beta^2  \sigma \\ \delta^2 \sigma & \alpha \delta - \beta \gamma - \beta \delta \sigma \end{bmatrix}.
\end{align*}
Since $x s_\sigma x^{-1} \in S_0$ we have $- \beta^2  \sigma = 0$ and since $\sigma \neq 0$, we have $\beta = 0$. Since $s_\sigma$ was chosen arbitrarily, 
any element which normalizes $S_0$ is a lower triangular matrix and is therefore in $L$ by \ref{mem_L_iff_lower_triangular}. Thus $N_{\SL_2(F)}(S_0) \subset L$ as required. \\
\end{proof}


\begin{lemma}
    \label{ex_of_card_D_gt_two}
    \lean{ex_of_card_D_gt_two}
    \leanok
    If the cardinality of finite subgroup of $D_0 \le D$ is greater than $2$ then there exists an element $x \in D_0$ which does not belong to the center $Z$, that is, $x \ne d_1 = I$ and $x \ne d_{-1} = -I$.
\end{lemma}
\begin{proof}
    \leanok
 Suppose for a contradiction that if $\delta \ne \pm 1$ then $d_\delta \notin D_0$. We show that $D_0 \le Z$ and therefore, $|D_0| \le 2$, a contradiction.
 
 Let $d_\delta \in D_0 \le D$ then given $d_\delta \notin D_0$ if $\delta \ne \pm 1$ and $Z = \langle -I\rangle = \{I, -I\}$. It immediately follows that $D_0 \le Z$.

\end{proof}


\begin{proposition}[Normalizers of subgroups of $D$ are contained in $L$]
\label{normalizer_subgroup_D_eq_DW}
\uses{SpecialSubgroups.D, SpecialSubgroups.DW}
\lean{normalizer_subgroup_D_eq_DW}
\leanok
    $N_{\SL_2(F)}(D_0) = \langle D , w \rangle$, where  $D_0$ is any subgroup of $D$ with order greater than 2. \\
    \end{proposition}

\begin{proof}
    \uses{SpecialLinearGroup.fin_two_diagonal_iff, SpecialLinearGroup.fin_two_antidiagonal_iff, ex_of_card_D_gt_two}
    \leanok
    Since $|D_0| > 3$, we can choose a $d_\delta \in D_0 \! \setminus \! Z$, that is where $\delta \neq 1$. To determine the normalizer of $D_0$ in $\SL_2(F)$we consider which $x \in \SL_2(F)$satisfy $x d_\delta x^{-1} \in D_0$.
    \begin{align}\label{6.3proof3} xd_\delta x^{-1} &= \begin{bmatrix} \alpha & \beta \\ \gamma & \delta \end{bmatrix} \begin{bmatrix} \delta & 0 \\ 0 & \delta^{-1} \end{bmatrix} \begin{bmatrix} \delta & - \beta \\ - \gamma & \alpha \end{bmatrix} \nonumber \\[1.5ex]
    &= \begin{bmatrix} \alpha & \beta \\ \gamma & \delta \end{bmatrix} \begin{bmatrix} \delta \delta & - \beta \delta \\ - \gamma \delta^{-1} & \alpha \delta^{-1} \end{bmatrix} \nonumber \\[1.5ex]
    &= \begin{bmatrix} \alpha \delta \delta - \beta \gamma \delta^{-1} & \alpha \beta (\delta^{-1} - \delta) \\ \gamma \delta (\delta - \delta^{-1}) & \alpha \delta \delta^{-1} - \beta \gamma \delta \end{bmatrix} \in D_0.
    \end{align}
    
    Since (\ref{6.3proof3}) is in $D_0$, the top right and bottom left entries must be 0. Since  $\delta \neq \pm 1$, we have $\delta \neq \delta^{-1}$ and so $\alpha \beta = 0 = \gamma \delta$. \\
    \\
     \space If $\alpha = 0$, then $\beta$ and $\gamma$ are non-zero since det$(x) = 1$, thus $\delta = 0$. So det$(x) = - \gamma \beta = 1$  and $- \gamma = \beta^{-1}$. (\ref{6.3proof3}) becomes $$\begin{bmatrix} \delta^{-1} & 0 \\ 0 & \delta \end{bmatrix} = d^{-1}_\delta.$$Since $D_0$ is a group, it contains the inverse of each of it's elements, so $d^{-1}_\delta \in D_0$ as required. In this case we have $x \in wD$. \\
    \\
     \space If $\alpha \neq 0$, then similarly $\beta = 0$, $\delta = \alpha^{-1}$ and $\gamma = 0$. (\ref{6.3proof3}) now becomes $$\begin{bmatrix} \delta & 0 \\ 0 & \delta^{-1} \end{bmatrix} = d_\delta \in D_0.$$This time we have $x \in D$. So $x \in D \cup wD = \langle D , w \rangle$ and any element which normalises $D_0$ is in $\langle D , w \rangle$, thus $N_{\SL_2(F)}(D_0) \subset \langle D , w \rangle$. \\
    \\
    Now take an arbitrary $y \in \langle D , w \rangle = D \cup wD$. If $y \in D$ then $y = d_{\rho 1}$, for some $\rho 1 \in F^\times$.
    \begin{align*} d_{\rho 1} d_\delta d^{-1}_{\rho 1} = d_\delta \in D_0.
    \end{align*}
    
    If $y \in wD$ then $y = w d_{\rho 2}$, for some $ d_{\rho 2} \in F^\times$.
    \begin{align*} (w d_{\rho 2}) d_\delta (w d_{\rho 2})^{-1} &= w d_{\rho 2} d_\delta d^{-1}_{\rho 2} w^{-1}
    \\ &= w d_\delta w^{-1}
    \\ &= d^{-1}_\delta \in D_0.
    \end{align*}
    
    Thus $y$ indeed the whole of $\langle D , w \rangle$ is contained in $N_{\SL_2(F)}(D_0)$. This inclusion gives the desired result, $N_{\SL_2(F)}(D_0) = \langle D , w \rangle$. \\
    
\end{proof}


%-----------------------------

\subsubsection{Centralisers}

\begin{definition}[Centralizer]
The \textbf{centralizer} $C_G(H)$ of a subset $H$ of a group $G$ is the set of elements of $G$ which commute with each element of $H$.
\begin{equation*} 
    C_G(H) = \{ g \in G  : gh=hg, \quad \forall h\in L \}. \end{equation*} 
\end{definition}

\begin{corollary}
    \label{centralizer_neg_eq_centralizer}
    \lean{centralizer_neg_eq_centralizer}
    \leanok
    Let $x \in \SL_2(F)$ then the centralizer of the negative equals $C_{SL_2(F)}(x) = C_{SL_2(F)}(-x)$.
\end{corollary}
\begin{proof}
    \leanok
 An element $y \in \SL_2(F)$ belongs to $C_{SL_2(F)}$ if and only if
  $1 = x y x^{-1} y^{-1} = (-x) y (-x^{-1}) y^{-1}$  if and only if $y$ belongs to $C_{\SL_2(F)}(-x)$.
\end{proof}

    

\begin{proposition}[Centralizer of noncenter $s_\sigma$]
\label{centralizer_s_eq_SZ}
\uses{SpecialSubgroups.S, SpecialSubgroups.Z, SpecialMatrices.s}
\lean{centralizer_s_eq_SZ}
\leanok
The centralizer $C_{\SL_2(F)}(\pm s_\sigma) =  S \times Z $ where $\sigma \neq 0$.
\end{proposition}

\begin{proof}
    \uses{SpecialLinearGroup.fin_two_shear_iff, centralizer_neg_eq_centralizer}
    \leanok
To determine the centralizer of $s_\sigma$ in $L$, we consider which $y \in \SL_2(F)$satisfy $y s_\sigma = s_\sigma y$ for an arbitrarily chosen $s_\sigma$, with $\sigma \neq 0$. \\
\vspace{-0.5mm}
\begin{align}\label{6.3proof2} y s_\sigma &= s_\sigma y, \nonumber \\[1.5ex]
\begin{bmatrix} \alpha & \beta \\ \gamma & \delta \end{bmatrix} \begin{bmatrix} 1 & 0 \\ \sigma & 1 \end{bmatrix} &= \begin{bmatrix} 1 & 0 \\ \sigma & 1 \end{bmatrix} \begin{bmatrix} \alpha & \beta \\ \gamma & \delta \end{bmatrix}, \nonumber \\[1.5ex]
\begin{bmatrix} \alpha + \beta \sigma & \beta \\ \gamma + \delta \sigma & \delta \end{bmatrix} &= \begin{bmatrix} \alpha & \beta \\ \gamma +  \alpha \sigma & \delta + \beta \sigma \end{bmatrix}.
\end{align}
\vspace{.5mm}

Equating the top left entries of (\ref{6.3proof2}) gives $\alpha + \beta \sigma = \alpha$ which means $\beta = 0$ since $\sigma \neq 0$ by assumption. Equating the bottom left entries gives that $\alpha = \delta$. Finally, since det$(y) = 1$, we have $\alpha \delta = 1$ so $\alpha = \pm 1$. Thus a $y \in C_{\SL_2(F)}(s_\sigma)$ is

\begin{align*} y &= \begin{bmatrix} \alpha & 0 \\ \gamma & \alpha \end{bmatrix}. \tag{where $\alpha = \pm 1$}
\end{align*}

So $y = \pm s_\sigma$ for some $\sigma \in F$, and $SZ = \{ \pm s_\sigma \} \subset C_{\SL_2(F)}(s_\sigma)$. Now take an arbitrary $s_\gamma z \in SZ$.
\begin{align*} (s_\gamma z) s_\sigma &= s_\sigma (s_\gamma z),
\\ s_\gamma s_\sigma z &= s_\sigma s_\gamma z, \tag{since $z \in Z$}
\\ s_{\gamma + \sigma} &= s_{\gamma + \sigma}.
\end{align*}

Thus $s_\gamma z$ and indeed the whole of $SZ$ is contained in $ C_{\SL_2(F)}(s_\sigma)$, so $C_{\SL_2(F)}(s_\sigma) = SZ$. \\
\\
Since $S$ commutes elementwise with $Z$ and $S \cap Z = \{ I_G \}$, we can apply Corollary \ref{directproductZ} and assert that $C_{\SL_2(F)}(s_\sigma) = SZ \cong S \times Z$ as required. The centralizer of $- s_\sigma$ is also $ S\times Z$, 
since an element $x$ commutes with $- s_\sigma$ if and only if it commutes with $s_\sigma$:
\begin{align*} 
    xs_\sigma = s_\sigma x \iff -(x s_\sigma) = - (s_\sigma x) \iff x(- s_\sigma) = (- s_\sigma)x.
\end{align*}

Note that in case of $\sigma = 0$, $\pm s_\sigma \in Z$ and thus it's centralizer is the whole of $L$.

\end{proof}


    

\begin{proposition}[Centralizer of noncenter $d_\delta$]
    \label{centralizer_d_eq_D}
    \uses{SpecialMatrices.d, SpecialSubgroups.D}
    \lean{centralizer_d_eq_D}
    \leanok
    The centralizer $C_{\SL_2(F)}(d_\delta) = D$ for $\delta \neq \pm 1$.
    \end{proposition}        
\begin{proof}
    \uses{SpecialLinearGroup.fin_two_diagonal_iff}
    \leanok
Now we consider which $y \in \SL_2(F)$satisfy $y d_\delta = d_\delta y$ for an arbitrarily chosen $d_\delta$, with $\delta \neq \pm 1$.
\begin{align}\label{6.3proof4} y d_ \delta &= d_\sigma y, \nonumber \\[1.5ex]
\begin{bmatrix} \alpha & \beta \\ \gamma & \delta \end{bmatrix} \begin{bmatrix} \delta & 0 \\ 0 & \delta^{-1} \end{bmatrix} &= \begin{bmatrix} \delta & 0 \\ 0 & \delta^{-1} \end{bmatrix} \begin{bmatrix} \alpha & \beta \\ \gamma & \delta \end{bmatrix}, \nonumber \\[1.5ex]
\begin{bmatrix} \alpha \delta & \beta \delta^{-1} \\ \gamma \delta & \delta \delta^{-1} \end{bmatrix} &= \begin{bmatrix} \alpha \delta & \beta \delta \\ \gamma \delta^{-1} & \delta \delta^{-1} \end{bmatrix}.
\end{align}

Equating the top right and bottom left entries of (\ref{6.3proof4}) gives that $\beta = 0 = \gamma$ since Since $\delta \neq \delta^{-1}$. Thus $\delta = \alpha^{-1}$ and 
\begin{align*} x = \begin{bmatrix} \alpha & 0 \\ 0 & \alpha^{-1} \end{bmatrix} \in D. 
\end{align*}

Thus $x$ and indeed the whole of $C_{\SL_2(F)}(d_\delta)$ is contained in $D$. Now take an arbitrary $d_\rho \in D$.
\begin{align*} d_\rho d_\delta = d_{\rho \delta} = d_\delta d_\rho.
\end{align*}
So clearly $D \subset C_{\SL_2(F)}(d_\delta)$ and thus $C_{\SL_2(F)}(d_\delta) = D$ as required.
\end{proof}


%---------------------------------------------------


\begin{proposition}[Centralizers of conjugate elements]
    \label{conjugate_centralizers_of_IsConj}
    \lean{conjugate_centralizers_of_IsConj}
    \leanok
    Let $a$ and $b$ be conjugate elements in a group $G$. Then $\exists \, x \in G$ such that $xC_G(a)x^{-1} = C_G(b)$. \vspace{3mm}
\end{proposition}
\begin{proof}
\leanok
This proposition essentially claims that conjugate elements have conjugate centralizers. Since $a$ and $b$ are conjugate there exists an $x \! \in \! G$ such that $b = xax^{-1}$. Let $g$ be an arbitrary element of $C_G(a)$. Then,

\begin{align*} (xgx^{-1})(xax^{-1}) &= xgax^{-1}\\
&= xagx^{-1} \tag{since $g \in C_G(a)$}\\
&= (xax^{-1})(xgx^{-1}). \end{align*}

Thus $xgx^{-1} \in C_G(xax^{-1})$. Since $g$ was chosen arbitrarily, $$xC_G(a)x^{-1} \subset C_G(xax^{-1}) = C_G(b).$$ 

Conversely, let $h$ be an arbitary element of $C_G(xax^{-1})$. Then,

\begin{align*} (x^{-1}hx)a &= x^{-1}h(xax^{-1})x \\
&= x^{-1}(xax^{-1})hx \tag{since $h \in C_G(xax^{-1})$} \\
&= a(x^{-1}hx). \end{align*}

So $x^{-1}hx \in C_G(a)$ and since $h$ was arbitrarily chosen from $C_G(xax^{-1})$, \linebreak $x^{-1}C_G(xax^{-1})x \subset C_G(a)$. Multiplication on the left by $x$ and on the right by $x^{-1}$ gives $C_G(b) =  C_G(xax^{-1}) \subset xC_G(a)x^{-1}$. Since we have shown that each set contains the other, $xC_G(a)x^{-1} = C_G(b)$ as required. \\
\end{proof}




\begin{corollary}[ Centralizer of non-central element is commutative]
    \label{IsCommutative_centralizer_of_not_mem_center}
    \lean{IsCommutative_centralizer_of_not_mem_center}
    \leanok
The centralizer of an element $x$ in $\SL_2(F)$ is abelian unless $x$ belongs to the centre of $L$. \vspace{3mm}
\end{corollary}

\begin{proof}
    \uses{SL2_IsConj_d_or_IsConj_s_or_IsConj_neg_s_of_AlgClosed, conjugate_centralizers_of_IsConj, centralizer_s_eq_SZ, centralizer_d_eq_D}
    \leanok
    This is almost an immediate consequence of the preceding results. Propositions \ref{centralizer_s_eq_SZ} and \ref{centralizer_d_eq_D} show that an element of the form $\pm s_\sigma$ which does not lie in the centre of $\SL_2(F)$ has centralizer $S \times Z$, whilst a non-central element of the form $d_\delta$ has centralizer $D$.
Both $S$ and $D$ are abelian since they are isomorphic to $F$ and $F^\times$ respectively. Let $s_\sigma z_1$ and $s_\gamma z_2$  be arbitrary elements of $S \times Z$.

\vspace{-.5mm}
\begin{align*} 
    (s_\sigma z_1)(s_\gamma z_2)  &= s_\sigma s_\gamma z_2 z_1  \tag{since  $z_1 \in Z$}
\\ &= s_\gamma s_\sigma z_2 z_1  \tag{since  $S$ is abelian}
\\ &= (s_\gamma z_2)(s_\sigma z_1).   \tag{since  $z_2 \in Z$}
\end{align*} 

Thus $S \times Z$ is also abelian. Since every element of $\SL_2(F)$ is conjugate to $d_\delta$ or $\pm s_\sigma$ by Proposition \ref{SL2_IsConj_d_or_IsConj_s_or_IsConj_neg_s_of_AlgClosed} and conjugate elements have conjugate centralizers by Proposition \ref{conjugate_centralizers_of_IsConj}, the centralizer of each $x \in \SL_2(F)\setminus Z$ is conjugate to either $\times Z$ or $D$. 
Since conjugate subgroups are isomorphic, they must have the same structure, and thus since both $S \times Z$ and $D$ are abelian, $C_{\SL_2(F)}(x)$ is also abelian. 
Note that in general this does hold for $x \in Z$, since its centralizer is the whole of $\SL_2(F)$ which is not abelian unless $\SL_2(F)= Z$.

\end{proof}




\section{The Projective Line \& Triple Transitivity}

It is convenient to sometimes take a geometric viewpoint and regard the elements of $\SL_2(F)$as pairs of vectors in the 2-dimensional vector space over $F$, which we will denote $V$. An element of $\SL_2(F)$is thus a linear transformation of $V$. 

\begin{definition} 
    Let $\mathscr{L}$ be the set of all 1-dimensional subspaces of $V$. A subset $\mathscr{S}$ of $\mathscr{L}$ is called a \textbf{subspace} of $\mathscr{L}$ if there is a subspace $U$ of $V$ such that $\mathscr{S}$ is the set of all 1-dimensional spaces of $U$. We have dim $U =$ dim $\mathscr{S} + 1$. The set $\mathscr{L}$ on which this concept of subspaces is defined is called the \textbf{projective line} on $V$ and an element of $\mathscr{L}$ is a 0-dimensional subspace of $\mathscr{L}$ and consequently called a \textbf{point}. The projective line can be considered as a straight line in the field, plus a point at infinity.
\end{definition}

Any 1-dimensional subspace of $V$ is a set of vectors of the form $\eta u$, where $u$ is a non-zero vector of $V$ and $\eta \in F^\times$. Thus the points of $\mathscr{L}$ are equivalence classes with the following relation defined on the set of vectors of $V$.
\begin{align*} u = \begin{bmatrix} u_1 \\ u_2 \end{bmatrix} \sim \begin{bmatrix} v_1 \\ v_2 \end{bmatrix} = v \iff u = \eta v, \qquad (\text{for $\eta \in F^\times$}).
\end{align*}

Notice that $u$ and $v$ are equivalent if and only if $u_1 v_2 = v_1 u_2$. Importantly each point $P_i$ of $\mathscr{L}$ can be represented by a corresponding equivalence class of vectors of $V$, that is, $P$ corresponds to $u$ if $P = u_1 / u_2$. In the case when $u_2 = 0$, this corresponds to the point at infinity.

\begin{definition} Let $S$ be a permutation group which acts on a set $X$ and $\{ x_1, x_2, x_3 \}$ and $\{ x_1', x_2', x_3' \}$ be two subsets of distinct elements of $X$. Then $S$ is said be \textbf{triply transitive} on $X$ if there is an element $\pi \in S$ such that,
\begin{align*} x^{\pi}_i = x'_i, \qquad(\text{$i$ = 1,2 or 3}).
\end{align*} 
\end{definition}

\begin{theorem} \label{6.6}
Let $\mathscr{L}$ be the projective line over the field $F$. Then $\SL_2(F)$is triply transitive on the set of the points of $\mathscr{L}$. \vspace{3mm}
\end{theorem}

\begin{proof} Let $P_1$, $P_2$ and $P_3$ be distinct points of $\mathscr{L}$ and $p_i$ be a vector in $V$ corresponding to $P_i$. Since each $P_i$ is distinct, $p_1$, $p_2$ and $p_3$ are thus pairwise linearly independent. Thus $p_1$ and $p_2$  form a basis for $V$ and it's clear that there exist $\alpha, \beta \in F^\times$ such that,
\begin{align*} p_3 = \alpha p_1 + \beta p_2.
\end{align*}

Now, let $Q_1$, $Q_2$ and $Q_3$ be three more distinct points of $\mathscr{L}$ and $q_i$ be a vector in $V$ corresponding to $Q_i$. Similarly, by the above argument, there exist $\gamma, \delta \in F^\times$ such that,
\begin{align*} q_3 = \gamma q_1 + \delta q_2.
\end{align*}

Let $\pi \in GL(2,F)$ be the linear transformation which sends $\alpha p_1$ to $\gamma q_1$  and $\beta p_2$ to $\delta q_2$. Thus,
\begin{align*} \pi(p_3) = \pi(\alpha p_1 + \beta p_2) = \pi(\alpha p_1) + \pi(\beta p_2) = \gamma q_1 + \delta q_2 = q_3 
\end{align*}

Hence we get $P^\pi_1 = Q_1$, $P^\pi_2 = Q_2$ and $P^\pi_3 = Q_3$ and $GL(2,F)$ is triply transitive. Now set,
\begin{align*} \eta = \sqrt{\frac{1}{\text{det }\pi}}.
\end{align*}

Consider the mapping $\theta$ which sends $\alpha p_1$ to $\eta \gamma q_1$ and $\beta p_2$ to $\eta \delta q_2$. Observe that,
\begin{align*} \text{det }\theta = \eta^2 \, \text{det } \pi = 1
\end{align*}

So $\theta \in SL(2,F) = \SL_2(F)$and since $P^\theta_1 = Q_1$, $P^\theta_2 = Q_2$ and $P^\theta_3 = Q_3$, we have that $\SL_2(F)$is also triply transitive. 

\end{proof}

The following proposition looks at what happens when the group $\SL_2(F)$acts on the projective line $\mathscr{L}$.

\begin{proposition} \label{6.7} (i) Each element of the form $d_\delta$ (with $\delta \neq \pm 1$), fixes the same two points on the projective line $\mathscr{L}$ and fix no other point. \vspace{3mm} \\
(ii) Each element of the form $\pm s_\sigma$ (with $\sigma \neq 0$), fixes the same point $P$ on $\mathscr{L}$ and fix no other point. Furthermore, \emph{Stab}$(P) = H$. \vspace{3mm} \\
(iii) All conjugate elements have the same number of fixed points on $\mathscr{L}$. \vspace{3mm} \\
(iv) Any noncentral element of $\SL_2(F)$has at most 2 fixed points on $\mathscr{L}$.
\end{proposition}

\begin{proof} 
(i) Let $P$ be a fixed a point of an arbitrary $d_\delta \in D$, with $\delta \neq \pm 1$ and let $u$ belong to the corresponding equivalence class of vectors of $V$ to $P$. \\
\begin{align*} d_\delta u = \begin{bmatrix} \delta & 0 \\ 0 & \delta^{-1} \end{bmatrix} \begin{bmatrix} u_1 \\ u_2 \end{bmatrix} &= \begin{bmatrix} u_1 \delta \\ u_2 \delta^{-1} \end{bmatrix} \sim \begin{bmatrix} u_1 \\ u_2 \end{bmatrix}, 
\\[1.5ex] u_1 u_2 \delta &= u_1 u_2 \delta^{-1}.
\end{align*}

Since $\delta \neq \pm 1$, $\delta$ does not equal $\delta^{-1}$, and so either $u_1 = 0$ or $u_2 = 0$. Thus $u$ is equivalent to either the vector $\begin{bmatrix} 0 \\ 1 \end{bmatrix}$ or $\begin{bmatrix} 1 \\ 0 \end{bmatrix}$ and these correspond to 2 distinct points of $\mathscr{L}$ which are fixed by $d_\delta$. \\
\\
(ii) Let $P$ be a fixed a point of an arbitrary $s_\sigma$, with $\sigma \neq 0$, and let $u$ be the corresponding element of $V$ to $P$. \\
\begin{align*} s_\sigma u = \begin{bmatrix} 1 & 0 \\ \sigma & 1 \end{bmatrix} \begin{bmatrix} u_1 \\ u_2 \end{bmatrix} &= \begin{bmatrix} u_1 \\ u_1 \sigma + u_2 \end{bmatrix} \sim \begin{bmatrix} u_1 \\ u_2 \end{bmatrix}, 
\\[1.5ex] u_1 u_2 &= {u_1}^2 \sigma + u_1 u_2.
\end{align*}

This gives ${u_1}^2 \sigma = 0$ and since $\sigma \neq 0$ we have $u_1 = 0$. Thus $s_\sigma$ has just one fixed point, $P$ which corresponds to the equivalence class of $\begin{bmatrix} 0 \\ 1 \end{bmatrix}$ in $V$. We show also that $P$ is also the only fixed point of $-s_\sigma$, with $\sigma \neq 0$.
\begin{align*} -s_\sigma u = \begin{bmatrix} -1 & 0 \\ \sigma & -1 \end{bmatrix} \begin{bmatrix} u_1 \\ u_2 \end{bmatrix} &= \begin{bmatrix} -u_1 \\ u_1 \sigma - u_2 \end{bmatrix} \sim \begin{bmatrix} u_1 \\ u_2 \end{bmatrix}, 
\\[1.5ex] -u_1 u_2 &= {u_1}^2 \sigma - u_1 u_2.
\end{align*}

So again $u_1 =0$ and $-s_\sigma$ fixes $P$ and no other point. We now calculate the stabiliser of $P$ in $L$, by considering which $x \in \SL_2(F)$fix $P$. \\
\begin{align*} x u = \begin{bmatrix} \alpha & \beta \\ \gamma & \delta \end{bmatrix} \begin{bmatrix} 0 \\ 1 \end{bmatrix} &= \begin{bmatrix} \beta \\ \delta \end{bmatrix} \sim \begin{bmatrix} 0 \\ 1 \end{bmatrix}.
\end{align*}

Thus $\beta = 0$ and $x \in H$. Since $x$ was chosen arbitrarily from Stab$(P)$, we have Stab$(P) \subset H$. Now let an arbitrarily chosen $y \in H$ act on $P$. \\
\begin{align*} y u = \begin{bmatrix} \alpha & 0 \\ \gamma & \alpha^{-1} \end{bmatrix} \begin{bmatrix} 0 \\ 1 \end{bmatrix} &= \begin{bmatrix} 0 \\ \alpha^{-1} \end{bmatrix} \sim \begin{bmatrix} 0 \\ 1 \end{bmatrix}.
\end{align*}

Thus $y$ and indeed $H$ is contained in Stab$(P)$, so Stab$(P) = H$ as desired. \\
\\
(iii) Let $P_i$ $(i = 1,2,...)$ be the fixed points of $x\in \SL_2(F)$and let $y$ be conjugate to $x$ in $L$. That is, there exists a $g \in \SL_2(F)$such that $x = gyg^{-1}$.
\begin{align*} x P_i &= P_i,
\\ gyg^{-1} P_i &= P_i,
\\ y(g^{-1} P_i) &= (g^{-1} P_i).
\end{align*}

This shows that $P_i$ is a fixed point of $x$ if and only if $g^{-1} P_i$ is a fixed point of $y$. Thus conjugate elements have the same number of fixed points. \\
\\
(iv) By Proposition \ref{ISL2_IsConj_d_or_IsConj_s_or_IsConj_neg_s_of_AlgClosed} every  element of $\SL_2(F)$is conjugate to either $d_\delta$ or $\pm s_\sigma$, so since conjugate elements have the same number of fixed points, every element of $\SL_2(F)\! \setminus \! Z$ has either the same number of fixed points as $d_\delta$ (with $\delta \neq \pm 1$), namely 2, or the same number as $\pm s_\sigma$, (with $\sigma \neq 0$), namely 1.

\end{proof}



\chapter[The Maximal Abelian Subgroup Class Equation]{The Maximal Abelian Subgroup Class Equation}\label{Ch6_MaximalAbelianSubgroupClassEquation}
% \chaptermark{The Class Equation}

\section[A finite subgroup of $\SL_2(F)$]{A Finite Subgroup of $\pmb{L}$}

We now return to the realm of finite groups and consider $G$ to be an arbitrary finite subgroup of $\SL_2(F)$. We will still continue to use $Z$ to denote the centre of $\SL_2(F)$, and will use $Z(G)$ whenever we refer to the centre of $G$. \\
\\
Observe that if $Z$ is not contained in $G$, then $Z$ must contain a non-identity element, thus $|Z| = 2$ and $p \neq 2$ by Lemma \ref{6.2}. Recall that $\SL_2(F)$ has a unique element of order 2 by Lemma \ref{6.2b}, $- I_L$, which is not in $G$, therefore $G$ has no element of order 2. \\
\\
By Cauchy's Theorem, which says that if a prime $p$ divides the order of a finite group, then the group contains an element of order $p$, we deduce that 2 does not divide the order of $G$. \\
\\
This means that $|G|$ and $|Z|$ are relatively prime, so $G \cap Z = \{ I_L \}$ and we can use Corollary \ref{directproductZ} to show that $GZ \cong G \times Z$. This shows that regardless of whether $G$ contains $Z$ or not, its structure is uniquely determined by $GZ$, so it suffices to only consider the case when $Z \subset G$. 

\section{Maximal Abelian Subgroups}

\begin{definition}[Maximal Abelian Subgroup]
\label{IsMaximalAbelian}
\lean{IsMaximalAbelian}
\leanok
Let $H$ and $J$ be subgroups of a group $G$ where $H$ is abelian. $H$ is called \textbf{maximal abelian} if $J$ is not abelian whenever $H \subsetneq J$.
\end{definition}

\begin{remark}
The definition was stated in positive form:

A subgroup $H$ is said to be a maximal abelian subgroup of $G$ if for every $J$ subgroup of $G$ satisfying $H \le J$ we have that $J \le H$. Which overall implies $H = J$ by antisymmetry of the preorder.

In Lean this statement looks like the following:

\begin{verbatim}
  def IsMaximalAbelian {L : Type*} [Group L] (G : Subgroup L) : Prop := Maximal (IsCommutative) G
\end{verbatim}

  where the definition of \texttt{Maximal} in mathlib implicitly recognises the existence of a $\le$ operator (a more primitive notion of a partial order) and is:

  \begin{verbatim}
    def Maximal (P : α → Prop) (x : α) : Prop := P x ∧ ∀ ⦃y⦄, P y → x ≤ y → y ≤ x
  \end{verbatim}

  Which informally means that an object $M$ that satisfies a property is maximal if any other object $K$ that also satisfies the property and is related to $M$ by $M \le K$ then in fact we must have the symmetric relation $K \le M$.

  When $\le := \subseteq$ then this is the natural notion of maximal.
\end{remark}


\begin{definition}[Elementary Abelian]
\label{IsElementaryAbelian}
\lean{IsElementaryAbelian}
\leanok
A group $G$ is said to be \textbf{elementary abelian} if it is abelian and every non-trivial element has order $p$, where $p$ is prime.
\end{definition}

\begin{remark}
  In Lean we define the notion of a subgroup of $H$ of $G$ being elementary abelian the following way:
  
  \begin{verbatim}
  def IsElementaryAbelian {G : Type*} [Group G] (p : ℕ) (H : Subgroup G) : Prop :=
  IsCommutative H ∧ ∀ h : H, h ≠ 1 → orderOf h = p
  \end{verbatim}
\end{remark}

\begin{definition}
\label{MaximalAbelianSubgroupsOf}
\uses{IsMaximalAbelian}
\lean{MaximalAbelianSubgroupsOf}
\leanok
Let $\mathfrak{M}$ denote the set of all maximal abelian subgroups of $G$.
\end{definition}

\begin{remark}
  When a set/object with some additional structure has been defined informally, when one wants to formalise results about the object it is often the case a decision has to be made
  about whether the set is defined in Lean as a set or whether it is defined as its own type. In this case, I have opted to define it as a set but later on when using quotients we shall see an example
  of how it is beneficial to define an object as a type/subtype in its own right.
\end{remark}

\vspace{3mm}

Maximal abelian subgroups play an important role in determining the structure of $G$. In particular, every element in $G$ must be contained in some maximal abelian subgroup, since every element commutes at least with itself and $Z$. This will allow us to decompose $G$ into the conjugacy classes of these maximal abelian subgroups. Note also that unless $G=Z$, $Z$ is not a maximal abelian subgroup, because for each $x \in G \! \setminus \! Z$, $\langle Z,x \rangle$ is clearly a larger abelian subgroup than $Z$. \\
\\
We will shortly prove an important theorem regarding the maximal abelian subgroups of $G$, but in order to do so we require the following two lemmas. \\

\begin{lemma}
  \label{IsElementaryAbelian.dvd_card}
  \lean{IsElementaryAbelian.dvd_card}
  \leanok
If $G$ is a finite group of order $p^m$ where $p$ is prime and $m > 0$, then $p$ divides $|Z(G)|$. 
\end{lemma}

\begin{proof}
Let $C(x)$ be the set of elements of $G$ which are conjugate in $G$ to $x$, we call this the conjugacy class of $x$. Bhattacharya shows that the set of all conjugacy classes form a partition of $G$ \cite[p.112]{bhattacharya}. Now consider the following rearranged class equation of $G$, where $S$ is a subset of $G$ containing exactly one element from each conjugacy class not contained in $Z(G)$. 
 
\begin{equation} \label{cen2}
|G| - \sum_{x \in S} [G:N_G(x)] = |Z(G)|.
\end{equation}

Since $|G| = p^m$, each subgroup of $G$ is of order $p^k$ for some $k \leq m$. In particular each $N_G(x)$ has order $p^k$ and is strictly contained in $G$ since $x \not \in Z(G)$ by assumption. Thus each $[G:N_G(x)] > 1$, and are therefore divisible by $p$. Since $p$ divides the left hand side of (\ref{cen2}), it must also divide the right, thus $p$ divides $|Z(G)|$. 

\end{proof}



\begin{lemma}
  \label{coprime_card_fin_subgroup_of_inj_hom_group_iso_units}
  \lean{coprime_card_fin_subgroup_of_inj_hom_group_iso_units}
  \leanok
Every finite subgroup of a multiplicative group of a field is cyclic.
\end{lemma}
%PROOF AND DEPENDENCIES

\begin{proof} 
  See \cite[p.41]{suzuki}.
\end{proof}

\begin{theorem}
  \label{MaximalAbelianSubgroup.centralizer_meet_G_in_MaximalAbelianSubgroups_of_noncentral}
  \uses{IsCommutative_centralizer_of_not_mem_center, MaximalAbelianSubgroupsOf}
  \lean{MaximalAbelianSubgroup.centralizer_meet_G_in_MaximalAbelianSubgroups_of_noncentral}
  \leanok 
  Let $G$ be an arbitrary finite subgroup of $\SL_2(F)$ containing $Z$. \\
If $x \in G \! \setminus \! Z$ then we have $C_G(x) \in \mathfrak{M}$. \vspace{3mm} \\
\end{theorem}
\begin{proof}
  Let $x$ be chosen arbitrarily from $G \! \setminus \! Z$. Then by Corollary \ref{6.5}, $C_{\SL_2(F)}(x)$ is abelian. By definition, $C_G(x) = C_{\SL_2(F)}(x) \cap G$, and using the elementary fact that the intersection of two subgroups is itself a subgroup, we have $C_G(x) < C_{\SL_2(F)}(x)$. Now since every subgroup of an abelian group is abelian, $C_G(x)$ is also abelian. \\
  \\
  Now let $J$ be a maximal abelian subgroup of $G$ containing $C_G(x)$. Since $J$ is abelian and $x \in C_G(x) \subset J$, we have $jx=xj$, $\forall j \in J$, thus $J \subset C_G(x)$. Therefore $J=C_G(x)$ and $C_G(x) \in \mathfrak{M}$. \\
\end{proof}

Before we continue proving properties about Maximal Abelian Subgroups we first need to understand how commutative subgroups interact with other subgroups. 
We now list a few results about commutative subgroups and their interaction with other subgroups:

\begin{remark}
\label{IsCommutative_of_IsCommutative_subgroupOf}
\lean{IsCommutative_of_IsCommutative_subgroupOf}
\leanok
Let $H, K$ be two subgroups of a group $G$ then $H \sqcap K = H \cap K$ is commutative if $H \sqcap K$ regarded as a subgroup of $K$ is commutative.
\end{remark}

\begin{remark}
  The remark above \ref{IsCommutative_of_IsCommutative_subgroupOf} probably seems ridiculous, but Lean genuinely understands both objects as belonging to completely different types and 
  this result is necessary to be able to jump between the corresponding contexts.
\end{remark}

\begin{definition}
  \label{center_mul}
  \lean{center_mul}
  \leanok
  Let $H$ be a subgroup of a group $G$ then the pointwise set product $Z(G) H$ is a subgroup of $G$
\end{definition}
\begin{proof}
\begin{enumerate}
  \item \texttt{one_mem'}: Both $Z(G)$ and $H$ are subgroups of $G$ so they contain the identity element, thus $1 \cdot 1 \in Z(G) H$.
  \item \texttt{mul_mem'}: Let $z_1 h_1, z_2 h_2 \in Z(G) H$ then $z_1h_1z_2h_2 = z_1z_2 h_1h_2 \in Z(G) H$ as $z_i$ is in the center.
  \item \texttt{inv_mem'}: Let $zh \in Z(G) H$ then $z^{-1} h^{-1} \in Z(G) H$ and $z h z^{-1} h^{-1} = zz^{-1}h h^{-1} = 1$.

\end{enumerate}
\end{proof}


\begin{lemma}
  \label{center_mul_subset_center_mul}
  \uses{center_mul}
  \lean{center_mul_subset_center_mul}
  \leanok
\end{lemma}

\begin{lemma}[ The join of a commutative subgroup with the center of a group is commutative]
  \label{IsComm_of_center_join_IsComm}
  \uses{center_mul_subset_center_mul, center_mul}
  \lean{IsComm_of_center_join_IsComm}
  \leanok

  Let $H$ be a commutative subgroup of $G$ then the subgroup $Z(G) \sqcup H$  is a commutative subgroup of $G$.
\end{lemma}
\begin{proof}
  Let $x, y \in Z(G) \sqcup H$ recalling that the supremum can be thought of taking the closure
  we know that if $x$ and $y$ belong to the closure then since $Z(G) H$ is a subgroup of $G$ and $Z(G) \sqcup H \subseteq Z(G) H$
  we know that $x, y \in Z(G) H$ and thus there exist $z_1 h_1 = x$ and $z_2 h_2 = y$. Therefore, we can now show that $x$ and $y$ commute:

  \begin{align*}
  x y &= z_1 h_1 z_2 h_2\\
  & = z_1 z_2 h_1 h_2 \tag{as $z_2$ is in the center}\\
  &= z_2 z_1 h_2 h_1 \tag{as $H$ is a commutative subgroup}\\
  &= z_2 h_2 z_1 h_1 \tag{as $z_1$ is in the center}
  \end{align*}
\end{proof}

\begin{lemma}[$Z$ is contained within any Maximal Abelian Subgroup of a subgroup containing $Z$]
  \label{MaximalAbelianSubgroup.center_le}
  \uses{IsCommutative_of_IsCommutative_subgroupOf, IsComm_of_center_join_IsComm}
  \lean{MaximalAbelianSubgroup.center_le}
  Let $H$ be a subgroup of $G$, if $Z(G) \le H$ then for any maximal abelian subgroup of $H$, $A$ we have that $Z(G) \le A$ 
  \leanok
\end{lemma}
\begin{proof}
  
\end{proof}

\begin{lemma}
\label{MaximalAbelianSubgroup.le_centralizer_of_mem}
\lean{MaximalAbelianSubgroup.le_centralizer_of_mem}
\leanok
Let $H$ be a subgroup of $G$ and let $A$ be a maximal abelian subgroup of $H$, and let $x \in A$ then $A \le C_G(x)$.
\end{lemma}

\begin{lemma}
  \label{MaximalAbelianSubgroup.not_le_of_ne}
  \lean{MaximalAbelianSubgroup.not_le_of_ne}
  \leanok
  Let $H$ be a subgroup of a group $G$ and let $A \ne B$ be maximal abelian subgroups of $H$ then $B \not\le A$.
\end{lemma}
\begin{proof}
Suppose for a contradiction that $B \le A$, then by the maximality of $B$ and because $A$ is commutative as it is maximal abelian we must have that $A \le B$.
But this shows $A = B$ by antisymmetry, a contradiction.
\end{proof}

\begin{lemma}
  \label{MaximalAbelianSubgroup.lt_cen_meet_G}
  \uses{MaximalAbelianSubgroup.not_le_of_ne, MaximalAbelianSubgroup.le_centralizer_of_mem}
  \lean{MaximalAbelianSubgroup.lt_cen_meet_G}
  \leanok
  Let $H$ be a subgroup of $G$, let $A \ne B$ be maximal abelian subgroups of $H$ and let $x \in A \cap B$ then $A < C_G(x) \sqcap H$.
\end{lemma}

\begin{theorem}
  \label{MaximalAbelianSubgroup.center_eq_meet_of_ne_MaximalAbelianSubgroups}
  \uses{MaximalAbelianSubgroup.centralizer_meet_G_in_MaximalAbelianSubgroups_of_noncentral, MaximalAbelianSubgroup.center_le, MaximalAbelianSubgroup.lt_cen_meet_G, MaximalAbelianSubgroup.le_centralizer_of_mem}
  \lean{MaximalAbelianSubgroup.center_eq_meet_of_ne_MaximalAbelianSubgroups}
For any two distinct subgroups $A$ and $B$ of $\mathfrak{M}$, we have
\begin{align*} A \cap B = Z. \end{align*}
\end{theorem}

\begin{proof}
  Consider $x \in A \cap B$. Since both $A$ and $B$ are abelian, $x$ commutes with each $a \in A$ and $b \in B$ and thus $C_G(x)$ contains both $A$ and $B$.  If $x \in G \setminus Z$, then $C_G(x) \in \mathfrak{M}$ by \ref{MaximalAbelianSubgroup.centralizer_meet_G_in_MaximalAbelianSubgroups_of_noncentral} and because $A$ and $B$ are distinct we have $A \subsetneq A \cup B \subset C_G(x)$. 
  This contradicts the fact that $A$ is maximum abelian and thus $x \in Z$. Finally, note that Z is contained in every maximal abelian subgroup, since otherwise we would have the contradiction that $\langle A, Z \rangle$ would generate a larger abelian subgroup than $A$. Hence $A \cap B = Z$. \\
\end{proof}

%---------------------------------------

\begin{lemma}
\label{MaximalAbelianSubgroup.singleton_of_cen_eq_G}
\uses{MaximalAbelianSubgroup.center_le, MaximalAbelianSubgroupsOf, IsMaximalAbelian}
\lean{MaximalAbelianSubgroup.singleton_of_cen_eq_G}
\leanok
Let $H$ be a subgroup of $G$ and suppose $H = Z(G)$ then the maximal abelian subgroups are $\mathfrak{M} = \{Z(G)\}$.
\end{lemma}
\begin{proof}
  We show that $A \in \mathfrak{M}$ if and only if $A = Z(G)$
  \begin{itemize}
    \item[$\Rightarrow$] Suppose $A$ is a maximal abelian subgroup of $H$, then by \ref{MaximalAbelianSubgroup.center_eq_meet_of_ne_MaximalAbelianSubgroup.center_le} $Z(G) \le A$. Furthermore, $A \le H = Z(G)$; which overall shows $A = Z(G)$ as required.
    \item[$\Leftarrow$] Suppose $A = Z(G)$ we now show that $A$ is a maximal abelian subgroup. 
      On the one hand, $A = Z(G)$ so it follows that $A$ is abelian.
      On the other hand, we need to show that $Z(G)$ is maximal. Let $B$ be a subgroup of $H$ that is commutative and such that $Z(G) \sqcap H \le B$, we show that it follows that $B \le Z(G) \sqcap H$. But this follows trivially as 
      $B \leq H = Z(G) \sqcap H = \top$.
  \end{itemize}
\end{proof}

\begin{lemma}
  \label{MaximalAbelianSubgroup.IsCyclic_and_card_Coprime_CharP_of_center_eq}
  \uses{MaximalAbelianSubgroup.singleton_of_cen_eq_G, SpecialSubgroups.card_Z_eq_two_of_two_ne_zero, SpecialSubgroups.IsCyclic_Z, SpecialSubgroups.card_Z_eq_one_of_two_eq_zero}
  \lean{MaximalAbelianSubgroup.IsCyclic_and_card_Coprime_CharP_of_center_eq}
  \leanok
  If $G = Z(G)$ then an element $A$ of $\mathfrak{M}$, the maximal abelian subgrups of $G$ is a cyclic group whose order is relatively prime to $p$.
\end{lemma}
\begin{proof}
  Here $G$ is the only element of $\mathfrak{M}$. If $p \neq 2$ then $|G|=2$ and $G$ is a cyclic group whose order is relatively prime to $p$. If $p=2$ then $G = I_G$ which is trivially a $S_p$-subgroup. \\
\end{proof}

\begin{remark}
  \label{mem_centralizer_self}
  \lean{mem_centralizer_self}
  \leanok
 Let $G$ be a group then centralizer of an element $x \in G$, $C_G(x)$ contains $x$. 
\end{remark}

\begin{lemma}
  \label{MaximalAbelianSubgroup.center_not_mem}
  \uses{mem_centralizer_self, MaximalAbelianSubgroup.centralizer_meet_G_in_MaximalAbelianSubgroups_of_noncentral}
  \lean{MaximalAbelianSubgroup.center_not_mem}
  \leanok
  Let $F$ be an algebraically closed field, let $G$ be a subgroup of $\SL_2(F)$ where $G \ne Z(\SL_2(F))$ then the center is not a maximal abelian subgroup of $G$, $Z(G) \notin \mathfrak{M}$.
\end{lemma}
% PROOF

\begin{lemma}
  \label{MaximalAbelianSubgroup.le_centralizer_meet}
  \uses{IsCommutative_of_IsCommutative_subgroupOf}
  \lean{MaximalAbelianSubgroup.le_centralizer_meet}
  \leanok

  Let $H$ be a subgroup of a group $G$, let $A$ be a maximal abelian subgroup of $H$, and suppose $x \in A \subseteq G$ then 
  $A \le C_{\SL_2(F)} \sqcap H$.
\end{lemma}
%PROOF

\begin{lemma}
  \label{MaximalAbelianSubgroup.eq_centralizer_meet_of_center_lt}
  \uses{MaximalAbelianSubgroup.centralizer_meet_G_in_MaximalAbelianSubgroups_of_noncentral, MaximalAbelianSubgroup.le_centralizer_meet}
  \lean{MaximalAbelianSubgroup.eq_centralizer_meet_of_center_lt}
  \leanok
  Let $F$ be an algebraically closed field, let $G$ and $A$ be a subgroup of $\SL_2(F)$ where $A$ is a maximal abelian subgroup of $G$ and $Z(\SL_2(F)) < A$ then there exists an element $x \in G \setminus Z(SL(2)) \subseteq \SL_2(F)$ such that
  $A = C_{\SL_2(F)}(x) \sqcap G = C_{G}(x)$.
\end{lemma}
%PROOF

\begin{theorem}
  \label{MaximalAbelianSubgroup.IsCyclic_and_card_coprime_CharP_of_IsConj_d}
  \uses{SpecialSubgroups.center_SL2_eq_Z, conjugate_centralizers_of_IsConj, centralizer_d_eq_D, SpecialMatrices.d, SpecialSubgroups.D_iso_units, 
    coprime_card_fin_subgroup_of_inj_hom_group_iso_units}
  \lean{MaximalAbelianSubgroup.IsCyclic_and_card_coprime_CharP_of_IsConj_d}
  \leanok
  Let $F$ be an algebraically closed field of characteristic $p$ and let $G$ be a finite subgroup of $\SL_2(F)$ containing $Z$, let $A$ be a subgroup of $\SL_2(F)$ which is a maximal abelian subgroup of $G$ and furthermore suppose 
  that $A = C_{\SL_2(F)}(x) \sqcap G$ where $x \in \SL_2(F) \setminus Z$ and that $x$ is conjugate to $d_\delta$ for some $\delta \in F^\times$ then $A$ is cyclic and the cardinality of $A$ is coprime to $p$.
\end{theorem}
%PROOF

To prove the statement when $x$ is conjugate to $s_\sigma$ for some $\sigma \in F$ we first need the following lemmas:

\begin{lemma}
  \label{MaximalAbelianSubgroup.centralizer_eq_conj_SZ_of_IsConj_s_or_IsConj_neg_s}
  \lean{MaximalAbelianSubgroup.centralizer_eq_conj_SZ_of_IsConj_s_or_IsConj_neg_s}
  \leanok
\end{lemma}

We need the following computations which essentially makes allowances which let us think of the complete lattice structure with the further property of
being a distributive lattice, that is, $(H \sqcup K) \sqcap L = (H \sqcap L) \sqcup (K \sqcap L)$.


\begin{lemma}
  \label{MaximalAbelianSubgroup.conj_T_join_Z_meet_G_eq_conj_T_meet_G_join_Z}
  \uses{SpecialSubgroups.center_SL2_eq_Z, SpecialSubgroups.S, SpecialSubgroups.Z}
  \lean{MaximalAbelianSubgroup.conj_T_join_Z_meet_G_eq_conj_T_meet_G_join_Z}
  \leanok

  Let $c \in \SL_2(F)$  and $G$ be a subgroup of $\SL_2(F)$ then $c(S \sqcup Z)c^{-1} \sqcap G = (cSc^{-1} \sqcap G) \sqcup Z$
\end{lemma}
% PROOF AND DEPENDENCIES

We also need the following computation:
\begin{lemma}
\label{MaximalAbelianSubgroup.conj_inv_conj_eq}
\uses{SpecialSubgroups.center_SL2_eq_Z}
\lean{MaximalAbelianSubgroup.conj_inv_conj_eq}
\leanok
Let $c \in \SL_2(F)$ and $G$ be a subgroup of $\SL_2(F)$ then 
\[
c^{-1}(c(S \sqcap G)c^{-1} \sqcup Z)c = (S \sqcap c^{-1}Gc) \sqcup Z
\]
\end{lemma}
% PROOF AND DEPENDENCIES

\begin{remark}
  \label{IsElementaryAbelian.subgroupOf}
  \lean{IsElementaryAbelian.subgroupOf}
  \leanok
  If a subgroup $H$ of a group $G$ is an elementary abelian subgroup then for any subgroup $K$ we have that $H \sqcap K$ is also an elementary abelian subgroup.
\end{remark}


\begin{lemma}
  \label{MaximalAbelianSubgroup.exists_noncenter_of_card_center_lt_card_center_Sylow}
  \uses{SpecialSubgroups.center_SL2_eq_Z, SpecialSubgroups.card_Z_eq_one_of_two_eq_zero, SpecialSubgroups.card_Z_eq_two_of_two_ne_zero, SpecialSubgroups.Z }
  \lean{MaximalAbelianSubgroup.exists_noncenter_of_card_center_lt_card_center_Sylow}
  \leanok
  Let $G$ be a finite subgroup of $\SL_2(F)$, let $S$ be a $p$-Sylow subgroup of $G$ where $p$ is the characteristic of the field $F$ and furthermore suppose $p \le |Z|$ then
  there exists a noncentral element in $S$, that is, $S \setminus Z \ne \varnothing$.
\end{lemma}
%PROOF AND DEPENDENCIES

To show the Sylowness of the subgroup we shall construct we need the following lemma:

\begin{lemma}
 \label{MaximalAbelianSubgroup.mul_center_inj}
 \uses{SpecialSubgroups.center_SL2_eq_Z}
 \lean{MaximalAbelianSubgroup.mul_center_inj}
 \leanok
 Let $S$ and $Q$ be subgroups of a group $\SL_2(F)$ where $S \le Q$ and furthermore, we have the added condition that either $I = -I$ or $-I \notin S$ and suppose $SZ = QZ$ then
 $S = Q$
\end{lemma}
%PROOF AND DEPENDENCIES

\begin{theorem}
\label{MaximalAbelianSubgroup.A_eq_Q_join_Z_of_IsConj_s_or_neg_s}
\uses{MaximalAbelianSubgroup.centralizer_eq_conj_SZ_of_IsConj_s_or_IsConj_neg_s, SpecialSubgroups.S_join_Z_eq_SZ, MaximalAbelianSubgroup.conj_T_join_Z_meet_G_eq_conj_T_meet_G_join_Z, MaximalAbelianSubgroup.conj_inv_conj_eq, SpecialSubgroups.center_SL2_eq_Z,
  SpecialMatrices.order_s_eq_char, orderOf_injective, MaximalAbelianSubgroup.center_le, 
  MaximalAbelianSubgroup.IsCyclic_and_card_coprime_CharP_of_IsConj_d, IsElementaryAbelian.subgroupOf, IsPGroup.exists_le_sylow,
  MaximalAbelianSubgroup.exists_noncenter_of_card_center_lt_card_center_Sylow, MaximalAbelianSubgroup.mul_center_inj}
\lean{MaximalAbelianSubgroup.A_eq_Q_join_Z_of_IsConj_s_or_neg_s}
\leanok
Let $F$ be an algebraically closed field of characteristic $p$ and let $G$ be a finite subgroup of $\SL_2(F)$ containing $Z$, let $A$ be a subgroup of $\SL_2(F)$ which is a maximal abelian subgroup of $G$ and furthermore suppose $Z < A$ and $A = C_{\SL_2(F)}(x) \sqcap G$ where $x \in G \setminus Z \subseteq \SL_2(F)$ and $x$ is conjugate to $s_\sigma$ for some $\sigma \in F$
then there exists a finite nontrivial elementary abelian Sylow $p$-subgroup of $G$ such that $A = Q \sqcup Z$.
\end{theorem}
%PROOF


\begin{lemma}
\label{MaximalAbelianSubgroup.IsCyclic_and_card_coprime_CharP_or_eq_Q_join_Z_of_center_ne}
\uses{MaximalAbelianSubgroup.center_not_mem, MaximalAbelianSubgroup.eq_centralizer_meet_of_center_lt, SL2_IsConj_d_or_IsConj_s_or_IsConj_neg_s_of_AlgClosed, MaximalAbelianSubgroup.A_eq_Q_join_Z_of_IsConj_s_or_neg_s}
\lean{MaximalAbelianSubgroup.IsCyclic_and_card_coprime_CharP_or_eq_Q_join_Z_of_center_ne}
\leanok
If $G \ne Z(G)$ then an element of $A$ of $\mathfrak{M}$, the maximal abelian subgroups of $G$, is either cyclic group whose order is relatively prime to $p$, the characteristic of the field $F$; or of the form $Q \times Z = Q \sqcup Z$ where $Q$ is an elementary abelian Sylow $p$-subgroup of $G$.
\end{lemma}
\begin{proof}
  Since $Z \not \in \mathfrak{M}$, each $A \in \mathfrak{M}$ contains at least one $x \not \in Z$. By Proposition  \ref{6.3} this $x$ is conjugate to either $d_\delta$ or $\pm s_\sigma$ in $\SL_2(F)$. It suffices to only consider these cases: \\
  \\
   \space $\pmb{x}$ \textbf{conjugate to} $\pmb{d_\delta}$ \textbf{in} $\pmb {L}$. There is a $y \in L$ such that $x = y d_\delta y^{-1}$. Since $x \not \in Z$, we have $d_\delta \not \in Z$, because otherwise we get the contradiction,
  \begin{align*} x =  y d_\delta y^{-1} = d_\delta \in Z.
  \end{align*}
  Thus $\omega \neq \pm 1$. Let $A = C_G(x)$, since $C_G(x) \in \mathfrak{M}$ by part (i). Observe that
  \begin{align*}  C_G(d_\delta) &<  C_{\SL_2(F)}(d_\delta)  \tag{see proof of (i)}
  \\ &= D  \tag{by Lemma \ref{6.4ii}}
  \\ &\cong F^*.  \tag{by Lemma \ref{6.1b}}
  \end{align*}
  
  Since $A$ is conjugate to $C_G(d_\delta)$ by Proposition \ref{conjcent}, we have that $A$ is isomorphic to a finite subgroup of $F^*$ and by Lemma \ref{finsubcyc}, $A$ is cyclic. By Lagrange's Theorem any finite subgroup of $F^*$ has an order which divides $p^m - 1$ for some $m \in \mathbb{Z}^+$, and since $p \nmid (p^m - 1)$, $|A|$ is relatively prime to $p$. \\
  \\
   \space $\pmb{x}$ \textbf{conjugate to} $\pmb{\pm s_\sigma}$ \textbf{in} $\pmb{L}$. Again let $A = C_G(x) \in \mathfrak{M}$. $A$ is conjugate to $C_G({\pm s_\sigma})$ in $\SL_2(F)$ by Proposition \ref{conjcent}. Since $x \notin Z$, we have $\lambda \neq 0$. Observe that
  \begin{align*}  C_G({\pm s_\sigma}) &<  C_{\SL_2(F)}({\pm s_\sigma})
  \\&= T \times Z  \tag{by Lemma \ref{6.4i}}
  \\&\cong F \times Z. \tag{by Lemma \ref{6.1b}}
  \end{align*}
  
  So $A$ is isomorphic to a finite subgroup of $F \times Z$, call it $Q \times Z$. Now $A = Q \times Z \cong QZ$ by Corollary \ref{directproductZ}, which means that an arbitrary element of $A$ is of the form $q_1z_1$, where $q_1 \in Q$, $z_1 \in Z$.
  \begin{align*} q_1z_1q_2z_2 &= q_2z_2 q_1z_1, \tag{$A \in \mathfrak{M}$}
  \\ q_1q_2z_1z_2 &= q_2q_1z_1z_2, \tag{$z_1$, $z_2 \in Z$}
  \\  q_1q_2z_1z_2(z_1z_2)^{-1} &= q_2q_1z_1z_2(z_1z_2)^{-1},
  \\ q_1q_2 &= q_2q_1.
  \end{align*}
  Thus $Q$ is also abelian. Recall from the proof of Proposition \ref{6.3}(ii) that all non-trivial elements of $S$ have order $p$, so each non-trivial element of $Q$ has order $p$ which means that $Q$ is elementary abelian. Thus $Q$ has order $p^m$, for some $m \in \mathbb{Z}^+$. \\
  \\
  Now let $S$ be a Sylow $p$-subgroup containing $Q$. We apply Lemma \ref{IsElementaryAbelian.dvd_card} to determine that $p$ divides $|Z(S)|$, moreover $|Z(S)| \geq p$. \\
  \\
  If $p=2$, then $Z=I_L$ by Lemma \ref{6.2}. So $|Z| = 1$ and hence $|Z(S)| \geq 2 > |Z|$.\\
  If $p > 2$, then  $Z = \langle - I_L \rangle$ also by Lemma \ref{6.2}. So $|Z| = 2$ and again we get $|Z(S)| > 2 = |Z|$. \\
  \\
  So $Z(S)$ must contain at least one element which is not in $Z$, let $y$ be one such element. Let $s_1z_1$ be an arbitrary element of $S \times Z$.
  \begin{align*}
  (s_1z_1)y(s_1z_1)^{-1} &= (s_1z_1)y(z_1^{-1}s_1^{-1})
  \\ &= s_1y(z_1z_1^{-1})s_1^{-1} \tag{since $y \in L$, $z_1 \in Z$}
  \\ &= y(s_1s_1^{-1}) \tag{since $s_1 \in S$, $y \in Z(S)$}
  \\ &= y
  \end{align*}
  
  Thus $s_1z_1 \in C_G(y)$ and since it was chosen arbitrarily, $S \times Z \subset C_G(y)$. Also since $y \in G \! \setminus \! Z$ we have $C_G(y) \in \mathfrak{M}$ by part (i).
  
  \begin{equation*}
  A = Q \times Z \subset S \times Z \subset C_G(y).
  \end{equation*}
  
  Since $A$ and $C_G(y)$ are both in $\mathfrak{M}$ it must be that $A = C_G(y)$. This means $Q = S$ and $Q$ is a Sylow $p$-subgroup of G.\\
\end{proof}

\begin{theorem}
\label{MaximalAbelianSubgroup.IsCyclic_and_card_coprime_CharP_or_eq_Q_join_Z}
\uses{MaximalAbelianSubgroup.IsCyclic_and_card_coprime_CharP_or_eq_Q_join_Z_of_center_ne, MaximalAbelianSubgroup.IsCyclic_and_card_Coprime_CharP_of_center_eq}
\lean{MaximalAbelianSubgroup.IsCyclic_and_card_coprime_CharP_or_eq_Q_join_Z}
\leanok
An element $A$ of $\mathfrak{M}$ is either a cyclic group whose order is relatively prime to $p$, or of the form $Q \times Z$ where $Q$ is an elementary abelian Sylow $p$-subgroup of $G$. \vspace{3mm}
\end{theorem}
\begin{proof}
  First consider the trivial case of $G=Z$.
  By \ref{MaximalAbelianSubgroup.IsCyclic_and_card_Coprime_CharP_of_center_eq} we yield that $A$ is cyclic and has cardinality coprime to $p$.
  \\
  Now assume $G \neq Z$.
  By \ref{MaximalAbelianSubgroup.IsCyclic_and_card_coprime_CharP_or_eq_Q_join_Z_of_center_ne} we yield that $A$ is either a cyclic group whose order is relatively prime to $p$, or of the form $Q \times Z$ where $Q$ is an elementary abelian Sylow $p$-subgroup of $G$.
\end{proof}

\begin{theorem}
  \label{MaximalAbelianSubgroup.index_normalizer_le_two}
  \uses{MaximalAbelianSubgroup.IsCyclic_and_card_coprime_CharP_or_eq_Q_join_Z, normalizer_subgroup_D_eq_DW}
  \lean{MaximalAbelianSubgroup.index_normalizer_le_two}
  \leanok
If $A \in \mathfrak{M}$ and $|A|$ is relatively prime to $p$, then we have $[N_G(A): A] \leq 2$. 

\end{theorem}
\begin{proof}
  (iv) If $|A| \leq 2$ then $A=Z=G$. So $A$ is trivially normal in $G$ and $[N_G(A): A] = 1$. \\
  \\
  Now assume that $|A| > 2$. Since $|A|$ is relatively prime to $p$, we have that $A$ is a cyclic group conjugate to a finite subgroup of $D$ in $\SL_2(F)$ by the proof of part \ref{MaximalAbelianSubgroup.IsCyclic_and_card_coprime_CharP_or_eq_Q_join_Z}, call this subgroup ${\widetilde{A}}$. Thus both ${\widetilde{A}}$ and $D$ have orders greater than 2. Applying Proposition \ref{normalizer_subgroup_D_eq_DW} we observe that
  \begin{align}\label{norm1}  N_{\SL_2(F)}({\widetilde{A}}) = \langle D , w \rangle = N_{\SL_2(F)}(D).
  \end{align}
  
  Since $A$ and ${\widetilde{A}}$ are conjugate in $\SL_2(F)$, there exists an element $z \in L$ such that $zAz^{-1} = {\widetilde{A}}$. This $z$ determines an inner automorphism of $\SL_2(F)$ defined by
  \begin{align*} 
      i_z: L \longrightarrow L,  \qquad \text{where} \quad  i_z(t) = z t z^{-1}  \quad \forall \; t \in L.
  \end{align*}
  
  Let $i_z(G) = {\widetilde{G}}$ denote the image of $G$ under $i_z$. Since $A$ is a maximal abelain subgroup of $G$ it's a simple task to show that ${\widetilde{A}}$ is a maximal abelian subgroup of ${\widetilde{G}}$ and I will leave this to the reader to verify. We now show that $i_z(N_G(A)) = N_{\widetilde{G}}({\widetilde{A}})$ . Take an arbitrary $g \in N_G(A)$.
  \begin{align*} (z g z^{-1}) {\widetilde{A}} (z g z^{-1})^{-1} &= z g (z^{-1} {\widetilde{A}} z) g^{-1} z^{-1}
  \\ &=  z (g A g^{-1}) z^{-1} \tag{since $zAz^{-1} = {\widetilde{A}}$ }
  \\ &= z A z^{-1} \tag{since $g \in N_G(A)$}
  \\ &= {\widetilde{A}}.
  \end{align*}
  
  So $z g z^{-1} = i_z(g) \in N_{\widetilde{G}}({\widetilde{A}})$ and since it was chosen arbitrarily, $i_z(N_G(A)) \subset N_{\widetilde{G}}({\widetilde{A}})$. Now take an arbitrary $z h z^{-1} \in N_{\widetilde{G}}({\widetilde{A}})$.
  \begin{align*} {\widetilde{A}} &= (z h z^{-1}) {\widetilde{A}} (z h z^{-1})^{-1}
  \\ &= z h (z^{-1} {\widetilde{A}} z) h^{-1} z^{-1}
  \\ &= z h A h^{-1} z^{-1}. \tag{since $A = z^{-1} {\widetilde{A}} z$}
  \end{align*}
  
  Now multiplication on the left by $z^{-1}$ and right by $z$ gives:
  \begin{align*} A = z^{-1} {\widetilde{A}} z = h A h^{-1},
  \end{align*}
  
  so $h \in N_G(A)$. Furthermore, $z h z^{-1}$ and indeed the whole of $N_{\widetilde{G}}({\widetilde{A}})$ is contained in $i_z(N_G(A))$. Thus $ i_z(N_G(A)) = N_{\widetilde{G}}({\widetilde{A}})$. In particular, we have,
  \begin{align}\label{6.8iv1} [N_G(A): A] = [N_{\widetilde{G}}({\widetilde{A}}): {\widetilde{A}}].
  \end{align}
  
  Since ${\widetilde{G}} < L$, the normaliser of ${\widetilde{A}}$ in ${\widetilde{G}}$ is simply the normaliser of ${\widetilde{A}}$ in $\SL_2(F)$ restricted to ${\widetilde{G}}$, thus $N_{\widetilde{G}}({\widetilde{A}}) < N_{\SL_2(F)}({\widetilde{A}}) = N_{\SL_2(F)}(D)$ by (\ref{norm1}). Now since $D \vartriangleleft N_{\SL_2(F)}(D)$, the Second Isomorphism Theorem shows that,
  \begin{align}\label{2iso} N_{\widetilde{G}}({\widetilde{A}})/( N_{\widetilde{G}}({\widetilde{A}}) \cap D) \; \cong \; DN_{\widetilde{G}}({\widetilde{A}}) / D.
  \end{align}
  \\
  Clearly ${\widetilde{A}} \subset {\widetilde{G}} \cap D$. We show that this inclusion is infact an equality. Assume that there exists some $d_\delta \in  {\widetilde{G}} \cap D$ which is not in ${\widetilde{A}}$. The group $\langle d_\delta , {\widetilde{A}} \rangle$ is thus an abelian subgroup of ${\widetilde{G}}$, strictly larger than ${\widetilde{A}}$ and contradicting the fact that ${\widetilde{A}}$ is maximal abelian in ${\widetilde{G}}$. Thus ${\widetilde{A}} =  {\widetilde{G}} \cap D$. It is trivial to see that ${\widetilde{A}} \subset N_{\widetilde{G}}({\widetilde{A}}) \cap D$. Also $N_{\widetilde{G}}({\widetilde{A}}) \cap D \subset {\widetilde{G}} \cap D = {\widetilde{A}}$. So,
  \begin{align}\label{parti} {\widetilde{A}} =  N_{\widetilde{G}}({\widetilde{A}}) \cap D.
  \end{align}
  
  Observe also that, 
  \begin{align}\label{index1or2} DN_{\widetilde{G}}({\widetilde{A}}) = \{ D, \langle D, w \rangle \} \subset \langle D, w \rangle = N_{\SL_2(F)}(D).
  \end{align}
  
  Now we piece the preceding results together to give the desired result.
  \begin{align*}  N_{\widetilde{G}}({\widetilde{A}}) / {\widetilde{A}} \; & \cong \;  N_{\widetilde{G}}({\widetilde{A}})/( N_{\widetilde{G}}({\widetilde{A}}) \cap D) \tag{by (\ref{parti})}
  \\ & \cong \; DN_{\widetilde{G}}({\widetilde{A}}) / D \tag{by (\ref{2iso})}
  \\ & \subset N_{\SL_2(F)}(D) / D \tag{by (\ref{index1or2})}
  \\ &= \langle D, w \rangle / D \; \cong \; \mathbb{Z}_2.
  \end{align*}
  
  We have shown that $N_{\widetilde{G}}({\widetilde{A}}) / {\widetilde{A}}$ is isomorphic to a subset of $\mathbb{Z}_2$. Thus by (\ref{6.8iv1}) we have established that, $$[N_G(A): A] = [N_{\widetilde{G}}({\widetilde{A}}): {\widetilde{A}}] \leq 2.$$
  \vspace{-2mm}
\end{proof}


\begin{theorem}
  \label{MaximalAbelianSubgroup.of_index_normalizer_eq_two}
  \uses{MaximalAbelianSubgroup.index_normalizer_le_two}
  \lean{MaximalAbelianSubgroup.of_index_normalizer_eq_two}
  If $A \in \mathfrak{M}$, $|A|$ is relatively prime to $p$, and if $[N_G(A): A] = 2$, then there is an element $y$ of $N_G(A) \! \setminus \! A$ such that, 
  \vspace{-1mm}
  \begin{align*} yxy^{-1} = x^{-1} \qquad \forall x \in A.\end{align*}
  \end{theorem}
\end{theorem}
\begin{proof}
  
  If $[N_G(A): A] = 2$, then the above argument at \ref{MaximalAbelianSubgroup.index_normalizer_le_two} shows that $N_{\widetilde{G}}({\widetilde{A}}) / {\widetilde{A}} \; \cong \; \mathbb{Z}_2$. Thus $DN_{\widetilde{G}}({\widetilde{A}}) = N_{\SL_2(F)}(D) = \langle D, w \rangle$. This means that $N_{\widetilde{G}}({\widetilde{A}})$ contains some element $wd_\omega$. In fact, since $w d_\delta \not \in D$, we have $w d_\delta \in N_{\widetilde{G}}({\widetilde{A}}) \! \setminus \! {\widetilde{A}}$. Take any element $x \in A$. Since ${\widetilde{A}} = zAz^{-1}$, $zxz^{-1} \in {\widetilde{A}}$, call it $d_\sigma$. Let $y = z^{-1}w d_\delta z$. Since $wd_\omega \in N_{\widetilde{G}}({\widetilde{A}}) \! \setminus \! {\widetilde{A}}$ it follows that $y \in N_G(A)\! \setminus \! A$. We show that this $y$ inverts $x$:
  \begin{align*} yxy^{-1} &= (z^{-1}w d_\delta z)(z^{-1} d_\sigma z)(z^{-1}d^{-1}_\omega w^{-1} z)
  \\ &= z^{-1} w d_\delta  d_\sigma d^{-1}_\omega w^{-1} z
  \\ &=  z^{-1} w  d_\sigma  w^{-1} z 
  \\ &=  z^{-1}  d^{-1}_\sigma z  \tag{by Lemma \ref{6.1}}
  \\ &= x^{-1}.
  \end{align*}
\end{proof}


\begin{theorem}
  \label{MaximalAbelianSubgroup.exists_IsCyclic_K_normalizer_eq_Q_join_K}
  \uses{normalizer_subgroup_S_le_L, MaximalAbelianSubgroup.IsCyclic_and_card_coprime_CharP_or_eq_Q_join_Z}
  \lean{MaximalAbelianSubgroup.exists_IsCyclic_K_normalizer_eq_Q_join_K}
  Let $Q$ be a Sylow $p$-subgroup of $G$. If $Q \neq \{I_G\}$, then there is a cyclic subgroup $K$ of $G$ such that $N_G(Q) = Q \sqcup K = QK$. \\
\end{theorem}
\begin{proof}
By part \ref{MaximalAbelianSubgroup.IsCyclic_and_card_coprime_CharP_or_eq_Q_join_Z}, $Q$ is conjugate to a finite subgroup of $S$ in $\SL_2(F)$. In fact, without loss of generality we can assume that $Q \subset S$, moreoever $Q \subset S \cap G$. We show that this is in fact an equality by showing that the reverse inclusion also holds. 
Let $s_\sigma$ be an arbitrary element of $S \cap G$. Then $\langle s_\sigma, Q \rangle$ is a $p$-group of $G$ which must be equal to $Q$ since it is a Sylow $p$-subgroup of $G$. Thus $s_\sigma \in Q$ and
\begin{align}\label{Q=TNG} Q = S \cap G.
\end{align}

Since $|Q| > 1$, Proposition \ref{normalizer_subgroup_S_le_L} gives that $N_G(Q) \subset N_{\SL_2(F)}(Q) \subset H$. So $N_G(Q) \subset H \cap G$. Now take an arbitrarily chosen $d_\delta s_\sigma \in H \cap G$ and $s_\gamma \in Q$.
\begin{align*} (d_\delta s_\sigma) s_\gamma (d_\delta s_\sigma)^{-1} &= d_\delta ( s_\sigma s_\gamma  s_{-\sigma}) d^{-1}_\delta
\\ &=  d_\delta s_\gamma d^{-1}_\delta \tag{by Lemma \ref{6.1}}
\\ &= t_\sigma. \tag{where $\sigma = \mu \omega^{-2}$, by Lemma \ref{6.1}}
\end{align*}

Since it is a product of elements of $G$, $s_\sigma \in S \cap G = Q$ by (\ref{Q=TNG}). Thus $d_\delta s_\sigma \in N_G(Q)$ and indeed the whole of $H \cap G$ is contained in $N_G(Q)$ and
\begin{align}\label{normQ=HNG} N_G(Q) = H \cap G.
\end{align}

We now define a map $\phi$ by,
\begin{align*} \phi : N_G(Q) \longrightarrow D, \qquad \text{where} \quad \! \phi(d_\delta s_\sigma) = d_\delta \quad \forall \; d_\delta s_\sigma \in N_G(Q).
\end{align*}

Next we determine the kernel of $\phi$.
\begin{align*} \ker(\phi) &= \{ d_\delta s_\sigma \in N_G(Q) : \phi(d_\delta s_\sigma) = I_G \}
\\ &= N_G(Q) \cap T
\\ &= H \cap G \cap T \tag{by (\ref{normQ=HNG})}
\\ &= T \cap G = Q. \tag{by (\ref{Q=TNG})}
\end{align*}

We show that $\phi$ is a group homomorphism. Take $d_\delta s_\sigma$, $d_\rho s_\gamma$ from $ N_G(Q)$.
\begin{align*} \phi(d_\delta s_\sigma d_\rho s_\gamma) &= \phi(d_\delta d_\rho t_\sigma s_\gamma) \tag{where $\sigma = \lambda \rho^2$, by Lemma \ref{6.1}}
\\ &= d_\delta d_\rho
\\ &= \phi(d_\delta s_\sigma) \phi(d_\rho s_\gamma).
\end{align*}

Thus by the First Isomorphism Theorem,
\begin{align}\label{6.8viso} N_G(Q) / Q &\cong \phi(N_G(Q)),
\end{align}

Since $N_G(Q)$ is a finite group, it's image under $\phi$ is thus a finite subgroup of $D$. Furthermore, since $D \cong F^*$ (by Lemma \ref{6.1b}), $\phi(N_G(Q))$ is a cyclic group whose order divides $p^m-1$ and is therefore relatively prime to $p$, and by \eqref{6.8viso}, so too is $N_G(Q) / Q$. \\
\\
Let $r$ be the order of $N_G(Q) / Q$. Since it is cyclic, $N_G(Q)/Q$ is generated by a single element, namely a coset of $Q$ in $N_G(Q)$, call it $kQ$. So $|kQ| = r$. Observe that,
\begin{align*} (kQ)^r &= Q,
\\ k^rQ &= Q,
\\ k^r &\in Q.
\end{align*}
Since $Q$ is elementary abelian, each of it's non-trivial elements has order $p$, so $k$ has order $r$ or $rp$. In either case, since gcd$(r,p)=1$, the order of $k^p$ is $r$. Let $K = \langle k^p \rangle$. Now $|K| = r$ and
\begin{align*} |N_G(Q)| &= r|Q|
\\ &= |K||Q|
\\ &= |QK|. \tag{since $Q \cap K = I_G$} 
\end{align*}
Thus,
\begin{align}\label{QK} N_G(Q) &= QK.
\end{align}
\end{proof}



\begin{theorem}
  \label{MaximalAbelianSubgroup.K_mem_MaximalAbelianSubgroups_of_center_lt_card_K}
  \uses{MaximalAbelianSubgroup.IsCyclic_and_card_coprime_CharP_or_eq_Q_join_Z}
  \lean{MaximalAbelianSubgroup.K_mem_MaximalAbelianSubgroups_of_center_lt_card_K}
  Let $Q$ be a Sylow $p$-subgroup of $G$. If $Q \neq \{I_G\}$, then there is a cyclic subgroup $K$ of $G$ such that $N_G(Q) = Q \sqcup K = QK$. Furthermore, If $|K| > |Z|$, then $K \in \mathfrak{M}$
\end{theorem}
\begin{proof} 
Assume $|K| > |Z|$. Since $K$ is abelian, it must be contained in some maximal abelian group $A \in \mathfrak{M}$. By part \ref{MaximalAbelianSubgroup.IsCyclic_and_card_coprime_CharP_or_eq_Q_join_Z}, $A$ must also be a cyclic group whose order is relatively prime to $p$. \\
\\
Since $A$ is conjugate in $\SL_2(F)$ to a subgroup of $D$, each non-central element of $A$ has exactly 2 fixed points on the projective line $\mathscr{L}$ by Proposition \ref{6.7}. Let $A = \langle x \rangle$ and let $P_1$ and $P_2$ be the points fixed by $x$. We show by induction on $n$ that $x^n$ also fixes $P_1$ and $P_2$, for all $n \in \mathbb{Z^+}$. We do this by assuming first that $x^{n-1}$ fixes $P_i$.
\begin{align*} x^n P_i = x(x^{n-1} P_i) = x (P_i) = P_i.
\end{align*}

The importance of this is that since each element of $A$ can be expressed as some power of $x$, they must have the same two fixed points, namely $P_1$ and $P_2$. In other words, 
\begin{align}\label{stab} A \subset S_L(P_i), \qquad (\text{$i$ = 1 or 2})
\end{align}

By Proposition \ref{6.7}(ii), each element of $S$ has a common fixed point $P$ and Stab$(P) = H$. Since $K \subset H$, each element in $K$ fixes $P$. Also, since $K \subset A$, this $P$ must be equal to either $P_1$ or $P_2$. Therefore by (\ref{stab}), $A \subset \text{Stab}(P) = H$. We arrive at the following result:
\begin{align*} A &\subset H \cap G 
\\ &= N_G(Q) \tag{by (\ref{normQ=HNG})}
\\ &= QK. \tag{by (\ref{QK})}
\end{align*}

Furthermore, we get,
\begin{align*} A &= QK \cap A
\\ &= QK \cap AK \tag{$K \subset A$ so $A = AK$}
\\ &= (Q \cap A)K
\\ &= K \tag{$Q \cap A = I_G$}
\end{align*}

Thus $K \in \mathfrak{M}$.\\
\\
\end{proof}

For the duration of this paper, unless otherwise stated, $Q$ will denote a Sylow $p$-subgroup of $G$ and $K$ will be as described above. 


\section{Conjugacy of Maximal Abelian Subgroups}

\begin{definition}
  \label{ConjClassOfSet}
  \uses{MaximalAbelianSubgroupsOf}
  \lean{ConjClassOfSet}
  \leanok
  Let $G$ be a subgroup of $\SL_2(F)$ and let $A \in \mathfrak{M}$ then define the conjugacy class of $A$ to be 
  \[
  \mathcal{C}(A) = \{ x A x^{-1} : x \in G \}.
  \]
\end{definition}

\begin{definition}[Noncenter of a subgroup]
  \label{Subgroup.noncenter}
  \lean{Subgroup.noncenter}
  \leanok
  Let $A$ be a subgroup of a group $G$ let $A^* = A \setminus Z(G)$ be the "noncenter" part of $A$.
\end{definition}

Now we define the noncenter version of \ref{ConjClassOfSet}

\begin{definition}
  \label{noncenter_ConjClassOfSet}
  \uses{MaximalAbelianSubgroupsOf, Subgroup.noncenter}
  \lean{noncenter_ConjClassOfSet}
  \leanok
  Let $G$ be a subgroup of $\SL_2(F)$ and let $A^* \in \mathfrak{M}^*$ then define the conjugacy class of $A^*$ to be
  \[
  \mathcal{C}(A^*) = \left\{x A^* x^{-1} \; | \; x \in G \right\}
   \]
\end{definition}

\begin{definition}
\label{noncenter_MaximalAbelianSubgroupsOf}
\uses{MaximalAbelianSubgroupsOf, Subgroup.noncenter}
\lean{noncenter_MaximalAbelianSubgroupsOf}
\leanok
Let $\mathfrak{M}^*$ be the set of all $A_i^*$ and let $\mathcal{C}_i^*$ be the conjugacy class of $A_i^*$. \\
\end{definition}

\begin{definition}[Conjugacy class of subgroup]
\label{C}
\lean{C}
\leanok
Let $A \in \mathfrak{M}$ and define the union of the conjugacy classes of $A$ to
\begin{align*} 
  C(A) = \bigcup_{x \in G} x A x^{-1}
\end{align*}
\end{definition}

Similarly, we define the analogous for the noncenter part of a maximal abelian subgroup:

\begin{definition}[Union of conjugacy class of noncenter part of a subgroup]
  \label{noncenter_C}
  \lean{noncenter_C}
  \leanok
  Let $A^* \in \mathfrak{M}^*$ then denote union of the conjugacy class of $A^*$ to be
\begin{align*}
  C(A^*) = \bigcup_{x \in G} x A^* x^{-1} = \bigcup_{B \in \mathcal{C}(A^*)} B.
\end{align*}
\end{definition}


In other words, $C_i$ denotes the set of elements of $G$ which belong to some element of $\mathcal{C}_i$. It's evident that $C_i^* = C_i \setminus Z$ and that there is a $C_i$ corresponding to each $\mathcal{C}_i$. Clearly we have the relation,

\begin{lemma}
\label{card_noncenter_C_eq_noncenter_MaximalAbelianSubgroup_mul_noncenter_ConjClassOfSet}
\uses{noncenter_MaximalAbelianSubgroupsOf, card_noncenter_C, card_noncenter_MaximalAbelianSubgroupsOf, card_noncenter_ConjClassOfSet}
\lean{card_noncenter_C_eq_noncenter_MaximalAbelianSubgroup_mul_noncenter_ConjClassOfSet}
\begin{align} |C_i^*| = |A_i^*||\mathcal{C}_i^*|.
\end{align}
\end{lemma}

Here the argument from Christopher Butler's exposition has been modified, it turns out to be significantly more
idiomatic to lean to first define the following equivalence relation and its corresponding quotient to eventually set up
the maximal abelian class equation.

\begin{lemma}[Equivalence relation on $\mathfrak{M}^*$]
\label{lift_noncenter_MaximalAbelianSubgroupsOf}
\uses{MaximalAbelianSubgroupsOf}
\lean{lift_noncenter_MaximalAbelianSubgroupsOf}
\leanok
 Let $G$ be a finite subgroup of $\SL_2(F)$, then the relation $\sim$ on the set of noncenter part of maximal abelian subgroups of $G$, $\mathfrak{M}^*$ defined by
 \[
 A \sim B \text{ if and only if } \exists x \in G \text{ such that } x A x^{-1} = B
 \]
 gives an equivalence relation.
\end{lemma}

\begin{proof}
 We show the relation $\sim$ defined above is in fact an equivalence relation on  $\mathfrak{M}^*$:

\begin{itemize}
\item $\sim$ is reflexive:

For any $x \in A$ as conjugation by an element in the subgroup defines an automorphism and so $A = x A x^{-1}$ as this automorphism fixes the subgroup.

Therefore, $A \sim A$ and $\sim$ is thus reflexive.

\item $\sim$ is symmetric:

If $A \sim B$, then $\exists \; x \in G$ such that,
\begin{align*} A = xBx^{-1} \iff x^{-1}Ax = B \iff B = yAy^{-1} \quad \text{for} \; y = x^{-1} \in G.
\end{align*}

Thus $B \sim A$ and $\sim$ is symmetric.\\

\item $\sim$ is transitive:

If $A \sim B$ and $B \sim C$, then $\exists \; x, y \in G$  such that,
\begin{align*} A = xBx^{-1} \; \text{and} \; B = yCy^{-1} \Rightarrow A = xyCy^{-1}x^{-1} = (xy)C(xy)^{-1}.
\end{align*}
Thus $A \sim C$ (since $xy \in G$), which shows that $\sim$ is transitive. \\
\end{itemize}

Therefore, we have shown that $\sim$ relation is in fact an equivalence relation on $\mathfrak{M}$
\end{proof}

\begin{remark}[Setoid type in Lean]
  TODO
\end{remark}

Now that we have set up the equivalence relation on maximal abelian subgroups we proceed to lift particular functions that will be of interest to set up the maximal abelian class equation and 
other suitable results.

\begin{lemma}[Equivalent noncenter subgroups of $\mathfrak{M}^*$ have the equal union of their conjugacy class]
  \label{noncenter_C_eq_of_related}
  \uses{noncenter_C, noncenter_MaximalAbelianSubgroupsOf}
  \lean{noncenter_C_eq_of_related}
  Let $G$ be a subgroup of $\SL_2(F)$ and let $A , B \in \mathfrak{M}*$ be a noncenter maximal abelian subgroups of $G$ where $A \sim B$
  then 
  \[
  \bigcup_{x \in G} x A x^{-1} = \bigcup_{x \in G} x B x^{-1}
  \]
\end{lemma}
% PROOF

\begin{definition}[Lift of the union of the conjugacy class of noncenter of a subgroup]
\label{lift_noncenter_C}
\uses{noncenter_C, noncenter_C_eq_of_related, noncenter_MaximalAbelianSubgroupsOf}
\lean{lift_noncenter_C}
\leanok
 Let $[A^*] \in \mathfrak{M}^* / \sim$, given for all $A^* \sim B^*$ we have that $C(A^*) = C(B^*)$ by \ref{noncenter_C_eq_of_related} we can define the lift of $C : \mathfrak{M}^* \rightarrow \mathcal{P}(\SL_2(F))$ to be 
  $\tilde{C}([A^*]) = \bigcup_{x \in G} x A^* x^{-1}$ where this map is well-defined for any choice of a representative of $[A^*]$.
\end{definition}

\begin{theorem}[The union of conjugacy classes of the set representatives of $\mathfrak{M}^* / \sim$ cover $G \setminus Z(\SL_2(F))$]
\label{union_lift_noncenter_C_eq_G_diff_center}
\uses{lift_noncenter_MaximalAbelianSubgroupsOf, lift_noncenter_C}
\lean{union_lift_noncenter_C_eq_G_diff_center}
  Let $G$ be a finite subgroup of $\SL_2(F)$ provided $\mathfrak{M}^* / \sim$ is a finite then we have the set equality
  \[
   G \setminus Z(\SL_2(F)) = \bigcup_{[A^*] \in \mathfrak{M}^* / \sim} C([A^*])
  \]
\end{theorem}
% PROOF

\begin{theorem}[Distinct elements of $\mathfrak{M}^* / \sim$ are mapped to disjoint sets through $\tilde{C}$]
  \label{disjoint_of_ne_lift_noncenter_MaximalAbelianSubgroupsOf}
  \uses{lift_noncenter_MaximalAbelianSubgroupsOf, lift_noncenter_C}
  \lean{disjoint_of_ne_lift_noncenter_MaximalAbelianSubgroupsOf}
  Let $[A^*], [B^*] \in \mathfrak{M}^* / \sim$ then
  \[
  \tilde{C}([A^*]) = \tilde{C}([B^*]) \iff [A^*] = [B^*]
  \]
  Or equivalently,
  \[ 
  C(A^*) \cap C(B^*) = \varnothing, \qquad \forall \;  A^* \not\sim B^* 
  \]
\end{theorem}
% PROOF

\begin{theorem}
  \label{card_noncenter_ConjClassOfSet_eq_card_ConjClassOfSet}
  \uses{MaximalAbelianSubgroupsOf, noncenter_MaximalAbelianSubgroupsOf, noncenter_ConjClassOfSet, ConjClassOfSet}
  \lean{card_noncenter_ConjClassOfSet_eq_card_ConjClassOfSet}
  For all maximal abelian subgroups $A \in \mathfrak{M}$ we have that 
  \[
  |\mathcal{C}(A)| = |\mathcal{C}(A^*)|
  \]
\end{theorem}
% PROOF


\begin{theorem}
\label{card_ConjClassOfSet_eq_index_normalizer}
\uses{MaximalAbelianSubgroupsOf, ConjClassOfSet}
\lean{card_ConjClassOfSet_eq_index_normalizer}

Let $G$ be a finite subgroup of $\SL_2(F)$ and let $A$ be a maximal abelian subgroup of $G$, 
$A \in \mathfrak{M}$ then $|\mathcal{C}(A)| = [G : N_G(A)]$.
\end{theorem}
%PROOF

\begin{theorem}[The maximal subgroup class equation]
  \label{card_noncenter_fin_subgroup_eq_sum_card_noncenter_mul_index_normalizer}
  \uses{lift_noncenter_MaximalAbelianSubgroupsOf, lift_card_noncenter, lift_card_noncenter_C}
  \lean{card_noncenter_fin_subgroup_eq_sum_card_noncenter_mul_index_normalizer}

Let $G$ be a finite subgroup of $\SL_2(F)$, define the equivalence relation on the maximal abelian subgroups of $G$, $\mathfrak{M}^*$ as above in \ref{lift_noncenter_MaximalAbelianSubgroupsOf}
then 
$|G \! \setminus  \! Z| = \sum_{[A^*] \in \mathfrak{M}^* / \sim} |A^*| [\tilde{C}([A^*])].$

\end{theorem}
% PROOF

\begin{proof}
(i)
\\
The equivalence class of $A_i^*$ in $\mathfrak{M}^*$ therefore coincides with the set $\mathcal{C}_i^* = \{ xA_i^*x^{-1} : x \in G \}$. Furthermore, this tells us that each $A_i^*$ belongs to exactly one conjugacy class. Thus the conjugacy classes $\mathcal{C}_i^*$ form a partition of $\mathfrak{M}^*$,
\begin{align*} \mathfrak{M}^* = \bigcup\limits_{A_i^* \in S} \mathcal{C}_i^*,  \qquad \text{and}  \qquad \mathcal{C}_i^* \cap \mathcal{C}_j^* = \varnothing, \qquad \forall \; i \neq j.
\end{align*}

Since the set of $\mathcal{C}_i^*$ are pairwise disjoint, it follows that the set of $C_i^*$ are also pairwise disjoint and we get the desired result,

\begin{align*} G \! \setminus \! Z = \bigcup\limits_{A_i^* \in S} C_i^*,  \qquad \text{and}  \qquad C_i^* \cap C_j^* = \varnothing, \qquad \forall \; i \neq j.
\end{align*}

(ii) Let $x A_i x^{-1} \in \mathcal{C}_i$ and $x A_i^* x^{-1} \in \mathcal{C}_i^*$. Since $x A_i x^{-1} \! \setminus \! Z = x A_i^* x^{-1}$, it is quite clear that,
\begin{align*} x A_i x^{-1} \in \mathcal{C}_i \iff x A_i^* x^{-1} \in \mathcal{C}_i^*.
\end{align*}
Thus $|\mathcal{C}_i^*| = |\mathcal{C}_i|$ as desired. \\
\\
(iii) Now we define a map $\phi$ by:
\begin{align*} \phi: \mathcal{C}_i &\longrightarrow G / N_G(A_i),
\\ \phi(xA_ix^{-1}) &= xN_G(A_i). \tag{$\forall \; x \in G, \; A_i \in \mathfrak{M}$}
\end{align*}

Clearly $\phi$ is trivially surjective. We now show that it is both well-defined and injective.
\begin{align*} xN_G(A_i) = yN_G(A_i) &\iff y^{-1}xN_G(A_i) = N_G(A_i) \\
&\iff y^{-1}x \in N_G(A_i) \\
&\iff (y^{-1}x)A_i(y^{-1}x)^{-1} = A_i \\
&\iff y^{-1}xA_ix^{-1}y = A_i \\
&\iff xA_ix^{-1} = yA_iy^{-1}.
\end{align*}

Hence $\phi$ is well-defined and injective. This shows that $\phi$ is a bijection proving that $|\mathcal{C}_i| = [G:N_G(A_i)]$. This is a crucial result which shows that the number of maximal abelian subgroups conjugate to $A_i$ is equal to the index of the normaliser of $A_i$ in $G$. \\
\\
(iv) This follows directly from parts (i), (ii) and (iii) and \eqref{orderorder}.
\begin{align*} G \! \setminus \! Z &= \bigcup\limits_{A_i^* \in S} C_i^*,  \qquad \text{and}  \qquad C_i^* \cap C_j^* = \varnothing, \qquad \forall \;  i \neq j, \\
 |G \! \setminus \! Z| &=  \sum_{A_i^* \in S} |C_i^*| = \sum_{A_i^* \in S} |A_i^*||\mathcal{C}_i^*| = \sum_{A_i^* \in S} |A_i^*||\mathcal{C}_i|
\\ &= \sum_{A_i^* \in S} |A_i^*| [G:N_G(A_i)].
\end{align*}

\end{proof}

This theorem proves that the non-central parts of the maximal abelian subgroups form a partition of the non-central part of $G$. This will serve as a powerful tool in decomposing $G$ and counting its elements.

\section{Constructing The Class Equation}

It is necessary to prove the following 2 short lemmas before we proceed further.
 
\begin{lemma}
\label{normalizer_noncentral_eq}
\lean{normalizer_noncentral_eq} $N_G(A) =N_G(A^*)$.
\end{lemma}

\begin{proof}
(iii) Let $x \in N_G(A^*)$. Take an arbitary $a \in A = A^* \cup Z$. If $a \in A^*$, then since  $x \in N_G(A^*)$, we have $xax^{-1} \in A^* \subset A$. If $a \in Z$, then $xzx^{-1} = zxx^{-1} = z \in A$. Therefore $x$ is in the normaliser of $A$ and $N_G(A^*) \subset N_G(A)$. \\
\\
Conversely, take $y \in N_G(A)$ and $a \in A^*$. $yay^{-1} \in A = A^* \cup Z$. If  $yay^{-1} \in Z$, then
\begin{align*} yay^{-1} &= z, \tag{some $z \in Z$}
\\ a &= y^{-1}zy = y^{-1}yz = z \not \in A^*.
\end{align*}
This contradicts the fact that $a \in A^*$. Therefore $yay^{-1} \in A^*$ and $y \in N_G(A^*)$. Since $y$ was chosen arbitrarily we get $N_G(A) \subset N_G(A^*)$ and hence $N_G(A) =N_G(A^*)$.

\end{proof}

\begin{lemma}
\label{normalizer_Sylow_join_center_eq_normalizer_Sylow}
\uses{MaximalAbelianSubgroup.IsCyclic_and_card_coprime_CharP_or_eq_Q_join_Z, MaximalAbelianSubgroup.index_normalizer_le_two}
\lean{normalizer_Sylow_join_center_eq_normalizer_Sylow}
$N_G(Q \times Z) = N_G(Q)$.
\end{lemma}

\begin{proof} 

If $p = 2$ then $Z = I_G$ and the result is trivial. Now assume $p \neq 2$. Thus $|Z| = 2$. Let $x$ and $q_1$ be arbitrarily chosen elements of $N_G(Q)$ and $Q$ respectively.
\begin{align*} xq_1x^{-1} &= q_2, \tag{for some $q_2 \in Q$}
\\ xq_1x^{-1}z_1 &= q_2z_1,
\\ xq_1z_1x^{-1} &= q_2z_1 \in Q \times Z.
\end{align*}
Thus any element $x$ which is in $N_G(Q)$ is also in $N_G(Q \times Z)$ so we have $N_G(Q) \subset N_G(Q \times Z)$. \\
\\
Let $q_1 z_1$ be an arbitrarily chosen element of $Q \times Z$ such that $q_1 \in Q$ and $z_1 \in Z$. Now let $y$ be an arbitrarily chosen element of $N_G(Q \times Z)$.
\begin{align*} y q_1 z_1 y^{-1} = q_2 z_2 \in Q \times Z. \qquad (\text{where $q_2 \in Q$ and $z_2 \in Z$}) 
\end{align*}

Consider now the order of $q_1z_1$ in $G$. Since $p \neq 2$, $Q \cap Z = I_G$ and $|q_1 z_1| = |q_1| |z_1|$. Note that $q_1 z_1$ and $q_2 z_2$ are conjugate in $G$, and thus their orders are equal. This means that $|z_1| = |z_2|$, because otherwise 2 would divide one of them and not the other. Thus $z_1 = z_2$ and,
\begin{align*} y q_1z_1 y^{-1} &=  q_2z_2 = q_2z_1
\\ y q_1 y^{-1} z_1 &= q_2z_1,
\\ y q_1 y^{-1} &= q_2 \in Q
\end{align*}
Hence $y \in N_G(Q)$. Furthermore, since $y$ was chosen arbitrarily, any element which is in $N_G(Q \times Z)$ is also in $N_G(Q)$, so $N_G(Q \times Z) = N_G(Q)$ as desired.

\end{proof}

We now start to count the elements of the seperate components of $G$ and use the preceeding 2 theorems to construct what will be an invaluable formula in determining the structure of $G$, something we will call the \textbf{Maximal Abelian Subgroup Class Equation} of $G$. \\
\\
First we split $\mathfrak{M}$ into the conjugacy classes of it's elements. Theorem \ref{MaximalAbelianSubgroup.IsCyclic_and_card_coprime_CharP_or_eq_Q_join_Z} tells us that every maximal abelian subgroup is either a cyclic subgroup whose order is relatively prime to $p$ or of the form $Q \times Z$ where $Q$ is a Sylow $p$-subgroup. Let $\mathcal{C}_1, \mathcal{C}_2,...,\mathcal{C}_s, \mathcal{C}_{s+1},..., \mathcal{C}_{s+t}$ (where $s, t \in \mathbb{Z}^+$) denote the conjugacy classes of the cyclic subgroups whose order is relatively prime to $p$. Recall that part (iv) of Theorem \ref{MaximalAbelianSubgroup.index_normalizer_le_two} tells us that $[N_G(A): A] = 1$ or 2. Let $A_i$ be a representative from each $\mathcal{C}_i$ such that,
\begin{align*} [N_G(A_i) : A_i] &= 1, \tag{for  $i \leq s$} \\[2mm]
[N_G(A_i) : A_i] &= 2. \tag{for  $s < i \leq s+t$}, \end{align*}

Now let $Q_1$ and $Q_2$ be any two Sylow $p$-subgroups of $G$. By the Second Sylow Theorem, $Q_1$ and $Q_2$ are conjugate to each other in $G$. That is, there exists a $g \in G$ such that $gQ_1g^{-1} = Q_2$.

\begin{align*} gQ_1g^{-1} = Q_2 &\iff gQ_1g^{-1}Z = Q_2Z 
\\ &\iff gQ_1Zg^{-1} = Q_2Z
\\ &\iff g(Q_1 \times Z)g^{-1} = (Q_2 \times Z). \tag{by Corollary \ref{directproductZ}}
\end{align*} 

So $Q_1 \times Z$ and $Q_2 \times Z$ belong to the same conjugacy class, furthermore there is thus only 1 conjugacy class of elements of this form in $\mathfrak{M}$. Let $\mathcal{C}_{Q \times Z}$ denote this conjugacy class and let $Q \times Z$ be a representative from it. The following diagram provides a visual representation of $G$ divided into it's maximal abelian subgroups.

% \begin{center}
% \begin{tikzpicture}[thick, scale=0.4]

% \draw (0,0) ellipse (22pt and 22pt); 

% \draw[dashed][rotate around={308:(0,0)},red] (3,0) ellipse (108pt and 41pt);  
% \draw[dashed][rotate around={318:(0,0)},red] (3,0) ellipse (108pt and 41pt);  
% \draw[rotate around={328:(0,0)},red] (3,0) ellipse (108pt and 41pt); 
% \draw[dashed][rotate around={338:(0,0)},red] (3,0) ellipse (108pt and 41pt);  

% \draw[dashed][rotate around={301:(0,0)},lightgray] (3,0) ellipse (94pt and 37pt); 
% \draw[dashed][rotate around={296:(0,0)},lightgray] (3,0) ellipse (94pt and 37pt); 
% \draw[dashed][rotate around={291:(0,0)},lightgray] (3,0) ellipse (94pt and 37pt);  

% \draw[dashed][rotate around={258:(0,0)},orange] (2,0) ellipse (79pt and 37pt);  
% \draw[rotate around={270:(0,0)},orange] (2,0) ellipse (79pt and 37pt);  
% \draw[dashed][rotate around={282:(0,0)},orange] (2,0) ellipse (79pt and 37pt); 

% \draw[dashed][rotate around={198:(0,0)},cyan] (3.4,0) ellipse (120pt and 35pt);  
% \draw[rotate around={203:(0,0)},cyan] (3.4,0) ellipse (120pt and 35pt);
% \draw[dashed][rotate around={208:(0,0)},cyan] (3.4,0) ellipse (120pt and 35pt);
% \draw[dashed][rotate around={213:(0,0)},cyan] (3.4,0) ellipse (120pt and 35pt);
% \draw[dashed][rotate around={218:(0,0)},cyan] (3.4,0) ellipse (120pt and 35pt);

% \draw[dashed][rotate around={128:(0,0)},blue] (2,0) ellipse (79pt and 37pt);  
% \draw[rotate around={148:(0,0)},blue] (2,0) ellipse (79pt and 37pt);
% \draw[dashed][rotate around={168:(0,0)},blue] (2,0) ellipse (79pt and 37pt);

% \draw[dashed][rotate around={108:(0,0)},lightgray] (3,0) ellipse (94pt and 37pt); 
% \draw[dashed][rotate around={113:(0,0)},lightgray] (3,0) ellipse (94pt and 37pt); 
% \draw[dashed][rotate around={118:(0,0)},lightgray] (3,0) ellipse (94pt and 37pt); 

% \draw[dashed][rotate around={82:(0,0)},teal] (3,0) ellipse (108pt and 41pt);  
% \draw[rotate around={86:(0,0)},teal] (3,0) ellipse (108pt and 41pt);  
% \draw[dashed][rotate around={90:(0,0)},teal] (3,0) ellipse (108pt and 41pt);  
% \draw[dashed][rotate around={94:(0,0)},teal] (3,0) ellipse (108pt and 41pt);  
% \draw[dashed][rotate around={98:(0,0)},teal] (3,0) ellipse (108pt and 41pt);  

% \draw[dashed][rotate around={18:(0,0)},green] (3.4,0) ellipse (120pt and 35pt);
% \draw[rotate around={26:(0,0)},green] (3.4,0) ellipse (120pt and 35pt);
% \draw[dashed][rotate around={34:(0,0)},green] (3.4,0) ellipse (120pt and 35pt); 

% \node[] at (0,-10) {\resizebox{8cm}{!}{Fig 1: $G$ arranged into it's maximal abelian subgroups}};
% \node[] at (0,0) {\resizebox{.3cm}{!}{$Z$}};

% \node[] at (6.1,-4.5) {\resizebox{.5cm}{!}{$A_1$}};
% \node[] at (-0.2,-5.6) {\resizebox{.5cm}{!}{$A_s$}};
% \node[] at (-7.8,-4.1) {\resizebox{.9cm}{!}{$A_{s+1}$}};
% \node[] at (-5.0,3.3) {\resizebox{.9cm}{!}{$A_{s+2}$}};
% \node[] at (0.2,7.6) {\resizebox{.9cm}{!}{$A_{s+t}$}};
% \node[] at (8.0,4.0) {\resizebox{1.1cm}{!}{$Q \times Z$}};

% \node[] at (7.9,-6.0) {\resizebox{.5cm}{!}{$\mathcal{C}_1$}};
% \node[] at (-0.2,-7.9) {\resizebox{.5cm}{!}{$\mathcal{C}_s$}};
% \node[] at (-10.9,-4.7) {\resizebox{1.0cm}{!}{$\mathcal{C}_{s+1}$}};
% \node[] at (-8.2,4.9) {\resizebox{1.0cm}{!}{$\mathcal{C}_{s+2}$}};
% \node[] at (-0.1,10.0) {\resizebox{1.0cm}{!}{$\mathcal{C}_{s+t}$}};
% \node[] at (11.6,5.1) {\resizebox{1.2cm}{!}{$\mathcal{C}_{Q \times Z}$}};

% \node[scale=1.6, rotate=143,gray] at (6.9,-5.1) { $\Bigg\{$ };
% \node[scale=1.1, rotate=90,gray] at (0,-6.6) { $\Bigg\{$ };
% \node[scale=1.3, rotate=28,gray] at (-8.9,-4.8) { $\Bigg\{$ };
% \node[scale=1.4, rotate=328,gray] at (-6.3,3.9) { $\Bigg\{$ };
% \node[scale=1.2, rotate=270,gray] at (0.0,8.7) { $\Bigg\{$ };
% \node[scale=1.2, rotate=206,gray] at (9.6,4.6) { $\Bigg\{$ };

% \end{tikzpicture}
% \end{center}

We can reformulate the counting formula in Theorem \ref{card_noncenter_fin_subgroup_eq_sum_card_noncenter_mul_index_normalizer} using the notation we have introduced to show that it agrees with the intuitive approach that Fig 1 suggests.
% MODIFY NOTATION

\begin{align*} 
  |G \! \setminus \! Z| = \sum_{A_i^* \in S} |A_i^*| [G:N_G(A_i)] = \sum_{A_i^* \in S} |C_i^*| = |C_{Q \times Z}^*| + \sum_{i=1}^{s+t} |C_i^*|.
\end{align*}

We are now able to begin to evaluate $G$. Firstly, let $|Z| = e$ and $|G| = eg$. We know well by now that $e = 1$ or 2 depending on whether $p$ equals 2 or not, and by Lagrange's Theorem, the order of a subgroup divides the order of the group, so $e$ divides $|G|$ since $Z < G$. \\
\\
We consider the cyclic case first. Again, by Lagrange's Theorem, since $Z$ is a subgroup of each $A_i$, $e$ divides $|A_i|$. So set $|A_i| = eg_i$. Since $Z \notin \mathfrak{M}$, each $A_i$ is therefore strictly larger than $Z$ and so each $g_i$ is an integer greater than or equal to 2. \\
\\
To determine the order of each $C_i$, we return to the set $\mathfrak{M}^*$. The size of one representative of each class is,
\begin{align*} |A_i^*| = |A_i \! \setminus \! Z| = eg_i-e = e(g_i-1). \end{align*}
The number of $A_i^*$ in each conjugacy class $\mathcal{C}_i$ for $i \leq s$ is thus,
\begin{align*} |\mathcal{C}_i^*| = |\mathcal{C}_i| = [G:N_G(A_i)] = \frac{|G|}{|A_i|} = \frac{eg}{eg_i} = \frac{g}{g_i}. \end{align*}
\\
Therefore the total number of elements of $G$ in the noncentral part of $C_i$ for $i \leq s$ is,
\begin{align} \label{classeq1of3} \sum_{i=1}^{s} |C_i^*| = \sum_{i=1}^{s} |A_i^*| |\mathcal{C}_i^*| = \sum_{i=1}^{s} \frac{eg(g_i-1)}{g_i}.
\end{align}
\\
The number of $A_i^*$ in each conjugacy class $\mathcal{C}_i$ for $s < i \leq s+t$ is thus,
\begin{align*} |\mathcal{C}_i^*| = |\mathcal{C}_i| = [G:N_G(A_i)] = \frac{|G|}{2|A_i|} = \frac{eg}{2eg_i} = \frac{g}{2g_i}. \end{align*}
\\
Therefore the total number of elements of $G$ in the noncentral part of $C_i$ for $s < i \leq s+t$ is,
\begin{align}\label{classeq2of3} \sum_{i=s+1}^{s+t} |C_i^*| = \sum_{i=s+1}^{s+t} |A_i^*| |\mathcal{C}_i^*| = \sum_{i=s+1}^{s+t} \frac{eg(g_i-1)}{2g_i}.
\end{align}
We next determine the order of $C_{Q \times Z}$. Let $|Q| = q$. If $p \nmid |G|$ then $q=1$ and if $p = 0$, then we consider a Sylow $p$-subgroup to simply be $I_G$. So $q$ is always at least 1. Since $Z < K$, we can let $|K| = ek$. Observe that if $K \in \mathfrak{M}$, then by Theorem \ref{6.8}(v), $K = A_i$ for some $0 < i \leq t$ and $k = g_i$. Recall that $N_G(Q) = QK$ and so,
\begin{align*} |N_G(Q \times Z)^*| &= |N_G(Q \times Z)|  \tag{by Lemma \ref{normalizer_noncentral_eq}}
\\ &= |N_G(Q)| \tag{by Lemma \ref{unsure}}
\\ &= |QK| = eqk.
\end{align*}

Again we count the size and number of these maximal abelian groups.
\begin{align*} |(Q \times Z)^*| = |QZ| - |Z| = e(q-1).
\end{align*}

Since there is only one conjugacy class of $Q \times Z$, the number of $(Q \times Z)^*$ in $\mathfrak{M}^*$ is thus,
\begin{align*} 
  |\mathcal{C}_{Q \times Z}^*| =  |\mathcal{C}_{Q \times Z}| =  [G: N_G(Q \times Z)] = \frac{|G|}{|N_G(Q \times Z)^*|} = \frac{eg}{eqk} = \frac{g}{qk}.
\end{align*}

Therefore the total number of elements of $G$ in the noncentral parts of each $Q \times Z$ is,
\begin{align} \label{classeq3of3} 
  |C_{Q \times Z}^*| = |(Q \times Z)^*| |\mathcal{C}_{Q \times Z}^*| = \frac{eg(q-1)}{qk}.
\end{align}

We now sum together (\ref{classeq1of3}), (\ref{classeq2of3}) and (\ref{classeq3of3}) to create the \textbf{Maximal Abelian Subgroup Class Equation} of $G$.

\begin{align}\label{classeq} |G \! \setminus \! Z| &= |C_{Q \times Z}^*| + \sum_{i=1}^{s+t} |C_i^*|, \nonumber \\
|G \! \setminus \! Z| &= |(Q \times Z)^*| |\mathcal{C}_{Q \times Z}^*| + \sum_{i=1}^{s} |A_i^*| |\mathcal{C}_i^*| + \sum_{i=s+1}^{s+t} |A_i^*| |\mathcal{C}_i^*|, \nonumber \\
eg - e &= \frac{eg(q-1)}{qk} + \sum_{i=1}^{s} \frac{eg(g_i-1)}{g_i} + \sum_{i=s+1}^{s+t} \frac{eg(g_i-1)}{2g_i}, \nonumber \\
1 &= \frac{1}{g} + \frac{q-1}{qk} + \sum_{i=1}^{s} \frac{g_i-1}{g_i} + \sum_{i=s+1}^{s+t} \frac{g_i-1}{2g_i}.
\end{align}

Since $g,k,q \in \mathbb{Z}^+$ this implies that,
\begin{align*} \frac{1}{g} > 0 \quad \text{and} \quad \frac{q-1}{qk} \geq 0.
\end{align*} 

Also, since $g_i \geq 2$ for $1 \leq i \leq s + t$, we have,
\begin{align*} 
  \frac{g_i-1}{g_i} \geq \frac{1}{2}, \quad \sum_{i=1}^{s} \frac{g_i-1}{g_i} \geq \frac{s}{2} \quad \text{and} \quad \sum_{i=s+1}^{s+t} \frac{g_i-1}{2g_i} \geq \frac{t}{4}.
\end{align*}

Thus we can find a lower bound for (\ref{classeq}) which limits the possible number of conjugacy classes somewhat,
\begin{align*} 1 > \frac{s}{2} + \frac{t}{4}.
\end{align*}

There are only 6 possible different pairs of values which $s$ and $S$ can take: \vspace{3mm}

\begin{center}
\centering
  \begin{tabular}{||c||c|c|c|c|c|c||}
\hline
Case & I & II & III & IV & V & VI \\ [1ex]
\hline\hline
 $s$ & 1 & 1 & 0 & 0 & 0 & 0 \\ [1ex]
\hline
$S$ & 0 & 1 & 0 & 1 & 2 & 3 \\ [1ex]
 \hline
\end{tabular}
\end{center}
\vspace{2mm}

Each case will be examined individually in the next chapter.
\chapter{Dickson's Classification Theorem for finite subgroups of $\SL_2(F)$}\label{Ch7_DicksonsClassificationTheorem}

\section{Five Lemmas}

Before we detemine the structure of $G$ in each of the 6 cases, it is necessary to prove a number of lemmas which will be used.

\begin{lemma}
    \label{IsPGroup.not_le_normalizer} 
    \lean{IsPGroup.not_le_normalizer}
    Let $H$ be a proper subgroup of a $p$-group $G$. Then $H \subsetneq N_G(H)$.
\end{lemma}
% DEPENDENCIES

\begin{proof} Let $S$ denote the set of left cosets of $H$ in $G$. That is,
\begin{align*} S = \{ x H : x \in G \}, \quad \text{and} \;\;\; |S| = [G : H] = p^k. \quad \text{ (for some $k \geq 1$)}
\end{align*}

Consider the action of $H$ on $S$ by left multiplication. We calculate the stabiliser of $xH \in S$ in $H$.
\begin{align*} \text{Stab}(xH) &= \{ y \in H : yxH = xH \}
\\ &= \{ y \in H : x^{-1}yx \in H \}.
\end{align*}

If $x \in H$ then $x^{-1}yx \in H$ for all $y \in H$. Thus the Stab$(xH) = H$ and by the Orbit-Stabiliser Theorem,
\begin{align*} |\text{Orb}(xH)| = [H : \text{Stab}(xH)] = 1.
\end{align*}

Observe that,
\begin{align*} S = \bigcup\limits_{xH \in S} \text{Orb}(xH),
\end{align*}

where the orbits are pairwise disjoint. Now since $p$ divides $|S|$, $p$ divides the sum of all the orbit sizes. Furthermore, since each orbit size is 1 or a multiple of $p$, there must be at least $p$ elements of $S$ which have an orbit of 1. In particular, there exists an $x_1 H \in S$ which has an orbit of 1 and $x_1 \not \in H$. That is,
\begin{align*} y x_1 H &= x_ 1 H, \tag{$\forall y \in H$}
\\ x_1^{-1} y x_1 &\in H,
\\ x_1^{-1} H x_1 &\subset H,
\\ x_1 &\in N_G(H) \! \setminus \! H. \qedhere
\end{align*} 

\end{proof}

\begin{lemma}
    \label{Sylow.not_normal_subgroup_of_G}
    \lean{Sylow.not_normal_subgroup_of_G}
Let $Q$ be a Sylow $p$-subgroup and $K$ a maximal abelian subgroup of $G$ such that $N_G(Q) = QK$ and $Q \cap K = \{ I_G \}$. If $[N_G(K) : K] = 2$, then $Q$ is not a normal subgroup of $G$.
\end{lemma}
% DEPENDENCIES

\begin{proof} The approach here is proof by contradiction, so we begin by assuming that $Q \vartriangleleft G$. Thus $N_G(Q) = G$ and $N_G(K) \subset N_G(Q)$. Consider the natural homomorphism of $N_G(Q)$ onto $N_G(Q)/Q$,
\begin{align*} \phi : N_G(Q) &\longrightarrow N_G(Q)/Q, \\
\phi(x) &= xQ, \\
ker(\phi) &= \{ x \in N_G(Q) : \phi(x) = I_G Q \} = Q.
\end{align*}

Let $\phi '$ be the restiction of $\phi$ to $N_G(K)$: 

\begin{equation*} \phi ' = \left.\phi\right|_{N_G(K)} : N_G(K) \longrightarrow N_G(Q)/Q.
\end{equation*}

Thus $ker(\phi ') = ker(\phi) \cap N_G(K) = Q \cap N_G(K)$. By the 1st Isomorphism Theorem,
\begin{align*} \text{Im}(\phi ') &\cong N_G(K) / ker(\phi '), \\
N_G(Q)/Q &\cong N_G(K) / (Q \cap N_G(K)), \\
K &\cong N_G(K) / (Q \cap N_G(K)) \tag{$N_G(Q) = QK$}, \\
|Q \cap N_G(K)| &= [N_G(K) : K] = 2. \tag{by assumption}
\end{align*}

So $2$ divides $|Q|$, which implies that $2 \nmid |K|$ since $Q \cap K = \{ I_G \}$. Moreover, $|Q \cap N_G(K)|$ and $|K|$ are relatively prime. \\
\\
Take $a \in ker(\phi') = Q \cap N_G(K)$ and $b \in N_G(K)$.
\begin{align*} \phi'(bab^{-1}) &= \phi'(b)\phi'(a)\phi'(b^{-1}) \\
&= \phi'(b)(I_G Q) \phi'(b^{-1}) \\
&=  \phi'(b)\phi'(b^{-1})(I_G Q) =  I_G Q. \end{align*}

Thus $bab^{-1} \in ker(\phi') = Q \cap N_G(K)$ and so $Q \cap N_G(K) \vartriangleleft N_G(K)$. \\
\\
Now let $x \in Q \cap N_G(K)$ and $y \in K$. Notice that both $x$ and $y$ are elements of $N_G(K)$,

\begin{align*} xyx^{-1}y^{-1} &=  (xyx^{-1})y^{-1} \in K, \tag{since $K \vartriangleleft N_G(K)$} \\
xyx^{-1}y^{-1} &= x(yx^{-1}y^{-1}) \in Q \cap N_G(K), \tag{since $Q \cap N_G(K) \vartriangleleft N_G(K)$} \\
xyx^{-1}y^{-1} &\in K \cap ( Q \cap N_G(K)) \\
&= I_G, \tag{since gcd$(|Q \cap N_G(K)|,|K|) = 1$} \\
xy &= yx. \\
\end{align*}

Therefore $(Q$ $\cap$ $N_G(K)) \times K$ is an abelian subgroup of which $K$ is a proper subgroup. This contradicts the fact that $K$ is a maximal abelian subgroup, thus $Q$ is not a normal subgroup of $G$.

\end{proof}

% MAY NOT BE NEEDED
\begin{lemma}
    \label{subfield} Let $p$ be the prime characteristic of $F$ and let $q= p^k$ for some $k>0$. Set,
\begin{align}\label{RRR} R = \{ \lambda \in F : \lambda^q -\lambda = 0 \}.
\end{align}
Then $R$ is a subfield of $F$.
\end{lemma}

\begin{proof} Since $R$ is a subset of $F$ it suffices to show that the following 3 criteria are met: \\
\\
(i) $0, 1 \in R$. \\
(ii) If $\lambda_1, \lambda_2 \in R$, then $\lambda_1 - \lambda_2 \in R$. \\
(iii) If $\lambda_1, \lambda_2 \in R$ and $\lambda_1 \neq 0 \neq \lambda_2$, then $\lambda_1 \lambda^{-1}_2 \in R$. \\
\\
We see immediately that (i) is satified. Since $p$ is the characteristic of $F$, any coeffiecients which are a multiple of $p$ vanish. We get,
\begin{align*} (\lambda_1 - \lambda_2)^q = (\lambda^p_1 - \lambda^p_2)^{p^{k-1}} = ... = \lambda^q_1 - \lambda^q_2 = \lambda_1 - \lambda_2.
\end{align*}

Thus $\lambda_1 - \lambda_2 \in R$ and (ii) is also satisifed. Finally observe that if $\lambda_2$ is a non-zero element of $R$, then $\lambda^{-1}_2 = \lambda^{-q}_2$ and,
\begin{align*} (\lambda_1 \lambda^{-1}_2)^q = \lambda^q_1 \lambda^{-q}_2 = \lambda_1 \lambda^{-1}_2.
\end{align*}

So $\lambda_1 \lambda^{-1}_2 \in R$ and $R$ is a subfield of $F$.

\end{proof}

Each finite field is uniquely determined up to isomorphism by the number of elements it contains \cite[p.227]{stewart}. Since the $R$ defined in \eqref{RRR} has $q$ elements, from now on when we use the notation $\mathbb{F}_q$ to denote a field of $q$ elements, we shall actually mean,
\begin{align}
    \label{subfield} \mathbb{F}_q = R \subset F.
\end{align}

\begin{lemma}
    \label{Matrix.card_GL_field}
    \lean{Matrix.card_GL_field}
    \leanok
    Let $\mathbb{F}_q$ be the field of $q$ elements, where $q$ is the power of a prime. The order of $GL(2,\mathbb{F}_q)$ is $(q^2-1)(q^2-q)$.
\end{lemma}
\begin{proof}
    In order to prove this, we again take a geometric viewpoint. Recall that $GL(2,\mathbb{F}_q)$ is the group of 2 x 2 invertible matrices over $\mathbb{F}_q$ under ordinary matrix multiplication. The order of $GL(2,\mathbb{F}_q)$ is thus equal to the number of ordered pairs $\{u,v\}$ of linearly independent vectors in a 2-dimensional vector space over $\mathbb{F}_q$. \\
    \\
    There are clearly $q^2$ different vectors in the 2-dimensional vector space over $\mathbb{F}_q$. The only restriction on the first vector $u$, is that it must be non-zero, so there are $(q^2 - 1)$ choices for $u$. To ensure the second vector $v$ is linearly independent of $u$, it must not be of the form $\alpha u$, where $\alpha \in \mathbb{F}_q$. Since there are $q$ choices for $\alpha$, there are $(q^2-q)$ choices for $v$. \\
    \\
    Thus the order of $GL(2,\mathbb{F}_q)$ is the product of the number of choices of $u$ and the number of choices of $v$, that is, $(q^2-1)(q^2-q)$ as required.
\end{proof}

\begin{lemma}
\label{card_SL_field}
\uses{Matrix.card_GL_field}
\lean{card_SL_field}

The order of $\SL_2(\mathbb{F}_q)$ is $q(q^2-1)$
\end{lemma}

\begin{proof} 
Consider the map $\phi$ defined as,
\begin{align*} \phi : GL(2,\mathbb{F}_q) \longrightarrow \mathbb{F}^*_q, \qquad \text{where} \quad \! \! \phi(x) = \text{det}(x), \quad \forall \; x \in GL(2,\mathbb{F}_q).
\end{align*}

Next we determine the kernel of $\phi$.
\begin{align*} ker(\phi) &= \{  GL(2,\mathbb{F}_q) : \text{det}(x) = 1 \} = \SL_2(\mathbb{F}_q).
\end{align*}

We show that $\phi$ is a group homomorphism. Take $x,y \in GL(2,\mathbb{F}_q)$,
\begin{align*} 
\phi(xy) = \text{det}(xy) = \text{det}(x) \text{det}(y) = \phi(x) \phi(y).
\end{align*}

Clearly $\phi$ is surjective, since $\alpha \in \mathbb{F}^*_q$ is the determinant of $\begin{bmatrix} \alpha & 0 \\ 0 & 1 \end{bmatrix} \in GL(2,\mathbb{F}_q)$. Therefore because $\SL_2(F) \lhd \GL_2(F)$, by the First Isomorphism Theorem,
\begin{align*} GL(2,\mathbb{F}_q) / \SL_2(\mathbb{F}_q) \cong \mathbb{F}^*_q.
\end{align*}
Thus,
\begin{align*} |\SL_2(\mathbb{F}_q)| =  \frac{|GL(2,\mathbb{F}_q)|}{|\mathbb{F}^*_q|} = \frac{(q^2-1)(q^2-q)}{q-1} = q(q^2-1).
\end{align*}

\end{proof}

\begin{lemma}
    \label{QuotientGroup.comapMk'OrderIso}
    \lean{QuotientGroup.comapMk'OrderIso}
Let $N$ be a normal subgroup of a group $G$ and let $H$ be a subgroup of $G$ which contains $N$.Then,
\begin{align*} H / N \vartriangleleft G / N \iff H \vartriangleleft G
\end{align*} 
\end{lemma}

\begin{proof} If $H \vartriangleleft G$, then it follows from the Third Isomorphism Theorem that $ H / N \vartriangleleft G / N$. Conversely, assume that $H / N$ is normal in $G / N$. Let $x$ be an arbitrary element of $G$ and $h$ be an arbitrary element of $H$. Since $H / N$ is normal in $G / N$ we have,
\begin{align*} x h x^{-1}N = (xN)(hN)(x^{-1}N) = (xN)(hN)(xN)^{-1} \in H / N.
\end{align*}
Thus $x h x^{-1} \in H$. Since $x$ and $h$ were chosen arbitrarily, we have that $H \vartriangleleft G$.

\end{proof}

\section {The Six Cases}

We now address individually the 6 possible combinations of $s$ and $t$ in \eqref{classeq} and determine the structure of $G$ in each case. \\
\\
\begin{theorem}[Case I]
\label{case_I}
\uses{card_noncenter_fin_subgroup_eq_sum_card_noncenter_mul_index_normalizer, MaximalAbelianSubgroup.K_mem_MaximalAbelianSubgroups_of_center_lt_card_K}
\lean{case_I}
Claim: \textit{In this case, the Sylow $p$-subgroup $Q$ is different from $G$ and is an elementary abelian normal subgroup of $G$. The factor group $G/Q$ is a cyclic group whose order is relatively prime to $p$.} \\
\\
\end{theorem}
% DEPENDENCIES AND STATEMENT
\begin{proof} Here, $s = 1$ and $t = 0$. Equation (\ref{classeq}) simplifies to:
\begin{align}\label{case1a} 1 &= \frac{1}{g} + \frac{q-1}{qk} + \frac{g_1-1}{g_1}, \nonumber
\\ 1 &= \frac{1}{g} + \frac{1}{k} - \frac{1}{qk}  + 1 - \frac{1}{g_1}, \nonumber
\\ \frac{1}{qk}  + \frac{1}{g_1} &= \frac{1}{g} + \frac{1}{k}.
\end{align}
 \space \textbf{Case Ia:} $\pmb{q = 1}$. Here we have $Q = I_G$ and is trivially an elementary abelian normal subgroup of $G$. Equation (\ref{case1a}) gives $g=g_1$, thus $G/Q = G = A_1$, which indeed is a cyclic group whose order is relatively prime to $p$. \\
\\
 \space \textbf{Case Ib:} $\pmb{q > 1}$. If $k=1$ then (\ref{case1a}) gives,
\begin{align*} \frac{1}{q}  + \frac{1}{g_1} &= \frac{1}{g} + 1 \; > \; 1.
\end{align*}
But since both $1/q$ and $1/g_i$ are at most $1/2$ each, this is a contradiction. Thus $k > 1$. This means that $|K| = ek > e = |Z|$, so $k = g_1$ by 
Theorem \ref{MaximalAbelianSubgroup.K_mem_MaximalAbelianSubgroups_of_center_lt_card_K}. Equation (\ref{case1a}) now gives $qk = g$.
\begin{align*} |G| = eg = eqk = |N_G(Q)|.
\end{align*}
Thus $G = N_G(Q)$ and so $Q \vartriangleleft G$. Therefore $Q \neq G$ and is an elementary abelian normal subgroup of $G$. Also,
\begin{align*} G/Q = N_G(Q)/Q \cong K = A_1.
\end{align*}
Thus $G/Q$ is a cyclic group whose order is relatively prime to $p$.

\end{proof}

\begin{theorem}[Case II]
\label{case_II}
\uses{card_noncenter_fin_subgroup_eq_sum_card_noncenter_mul_index_normalizer, MaximalAbelianSubgroup.of_index_normalizer_eq_two}
\lean{case_II}
Claim: \textit{The order of $G$ is relatively prime to $p$ and either $G \cong \SL_2(3)$ or $G$ is the group of order $4n$, where $n$ is odd, defined by the presentation:}
\begin{align*} \langle \, x,y \, | \, x^n = y^2, \, yxy^{-1} = x^{-1} \, \rangle. \\
\end{align*}
\end{theorem}
% DEPENDENCIES AND STATEMENT
\begin{proof} Here, $s = 1 = t$. Equation (\ref{classeq}) simplifies to:
\begin{align}\label{case2a} 1 &= \frac{1}{g} + \frac{q-1}{qk} + \frac{g_1-1}{g_1} +  \frac{g_2-1}{2g_2}, \nonumber
\\ 1 &= \frac{1}{g} + \frac{q-1}{qk} + 1 - \frac{1}{g_1} + \frac{1}{2} - \frac{1}{2g_2}, \nonumber
\\ \frac{1}{g_1}  + \frac{1}{2g_2} &= \frac{1}{2} + \frac{1}{g} + \frac{q-1}{qk}.
\end{align}

First assume that $q>1$. This means $(q-1)/qk \geq 1/2k$ and consequently we bound (\ref{case2a}) from below:
\begin{align*} \frac{1}{2g_2} &= \frac{1}{2} - \frac{1}{g_1} + \frac{1}{g} + \frac{q-1}{qk} \; > \; \frac{1}{2k}.
\end{align*}

Thus $k > g_2 \geq 2$. So $K \in \mathfrak{M}$ and $k=g_i$ for some $i$. Since it is strictly greater than $g_2$, we have $k=g_1$. Equation (\ref{case2a}) now becomes
\begin{align*} \frac{1}{g_1}  + \frac{1}{2g_2} \; &= \; \frac{1}{2} + \frac{1}{g} + \frac{q-1}{qg_1},
\\ \frac{1}{g_1}  + \frac{1}{2g_2} \; &> \; \frac{1}{2} + \frac{1}{2g_1},
\\ \frac{1}{4} + \frac{1}{4} \; \geq \; \frac{1}{2g_1}  + \frac{1}{2g_2} \; &> \; \frac{1}{2}.
\end{align*}

This contradiction disproves the assumption that $q > 1$, so we have that $q = 1$. This means that $Q$, a Sylow $p$-subgroup of $G$, is simply the identity element and so $|G|$ is relatively prime to $p$. Also, Equation (\ref{case2a}) now reduces to:
\begin{align}\label{case2b} \frac{1}{g_1}  + \frac{1}{2g_2} &= \frac{1}{2} + \frac{1}{g}.
\end{align}

If $g_1 \geq 4$ we get
\begin{align*} \frac{1}{2g_2} &= \frac{1}{2} + \frac{1}{g} - \frac{1}{g_1} \; > \; \frac{1}{4}.
\end{align*}

Since $g_2 > 1$  this gives a contradiction and thus $g_1 < 4$. We now have two seperate cases to consider.\\
\\
 \space \textbf{Case IIa:} $\pmb{g_1 = 2}$. Equation (\ref{case2b}) becomes
\begin{align*} \frac{1}{2g_2} &= \frac{1}{g}, \; \; \Longrightarrow \; \; g = 2g_2.
\end{align*}

If $e=1$, then $p=2$. Also since $q=1$, 2 does not divide $|G|$, but $|G| = eg = e2g_2$ which is a contradiction. So $e=2$ and $p \neq 2$. We now have:
\begin{align*} |N_G(A_2)| &= 2|A_2|  = 2eg_2 = eg = |G|,  \tag{since $s+t = 2$}
\\ |N_G(A_1)| &= |A_1| = eg_1 = 4. \tag{since $s=1$} 
\end{align*}
Thus $G = N_G(A_2)$, that is $A_2 \vartriangleleft G$.\\
\\
By Corollary \ref{5thsylow}, $A_1$ is contained in a Sylow 2-subgroup of $G$, call it $S$. If $S$ is strictly larger than $A_1$, then by Lemma \ref{case2q}, $A_1 \subsetneq N_S(A_1) \subset N_G(A_1)$. Since $A_1 = N_G(A_1)$ we conclude that $A_1$ is a Sylow 2-subgroup of $G$. This means that 8 does not divide $|G| = 4g_2$ and so $g_2 = n$, where $n$ is odd. \\
\\
Since $A_2$ is cyclic it is generated by a single element, so let $A_2 = \langle x \rangle$ and thus $x^{2n}= I_G$.  Recall that because $[N_G(A_2): A_2] = 2$, Theorem \ref{MaximalAbelianSubgroup.of_index_normalizer_eq_two} tells us that there exists a $y \in N_G(A_2) \! \setminus \! A_2$ such that $yxy^{-1} = x^{-1}$. \\
\\
Recall from Chapter 2 that the number of $A_i$ in each conjugacy class $\mathcal{C}_i$ is equal to $[G : N_G(A_i)]$ so,
\begin{align*}  |\mathcal{C}_2| = [G:N_G(A_2)] &= 1.
\end{align*}

Due to the fact that $y$ belongs to some maximal abelian subgroup of $G$, and since $y \not \in A_2$ and $|\mathcal{C}_2| = 1$, it must be that $y$ belongs to $A_1$ or one of its conjugate subgroups. Thus $y$ has an order which divides $|A_1| = 4$ and since the only elements of order 1 and 2 lie in $Z$, the order of $y$ is 4. Furthermore, both $x^n$ and $y^2$ have order 2. Recalling that $G$ has at most 1 element of order 2, this gives the relation $x^n = y^2$. \\ 
\\
Let $H$ be the group generated by $x$ and $y$ and the above relations:
\begin{align*} H = \langle \, x,y \, | \, x^n = y^2, \, yxy^{-1} = x^{-1} \rangle.
\end{align*}

Notice that the second relation gives that $y x^n y^{-1} = x^{-n}$, so
\begin{align*} x^{-n} = y x^n y^{-1} = y y^2 y^{-1} = y^2 = x^n.
\end{align*}

This shows that $y^4 = x^{2n} = I_G$ and that $H$ is finite. Moreoever,
\begin{align*} H = \{ x^k, x^ky :  0 < k \leq 2n \}.
\end{align*}

 Thus $|H| = 4n = |G|$ and $H = G$. \\
\\
 \space \textbf{Case IIb:} $\pmb{g_1 = 3}$.  Equation (\ref{case2b}) becomes
\begin{align*} \frac{1}{2g_2} &= \frac{1}{6} + \frac{1}{g} \; > \; \frac{1}{6}.
\end{align*}
Therefore $g_2 = 2$ and $g = 12$. Again, since $q=1$ and 2 divides $|G|$, we have $p \neq 2$ and so $e = 2$. Thus we have,
\begin{align*} |G| = eg = 24, \qquad |A_1| = eg_1 = 6, \qquad |A_2| = eg_2 = 4.
\end{align*}
Again we determine the number of maximal abelian subgroups in each conjugacy class.
\begin{align*}  |\mathcal{C}_1| = [G:N_G(A_1)] &= \frac{|G|}{|A_1|} = \frac{24}{6} = 4, 
\\[1.5ex] |\mathcal{C}_2| = [G:N_G(A_2)] &= \frac{|G|}{2|A_2|} = \frac{24}{8} = 3.
\end{align*}

% \newpage
The figure below shows $G$ divided into it's maximal abelian subgroups:


% \begin{center}
% \begin{tikzpicture}[thick, scale=0.4]

% \draw[dashed][rotate around={0:(0,0)},red] (3,0) ellipse (108pt and 41pt);  
% \draw[dashed][rotate around={20:(0,0)},red] (3,0) ellipse (108pt and 41pt);  
% \draw[rotate around={40:(0,0)},red] (3,0) ellipse (108pt and 41pt); 
% \draw[dashed][rotate around={60:(0,0)},red] (3,0) ellipse (108pt and 41pt);  

% \draw[dashed][rotate around={180:(0,0)},blue] (2,0) ellipse (79pt and 37pt);  
% \draw[rotate around={210:(0,0)},blue] (2,0) ellipse (79pt and 37pt);
% \draw[dashed][rotate around={240:(0,0)},blue] (2,0) ellipse (79pt and 37pt);

% \draw (0,0) ellipse (22pt and 22pt); 

% \node[] at (0,-8) {\resizebox{9cm}{!}{Fig 2: The elements of $G$ arranged into maximal abelian subgroups.}};
% \node[] at (0,0) {\resizebox{.3cm}{!}{$Z$}};
% \node[] at (5.7,4.9) {\resizebox{.5cm}{!}{$A_1$}};
% \node[] at (-4.6,-2.8) {\resizebox{.5cm}{!}{$A_2$}};
% \node[] at (8.6,5) {\resizebox{.5cm}{!}{$\mathcal{C}_1$}};
% \node[] at (-6.8,-3.6) {\resizebox{.5cm}{!}{$\mathcal{C}_2$}};

% \node[scale=1.8, rotate=30,gray] at (-5.4,-3.2) { $\Bigg\{$ };
% \node[scale=2, rotate=210,gray] at (7.3,4) { $\Bigg\{$ };

% \node[scale=2, black] at (-.45,0) {.};
% \node[scale=2, black] at (.45,0) {.};

% \node[scale=3, red] at (4,4) {.};
% \node[scale=3, red] at (4.7,4.2) {.};
% \node[scale=3, red] at (4.8,3.3) {.};
% \node[scale=3, red] at (3.9, 3.2) {.};
% \node[scale=2, red] at (4.8, 1.7) {.};
% \node[scale=2, red] at (5.2, 2.3) {.};
% \node[scale=2, red] at (5.9, 2.2) {.};
% \node[scale=2, red] at (5.6, 1.5) {.};
% \node[scale=2, red] at (6, 0.2) {.};
% \node[scale=2, red] at (5.5, -0.5) {.};
% \node[scale=2, red] at (4.6, -0.8) {.};
% \node[scale=2, red] at (3.7, -1) {.};
% \node[scale=2, red] at (3, 5.2) {.};
% \node[scale=2, red] at (2.2,4.5) {.};
% \node[scale=2, red] at (1.5, 4.0) {.};
% \node[scale=2, red] at (0.9, 3.3) {.};

% \node[scale=3, blue] at (-3.5,-1.6) {.};
% \node[scale=3, blue] at (-3.2,-2.4) {.};
% \node[scale=2, blue] at (-3.6,0.4) {.};
% \node[scale=2, blue] at (-2.4,0.6) {.};
% \node[scale=2, blue] at (-2,-3.3) {.};
% \node[scale=2, blue] at (-1.0,-2.9) {.};

% \end{tikzpicture}
% \end{center}

Let $A_2 = \langle x \rangle$. By Theorem \ref{MaximalAbelianSubgroup.of_index_normalizer_eq_two}, there is an element $y \in N_G(A_2) \! \setminus \! A_2$ such that $y x y^{-1} = x^{-1}$. Since $N_G(A_2)$ has order 8, the order of $y$ must divide 8. The order of $y$ cannot be 8 since $N_G(A_2)$ is not cyclic and the only elements with order 1 or 2  are found in $Z$, thus $y$ has order 4. By the uniqueness of the element of order 2, we have $x^2 = y^2$. So
\begin{align*} N_G(A_2) = \langle x, y \; | \; x^2 = y^2, y x y^{-1} = x^{-1} \rangle.
\end{align*}
For simplicity let $N = N_G(A_2)$ . Since $|A_1| = 6$, the only elements in $C_1$ with order $2^k$ are those in $Z$, so every element of $G$ with order $2^k$ must belong to $C_2$. Since $C_2$ has order 8 it is equal to $N$ because each element of $N$ has order $2^k$. Furthermore, $N$ is thus a unique Sylow $2$-subgroup of $G$ and by Corollary \ref{4thsylow}, we have $N \vartriangleleft G$. \\
\\
Now consider the quotient group $G / N$, that is the set of left (or right) cosets of $N$ in $G$.
\begin {align*} G / N = \{ N, rN, r^2N \} \cong \langle r \rangle \cong \mathbb{Z}_3,
\end{align*}
where $r$ is some element of $G\! \setminus \! N$ with order 3. Without loss of generality we may regard $r$ to be a generator of $H$, where $H$ is the cyclic subgroup of $A_1$ of order 3. \\
\\
Let $H$ act on $N$ by conjugation. Since $|H| = 3$ the orbit of $x \in N$ has size 1 or 3.
\begin{align*} \text{Orb}(x) =  \{ r^k x r^{-k} : r^k \in H \}.
\end{align*}

Since $H$ is not contained in the centraliser of $x$ we conclude that the orbit of $x$ has size 3. Let $A_2, A'_2$ and $A''_2$ be the 3 elements of $\mathcal{C}_2$. Without loss of generality we may assume $y \in A'_2$ and consequently $xy \in A''_2$. Using the two relations between $x$ and $y$ we observe that,
\begin{align*} (xy)^{-1} = y^{-1} x^{-1} = y^{-1} (y x y^{-1}) = x y^{-1} = x^{-1} x^2 y^{-1} = x^{-1} y = yx
\end{align*}

% \begin{center}
% \begin{tikzpicture}[thick, scale=0.8]

% \draw[rotate around={60:(0,0)},blue] (2,0) ellipse (79pt and 37pt);  
% \draw[rotate around={90:(0,0)},blue] (2,0) ellipse (79pt and 37pt);
% \draw[rotate around={120:(0,0)},blue] (2,0) ellipse (79pt and 37pt);

% \draw (0,0) ellipse (22pt and 22pt); 

% \node[] at (0,-2) {\resizebox{9cm}{!}{Fig 3: The elements of $N$ arranged into maximal abelian subgroups.}};
% \node[] at (0,0) {\resizebox{.3cm}{!}{$Z$}};
% \node[] at (-2.5,4.7) {\resizebox{.5cm}{!}{$A_2$}};
% \node[] at (0.0,5.4) {\resizebox{.5cm}{!}{$A'_2$}};
% \node[] at (2.3,4.8) {\resizebox{.5cm}{!}{$A''_2$}};

% \node[scale=3, black] at (-.45,0) {.};
% \node[scale=3, black] at (.45,0) {.};

% \node[scale=3, blue] at (-1.7, 3.3) {.};
% \node[] at (-1.7,3.6) {\resizebox{.22cm}{!}{$x$}};
% \node[scale=3, blue] at (-2.2, 2.5) {.};
% \node[] at (-2.2,2.9) {\resizebox{.6cm}{!}{$x^{-1}$}};
% \node[scale=3, blue] at (-0.5,3.8) {.};
% \node[] at (0.5,4.2) {\resizebox{.21cm}{!}{$y$}};
% \node[scale=3, blue] at (0.5,3.8) {.};
% \node[] at (-0.25,4.3) {\resizebox{.6cm}{!}{$y^{-1}$}};
% \node[scale=3, blue] at (1.7,3.3) {.};
% \node[] at (1.7,3.6) {\resizebox{.4cm}{!}{$xy$}};
% \node[scale=3, blue] at (2.2,2.5) {.};
% \node[] at (2.2,2.8) {\resizebox{.4cm}{!}{$yx$}};

% \end{tikzpicture}
% \end{center}

The elements of $Z$ are fixed points under this group action and the remaining 6 elements of $N$ form 2 orbit cycles of order 3, with each cycle containing exactly one element from the noncentral parts of $A_2, A'_2$ and $A''_2$ in some order. If $y$ inverts $x$, then $y$ inverts all powers of $x$ including $x^{-1}$. Also, if $y$ inverts $x$, then $y^{-1}$ inverts $x^{-1}$ and thus inverts $x$ also. So the 2 relations we have established between $x$ and $y$ actually hold for any pair of elements of $N \! \setminus \! Z$ which belong to different elements of $\mathfrak{M}$. Therefore without loss of generality, we may assume that $x$ and $y$ are in the same orbit cycle and that $r x r^{-1} = y$. Fig 3 shows that there are only 2 elements which could complete this cycle, $xy$ and $yx$. If $r y r^{-1} = xy$, then we have the following 3 relations on $G$.
\begin{align}\label{3rel} r x r^{-1} = y, \qquad r y r^{-1} = xy, \qquad r xy x^{-1} = x.
\end{align}

Otherwise $r y r^{-1} = yx$. In this case, consider the orbit of $x$ under conjugation by $r^2$ instead. This gives the same orbit cycle but in the opposite direction:
\begin{align*} r^2 x r^{-2} = yx, \qquad r^2 yx r^{-2} = y, \qquad r^2 y r^{-2} = x.
\end{align*}
Observe that $x(yx) = x (x^{-1} y) = y$. Thus without loss of generality we can rename $r^2$ as $r$, $yx$ as $y$ and $y$ as $xy$. Notice that this now gives the same relations as in \eqref{3rel}. Since $x$ and $y$ generate a group of order 8 and $r$ has order 3, the group given by the following presentation has order at most 24 and is thus a presentation of $G$. 
\begin{align*} \langle x, y, r \, |  \, x^2= y^2, \, y x y^{-1} = x^{-1}, \, r^3 = I, \, r x r^{-1} = y, \, r y r^{-1} = xy, \, r xy r^{-1} = x \rangle,
\end{align*}

By Lemma \ref{ordersl2q}, we observe that the order of $\SL_2(3)$ is $3(3^2-1) = 24$. Now consider the following the elements of $\SL_2(3)$:
\begin{align*} a = \begin{bmatrix} 1 & 1 \\ 1 & 2 \end{bmatrix}, \qquad b = \begin{bmatrix} 0 & 2 \\ 1 & 0 \end{bmatrix}, \qquad c = \begin{bmatrix} 2 & 1 \\ 2 & 0 \end{bmatrix}.
\end{align*}

One can verify easily that each of the following relations hold:
\begin{align*} a^2 &= b^2, \qquad b a b^{-1} = a^{-1}, \qquad \quad \; c^3 = I, 
\\ c a c^{-1} &= b,  \qquad \; \: c b c^{-1} = ab, \qquad \! c ab c^{-1} = a.
\end{align*}

Since $G$ and $\SL_2(3)$ have the same order and since their respective generators satisfy the corresponding relations, there is an isomorphism mapping $x \mapsto a$, $y \mapsto b$ and $r \mapsto c$. Thus,
\begin{align*} G = \langle x, y, r \rangle \cong \langle a, b, c \rangle = \SL_2(3). 
\end{align*} 
\end{proof}


\begin{theorem}[Case III]
\label{case_III}
\uses{card_noncenter_fin_subgroup_eq_sum_card_noncenter_mul_index_normalizer, MaximalAbelianSubgroup.K_mem_MaximalAbelianSubgroups_of_center_lt_card_K}
\lean{case_III}
Claim: \textit{We have $G = Q \times Z$.}
\end{theorem}
% DEPENDENCIES AND STATEMENT
\begin{proof} Here, $s = 0 = t$. Equation (\ref{classeq}) simplifies to:
\begin{align}\label{case3a} 1 &= \frac{1}{g} + \frac{q-1}{qk}, \nonumber
\\ 1 &= \frac{1}{g} + \frac{1}{k} - \frac{1}{qk}, \nonumber
\\ 1 + \frac{1}{qk} &= \frac{1}{g} + \frac{1}{k}.
\end{align}

Since $s = 0 = t$, there are no cyclic maximal abelian subgroups whose order is relatively prime to $p$, so $K \not \in \mathfrak{M}$. Then by Theorem \ref{MaximalAbelianSubgroup.K_mem_MaximalAbelianSubgroups_of_center_lt_card_K} we have,
\begin{align*} ek = |K| \leq |Z| = e.
\end{align*} 
Thus $k = 1$ and equation (\ref{case3a}) reduces to $1/q = 1/g$, that is $g=q$.
\begin{align*} |G| =  eg &= eq = |Q \times Z|,
\\ G &= Q \times Z.
\end{align*}
\qedhere
\end{proof}


\begin{theorem}[Case IV]
\label{case_IV}
\lean{card_noncenter_fin_subgroup_eq_sum_card_noncenter_mul_index_normalizer}
\lean{case_IV}
Claim: \textit{Either $p=2$ and $G$ is isomorphic to the dihedral group of order $2n$, where $n$ is odd, or $p=3$ and $G \cong \SL_2(3)$.}
\end{theorem}
\begin{proof} Here, $s = 0$ and $t = 1$. Equation (\ref{classeq}) simplifies to:
\begin{align}\label{case4a} 1 &= \frac{1}{g} + \frac{q-1}{qk} +  \frac{g_1-1}{2g_1}, \nonumber
\\ 1 &= \frac{1}{g} + \frac{q-1}{qk} + \frac{1}{2} - \frac{1}{2g_1}, \nonumber
\\ \frac{1}{2} + \frac{1}{2g_1} &= \frac{1}{g} + \frac{q-1}{qk}.
\end{align}

Recall that $|A_1|=eg_1$ and $[N_G(A_1): A_1] = 2$ and so,
\begin{align*} eg = |G| \geq |N_G(A_1)| = 2eg_1.
\end{align*}

So $g \geq 2g_1$ and $1/2g_1 \geq 1/g$ and hence we can bound Equation (\ref{case4a}):
\begin{align*} \frac{1}{2} \; \leq \; \frac{1}{2} + \frac{1}{2g_1} - \frac{1}{g} &= \frac{q-1}{qk}.
\end{align*}

Clearly this forces $k = 1$ and also $q > 1$. We can now simplify and bound Equation (\ref{case4a}) as follows:
\begin{align*} \frac{1}{q} + \frac{1}{4} \; \geq \; \frac{1}{q} + \frac{1}{2g_1} &= \frac{1}{g} + \frac{1}{2} \; > \; \frac{1}{2}. 
\end{align*}

This gives $1/q > 1/4$ and so $q$ is equal to either 2 or 3. We examine each case individually. \\
\\
 \space \textbf{Case IVa:} $\pmb{q = 2}$. Equation (\ref{case4a}) becomes
\begin{align*} \frac{1}{2g_1} &= \frac{1}{g}, \; \; \Longrightarrow \; \; g = 2g_1,
\end{align*}

and we show that $A_1$ is a normal subgroup of $G$:
\begin{align*} |G| = eg = e2g_1 = 2|A_1| = |N_G(A_1)|. 
\end{align*}
In this case, a Sylow $p$-subgroup has order 2 so we have $p=2$ and also $e=1$. By it's definition, the order of $A_1$ is relatively prime to $p=2$, so we have that $|A_1|= g_1 = n$, where $n$ is odd, and consequently $G$ has order $2n$. \\  
\\
We now know enough about the structure of $G$ to establish some relations on it. Let $A_1 = \langle x \rangle$, so $x^n = I_G$. By Theorem \ref{MaximalAbelianSubgroup.of_index_normalizer_eq_two} there exists a $y \in N_G(A_1) \! \setminus \! A_1$ such that $y x y^{-1} = x^{-1}$.
\begin{align*} |\mathcal{C}_1| &= [G : N_G(A_1)] = 1.
\\ |\mathcal{C}_{Q \times Z}| &= [G : N_G(Q \times Z)] = \frac{|G|}{eqk} = \frac{2n}{2} = n.
\end{align*}
The only maximal abelian subgroups of $G$ are thus $A_1$ and the $n$ conjugate subgroups of $\mathcal{C}_{Q \times Z}$.

% \begin{center}
% \begin{tikzpicture}[thick, scale=0.4]

% \draw[rotate around={0:(0,0)},green] (3,0) ellipse (108pt and 41pt);  
% \draw[dashed][rotate around={20:(0,0)},green] (3,0) ellipse (108pt and 41pt);  
% \draw[dashed][rotate around={40:(0,0)},green] (3,0) ellipse (108pt and 41pt); 
% \draw[dashed][rotate around={60:(0,0)},lightgray] (3,0) ellipse (108pt and 41pt);  
% \draw[dashed][rotate around={80:(0,0)},lightgray] (3,0) ellipse (108pt and 41pt);  
% \draw[dashed][rotate around={100:(0,0)},lightgray] (3,0) ellipse (108pt and 41pt);  
% \draw[dashed][rotate around={120:(0,0)},green] (3,0) ellipse (108pt and 41pt);  

% \draw[rotate around={210:(0,0)},blue] (3,0) ellipse (108pt and 41pt);

% \draw (0,0) ellipse (22pt and 22pt); 

% \node[] at (0,-6) {\resizebox{9cm}{!}{Fig 4: The elements of $G$ arranged into maximal abelian subgroups.}};
% \node[] at (0,0) {\resizebox{.3cm}{!}{$Z$}};
% \node[] at (8.4,0.2) {\resizebox{1cm}{!}{$Q \times Z$}};
% \node[] at (-6.8,-3.1) {\resizebox{.5cm}{!}{$A_1$}};
% \node[] at (4.7,8.3) {\resizebox{1.1cm}{!}{$\mathcal{C}_{Q \times Z}$}};

% \node[scale=2.5, rotate=240,gray] at (4.0,6.7) { $\Bigg\{$ };

% \node[scale=2, black] at (-.45,0) {.};

% \node[scale=3, green] at (5.5, -0.1) {.};
% \node[scale=2, green] at (5.5, 2.0) {.};
% \node[scale=2, green] at (4.5,3.7) {.};
% \node[scale=1.3, gray] at (2.8,5.0) {.};
% \node[scale=1.3, gray] at (1.1,5.7) {.};
% \node[scale=1.3, gray] at (-0.9,5.7) {.};
% \node[scale=2, green] at (-2.8, 4.8) {.};

% \node[scale=1.6, blue] at (-5.1,-3.0) {.};
% \node[scale=3, blue] at (-3.4,-2.4) {.};
% \node[scale=1.6, blue] at (-3.6,-1.4) {.};
% \node[scale=1.6, blue] at (-2.4,-0.7) {.};
% \node[scale=1.6, blue] at (-2,-1.9) {.};
% \node[scale=1.6, blue] at (-0.8,-1.2) {.};

% \end{tikzpicture}
% \end{center}

Since $y$ belongs to some maximal abelian subgroup and $y \not \in A_1$, $y$ must belong to some element of $\mathcal{C}_{Q \times Z}$. Since $|Q \times Z|$ = 2, the order of $y$ is 2 and $y^2 = I_G$. We have established the following presentation of G.
\begin{align*} G = \langle x, y \; | \; x^n = I_G = y^2, \; y x y^{-1} = x^{-1} \rangle.
\end{align*}

Let $D_n$ denote the dihedral group of order $2n$, that is the group of symmetries of a regular polygon wih $n$ vertices. Let $r$ denote a clockwise rotation by $2\theta /n$ radians and $s$ denote a reflection. For $n$ odd, it can easily be verified that $D_n$ has the following presentation.
\begin{align*} D_n = \langle r, s \; | \; r^n = I = s^2, \; s r s^{-1} = r^{-1} \rangle.
\end{align*}

Since $G$ and $D_n$ have the same order and since their respective generators satisfy the corresponding relations, there is an isomorphism mapping $x \mapsto r$ and $y \mapsto s$. Thus,
\begin{align*} G = \langle x, y \rangle \cong \langle r, s \rangle = D_n.
\end{align*}

 \space \textbf{Case IVb:} $\pmb{q = 3}$. Now equation (\ref{case4a}) becomes
\begin{align*} \frac{1}{2g_1} &= \frac{1}{g} + \frac{1}{6} \; > \; \frac{1}{6}.
\end{align*}
This means that $g_1 = 2$ and $g = 12$. Since $q=3$ we have $p=3$ and $e=2$. Furthermore we have,
\begin{align*} |G| = 24, \quad |A_1| &= 4,  \quad |N_G(A_1)| = 8, \quad |Q \times Z| = 6 \quad |N_G(Q \times Z)| = 6
\end{align*}
\begin{align*} |\mathcal{C}_1| &= [G : N_G(A_1)] = \frac{24}{8} = 3
\\ |\mathcal{C}_{Q \times Z}| &= [G : N_G(Q \times Z)] = \frac{24}{6} = 4
\end{align*}
% \begin{center}
% \begin{tikzpicture}[thick, scale=0.4]

% \draw[dashed][rotate around={0:(0,0)},green] (3,0) ellipse (108pt and 41pt);  
% \draw[dashed][rotate around={20:(0,0)},green] (3,0) ellipse (108pt and 41pt);  
% \draw[rotate around={40:(0,0)},green] (3,0) ellipse (108pt and 41pt); 
% \draw[dashed][rotate around={60:(0,0)},green] (3,0) ellipse (108pt and 41pt);  

% \draw[dashed][rotate around={180:(0,0)},blue] (2,0) ellipse (79pt and 37pt);  
% \draw[rotate around={210:(0,0)},blue] (2,0) ellipse (79pt and 37pt);
% \draw[dashed][rotate around={240:(0,0)},blue] (2,0) ellipse (79pt and 37pt);

% \draw (0,0) ellipse (22pt and 22pt); 

% \node[] at (0,-7) {\resizebox{9cm}{!}{Fig 5: The elements of $G$ arranged into maximal abelian subgroups.}};
% \node[] at (0,0) {\resizebox{.3cm}{!}{$Z$}};
% \node[] at (5.4,5.2) {\resizebox{1cm}{!}{$Q \times Z$}};
% \node[] at (-4.6,-2.8) {\resizebox{.5cm}{!}{$A_1$}};
% \node[] at (9.3,5.3) {\resizebox{1.1cm}{!}{$\mathcal{C}_{Q \times Z}$}};
% \node[] at (-6.8,-3.6) {\resizebox{.5cm}{!}{$\mathcal{C}_1$}};

% \node[scale=1.8, rotate=30,gray] at (-5.4,-3.2) { $\Bigg\{$ };
% \node[scale=2, rotate=210,gray] at (7.7,4.4) { $\Bigg\{$ };

% \node[scale=2, black] at (-.45,0) {.};
% \node[scale=2, black] at (.45,0) {.};

% \node[scale=3, green] at (4,4) {.};
% \node[scale=3, green] at (4.7,4.2) {.};
% \node[scale=3, green] at (4.8,3.3) {.};
% \node[scale=3, green] at (3.9, 3.2) {.};
% \node[scale=2, green] at (4.8, 1.7) {.};
% \node[scale=2, green] at (5.2, 2.3) {.};
% \node[scale=2, green] at (5.9, 2.2) {.};
% \node[scale=2, green] at (5.6, 1.5) {.};
% \node[scale=2, green] at (6, 0.2) {.};
% \node[scale=2, green] at (5.5, -0.5) {.};
% \node[scale=2, green] at (4.6, -0.8) {.};
% \node[scale=2, green] at (3.7, -1) {.};
% \node[scale=2, green] at (3, 5.2) {.};
% \node[scale=2, green] at (2.2,4.5) {.};
% \node[scale=2, green] at (1.5, 4.0) {.};
% \node[scale=2, green] at (0.9, 3.3) {.};

% \node[scale=3, blue] at (-3.5,-1.6) {.};
% \node[scale=3, blue] at (-3.2,-2.4) {.};
% \node[scale=2, blue] at (-3.6,0.4) {.};
% \node[scale=2, blue] at (-2.4,0.6) {.};
% \node[scale=2, blue] at (-2,-3.3) {.};
% \node[scale=2, blue] at (-1.0,-2.9) {.};

% \end{tikzpicture}
% \end{center}

Notice that Fig 5 is almost identical to Fig 2 in the study of Case IIb. This is a strong indication that these 2 cases are isomorphic to each other and hence also to $\SL_2(3)$, albeit not a proof. However, an argument analogous to the one outlined in the proof of Case IIb can be directly applied here with a simple renaming of the conjugacy classes and representatives. It would be to repeat this argument again and I will leave it to the reader to verify.

\end{proof}
\begin{theorem}[Case V]
\label{case_V}
\uses{card_noncenter_fin_subgroup_eq_sum_card_noncenter_mul_index_normalizer, MaximalAbelianSubgroup.K_mem_MaximalAbelianSubgroups_of_center_lt_card_K, MaximalAbelianSubgroup.IsCyclic_and_card_coprime_CharP_or_eq_Q_join_Z}
\lean{case_V}
Claim: \textit{We have one of the following three cases: \\
\\
(i) $G \cong \SL_2(\mathbb{F}_q)$. \\
\\
(ii) $G \cong \langle \SL_2(\mathbb{F}_q), d_\pi \rangle$, where $\pi \in \mathbb{F}_{q^2} \setminus \mathbb{F}_q$, $\pi^2 \in \mathbb{F}_q$ and $\SL_2(\mathbb{F}_q) \vartriangleleft G$. \\
\\
(iii) $G \cong \SL_2(5)$ and $p=3=q$.}
\end{theorem}

\begin{proof} Here, $s = 0$ and $t = 2$. Equation (\ref{classeq}) simplifies to:
\begin{align} \label{case5a} 1 &= \frac{1}{g} + \frac{q-1}{qk} + \frac{g_1 -1}{2g_1} + \frac{g_2 -1}{2g_2}, \nonumber
\\ 
\frac{1}{2g_1} + \frac{1}{2g_2} &= \frac{1}{g} + \frac{q-1}{qk}. \end{align}

Recall that,
\begin{align*} eg = |G| \geq  |N_G(A_i)| \geq 2eg_i, \qquad \text{thus} \quad \! \frac{1}{g} \leq \frac{1}{2g_i}.
\end{align*}
Equation (\ref{case5a}) is therefore bounded from below:
\begin{align*}  \frac{2}{g} \leq \frac{1}{2g_1} + \frac{1}{2g_2} = \frac{1}{g} + \frac{q-1}{qk}. 
\end{align*}
Therefore $q>1$, since if $q=1$ we arrive at the contradiction $2/g \leq 1/g$. With this in mind we have $(q-1)/q \geq 1/2$ and since $g_i \geq 2$ this allows us to bound (\ref{case5a}) on either side.

\begin{align*} \frac{1}{2} &\geq \frac{1}{2g_1} + \frac{1}{2g_2} = \frac{1}{g} + \frac{q-1}{qk} > \frac{q-1}{qk} \geq \frac{1}{2k}.
\end{align*}

This gives $k > 1$ and so by Theorem \ref{MaximalAbelianSubgroup.K_mem_MaximalAbelianSubgroups_of_center_lt_card_K}, $k$ must equal $g_1$ or $g_2$ since the inequality $ek = |K| > |Z| = e$ holds. Without loss of generality we let $k=g_1$ and (\ref{case5a}) becomes,

\begin{align} \label{case5b} \frac{1}{2g_1} + \frac{1}{2g_2} &= \frac{1}{g} + \frac{q-1}{qg_1} = \frac{1}{g} + \frac{1}{g_1} - \frac{1}{qg_1}, \nonumber \\[1.5ex]
 \frac{1}{2g_2} &= \frac{1}{g} + \frac{1}{2g_1} - \frac{1}{qg_1}.
\end{align}
\\
Let $N_G(Q)$ act on $Q \! \setminus \! I_G$ by conjugation and consider the stabiliser in $N_G(Q)$ of an arbitrarily chosen $x \in Q \! \setminus \! I_G$.
\begin{align*} \text{Stab}(x) &= \{ g \in N_G(Q) : g x g^{-1} = x \}
\\ &= C_G(x) \cap N_G(Q)
\\ &= (Q \times Z) \cap N_G(Q) \tag{by Theorem \ref{MaximalAbelianSubgroup.IsCyclic_and_card_coprime_CharP_or_eq_Q_join_Z}}
\\ &= Q \times Z. \tag{since $Q \times Z \subset N_G(Q)$}
\end{align*}

Thus by the Orbit-Stabiliser Theorem,
\begin{align*} |\text{Orb}(x)| = [N_G(Q) : Q \times Z] = \frac{eqk}{eq} = k
\end{align*}

Since $x$ was chosen arbitrarily from $Q \! \setminus \! I_G$, each element of $Q \! \setminus \! I_G$ has an orbit in $N_G(Q)$ of size $k$. Considering also the fact that $Q \! \setminus \! I_G$ is equal to the union of the pairwise disjoint orbits of its elements, we conclude that $k = g_1$ divides $|Q \! \setminus \! I_G|$. Thus there exists some $d \in \mathbb{Z^+}$ such that,
\begin{align}\label{6.14} q-1 = d g_1.
\end{align}

Now set,
\begin{align} \label{6.14a} i = \frac{2 g_1 g_2 q}{g} > 0,
\end{align}
and multiply \eqref{case5b} by $ig$ to give,
\begin{align}\label{6.15} g_1 q &= i + (q-2) g_2.
\end{align}
Thus $i$ is an integer and since it is greater than zero by definition, \eqref{6.15} gives,
\begin{align}\label{6.16b} g_1 > \frac{(q-2) g_2}{q}.
\end{align}
Also, using \eqref{6.14} and \eqref{6.15} we get,
\begin{align}\label{6.16a} g_1 q &= i + (q-1) g_2 - g_2 \nonumber
\\ &= i + d g_1 g_2 - g_2, \nonumber
\\ g_2 &= i + (d g_2 - q) g_1.
\end{align}

Applying Lemma \ref{caseVlemma} we observe that $Q$ is not normal in $G$, and so 
\begin{align*} eg = |G| &> |N_G(Q)| = eqk = eqg_1, \\[1.5ex]
\frac{1}{qg_1} &> \frac{1}{g}.
\end{align*}
And (\ref{case5b}) gives us,
\begin{align}\label{6.13}  \frac{1}{2g_2} &= \frac{1}{g} - \frac{1}{qg_1} + \frac{1}{2g_1} < \frac{1}{2g_1}, \nonumber
\\[1.5ex] g_1 &< g_2.
\end{align}

Consider now,
\begin{align*} [G : N_G(Q)] = \frac{eg}{e q k} = \frac{g}{q g_1} = \frac{2 g_2}{i} \in \mathbb{Z}. \tag{by \eqref{6.14a}}
\end{align*}
Thus $i$ divides $2 g_2$. Recall that the order of $A_2$ is relatively prime to $p$ by Theorem \ref{MaximalAbelianSubgroup.IsCyclic_and_card_coprime_CharP_or_eq_Q_join_Z}, so $g_2$ is also relatively prime to $p$. Therefore if $p \neq 2$, $i$ is relatively prime to $p$ and if $p=2$ then $p$ divides $i$ but $p^2$ does not. Now since $Q$ is a Sylow $p$-subgroup of $G$, this means that greatest common denominator of $i$ and $q$ is either 1 or 2.
Now consider,
\begin{align*} [G : N_G(A_2)] = \frac{eg}{2 e g_2} = \frac{g_1 q}{i} \in \mathbb{Z}. \tag{by \eqref{6.14a}}
\end{align*}
Thus $i$ divides $g_1 q$ and since gcd$(i, q) = 1$ or 2, i must divide $2 g_1$. So there exists some $m \in \mathbb{Z^+}$ such that,
\begin{align}\label{6.17} i = \frac{2 g_1}{m}.
\end{align}

We consider now the separate cases which arise for different values of $q$. \\
\\
 \space \textbf{Cases Va and Vb:} $\pmb{q \geq 4}$. This condition gives us a lower bound for the inequality in \eqref{6.16b},
\begin{align*} g_1 > \frac{(q-2) g_2}{q} > \frac{g_2}{2}.
\end{align*}
Combining this with \eqref{6.13} we have,
\begin{align}\label{6.18} g_1 < g_2 < 2 g_1.
\end{align}

Substituting \eqref{6.17} into \eqref{6.16a} gives,
\begin{align*} g_2 = \left( \frac{2}{m} + d g_2 - q \right) g_1
\end{align*}
Thus \eqref{6.18} gives that,
\begin{align*} 1 < \frac{2}{m} + d g_2 - q < 2.
\end{align*}

This means that $2/m$ is some fraction between 0 and 1 and $d g_2 - q = 1$. So \eqref{6.16a} becomes,
\begin{align}\label{6.19} g_2 = g_1 + i.
\end{align}

Substituting this into \eqref{case5b} we find that,
\begin{align*} g_1 q &= i + (q - 2)(g_1 + i),
\\ 2 g_1 &= i(q - 1) = i d g_1, \tag{by \eqref{6.14}}
\\ 2 &= i d.
\end{align*}

We remark that since both $i$ and $d$ are positive integers, $i$ (and indeed $d$) must equal 1 or 2. Thus by \eqref{6.19} and \eqref{6.14a},
\begin{align*} g_1 &= \frac{i(q-1)}{2}, \qquad g_2 = \frac{i(q + 1)}{2}, \qquad g = \frac{2 g_1 g_2 q}{i} = \frac{iq(q^2 - 1)}{2}.
\end{align*}

Thus we have the following expressions for the orders of $K$ and $G$:
\begin{align}\label{orderGK} |K| = \frac{ei(q-1)}{2}, \qquad |G| = \frac{eiq(q^2-1)}{2}.
\end{align}

By Proposition \ref{6.7}, each noncentral element of $Q$ has a unique common fixed point on the projective line $\mathscr{L}$, call it $P_1$. Furthermore, we saw in the proof of Theorem \ref{MaximalAbelianSubgroup.K_mem_MaximalAbelianSubgroups_of_center_lt_card_K} that each noncentral element of $K$ also fixes $P_1$ as well as one other point, call it $P_2$. Let $u$ be a noncentral element of $Q$ and set $P_3 = P_2^u$. Clearly $P_3$ is different from $P_1$ and $P_2$ because otherwise a contradiction is reached. By Theorem \ref{6.6}, $PSL(\mathscr{L})$ is triply transitive, so there exists a $v \in L$ such that,
\begin{align*} P_1^v = R_1 = \begin{bmatrix} 0 \\ 1 \end{bmatrix}, \qquad P_2^v = R_2 = \begin{bmatrix} 1 \\ 0 \end{bmatrix}, \qquad P_3^v = R_3 = \begin{bmatrix} 1 \\ 1 \end{bmatrix}.
\end{align*} 

Observe that,
\begin{align*} vQv^{-1}R_1 &= vQP_1 = vP_1 = R_1,
\\ vKv^{-1}R_i &= vKP_i = vP_i = R_i. \qquad (i=1,2)
\end{align*} 

Thus $vQv^{-1}$ fixes $R_1$ whilst $vKv^{-1}$ fixes both $R_1$ and $R_2$. The only elements of $L$ that fix $R_1$ are the lower triangular matrices, thus  $vQv^{-1} \subset H$, whilst the only elements that fix $R_2$ are the upper triangular matrices, thus $vKv^{-1} \subset D$. Furthermore, each noncentral element of $vQv^{-1}$ has order $p$. The only elements of $H$ with order $p$ are those in $T$, thus $vQv^{-1} \subset T$. Since $u \in Q \setminus I_G$, we have that $v u v^{-1} = t_\gamma$ for some $\gamma \in F$.
\begin{align*} v u v^{-1}R_2 &= v u P_2 = v P_3 = R_3,
\\[1.5ex] \begin{bmatrix} 1 & 0\\ \gamma & 1 \end{bmatrix} \begin{bmatrix} 1 \\ 0 \end{bmatrix} &= \begin{bmatrix} 1 \\ \gamma  \end{bmatrix} \sim \begin{bmatrix} 1 \\ 1 \end{bmatrix}. \Longrightarrow \gamma = 1.
\end{align*}

So $v u v^{-1} = t_1$. If we now consider $\widetilde{G} = vGv^{-1}$ instead of $G$, we can assume without loss of generality that,
\begin{align*} Q \subset T, \qquad K \subset D, \qquad u = t_1.
\end{align*}

Let $x$ be a generator of $K$. By Theorem \ref{MaximalAbelianSubgroup.of_index_normalizer_eq_two} there exists a $y \in N_{\widetilde{G}}(K) \! \setminus \! K$ such that $y x = x^{-1} y$. Since $R_1$ is fixed by both $x$ and $x^{-1}$ we have,
\begin{align*} x^{-1} y R_1 =  y x R_1 = y R_1.
\end{align*}
Thus $x^{-1}$ fixes $y R_1$, that is $y R_1 \in \{ R_1, R_2 \}$. Similarly, $y R_2 \in \{ R_1, R_2 \}$. Assume $y R_1 = R_1$. Since $R_1$ and $R_2$ are distinct points in $\mathscr{L}$ this implies that $y R_2 = R_2$.

\begin{align*} y R_1 = \begin{bmatrix} \alpha & \beta \\ \gamma & \delta \end{bmatrix} \begin{bmatrix} 0 \\ 1 \end{bmatrix} &= \begin{bmatrix} \beta \\ \delta \end{bmatrix} \sim \begin{bmatrix} 0 \\ 1 \end{bmatrix} \Longrightarrow \beta = 0.
\\[1.5ex] y R_2 = \begin{bmatrix} \alpha & \beta \\ \gamma & \delta \end{bmatrix} \begin{bmatrix} 1 \\ 0 \end{bmatrix} &= \begin{bmatrix} \alpha \\ \gamma \end{bmatrix} \sim \begin{bmatrix} 1 \\ 0 \end{bmatrix} \Longrightarrow \gamma = 0.
\end{align*}

Thus $y \in D$, which is a contradiction since elements in $D$ do not invert $x \in D$, hence,
\begin{align}\label{yinterchange} y R_1 = R_2, \qquad \text{and} \quad y R_2 = R_1.
\end{align}
 
This allows us to determine more about $y$,
\begin{align*} y R_1 = \begin{bmatrix} \alpha & \beta \\ \gamma & \delta \end{bmatrix} \begin{bmatrix} 0 \\ 1 \end{bmatrix} &= \begin{bmatrix} \beta \\ \delta \end{bmatrix} \sim \begin{bmatrix} 1 \\ 0 \end{bmatrix} \Longrightarrow \delta = 0.
\\[1.5ex] y R_2 = \begin{bmatrix} \alpha & \beta \\ \gamma & \delta \end{bmatrix} \begin{bmatrix} 1 \\ 0 \end{bmatrix} &= \begin{bmatrix} \alpha \\ \gamma \end{bmatrix} \sim \begin{bmatrix} 0 \\ 1 \end{bmatrix} \Longrightarrow \alpha = 0.
\end{align*}

Thus $y$ is an anti-diagonal matrix. Recalling \eqref{antidiag}, for some $\rho \in F^*$ we have,
\begin{align*} y = d_\rho w = \begin{bmatrix} 0 & \rho \\ -\rho^{-1} & 0 \end{bmatrix}.
\end{align*}

Consider now the set of right cosets of $N_{\widetilde{G}}(Q)$ of the form $N_{\widetilde{G}}(Q) y q$, (where $q \in Q$) in $N_{\widetilde{G}}(Q) y Q$. For $q_1, q_2 \in Q$ we have,
\vspace{2mm}
\begin{align*} N_{\widetilde{G}}(Q) y q_1 = N_{\widetilde{G}}(Q) y q_2 &\iff y q_2 {q_1}^{-1} y^{-1} \in N_{\widetilde{G}}(Q)
\\ &\iff q_2 {q_1}^{-1} \in y^{-1} N_{\widetilde{G}}(Q) y
\\ &\iff (Q \cap y^{-1} N_{\widetilde{G}}(Q) y) q_2 = (Q \cap y^{-1} N_{\widetilde{G}}(Q) y) q_1. \\
\end{align*}

So the number of right cosets of $N_{\widetilde{G}}(Q)$ in $N_{\widetilde{G}}(Q) y Q$ is equal to the number of right cosets of $Q \cap y^{-1} N_{\widetilde{G}}(Q) y$ in $Q$. That is,
\vspace{2mm}
\begin{align}\label{doublecoset} [N_{\widetilde{G}}(Q) y Q : N_{\widetilde{G}}(Q)] = [Q : Q \cap y^{-1} N_{\widetilde{G}}(Q) y]. \\ \nonumber
\end{align}

Let $g$ be an arbitrary element of $N_{\widetilde{G}}(Q)$. By Theorems \ref{6.4i}(i) and \ref{6.7}(ii) we have $N_{\widetilde{G}}(Q) \subset H = \text{Stab}(R_1)$, thus $g$ fixes $R_1$. Using \eqref{yinterchange} we see that,
\vspace{2mm}
\begin{align*} y^{-1} g y R_2 = y^{-1} g R_1 = y^{-1} R_1 = R_2. \\
\end{align*}

Hence $R_2$ is a fixed point of $y^{-1} g y$. Since $g$ was chosen arbitrarily, we assert that each element of $y^{-1} N_{\widetilde{G}}(Q) y$ fixes $R_2$. On the contrary, the only element of $Q$ which fixes $R_2$ is $I_{\widetilde{G}}$, thus $Q \cap y N_{\widetilde{G}}(Q) y^{-1} = I_{\widetilde{G}}$.
\vspace{2mm}
\begin{align}\label{qwed} [N_{\widetilde{G}}(Q) y Q : N_{\widetilde{G}}(Q)] &= [Q : Q \cap y^{-1} N_{\widetilde{G}}(Q) y] = q, \nonumber
\\[1ex] |N_{\widetilde{G}}(Q) y Q| &= q|N_{\widetilde{G}}(Q)|. \\ \nonumber
\end{align}

We show next that $N_{\widetilde{G}}(Q) y Q \cap N_{\widetilde{G}}(Q) = \varnothing$. Let $t_\lambda d_\omega$ and $t_\mu$ be arbitrarily chosen from $N_{\widetilde{G}}(Q)$ and $Q$ respectively so that $t_\lambda d_\omega y t_\mu$ is an arbitrary element of $N_{\widetilde{G}}(Q) y Q$.
\begin{align}\label{onemore} t_\lambda d_\omega y t_\mu &= \begin{bmatrix} 1 & 0 \\ \lambda & 1 \end{bmatrix} \begin{bmatrix} \omega & 0 \\ 0 & \omega^{-1} \end{bmatrix} \begin{bmatrix} 0 & \rho \\ -\rho^{-1} & 0 \end{bmatrix}  \begin{bmatrix} 1 & 0 \\ \mu & 1 \end{bmatrix} \nonumber
\\[1.5ex] &= \begin{bmatrix} \omega & 0 \\ \omega \lambda & \omega^{-1} \end{bmatrix} \begin{bmatrix} \rho \mu & \rho \\ -\rho^{-1} & 0 \end{bmatrix} \nonumber
\\[1.5ex] &= \begin{bmatrix} \omega \rho \mu & \omega \rho  \\ \omega \lambda \rho \mu - \omega^{-1} \rho^{-1} & \omega \rho \lambda \end{bmatrix}.
\end{align}

Since $\omega$, $\rho \in F^*$, the top right entry of \eqref{onemore} is non-zero. Recall also that $N_{\widetilde{G}}(Q) \subset H$ by Theorem \ref{6.4i}(i) and that $H$ is the set of all lower triangular matrices of $L$. Since $t_\lambda d_\omega d_\rho w t_\mu$ was chosen arbitraily, no element of $N_{\widetilde{G}}(Q) y Q$ is in $H$ whilst the whole of $N_{\widetilde{G}}(Q)$ is contained in $H$, thus they are disjoint. Using \eqref{qwed} and \eqref{orderGK} we also observe that,
\begin{align*} |N_{\widetilde{G}}(Q) y Q| + |N_{\widetilde{G}}(Q)| = (q+1)|N_{\widetilde{G}}(Q)| = (q+1)eqg_1 = \frac{eiq(q^2-1)}{2} = |{\widetilde{G}}|.
\end{align*}
Since $N_{\widetilde{G}}(Q) y Q$ and $N_{\widetilde{G}}(Q)$ are disjoint and the sum of their orders is equal to the order of ${\widetilde{G}}$, they partition ${\widetilde{G}}$ into the set of elements that belong to $H$ and the set that don't.
\begin{align}\label{gsplit} {\widetilde{G}} = N_{\widetilde{G}}(Q) y Q \cup N_{\widetilde{G}}(Q).
\end{align}

Let $\mathbb{N} = \{ \lambda : t_\lambda \in Q \}$. We will show that $\mathbb{N} =\mathbb{F}_q$. For each $t_\lambda \in Q \setminus Z$, the element $y t_\lambda y^{-1} \notin H$, so by $\eqref{gsplit}$, $y t_\lambda y^{-1} \in N_{\widetilde{G}}(Q) y Q$. Thus there exists $t_\mu, t_\upsilon \in Q$ and $d_\omega \in K$ such that,
\begin{align*} y t_\lambda y^{-1} &= t_\mu d_\omega y t_\upsilon,
\\[1.5ex] \begin{bmatrix} 0 & \rho \\ -\rho^{-1} & 0 \end{bmatrix}\begin{bmatrix} 1 & 0 \\ \lambda & 1 \end{bmatrix}\begin{bmatrix} 0 & -\rho \\ \rho^{-1} & 0 \end{bmatrix} &= \begin{bmatrix} 1 & 0 \\ \mu & 1 \end{bmatrix}\begin{bmatrix} \omega & 0 \\ 0 & \omega^{-1} \end{bmatrix}\begin{bmatrix} 0 & \rho \\ -\rho^{-1} & 0 \end{bmatrix}\begin{bmatrix} 1 & 0 \\ \upsilon & 1 \end{bmatrix},
\\[1.5ex] \begin{bmatrix} 0 & \rho \\ -\rho^{-1} & 0 \end{bmatrix}\begin{bmatrix} 0 & -\rho \\ \rho^{-1} & -\rho \lambda \end{bmatrix} &= \begin{bmatrix} \omega & 0 \\ \omega \mu & \omega^{-1} \end{bmatrix}\begin{bmatrix} \rho \upsilon & \rho \\ -\rho^{-1} & 0 \end{bmatrix},
\\[1.5ex] \begin{bmatrix} 1 & -\rho^2 \lambda \\ 0 & 1 \end{bmatrix} &= \begin{bmatrix} \omega \rho \upsilon & \omega \rho \\ \omega \rho \mu \upsilon - \omega^{-1} \rho^{-1} & \omega \rho \mu \end{bmatrix}.
\end{align*}

Equating the top right entries gives,
\begin{align}\label{mattr} \omega = -\rho \lambda.
\end{align}

Since $t_1 \in Q$, so is it's inverse, thus $-1 \in \mathbb{N}$. Letting $\lambda = -1$ in \eqref{mattr} gives $\omega = \rho$, which means that $d_\rho \in K$. Consequently, this shows that $w = d_\rho^{-1} y \in {\widetilde{G}}$ and we may replace $y$ by $w$ in \eqref{gsplit} without it affecting the partition of ${\widetilde{G}}$. This is equivalent to letting $\rho = 1$, and \eqref{mattr} simplifies to,
\begin{align}\label{mattr2} \omega = -\lambda.
\end{align}

Let $\mathbb{M} = \{ \omega : d_\omega \in K \}$. Recall from \eqref{orderGK} that $|K| = i(q-1)$. We consider the different cases which arise depending on the values of $i$ and $e$. \\
\\
Let \textbf{Case Va} be the case when $e=1$ or $i = 1$. Observe that $i$ and $e$ cannot both equal 1, since this would imply that 2 divides $q-1$ (by \eqref{orderGK}), but if $e=1$ it follows that $q-1$ is even. Hence $ei = 2$ and $K$ has order $q-1$. Furthermore, the order of each element of $K$ divides $q-1$, so for each $\omega \in \mathbb{M}$,
\begin{align}
    \label{roots} \omega^{q-1} = 1.
\end{align}
Also, the following polynomial has at most $q-1$ roots in $F$.
\begin{align}\label{rootsx} x^{q-1} = 1.
\end{align}
By \eqref{subfield}, $\mathbb{F}_q \subset F$ and each element of $\mathbb{F}^*_q$ is a root of \eqref{rootsx}. Thus each $\omega$ of $\mathbb{M}$ is in $\mathbb{F}^*_q$ and since they have the same cardinality, $\mathbb{M} = \mathbb{F}^*_q$. By \eqref{mattr2}, $\lambda$ also ranges over $\mathbb{F}^*_q$ and considering also that $\lambda$ can be 0, we have $\mathbb{N} =\mathbb{F}_q$. \\
\\
Observe that each element of ${\widetilde{G}}$ is either of the form $t_\lambda d_\omega$ or $t_\lambda d_\omega w t_\mu$ (where $\lambda, \mu \in \mathbb{F}_q$, $\omega \in \mathbb{F}^*_q$), so ${\widetilde{G}} \subset \SL_2(\mathbb{F}_q)$. Also, Propostion \ref{ordersl2q} gives that, $|\SL_2(\mathbb{F}_q)| = q(q^2-1) = |{\widetilde{G}}|$, so ${\widetilde{G}} = \SL_2(\mathbb{F}_q)$. Since ${\widetilde{G}}$ is conjugate in $L$ to $G$, we have $G \cong \SL_2(\mathbb{F}_q)$  as desired. \\
\\
Let \textbf{Case Vb} be the case when $i = 2 = e$. This time the order of each element of $K$ divides $2(q-1)$, so for each $\omega \in \mathbb{M}$,
\begin{align}
    \label{roots} \omega^{2(q-1)} = 1.
\end{align}
As in the case of $i=1$, each element of $\mathbb{F}^*_q$ is a root of the polynomial in \eqref{rootsx}, as are each $\omega^2$. Thus $\omega^2$ ranges over $\mathbb{F}^*_q$ and by \eqref{subfield}, $\omega \in \mathbb{F}_{q^2} \setminus \mathbb{F}_q$. Simple matrix multiplication shows that, \\
\begin{align*} d_\omega^{-1} t_\lambda d_\omega = t_{\omega^2 \lambda}.
\end{align*}
Hence since $t_0, t_1 \in Q$, it follows that $t_{\omega^2} \in Q$ for each $\omega^2 \in \mathbb{F}^*_q$, thus $\mathbb{N} = \mathbb{F}_q$. Since $K$ is a cyclic group of order $2(q-1)$, so too is $\mathbb{M}$. Let $\pi$ be a generator of $\mathbb{M}$. It follows that $\pi^2$ has order $q-1$ and is therefore a generator of $\mathbb{F}^*_q$. Since $K = \langle d_\pi \rangle$, we have:
\begin{align*} {\widetilde{G}} = \langle t_\lambda, d_\pi, w : \lambda \in \mathbb{F}_q \rangle = \langle \SL_2(\mathbb{F}_q), d_\pi \rangle.
\end{align*}
Again, since ${\widetilde{G}}$ is conjugate in $L$ to $G$, we have $G \cong \langle \SL_2(\mathbb{F}_q), d_\pi \rangle$ as desired. Now we take an arbitrary $x$ from $\SL_2(\mathbb{F}_q)$ and conjugate it by $d_\pi$.
\begin{align*} d_\pi x d_\pi^{-1} &= \begin{bmatrix} \pi & 0 \\ 0 & \pi^{-1} \end{bmatrix} \begin{bmatrix} \alpha & \beta \\ \gamma & \delta \end{bmatrix}\begin{bmatrix} \pi^{-1} & 0 \\ 0 & \pi \end{bmatrix}
\\[1.5ex] &=  \begin{bmatrix} \pi & 0 \\ 0 & \pi^{-1} \end{bmatrix}  \begin{bmatrix} \alpha \pi^{-1} & \beta \pi \\ \gamma \pi^{-1} & \delta \pi \end{bmatrix}
\\[1.5ex] &= \begin{bmatrix} \alpha & \beta \pi^{-2} \\ \gamma \pi^{2} & \delta \end{bmatrix}. 
\end{align*}
Since $\pi^2 \in \mathbb{F}_q$, we have that $d_\pi x d_\pi^{-1} \in \SL_2(\mathbb{F}_q)$ and since $x$ was chosen arbitrarily, $d_\pi$ belongs to the normaliser of $\SL_2(\mathbb{F}_q)$ in $\langle \SL_2(\mathbb{F}_q), d_\pi \rangle$. This shows that $\SL_2(\mathbb{F}_q) \vartriangleleft \langle \SL_2(\mathbb{F}_q), d_\pi \rangle$ as desired. \\
\\
 \space \textbf{Cases Vc and Vd:} $\pmb{q \leq 3}$. Since $q - 1 = d g_1 \geq 2$ by \eqref{6.14}, $q$ cannot equal 2. So $q = 3 = p$, $e = 2$ and thus $g_1 = 2$. The inequalities in \eqref{6.13} and \eqref{6.16b} give,
\begin{align*} 2 < g_2 < 6.
\end{align*}
Also, since $g_2$ is relatively prime to $p=3$, we have $g_2 = 4$ or 5. Let \textbf{Case Vc} be the case when $g_2 = 4$. \eqref{case5b} becomes,
\begin{align*} \frac{1}{8} = \frac{1}{g} + \frac{1}{4} - \frac{1}{6},
\end{align*}

which gives $g = 24$. Observe that,
\begin{align*} |K| = 4 = i(q-1), \qquad |G| = 48 = iq(q^2-1),
\end{align*}
where $i=2$, thus we have the situation as described in Case Vb. That is, $G \cong \langle \SL_2(\mathbb{F}_q), d_\pi \rangle$ with $q=3$.\\
\\
Alternatively, \textbf{Case Vd} occurs when $g_2 = 5$. \eqref{case5b} becomes,
\begin{align*} \frac{1}{10} = \frac{1}{g} + \frac{1}{4} - \frac{1}{6}.
\end{align*}

Thus $g = 60 $ and $|G| = 120$. We verify, using Proposition \ref{ordersl2q}, that $\SL_2(5)$ has the same order as $G$, that is $|\SL_2(5)| = 5(5^2-1) =120$. Observe that,
\begin{align*} |\mathcal{C}_1| &= [G : N_G(A_1)] = \frac{eg}{2eg_1} = 15,
\\[1ex] |\mathcal{C}_2| &= [G : N_G(A_2)] = \frac{eg}{2eg_2} = 6,
\\[1ex] |\mathcal{C}_{Q \times Z}| &= [G : N_G(Q \times Z)] = \frac{eg}{ekq} = 10.
\end{align*}

Now consider the quotient group $G / Z$ of order 60. It's trivial that for all $A_i, A_j \in \mathfrak{M}$, $A_i / Z$ belongs to the same conjugacy class as $A_j / Z$ if and only $A_i$ and $A_j$ belong to the same conjugacy class. So the number of subgroups conjugate to $A_i / Z$ is $|\mathcal{C}_i|$. Similarly, the number of subgroups conjugate to $(Q\times Z) / Z$ is $|\mathcal{C}_{Q \times Z}|$. \\
\\
We now calculate the order of each maximal abelian subgroup of $G$ when we quotient out $Z$.
\begin{align*} |A_1 / Z| = 2, \qquad |A_2 / Z| = 5, \qquad |(Q \times Z) / Z| = 3.
\end{align*}

We now know enough about $G / Z$ to determine the order of each of it's elements: \\
\\
 \space The identity has order 1.gives \\
 \space The non-central element of $A_1 / Z$ has order 2, as does the non-central element in each of the $|\mathcal{C}_1| = 15$ subgroups conjugate to $A_1 / Z$. So there are $15$ elements of order 2. \\
 \space The 4 non-central elements of $A_2 / Z$ have order 5, as do the non-central elements in each of the $|\mathcal{C}_2| = 6$ subgroups conjugate to $A_2 / Z$. Thus there are $24$ elements of order 5. \\
 \space  The 2 non-central elements of $(Q \times Z) / Z$ have order 3, as do the non-central elements in each of the $|\mathcal{C}_{Q \times Z}| = 10$ subgroups conjugate to $(Q \times Z) / Z$. Thus there are $20$ elements of order 3. \\
\\
Since $1+15+24+20=60$, all elements of $G / Z$ are accounted for. \\
\\
Let $N$ be a normal subgroup of $G / Z$. Observe that each non-central element of $A_2 / Z$ is a generator of it, so if $N$ contains one non-central element of $A_2 / Z$, then it contains the whole of it, due to the closure of the group under multiplication and the fact that each element of $A_2 / Z$ is a power of any non-central element. Also, it can easily be seen that normal subgroups are composed of whole conjugacy classes, so since $N$ is normal in $G$, if it contains $A_2 / Z$, it must contain all subgroups conjugate to $A_2 / Z$. The consequence of this is that if $N$ has an element of order 5, then it contains all 24 elements of $G / Z$ of order 5. Similarly, if it contains an element of order 2, it contains all 15 of them and if it contains an element of order 3, it contains all 20 of them. This means that $|N|$ is partitioned by some or all of the elements in $\{ 1, 15, 20, 24 \}$. Bearing in mind that the order of $N$ divides 60 and that $N$ contains the identity element, this means that $N$ is equal to either the identity element or it is the whole of $G / Z$, since it's easy to see that no other partition of those numbers divides 60. Thus $G / Z$ has no non-trivial normal subgroups and is simple. \\
\\
By \cite[p.145]{dummit}, the only simple groups of order 60 are those isomorphic to the alternating group $A_5$ (not to be confused with an element of $\mathfrak{M}$), thus $G / Z \cong A_5$. Since $Z \cong \mathbb{Z}_2$, we have that $G$ is isomorphic to a central extension of $A_5$ which, according to Schur \cite{schur}, is unique and isomorphic to $\SL_2(5)$ as desired. The proofs of these 2 claims are beyond the scope of this thesis. \qedhere

\end{proof}

\begin{theorem}[Case VI]
\label{case_VI}
\uses{card_noncenter_fin_subgroup_eq_sum_card_noncenter_mul_index_normalizer, MaximalAbelianSubgroup.of_index_normalizer_eq_two}
\lean{case_VI}
Claim: \textit{We have one of the following three cases: \\
\\
(i) $G = \langle \, x,y \, | \, x^n = y^2, \, yxy^{-1} = x^{-1} \, \rangle$, where $n$ is even. \\
\\
(ii) $G = \widehat{S}_4$. \\
\\
(iii) $G \cong \SL_2(5)$ and $p$ does not divide $|G|$. \\
\\
Where $\widehat{S}_4$ is one of the representation groups of the symmetric group $S_4$ in which the transpositions correspond to the elements of order 4.} \\
\end{theorem}
% DEPENDENCIES AND STATEMENT

\begin{proof} Here, $s = 0$ and $t = 3$. Equation \eqref{classeq} simplifies to:
\begin{align} \label{case6a} 1 &= \frac{1}{g} + \frac{q-1}{qk} + \frac{g_1 -1}{2g_1} + \frac{g_2 -1}{2g_2} + \frac{g_3 -1}{2g_3}, \nonumber
\\[1ex] \frac{1}{2g_1} + \frac{1}{2g_2} + \frac{1}{2g_3} &= \frac{1}{g} + \frac{q-1}{qk} + \frac{1}{2}.
\end{align}

First assume that $q > 1$ and $k=1$. \eqref{case6a} is thus bounded as follows,
\begin{align*} \frac{3}{4} > \frac{1}{2g_1} + \frac{1}{2g_2} + \frac{1}{2g_3} &= \frac{1}{g} + \frac{q-1}{qk} + \frac{1}{2} > 1,
\end{align*}
which is a contradiction. Now assume that $q > 1$ and $k > 1$. This means that $k=g_i$ for some $i$. Without loss of generality we can assume that $k=g_1$. Now \eqref{case6a} becomes,
\begin{align*} \frac{1}{2} \geq \frac{1}{2g_2} + \frac{1}{2g_3} &\geq \frac{1}{g} + \frac{1}{2} > \frac{1}{2},
\end{align*}
which again is a contradiction, thus we conclude that $q=1$. \eqref{case6a} simplifies and we can now determine the possible values of each $g_i$.
 \begin{align} \label{case6b} \frac{1}{2g_1} + \frac{1}{2g_2} + \frac{1}{2g_3} &= \frac{1}{g} + \frac{1}{2}.
\end{align}

Without loss of generality we may assume that $2 \leq g_1 \leq g_2 \leq g_3$. If $g_1 \neq 2$ we arrive at the following contradiction
\begin{align*} \frac{1}{6} + \frac{1}{6} + \frac{1}{6} \geq \frac{1}{2g_1} + \frac{1}{2g_2} + \frac{1}{2g_3} &= \frac{1}{g} + \frac{1}{2}.
\end{align*}
Thus $g_1 = 2$ and we have,
\begin{align}\label{case6c} \frac{1}{2g_2} + \frac{1}{2g_3} > \frac{1}{4}.
\end{align}
\newpage
Clearly $g_2$ must equal either 2 or 3. If $g_2 = 2$ it is easily shown that $g=2 g_3$. If $g_2 = 3$ we see that $g_3 \in \{ 3,4,5 \}$. Assume that $g_2$ and $g_3 = 3$. Notice that since  $g_1 = 2$, 2 must divide the order of $G$. Recall also that a Sylow $p$-subgroup of $G$ has order 1, so we assert that $p \neq 2$ and $e=2$. We see from \eqref{case6b} that $|G| = 24$ and thus a Sylow $3$-subgroup has order 3. The maximal abelian subgroups conjugate to $A_2$ or $A_3$ have order 6 and therefore each contains a Sylow $3$-subgroup of $G$. Let $B_2$ and $B_3$ be the Sylow $3$-subgroups contained in $A_2$ and $A_3$ respectively. Observe that for $i = 2$ or 3,
\begin{align}\label{case6d} A_i \cong \mathbb{Z}_6 \cong \mathbb{Z}_3 \times \mathbb{Z}_2 \cong B_i \times Z \cong B_i Z. 
\end{align}
Let $b_2 \in B_2$, $b_3 \in B_3$ and $z \in Z$. Recall that $B_2$ and $B_3$ are conjugate in $G$ by Sylow's Second Theorem, so there exists an $x \in G$ such that,
\begin{align*} x b_2 x^{-1} &= b_3,
\\ x b_2 x^{-1} z &= b_3 z,
\\ x b_2 z x^{-1} &= b_3 z.
\end{align*} 
Since $b_2$, $b_3$ and $z$ were chosen arbitrarily, we observe that $B_2 Z$ is conjuagate to $B_3 Z$ and thus by \eqref{case6d}, $A_2 \cong A_3$. This contradicts the fact that $A_2$ and $A_3$ are representatives of different conjugacy classes of maximal abelian subgroups of $G$, which means that $g_2$ and $g_3$ cannot both equal 3. Thus we are left with the following three cases:
\begin{align*} g_1 = 2, \qquad g_2&=2, \qquad g=2 g_3.
\\[1ex] g_1 = 2, \qquad g_2&=3, \qquad g_3 = 4.
\\[1ex] g_1 = 2, \qquad g_2&=3, \qquad g_3 = 5.
\end{align*}
\\
 \space \textbf{Case VIa:} $\pmb{g_1 = 2, g_2 = 2, g=2 g_3}$. First observe that,
\begin{align*} [G : N_G(A_1)] = \frac{eg}{2eg_1} = \frac{g_3}{2}.
\end{align*}
Thus $g_3/2$ is an integer which means that $g_3$ must be even, call it $n$. Now let $A_3 = \langle x \rangle$. Since $|A_3| = eg_3$, the order of $x$ is $2n$ and $x^n$ has order 2. By Theorem \eqref{6.8}(iv) there exists a $y \in N_G(A_3) \! \setminus \! A_3$ such that $y x y^{-1} = x^{-1}$. Also,
\begin{align*} |\mathcal{C}_3| = [G : N_G(A_3)] = 1.
\end{align*}
Since $y \not \in A_3$ and $A_3$ has no conjugate subgroups (aside from itself), $y$ must lie in a maximal abelian subgroup conjugate to either $A_1$ or $A_2$. This means that since $|A_1| = 4 = |A_2|$ and $y \not \in Z$, the order of $y$ must be 4. By the uniqueness of the element of order 2, we have the relation $x^n = y^2$ and $G$ is given by the presentation,
\begin{align*} G = \langle \, x,y \, | \, x^n = y^2, \, yxy^{-1} = x^{-1} \, \rangle. \qquad \text{(where $n$ is even)}
\end{align*}

 \space \textbf{Case VIb:} $\pmb{g_1 = 2, g_2 = 3, g_3 = 4}$. In this case \eqref{case6b} becomes,
\begin{align*} \frac{1}{4} + \frac{1}{6} + \frac{1}{8} &= \frac{1}{g} + \frac{1}{2}.
\end{align*}
Thus $g = 24$ and $|G| = 48$. Consider the quotient group $G / Z$ of order 24 and the quotient group $N_G(A_2) / Z$ which, for convenience, we will call $H$.
\begin{align*} |H| = \frac{2eg_2}{e} = 6.
\end{align*}

Let $x$ be an element of order 6 from $A_2$. By Theorem \ref{MaximalAbelianSubgroup.of_index_normalizer_eq_two} there exists a $y \in N_G(A_2) \! \setminus \! A_2$ such that $y x = x^{-1} y$. Thus for $xZ, yZ, x^{-1}Z \in H$ we have,
\begin{align*} yZ xZ = yxZ =  x^{-1}yZ = x^{-1}Z yZ.
\end{align*}
If $H$ is abelian, then $xZ = x^{-1}Z$ and thus $x^2 \in Z$. Also, since $x$ has order 6, $x^2$ has order 3. This is contradiction since there is no element of order 3 in $Z$. Thus $H$ is non-abelian and is therefore isomorphic to the symmetric group $S_3$. \\
\\
Now we determine the normal subgroups of $H$. The identity and $H$ itself are trivially normal. Furthermore, the elementary result that any subgroup of index 2 is normal implies that $A_2 / Z$, the subgroup of $H$ of order 3, is normal. It remains to check the subgroups of order 2. Let r be a generator of one of the subgroups of order 2 and let $x$ be an arbitrary element of $H$. If $\langle r \rangle$ is normal in $H$, then $x r x^{-1} \in \{ I , r \}$. Since $r \neq I$ it follows that $x r x^{-1} \neq I$. Alternatively if $x r x^{-1} = r$, then $r \in Z(H)$. By the elementary result that $Z(S_n) = \{ I \}$ for $n > 2$, we have that $Z(H) = \{ I \}$ and the contradiction $r=I$. Thus $x r x^{-1} \not \in \langle r \rangle$ and $H$ has no normal subgroup of order 2. We conclude that the only normal subgroups of $H$ are those of order 1, 3 or 6. \\
\\
Note that the index of $H$ in $G / Z$ is 4. Let $G / Z$ act by left multiplication on the set of left cosets of $H$. By Theorem \ref{symhomoker}, this action induces a homomorphism $\phi : G / Z \longrightarrow S_4$ with kernel,
\begin{align*} ker(\phi) = \bigcap\limits_{x \in G / Z} x H x^{-1}  \subset H.
\end{align*}

Recall the elementary result that the kernel of a homomorphism is a normal subgroup of it's domain. Thus the kernel of $\phi$ is normal in $G / Z$ and consequently in $H$ as well, that is $ker(\phi) \in\{ I , A_2 / Z, H \}$. \\
\\
If $ker(\phi) = A_2 / Z$, then $A_2 / Z \vartriangleleft G / Z$ and by Lemma \ref{QuotientGroup.comapMk'OrderIso} $A_2 \vartriangleleft G$. This is a contradiction since the normaliser in $G$ of $A_2$ is a proper subgroup of $G$, thus $ker(\phi) \neq A_2 / Z$. \\
\\
If $ker(\phi) = H$, then $H \vartriangleleft G / Z$. Take an arbitrary $x \in G / Z$. Since $A_2 / Z$ is a subgroup of $H$ we get,
\begin{align*} x (A_2 / Z) x^{-1} \subset H.
\end{align*}
Furthermore, since $A_2 / Z$ has order 3, any subgroup conjugate to it has order 3. Since the only subgroup of $H$ of order 3 is $A_2 / Z$, and since $x$ was chosen arbitrarily, $A_2 / Z \vartriangleleft G / Z$. We have already shown that this leads to a contradiction, thus $ker(\phi) \neq H$. \\
\\
We conclude that $ker(\phi) = \{ I \}$ and so $\phi$ is injective. Since $G / Z$ has 24 elements, it's image under $\phi$ is the whole of $S_4$, that is $G / Z \cong S_4$. Thus $G$ is a \textit{representation group} of $S_4$, denoted by $\widehat{S}_4$ (for a full defintion of this, see \cite{suzuki}). Suzuki proves that $S_4$ has 2 distinct representation groups up to isomorphism \cite[p.301]{suzuki}, which are distinguished by the property that the elements corresponding to transpositions have either order 2 or order 4. Since $G$ has a unique element of order 2, it must be isomorphic to the representation group of $S_4$ in which the transpositions correspond to the elements of order 4, as desired.\\
\\
 \space \textbf{Case VIc:} $\pmb{g_1 = 2, g_2 = 3, g_3 = 5}$.  In this case \eqref{case6b} becomes,
\begin{align*} \frac{1}{4} + \frac{1}{6} + \frac{1}{10} &= \frac{1}{g} + \frac{1}{2}.
\end{align*}
Thus $|g| = 60$ and $|G| = 120$. Observe that a simple relabelling of the maximal abelian subgroups gives the same situation as described in \textbf{Case Vd:}. Thus $G \cong \SL_2(5)$, however in this case $p$ does not divide $|G|$.

\end{proof}

\section{Dickson's Classification Theorem}

We now state the main result of this paper, Dickson's classification of finite subgroups of $\SL_2(F)$. Observe that it is not the focus of this paper to determine whether the following groups actually exist, rather that this theorem can be regarded as an \textit{upper bound}, so to speak, of the only possible subgroups of $\SL_2(F)$.\\

\begin{theorem}[Class I]
    \label{dicksons_classification_theorem_class_I}
    \uses{case_I, case_II, case_III, case_VI}
    \lean{dicksons_classification_theorem_class_I} Let $F$ be an arbitary algebraically closed field of characteristic $p$. Any finite subgroup $G$ of $\SL_2(F)$ is isomorphic to one of the following groups. \vspace{3mm} \\
: When $p=0$ or $|G|$ is relatively prime to $p$: \vspace{1mm} \\
(i) A cyclic group. \vspace{3mm} \\
(ii) The group defined by the presentation:
\begin{equation*} \langle \, x,y \, | \, x^n = y^2, \, yxy^{-1} = x^{-1} \, \rangle.
\end{equation*}
(iii) The Special Linear Group $\SL_2(3)$. \vspace{3mm} \\
(iv) The Special Linear Group $\SL_2(5)$. \vspace{3mm} \\
(v) $\widehat{S}_4$, the representation group of $S_4$ in which the transpositions correspond to the elements of order $4$. \\
\\
\end{theorem}
% DEPENDENCIES
\begin{proof}
    Case Ia: This leads to Class I (i). \\
    Case IIa: This leads to Class I (ii) where $n$ is odd. \\
    Case IIb: This leads to Class I (iii). \\
    Case III where $G=Z$: This leads to Class I (i).\\
    Case VIa: This leads to Class I (ii) where $n$ is even. \\
    Case VIb: This leads to Class I (v). \\
    Case VIc: This leads to Class I (iv). \\
    
\end{proof}

\begin{theorem}[Class II]
    \label{dicksons_classification_theorem_class_II}
    \uses{case_I, case_III, case_IV, case_V}
    \lean{dicksons_classification_theorem_class_II}
    When $|G|$ is divisible by $p$: \vspace{1mm} \\
(vi) $Q$ is elementary abelian, $Q \vartriangleleft G$ and $G/Q$ is a cyclic group whose order is relatively prime to $p$. \vspace{3mm} \\
(vii) $p=2$ and $G$ is a dihedral group of order $2n$, where $n$ is odd. \vspace{3mm} \\
(viii) The Special Linear Group $\SL_2(5)$, where $p=3=q$. \vspace{3mm} \\
(ix) The Special Linear Group $\SL_2(\mathbb{F}_q)$. \vspace{3mm} \\
(x) The group $\langle \SL_2(\mathbb{F}_q), d_\pi \rangle$, where $\SL_2(\mathbb{F}_q) \vartriangleleft \langle \SL_2(\mathbb{F}_q), d_\pi \rangle$. \vspace{3mm} \\

Here, $Q$ is a Sylow $p$-subgroup of $G$ of order $q$, $\mathbb{F}_q$ is a field of $q$ elements, $\mathbb{F}_{q^2}$ is a field of $q^2$ elements, $\pi \in \mathbb{F}_{q^2} \setminus \mathbb{F}_q$ and $\pi^2 \in \mathbb{F}_q$. \vspace{3mm}
\end{theorem}

\begin{proof}% \\
% If $Z \subset G$, then $G$ has the same structure as one of the 6 cases previously discussed. We match the separate cases to the above classes. \\
% \\

Case Ib: This leads to Class II (vi). \\
Case III where $G \ne Z$: This leads to Class II (vi). \\
Case IVa: This leads to Class II (vii). \\
Case IVb: This leads to Class II (ix) with $q=3$. \\
Case Va: This leads to Class II (ix). \\
Case Vb: This leads to Class II (x). \\
Case Vc: This leads to Class II (x) with $q=3$. \\
Case Vd: This leads to Class II (viii). \\
\end{proof}

\begin{lemma}
    If $Z \not \subset G$, then $G$ has no element of order 2 and $|G|$ is therefore odd. Observe that in Cases II, IV, V and VI, $|G|$ is always even, thus we have either Case I or III. These correspond to Class I (i) or Class II (vi). \\
\end{lemma}

\section{Classification of finite subgroups of $\PGL_2(\Fbar_p)$}

\begin{theorem}[Classification of finite subgroup of $\PGL_2(\Fbar)$]
    \label{FLT_classification_fin_subgroups_of_PGL2_over_AlgClosure_ZMod}
    \uses{dicksons_classification_theorem_class_II, dicksons_classification_theorem_class_I, PGL_iso_PSL, SpecialSubgroups.center_SL2_eq_Z}
    \lean{FLT_classification_fin_subgroups_of_PGL2_over_AlgClosure_ZMod}
    Let $G$ be a finite subgroup of $\PGL_2(\Fbar_p)$ then $G$ is isomorphic to either a cyclic group, a dihedral group, $A_4$, $S_5$, $A_5$, or is isomorphic to $\PSL_2(k)$ or $\PGL_2(k)$ for some finite field $k$ of characteristic $p$.
\end{theorem}
% DEPENDENCY AND PROOF

\chapter{Bibliography}

% \bibliographystyle{plain} % We choose the "plain" reference style
% \bibliography{bibliography}

\begin{thebibliography}{3}

    \bibitem{butler}
    Butler, C. (2019). 
    \textit{Dickson's Classification of Finite Subgroups of the Two-dimensional Special Linear Group over an Algebraically Closed Field.}
    Master's Theses in Mathematical Sciences 2019: E63.
    
    \bibitem{sangwin}
    Sangwin, C. (2023). 
    \textit{Sums of the first n odd integers.}
    The Mathematical Gazette, 107(568), 10-24.
    
    \bibitem{dtt}
    Henri Darmon, Fred Diamond, and Richard Taylor. 
    \textit{Fermat’s last theorem}. 
    In Current developments in mathematics, 
    1995 (Cambridge, MA), pages 1–154. Int. Press, Cambridge, MA, 1994.
    \bibitem{alperin} 
    Alperin, J.L., Bell, R.B. 
    \textit{Groups and Representations}. 
    Springer,
    (1995).
    
    \bibitem{bhattacharya} 
    Bhattacharya, P.B., Jain, S.K., Nagpaul, S.R. 
    \textit{Basic Abstract Algebra, Second Edition}. 
    Cambridge University Press,
    (1994).
    
    \bibitem{dickson} 
    Dickson, L.E. 
    \textit{Linear Groups, with an Exposition of the Galois Field Theory}. 
    B.G.Teubner, Leipzig,
    (1901).
    
    \bibitem{dummit} 
    Dummit, D.S., Foote, R.M. 
    \textit{Abstract Algebra}. 
    Wiley,
    (2004).
    
    \bibitem{matrix} 
    Holst, A., Ufnarovski, V. 
    \textit{Matrix Theory}. 
    Studentlitteratur,
    (2014).
    
    \bibitem{hungerford} 
    Hungerford, T.W. 
    \textit{Abstract Algebra: An Introduction, Third Edition}. 
    Brooks/Cole, Cengage Learning,
    (2014).
    
    \bibitem{schur} 
    Schur, I. 
    \textit{Über die Darstellung der symmetrischen und der alternierenden Gruppe durch gebrochene lineare Substitutionen.} Journal für die reine und angewandte Mathematik (Crelles Journal) (139), p.155-250. 
    De Gruyter,
    (1911).
    
    \bibitem{stewart} 
    Stewart, I. 
    \textit{Galois Theory, Third Edition}. 
    Chapman \& Hall/CRC,
    (2003).
    
    \bibitem{suzuki}
    Suzuki, M. 
    \textit{Group Theory I}. 
    Spinger-Verlag, Berlin, Heidelberg, New York, 
    (1982).
    
\end{thebibliography}
\end{document}


\usepackage{xcolor}     % For coloring code elements

\home{https://AlexBrodbelt.github.io/ClassificationOfFiniteSubgroupsOfPGL}
\github{https://github.com/AlexBrodbelt/ClassificationOfFiniteSubgroupsOfPGL}
\dochome{https://AlexBrodbelt.github.io/ClassificationOfFiniteSubgroupsOfPGL/docs}

\title{Classification of finite subgroups of PGL}
\author{AlexBrodbelt}

\begin{document}
\maketitle
% In this file you should put the actual content of the blueprint.
% It will be used both by the web and the print version.
% It should *not* include the \begin{document}
%
% If you want to split the blueprint content into several files then
% the current file can be a simple sequence of \input. Otherwise It
% can start with a \section or \chapter for instance.
\chapter{Abstract and Acknowledgements}\label{Ch1_AbstractAndAcknowledgements}

\section{Acknowledgements}

I thank my supervisor Prof. David Jordan for his invaluable support and guidance throughout the project,

I would also like to thank Christopher Butler for providing the TeX code so I could easily set up the blueprint, and hopefully, improve and add to his amazing exposition of
\textbf{Dickson's Classification Theorem}.

I would also like to thank Prof. Kevin Buzzard for his support, patience and guidance throughout the project. His advice and comments on how I should go about formalising mathematics have been of utmost value. 

Finally, I would like to thank the many members of the Lean Zulip community who have provided insightful ideas and comments that have helped me progress much faster than otherwise, 
this also includes assistance with technical issues with setting up the blueprint and so forth. I am grateful to:

\begin{itemize}
    \item Artie Khovanov
    \item David Loeffler
    \item Mitchell Lee
    \item Yakov Pechersky
    \item Edward van de Meent
    \item Ruben Van de Velde
    \item Andrew Yang
    \item Johan Commelin
    \item Scott Carnahan
    \item Damiano Testa
    \item Aron Liu
\end{itemize}

\section{Abstract}


\section{Popular science summary}

In order to explain what this paper is about, it is necessary to first define a few of the mathematical concepts which it concerns. A \textit{group} is a set of objects, called \textit{elements}, together with a rule, called an \textit{operation}, which tells us how two elements combine with each other to make a third. Furthermore, to be considered a group it must also satisfy 4 conditions, called \textit{axioms}. One of which is that the group must be \textit{closed} under it's operation. This means that whenever any two elements in the group are combined, the resulting element is also part of the group. The remaining axioms require that the group must also be \textit{associative}, have an \textit{identity} element and each element must have an \textit{inverse}. The way in which the elements in a group act with each other is called the group's \textit{structure}. If 2 groups have the same number of elements and share the same structure, then they are regarded as being \textit{isomorphic} to each other, which essentially means that they equivalent. Many everyday things can be regarded as groups, such as the symmetries of geometrical objects, or the number systems we use. \\
\\
The set of 2 x 2 matrices whose \textit{determinant} is equal to 1, together with the operation of ordinary matrix multiplication, forms a group called the \textit{special linear group}. This is a group because the product of 2 matrices has a determinant equal to the product of the determinants of the 2 matrices, so since 1 x 1 = 1, this new element also belongs to the group, hence the axiom of being closed is satisfied. Furthermore, it is crucial that the entries in the matrices are taken from a specified \textit{ring} or \textit{field}. Rings and fields are, like groups, abstract mathematical objects, albeit they satisy even more axioms than groups do. Crucially, rings and fields have both an additive and a multiplicative identity. \\
\\
This paper focuses on $\SL_2(F)$, which is the two-dimensional special linear group whose entries are taken from an \textit{algebraically closed} field. Algebraically closed fields are infinite in size, which means that the resulting special linear group is also infinite. A \textit{subgroup} of a group is simply a group with the added requirement that each of it's elements must also belong to the original group. Thus a finite subgroup of $\SL_2(F)$ is any finite set of elements belonging to this infinite group $\SL_2(F)$, which satisfy the 4 axioms of being a group. \\
\\
This paper classifies all the possible structures which a finite subgroup of $\SL_2(F)$ could have. The result has implications within the study of finite \textit{simple} groups. This classification was first done by American mathematician Leonard Eugene Dickson in 1901. The purpose of this reformulation is to make it accessible to a wider audience by providing a more detailed explanation at the various stages of the proof.

\section{Abstract}

This paper is a reformulation of Leonard Dickson's complete classification of the finite subgroups of the two-dimensional special linear group over an arbitrary algebraically closed field, $\SL_2(F)$. The approach is to construct a class equation of the conjugacy classes of maximal abelian subgroups of an arbitrary finite subgroup of $\SL_2(F)$. In turn, this leads to only 10 possible classes of structures of this subgroup up to isomorphism.

\section{Acknowledgements from Christopher Butler}

I would like to take this opportunity to thank my advisor Arne Meurman. This paper would not have been possible without the guidance and insight he gave during our weekly discussions.

\cleardoublepage



\chapter{Introduction}\label{Ch2_Introduction}

\section{What is the formalization of mathematics?}

Formalization of mathematics is the art of teaching a computer what a piece of mathematics means.

That is, it is the process of carefully writing down a mathematical statement typically in first order logic or higher order logic and then scrutinously justifying each step of the proof to a computer program that checks the validity of every step of the reasoning. 

Typically one formalizes mathematics with the help of a proof assistant or interactive theorem prover, a piece of software which enables a human to write down mathematics and have the software verify the claims.

There exist many proof assistants, such examples are Lean, Isabelle, Coq, Metamath, etc.

For this project I have opted to use Lean due to its rapid growing mathematics library and its dependent type theory. I shall explain in more detail these last two reasons, but first I will comment on what Lean is.

\subsubsection{What is Lean?}

Lean is both a functional programming language and an interactive theorem prover that is being developed at Microsoft research and AWS by Leonardo de Moura and his team. It has been designed for both use in cutting-edge mathematics and the verification of software which is often essential to safety critical systems where correctness is of extreme
TODO:

- Brief explanation of type theory and curry-howard isomorphism.

- Example of formal proof and comparison with informal proof.


\begin{verbatim}
theorem add\textunderscore comm (a b : Nat) : a + b = b + a :=
  Nat.add\textunderscore comm a b
\end{verbatim}


\section{Fermat's Last Theorem}

TODO:

-History of Fermat's Equation

-Problem statements

-Developments in number theory that lead to the resolution of the conjecture.


\section{Formalizing Fermat's Last Theorem}

Following the sequence of success stories ranging from the Liquid Tensor Experiment to the formalization of the Polynomial Freiman-Rusza conjecture. 

Prof. Kevin Buzzard from Imperial College London has received a five-year grant that will allow him to lead the formalization of Fermat's Last Theorem. This grant kicked in in October of 2024. 

At the time of writing, since October of 2024, a digital blueprint has been set up to manage the project.

Alongside other infrastructure like the project dashboard, mathematicians around the world can claim tasks that are set by Prof. Kevin Buzzard and if in return a task is returned with a "sorry" free proof then one can claim the glory of having completed the task.

TODO:

- Current goal of the formalization

- Explain somewhat the modern approach and the highly sought after Modularity Lifting Theorem.

- My task: Classification of finite subgroups of $\PGL_2(\bar{\F}_p)$



\section{Classification of finite subgroups of the $\PGL_2(\Fbar_p)$}

TODO:

-Why are the finite subgroups of  $\PGL_2(\bar{\F}_p)$ relevant to number theory: i.e: Automorphic forms, Galois representations, etc.


The primary concern of this project is to formalise Theorem 2.47 of [DTT] which states:

\begin{enumerate}
    \item If $H$ is finite subgroup of $\PGL_2(\C)$ then $H$ is isomorphic to one of the following groups: the cyclic group $C_n$ of order $n$ ($n \in \Z_{>0}$), the dihedral group $D_{2n}$ of order $2n$ ($n \in \Z_{>1}$), $A_4$, $S_4$ or $A_5$.
\item If $H$ is a finite subgroup of $\PGL_2(\Fbar_p)$ then one of the following holds:
\begin{enumerate}
    \item $H$ is conjugate to a subgroup of the upper triangular matrices;
    \item $H$ is conjugate to $\PGL_2(\F_{\ell^r})$ and $\PSL_2(\F_{\ell^{r}})$ for some $r \in \Z_{>0}$;
    \item $H$ is isomorphic to $A_4$, $S_4$, $A_5$ or the dihedral group $D_{2r}$ of order $2r$ for some $r \in \Z_{>1}$ not divisible by $\ell$

\end{enumerate}
    Where $\ell$ is assumed to be an odd prime.
\end{enumerate}


By definition the Projective General Linear Group is:

\begin{equation}
    \PGL_n(F) = \GL_n(F) / (Z(\GL_n(F))) = \GL_n(F) / (F^\times I) 
\end{equation}

Similarly, the Projective Special Linear Group is:

\begin{equation}
    \PSL_n(F) = \SL_n(F) / (Z(\SL_n(F))) = \SL_n(F) / (\langle -I\rangle)
\end{equation}

Given we are working over an algebraically closed field $F$, it turns out that for any $n \in \N$, $\PGL_n(F)$ is isomorphic to $\PSL_n(F)$.

This isomorphism will be crucial as it will allow us to focus on classifying finite subgroups of $SL_2(F)$ to classify the finite subgroups of $\PGL_2(F)$.

The goal of the next chapter is to prove and formalize this result.
\chapter{Preliminaries}\label{Ch3_Preliminaries}

This section briefly outlines some standard group theory results which perhaps may not have been covered in a first course in Group Theory. Since they are not the main focus of this paper, most of the proofs have been omitted. A more
advanced reader may choose to skip this first chapter, using it only for reference purposes as and when the results are subsequently cited. 

\section{Some Elementary Theorems}

The following theorems are all well-known fundamental results in group theory. If the reader is interested in the proofs, they can be found in Hungerford \cite{hungerford}.

\begin{theorem}\label{lagrange} \textit{Let $G$ be a finite group. Then the order of any subgroup of $G$ divides the order of $G$.} \\
\end{theorem} 

\begin{theorem}\label{1stiso} \textit{Let $\phi  :G \rightarrow G'$ be a homomorphism of groups. Then, $$G/Ker \; \phi \cong Im \; \phi.$$ Hence, in particular, if $\phi$ is surjective then, $$G/Ker \; \phi \cong G'.$$} \\
\end{theorem} 

\vspace{-10mm}

\begin{theorem}\label{2ndiso} \textit{Let $H$ and $N$ be subgroups of $G$, and $N \vartriangleleft G$. Then, $$H/H \cap N \cong HN/N.$$} \\
\end{theorem} 

\vspace{-10mm}

\begin{theorem}\label{3rdiso} \textit{Let $H$ and $K$ be normal subgroups of $G$ and $K \subset H$. Then $H/K$ is a normal subgroup of $G/K$ and, $$(G/K)/(H/K) \cong G/H.$$} \\
\end{theorem} 

\vspace{-10mm}

\begin{theorem}\label{cauchy} \textit{If the order of a finite group $G$ is divisible by a prime number $p$, then $G$ has an element of order $p$.} \\
\end{theorem} 

\section{Sylow Theory}

In 1872, Norweigian mathematician Peter Ludwig Sylow published his theorems regarding the number of subgroups of a fixed order that a given finite group contains. Today these are collectively known as the Sylow Theorems and play a vital role in determining the structure of finite groups. I will use the results of these theorems several times throughout this paper and I state them here without proof. If the reader would like to read further, the proofs can be found in most introductory texts on group theory, such as Bhattacharya \cite{bhattacharya}, except Corollary \ref{5thsylow} which can be found in Alperin and Bell \cite[p.64]{alperin} . \\


\begin{definition}
\lean{Sylow}
\leanok 
Let $G$ be a finite group and $p$ a prime, a \textbf{Sylow $\pmb{p}$-subgroup} of $G$ is a subgroup of order $p^r$, where $p^{r+1}$ does not divide the order of $G$. \\
\\
Let $p$ be a prime. A group $G$ is called a \textbf{$\pmb{p}$-group} if the order of each of it's elements is a power of $p$. Similarly, a subgroup $H$ of $G$ is called a \textbf{$\pmb{p}$-subgroup} if the order of each of it's elements is a power of $p$.
\end{definition}

In each of the following results, $G$ is a finite group of order $p^r m$, where $p$ is a prime which does not divide $m$. \\
\\

\begin{theorem}[Sylow's first theorem]
\lean{Sylow.exists_subgroup_card_pow_prime}
\leanok
\textit{If $p^k$ divides $|G|$, then $G$ has a subgroup of order $p^k$.} \\

\end{theorem}

\begin{theorem}[Sylow's second theorem]
\lean{Sylow.equiv.proof_1}
\leanok
\textit{All Sylow $p$-subgroups of G are conjugate.} \\
\end{theorem}

\begin{theorem}[Sylow's third theorem]
\lean{card_sylow_modEq_one}
\leanok
\textit{The number of Sylow $p$-subgroups $n_p$ divides $m$ and satisfies $n_p \equiv 1 ($mod $p)$.} \\
\end{theorem}

\begin{corollary}[Sylow's fourth theorem]
\label{Sylow.unique_of_normal}
\lean{Sylow.unique_of_normal}
\leanok    
 \textit{A Sylow $p$-subgroup of $G$ is unique if and only if it is normal.} \\
\end{corollary}

\begin{corollary}[Sylow's fifth theorem]
\label{IsPGroup.exists_le_sylow}
\lean{IsPGroup.exists_le_sylow}
\leanok
\textit{Any $p$-subgroup of $G$ is contained in a Sylow $p$-subgroup.} \\
\end{corollary}

\section{Group Action}

\begin{definition} Let $G$ be a group and $X$ be a set. Then $G$ is said to \textbf{act} on $X$ if there is a map $\phi : G \times X \rightarrow X$, with $\phi(a,x)$ denoted by $a^*x$, such that for $a,b \in G$ and $x \in X$, the following 2 properties hold:
\begin{align*} &(i) \quad a\,^*(b\,^*x) = (ab)^*x,
\\  &(ii) \quad I_G\,^*x = x.
\end{align*}

The map $\phi$ is called the \textbf{group action} of $G$ on $X$.
\end{definition}

\begin{definition} Let $G$ be a group acting on a set $X$ and let $x \in X$. Then the set,
\begin{align*} Stab(x) = \{ g \in G  :  gx = x \},
\end{align*}
is called the \textbf{stabiliser} of $x$ in $G$. Each $g$ in $S_G(x)$ is said to \textbf{fix} $x$, whilst $x$ is said to be a \textbf{fixed point} of each $g$ in $S_G(x)$. Also, the set,
\begin{align*} \text{Orb}(x) = \{ gx : g \in G \},
\end{align*}
is called the \textbf{orbit} of $x$ in $G$.  
\end{definition} 

The orbit and the stabiliser of an element are closely related. The following theorem is a consequence of this relationship and it will be useful throughout this paper. \\

\begin{theorem} [Orbit-Stabilizer theorem]
    \textit{Let $G$ be a finite group acting on a set $X$. Then for each $x \in X$}, $$|G| = |\text{Orb}(x)| |\text{Stab}(x)|.$$ \\
\end{theorem}

The following standard theorem will all play a vital roll later on.

\begin{theorem}\label{symhomoker} Let $G$ be a group and $H$ a subgroup of $G$ of finite index $n$. Then there is a homomorphism $\phi : G \longrightarrow S_n$ such that,
\begin{align*} ker(\phi) = \bigcap\limits_{x \in G} x H x^{-1}.
\end{align*}
\end{theorem}

\begin{proof} See \cite[p.110]{bhattacharya} for proof.
\end{proof}

\section{Conjugation}

\begin{definition}[Conjugate elements]
\label{IsConj}
\lean{IsConj}
Let $G$ be a group and $a$ an element of $G$. An element $b \in G$ is said to be \textbf{conjugate} to $a$ if $b=xax^{-1}$ for some $x \in G$. \\
\end{definition}

\begin{remark}
\label{conj_elem}
In Lean, to state that two elements $g, h \in G$ where $G$ is a group, we use the slightly more general definition of conjugacy over monoids.

That is to say, given $g, h \in G$ where $G$ is a group (or more generally monoid) and impose that $g$ and $h$ are conjugate, instead of writing the equality which has type \texttt{Prop}:

\begin{verbatim}
∃ c : α, c * a * c⁻¹ = b
\end{verbatim}

We use the following statement of type \texttt{Prop} that has been defined in Mathlib under the name of \texttt{IsConj}.

The reason we would choose this over the naive statement is because Mathlib will contain a lot of very useful lemmas attached to this definition.

Saying two elements are conjugate is writing something like the following:

Assuming the terms \texttt{g : G} and \texttt{h : G} of the type \texttt{G} (which has the \texttt{Group} typeclass instance) are in scope.

\begin{verbatim}
IsConj g h
\end{verbatim}
\end{remark}


\begin{definition}[Conjugate subgroups]
Let $H_1$ be a proper subgroup of $G$ and fix $x \in G \setminus H_1$. The set $H_2 = \{g \in G : g= xh_1x^{-1}$, $\forall h_1 \in H_1\}$ is said to be a \textbf{conjugate subgroup} of $H_1$. We write $H_2 = xH_1x^{-1}$. It is trivial to show that $H_2$ is a subgroup of $G$.
\end{definition}

\begin{remark}
In Lean, to state that two subgroups $H, K$ of a group $G$ are conjugate subgroups similar to how is done in \ref{conj_elem} we can open the \texttt{MulAut} namespace to make use of the custom syntax:

\begin{verbatim}
conj c • H = K 
\end{verbatim}

This notation and API is useful because conjugation by a particular element is defined to be an element in the automorphism group of $G$, $\Aut(G)$. 

This becomes particularly crucial when formalizing the interactions of subgroups with the complete lattice structure on the set of subgroups of a group. 

These interactions and more discussion about this lattice structure will happen later on.
\end{remark}

Conjugation plays an important roll thoughout the paper, in particularly the following properties about conjugate elements and subgroups.

\begin{proposition}\label{conjugateprop} Let $a$, $b$ be conjugate elements of a group $G$ and $A$, $B$ be conjugate subgroups of $G$. Then the following properites hold: \vspace{3mm} \\
(i) If either $a$ or $b$ has finite order, then both $a$ and $b$ have the same order. \vspace{3mm} \\
\end{proposition}
\begin{proof}
    (i) Since $a$ and $b$ are conjugate elements in $G$, $b = xax^{-1}$ for some $x \in G$. Suppose that $b$ has finite order and $b^k = I_G$ for some $k \in \mathbb{Z}^+$,
    \begin{equation*} I_G = b^k = (xax^{-1})^k = xa^{k}x^{-1} \Rightarrow a^k = I_G.
    \end{equation*}
    Alternatively suppose that $a$ has finite order and $a^k = I_G$ for some $k \in \mathbb{Z}^+$,
    \begin{equation*} a^k = I_G \Rightarrow I_G = xa^{k}x^{-1} = (xax^{-1})^k = b^k.
    \end{equation*}
    Thus $a^k = I_G \iff b^k = I_G$. Thus $a$ and $b$ have the same order. \\
\end{proof}

\begin{proposition}
(ii) $A \cong B$. \\
\end{proposition}

\begin{proof}
\\
(ii) Since $A$ and $B$ are conjugate, there exists some $x \in G$ such that $B=xAx^{-1}$. Define the map $\phi$ by,
\begin{align*}
\phi:A &\longrightarrow xAx^{-1}, \\
a_1 &\longmapsto xa_1x^{-1} \tag{$\forall \; a_1 \in A$}. \end{align*}

We show that $\phi$ is a homomorphism between $A$ and $B=xAx^{-1}$.

\begin{equation*}
\phi(a_1a_2) = xa_1a_2x^{-1} = ( xa_1x^{-1})( xa_2x^{-1}) = \phi(a_1) \phi(a_2).
\end{equation*}
\\
Now consider an arbitrary $k \in ker(\phi)$.

\begin{equation*}
k \in ker(\phi) \iff \phi(k) = I_G \iff  xkx^{-1} = I_G \iff k = I_G.
\end{equation*}
\\
So $ker(\phi) = \{ I_G \}$ which means $\phi$ is injective. Now let $b_1 \in B = xAx^{-1}$. Thus $b_1 = xa_1x^{-1}$ for some $a_1 \in A$. Since $a_1 \in A$, $\phi(a_1) = xa_1x^{-1} = b_1$ and so $\phi$ is surjective. Thus $\phi$ is an isomorphism and $A$ and $B$ are isomorphic.

\end{proof}

The final part of this proposition is an important result which shows that since conjugate subgroups are isomorphic, conjugation preserves group structure and properties. In particular, conjugate subgroups have the same cardinality and if one is abelian or cyclic, then so is the other.

\section{Automorphism}

\begin{definition} An \textbf{automorphism} of a group $G$ is a isomorphism from $G$ onto itself. The set of all automorphisms of $G$ forms a group under composition and is denoted by $Aut(G)$.\\
\\
An \textbf{inner automorphism} is an automorphism whereby $G$ acts on itself by conjugation. That is, each $g \in G$ induces a map, $i_g : G \rightarrow G$, where $i_g(x) = g x g^{-1}$ for each $x \in G$. The set of all inner automorphisms is denoted by $Inn(G)$ and is a normal subgroup of $Aut(G)$ (For proof of this see \cite[p.104]{bhattacharya}.
\end{definition}

\section{Direct Product}

\begin{definition} If $G_1, G_2,...,G_n$ are groups, we define a coordinate operation on the Cartesian product $G_1 \times G_2 \times...\times G_n$ as follows:
\begin{align*} (a_1, a_2, ..., a_n) (b_1, b_2, ..., b_n) = (a_1 b_1, a_2 b_2, ..., a_n b_n),
\end{align*}
where $a_i, b_i \in G_i$. It is easy to verify that $G_1 \times G_2 \times...\times G_n$ is a group under this operation. This group is called the \textbf{direct product} of $G_1, G_2,...,G_n$.
\end{definition}

\begin{lemma} \label{directproductN} Let $A$ and $B$ be normal subgroups of $G$ with $A \cap B = \{ I_G \}$. Then $AB \cong A \times B$.
\end{lemma}

\begin{proof}

First note that the elements of $A$ commute with the elements of $B$, since $\forall \; a \in A$ and $b \in B$,
\begin{align*} aba^{-1}b^{-1} &=  a(ba^{-1}b^{-1}) \in A, \tag{since $A \vartriangleleft G$}
\\ aba^{-1}b^{-1} &=  (aba^{-1})b^{-1} \in B. \tag{since $B \vartriangleleft G$}
\end{align*}

Therefore $aba^{-1}b^{-1} \in A \cap B = \{ I_G \}$, and $ab = ba$. \\
\\
Define the operation $*$ on $A \times B$ by $(a_1 , b_1)*(a_2 , b_2) = (a_1 a_2 , b_1 b_2)$. Now define the map $\phi$ by,
\begin{align*}
\phi:A \times B &\longrightarrow AB, \\
(a,b) &\longmapsto ab \tag{$\forall \; a \in A, \; b\in B$}. \end{align*}

We show that $\phi$ is a homomorphism between $A \times B$ and $AB$.
\vspace{-0.5mm}
\begin{align*}
\phi((a_1,  b_1)*(a_2, b_2)) &= \phi (a_1 a_2 , b_1 b_2) \\
&=  a_1 a_2  b_1 b_2 \\
&=  a_1 b_1 a_2 b_2  \\
&= \phi(a_1 , b_1) \phi(a_2 , b_2). \end{align*}

Thus $\phi$ is a homomorphism and clearly surjective. It remains to show that it is injective. 
\vspace{-0.5mm}
\begin{align*} \phi(a_1 , b_1) &= \phi(a_2 , b_2), \\
a_1 b_1 &= a_2 b_2, \\
a_1 b_1 b_2^{-1} &= a_2, \\
b_1 b_2^{-1} &= a_1^{-1} a_2 \in A \cap B.
\end{align*}

Since $A \cap B = \{ I_G \}$, we have $b_1 b_2^{-1} = I_G = a_1^{-1} a_2$ and so $b_1 = b_2$, $a_1 = a_2$ and $\phi$ is injective. So $\phi$ is an isomorphism and $AB \cong A \times B$.
\\
\end{proof}

\begin{lemma}\label{directproductZ}
Let $A$ and $B$ be subgroups of $G$. If $A \cap B = \{ I_G \}$ and $ab = ba$ $\forall a \in A$, $b \in B$. Then $AB \cong A \times B$.
\end{lemma}

\begin{proof} Since $A$ and $B$ commute, the argument outlined in Lemma \ref{directproductN} also holds here.
\end{proof}

% \newpage



\chapter{Reduction of classification of finite subgroups of $\PGL_2(\Fbar_p)$ to classification of finite subgroups of $\PSL_2(\Fbar_p)$}\label{Ch4_ReductionOfProblem}

\section{Over an algebraically closed field $\PSL_n(F)$ is isomorphic to the projective $\PGL_n(F)$}


When $F$ is algebraically closed and $\textrm{char}(F) \neq 2$ it one can construct an isomorphism between 
the projective special linear group and the projective general linear group.

\begin{definition}
\label{SL_monoidHom_PGL}
\lean{SL_monoidHom_PGL}
\leanok
    Let $\varphi : \SL_n(R) \rightarrow \PGL_n(R)$ be the injection of $\PSL_n(R)$ into $\PGL_n(R)$ defined by
    \[
     S \mapsto i(S) \;  (R^\times I) 
    \]

    where $i : \SL_n(F) \hookrightarrow \GL_n(F)$ is the natural injection of the special linear group into the general linear group.
\end{definition}



We now prove a useful fact about elements that belong to the center of $\GL_n(R)$.

\begin{lemma}
    \label{GeneralLinearGroup.mem_center_general_linear_group_iff}
    \lean{GeneralLinearGroup.mem_center_general_linear_group_iff}
    \leanok
     Let $R$ be a commutative ring, then $G \in GL_n(F)$ belongs to center of $\GL_n(R)$, $Z(\GL_n(R))$ if and only if $G = r \cdot I$ where $r \in R^\times$.
    \end{lemma}
    
    \begin{proof}
        \leanok
        \begin{itemize}
        \item Suppose $G \in \GL_n(F)$ belongs to $Z(\GL_n(F))$ then for all $H \in \GL_n(F)$ we have that $G H = H G$. We will find it sufficient to only consider the case where $H$ is a transvection matrices.
        Let $1 \leq i < j \leq n$, then the transvection matrices are of the form $T_{ij} = I + E_{ij}$ where $E_{ij}$ is the standard basis matrix given by
        \[
        E_{{ij}_{kl}} = \begin{cases}
        1 & \text{if $i = k$ and $l = j$}\\
        0 & \text{otherwise}
        \end{cases}
        \] 
    
        Given $T_{ij} G = (I + E_{ij}) G = G T_{ij} (I + E_{ij})$, and addition is commutative we can use the cancellation law to yield that
        
        \[
        E_{ij} G = G E_{ij}
        \]
    
        But $G$ only commutes with $E_{ij}$ for all $i \neq j$ if $G = r \cdot I$ for some $r \in R^\times$.
        
        \item Suppose $G = r \cdot I$ for some $r \in R^\times$ then it is clear that for all $H \in \GL_n(F)$ that $r \cdot I  H = r \cdot H = H \cdot r = H (r \cdot I)$
        \end{itemize}
    \end{proof}


\begin{lemma}
\label{center_SL_le_ker}
\uses{SL_monoidHom_PGL}
\lean{center_SL_le_ker}
\leanok
Let $R$ be a non-trivial commutative ring, then $Z(\SL_n(R)) \le \ker (\varphi)$.
\end{lemma}
\begin{proof}
\uses{GeneralLinearGroup.mem_center_general_linear_group_iff}
\leanok
If $S \in Z(\SL_n(R)) \leq \SL_n(F)$ then $S = \omega I$ where $\omega$ is a primitive root of unity.

Because $\varphi = \pi_{Z(\GL_n(F))} \circ i$, the kernel of $\varphi$ is $i^{-1}(Z(\GL_n(F)))$, where we recall that $i : \SL_n(R) \hookrightarrow \GL_n(F)$ is the injection of $SL_n(F)$ into $\GL_n(F)$.

But given $i(S) = i(\omega \cdot I) = \omega \cdot I$ is of the form $r \cdot I$ where $r \in R^\times$ by \ref{GeneralLinearGroup.mem_center_general_linear_group_iff} it follows that $S \in \ker \varphi$, as desired.
\end{proof}




\begin{definition}
\label{PSL_monoidHom_PGL}
\uses{SL_monoidHom_PGL}
\lean{PSL_monoidHom_PGL}
\leanok
    Given $Z(\SL_n(F)) \leq \ker \varphi$ as shown in \ref{center_SL_le_ker}, by the universal property there exists a unique homomorphism $\bar{\varphi} : \PSL_n(F) \rightarrow \PGL_n(F)$ which is the lift of $\varphi$. 
    
    Where $\varphi = \bar{\varphi} \circ \pi_{Z(\SL_n(F))}$ and $\pi_{Z(\SL_n(F))} : \SL_n(F) \rightarrow \PSL_n(F)$ is the canonical homomorphism from the group into its quotient.
\end{definition}



\begin{lemma}
\label{Injective_PSL_monoidHom_PGL}
\lean{Injective_PSL_monoidHom_PGL}
\uses{PSL_monoidHom_PGL}
\leanok
    The homomorphism $\bar{\varphi}$ is injective.
\end{lemma}
\begin{proof}
\uses{GeneralLinearGroup.mem_center_general_linear_group_iff}
\leanok

To show $\bar{\varphi}$ is injective we must show that $\ker \bar{\varphi} \leq \bot_{\PSL_n(F)}$ where $\bot_{\PSL_n(F)}$ is the trivial subgroup of $\PSL_n(F)$.

Let $[S] \in \PSL_n(F)$ and suppose $[S] \in \ker \bar{\varphi}$. If $[S] \in \ker \bar{\varphi}$ then $\bar{\varphi} ([S]) = [1]_{\PGL_n(F)}$. But on the other hand, $\bar{\varphi} ([S]) = \varphi(s)$ and so $\varphi(S) = 1_{\PGL_n(F)}$, 
and thus $S \in Z(\GL_n(F))$, from \ref{GeneralLinearGroup.mem_center_general_linear_group_iff} it follows that $s = r \cdot I$ for some $r \in R^\times$. But given $S \in \SL_n(F)$ we know that 

\begin{equation*}
    \det(S) = \det(r \cdot I) = r^n \cdot 1 = 1 \implies \text{$r$ is a $n$th root of unity}
\end{equation*}

Therefore, given elements of $Z(\SL_n(F))$ are those matrices of the form $\omega \cdot I$ where $\omega$ is a $n$th root of unity, we can conclude that $[S] = [1]_{\PSL_n(F)}$ and thus $\ker \bar{\varphi} \leq \bot_{\PSL_n(F)}$ as required.

Which shows that the homomorphism $\bar{\varphi}$ is injective.
\end{proof}
    
    \begin{remark}[Quotients and their maps in Lean]
    When formalising results on quotient groups or for that matter any quotient type, it is valuable to appreciate which model Lean uses for quotients. 
    
    Typically, when one thinks of the elements of the quotient group say $\Z/2\Z$ there are two elements: 
    $[0]$ which represents the coset $\{\ldots, -2, 0, 2, \ldots\}$, and $[1]$ which represents the coset $\{\ldots, -3, -1, 1, 3, \ldots\}$
    since under the equivalence relation, $a \sim b$ if and only if $a - b \in 2\Z$. In this new setting all the elements belonging to the same coset, 
    or equivalence class, are now considered to be indistinguishable.

    Similarly, when defining a group homomorphism from $\theta : (\Z /2\Z, \dot{+}) \rightarrow G$, under this model one has to make sure that
    all the elements in $[0]$ are sent to the same element $g \in G$ in the target; and likewise, all the elements of $[1]$ are sent to the same element $h \in H$.

    Otherwise, should $\theta([0]) \ne \theta([2])$ then this would mean that $\theta$ would be treating what we thought were the indistinguishable elements $0$ and $2$, as different,
    This is the idea of showing the \textit{well-definedness} of a map on a quotient.

    In general, one of the ideas of quotients (not only group quotients) is to somehow eliminate redundant information.

    Let us run with the following amusing example in every day life:
    
    Suppose the lights in a room are on, and suppose Bob asks Alice what would happen should he press the light switch $n \in \N$ times. It then occurs to
    Alice that in this setting pressing the light switch on $12$ times or $1400$ times, or for that matter, any even number of times yields indistinguishable outcomes,
     the lights will be on; so in a sense the elements belonging to the set of even numbers are indistinguishable from each other, 
     what is more is that we are not interested in so much the number but the parity of the number.

    In this particular example, we defined a map $\psi : \N \rightarrow \{\text{On}, \; \text{Off}\}$ where we realised
    that both $12$ and $1400$ and all even natural numbers seem to behave equivalently under this map if and only if their difference is an even integer,
    that is, $a \sim b$ if and only if $a - b \in 2\Z$ where $a$ and $b$ are promoted to being elements of $\N \subset \Z$.    
    
    Given this map $\psi$ behaves the same on all elements which are indistinguishable, it seems natural to 
    define a map which now takes in the only relevant information which determines if the lights are on or not,
    the parity of the number of times the light switch has been pressed.
    
    \[
    \bar{\psi} : \N / \sim \rightarrow \{\text{On}, \; \text{Off}\}
    \]

    The quotient on $\N$ now allows us to say treat $[12] = [1400]$, as they are equal sets, and our new map $\bar{\psi}$ now recognises them to be the same
    under this new light. 

    However, we could have also phrased this observation as saying that $\psi$ respects the equivalence relation on $\sim$ and
    thus have defined a $\bar{\psi}$ to be the map which given the parity, an element of the new abstract object $\N / \sim = \{[0], [1]\}$,
    outputs whether the light is on or off. 

    The upshot of all of this is that when we define a quotient and a map from a quotient, we ultimately want such a function to respect the equivalence relation. Whether the elements of a quotient are
    modelled as a coset, a set of equivalent elements, or as an abstract object which satisfies our needs should not be the main concern. 

    This is what the definition of quotients in Lean recognises, but it also recognises that it would be rather strange to think of a term of a quotient type as a set, since it would be clunky to  constantly work with
    the type \texttt{Set (Set s)}, to define a quotient;  instead one simply modifies the definition of equality on terms, and in particular, when wanting to define in Lean the lift of an existing homomorphism $\gamma : G \rightarrow H$
    to $\bar{\gamma} : G/N \rightarrow H$, the most natural way to define/verify such a lift is sensible in Lean is to prove that equivalent elements map to the same output under $\gamma$.

    In fact, this is exactly what the general \texttt{Quot.lift} does:

 

    Similarly, \texttt{QuotientGroup.lift}, the universal property for factor groups, corresponds to:

    

    From this last definition one can see that there is no trace whatsoever to cosets. It is still possible to formalise a such a
    definition in a way that is akin to the notion of well-definedness, which is closer to the model of quotients as sets of subsets,
    since one can for example invoke \texttt{Quotient.exists_rep} which states:

   
    
    In fact, some of the theorems and definitions below heavily rely on this notion.
    Yet it becomes extremely useful later on to come to terms with this model of quotients and their maps which
    hinges on the universal property.
    \end{remark}

Before we can show that $\bar{\varphi}$ is surjective we need the following
lemma which allows us to find a suitable representative for an arbitrary element of $\PGL_n(F)$.

\begin{lemma}
\label{exists_SL_eq_scaled_GL_of_IsAlgClosed}
\lean{exists_SL_eq_scaled_GL_of_IsAlgClosed}
\leanok
If $F$ is an algebraically closed field then for every $G \in \GL_n(F)$ there exists a nonzero constant $\alpha \in F^\times$ and an element $S \in \SL_n(F)$ such that 
\begin{equation*}
    G = \alpha \cdot S
\end{equation*}
\end{lemma}

\begin{proof}
\leanok
Let $G \in \GL_n(R)$ then define
\begin{equation*}
    P(X) := X^n - \det(G)
\end{equation*}

By assumption $F$ is algebraically closed and $\det(G) \in F^\times$ thus there exists a root $\alpha \in F^\times$ such that 

\begin{equation*}
    \alpha^n - \det(G) = 0 \iff \alpha = \sqrt[n]{\det(G)} 
\end{equation*}

Let $S = \frac{1}{\alpha} \cdot G$, by construction $S \in \SL_n(F)$ as 

\begin{equation*}
    \det(S) = \left(\frac{1}{\alpha^n}\right) \cdot \det(G) = \frac{1}{\det(G)} \det(G) = 1
\end{equation*}
\end{proof}


\begin{lemma}
\label{Surjective_PSL_monoidHom_PGL}
\uses{PSL_monoidHom_PGL}
\lean{Surjective_PSL_monoidHom_PGL}
\leanok
    The map $\bar{\varphi}$ is surjective.
\end{lemma}
\begin{proof}
\uses{exists_SL_eq_scaled_GL_of_IsAlgClosed}
\leanok
    Let $G \; (F^\times I) = [G] \in \PGL_n(F)$, then $G \in \GL_n(F)$ we can find a representative of $[G']$ that lies within the special linear group.
    Given elements of the special linear group are matrices with determinant equal to one, we must scale $G$ to a suitable factor to yield a representative which lies within $\SL_n(F)$. Suppose $\det(G) \ne 1$ and let
    \[
    P(X) := X^n - \det(G) \in F[X]
    \]
    By assumption, $F$ is algebraically closed so there exists a root $\alpha \ne 0\in F$ such that 
    \[
    \alpha^n - \det(G) = 0 \iff \alpha^n = \det(G)
    \]
    We can define
    \[
    G' := \frac{1}{\alpha} \cdot G \quad \text{where} \quad \det(G') = \frac{1}{\alpha^n} \det(G) = 1.
    \]
    Thus $G' \in \SL_n(F) \leq \GL_n(F)$ and given $G' = \frac{1}{\alpha} G$ we have that $G'  \; (F^\times I) = G \; (F^\times I)$.
    
    Therefore, $\varphi(G') = i(G') (F^\times I) = G' (F^\times I) = G (F^\times I)$.
\end{proof}


\begin{lemma}
\label{Bijective_PSL_monoidHom_PGL}
\uses{PSL_monoidHom_PGL}
\lean{Bijective_PSL_monoidHom_PGL}
\leanok
    The map $\bar{\varphi}$ is bijective
\end{lemma}
\begin{proof}
\uses{Injective_PSL_monoidHom_PGL, Surjective_PSL_monoidHom_PGL}
\leanok
 We have shown that $\bar{\varphi}$ is injective in \ref{Injective_PSL_monoidHom_PGL} and have shown that $\bar{\varphi}$ is surjective in \ref{Surjective_PSL_monoidHom_PGL}. 
 Therefore, $\bar{\varphi}$ defines a bijection from $\PSL_n(F)$ to $\PGL_n(F)$.
\end{proof}


\begin{theorem}
\label{PGL_iso_PSL}
\uses{PSL_monoidHom_PGL}
\lean{PGL_iso_PSL}
\leanok
    If $F$ is an algebraically closed field, then the map $\bar{\varphi} : \PSL_n(F) \rightarrow \PGL_n(F)$ defines a group isomorphism between $\PSL_n(F)$ and $\PGL_n(F)$.
\end{theorem}

\begin{proof}
\uses{Bijective_PSL_monoidHom_PGL}
\leanok
    The map $\bar{\varphi}$ was shown to be a bijection in \ref{Bijective_PSL_monoidHom_PGL} and given $\bar{\varphi}$ is mulitplicative as it was defined to be the lift of the homomorphism $\varphi$, we can conclude that 
    $\bar{\varphi}$ defines a group isomorphism between $\PSL_n(F)$ and $ºPGL_n(F)$
\end{proof}


\begin{remark}[Noncomputable]
    Observe in the definition above it was necessary to add the \texttt{noncomputable} keyword before the definition, the reason for this is
    because the result \texttt{MulEquiv.ofBijective} implicitly uses the axiom of choice which means it is not possible for Lean to generate
    executable code.
\end{remark}



% \begin{center}
% \begin{tikzcd}
% 	{\SL_n(F)} && {\SL_n(F)} \\
% 	&& {} \\
% 	{\PSL_n(F)} && {\PGL_n(F)}
% 	\arrow["i", from=1-1, to=1-3]
% 	\arrow["{\textrm{can}_{\langle-I\rangle}}"', from=1-1, to=3-1]
% 	\arrow["{\textrm{can}_{F^\times I}}", from=1-3, to=3-3]
% 	\arrow[dotted, from=3-1, to=3-3]
% \end{tikzcd}
% \end{center}
% \end{proof}

This isomorphism will be essential to the classification of finite subgroups of $\PGL_2(\bar{\F}_p)$, as we only need understand a the classification of subgroups of $\PSL_2(\Fbar_p)$ to reach the desired result.


\section{Christopher Butler's exposition}

Following from the isomorphism defined in the previous section, we can now proceed to classify the finite subgroups of $\PGL_2(\bar{\F}_p)$ by classifying the finite subgroups of $\PSL_2(\bar{\F}_p)$. 
In turn, one can begin classifying the finite subgroups of $\PSL_2(\Fbar_p)$ by classifying the finite subgroups of $\SL_2(\Fbar_p)$ and then considering what happens after
quotienting by the center, $Z(\SL_2(F)) = \langle -I\rangle$.

We now turn our attention to the more general setting when $F$ is an arbitrary field that is algebraically closed, as this will turn out to be sufficient for our purposes.

Given $|\langle -I \rangle| = 2$ when $\textrm{char} F \ne 2$; and $\langle -I\rangle = \bot$ when $\textrm{char} F = 2$.
When a finite subgroup of $\SL_2(F)$ is sent through the canonical mapping $\pi_{Z(\SL_2(F))} : \SL_2(F) \rightarrow \PSL_2(F)$ 
the resulting subgroup will either shrink by a factor of two or it will remain intact should the center not be contained within the subgroup. 

We now proceed to classify all finite subgroups of $\SL_2(F)$ when $F$ is algebraically closed field. 
From now on, we follow Christopher Butler's exposition of Dickson's classification of finite subgroups of $\SL_2(F)$ over an algebraically closed field $F$. 

Christopher has been kind enough to provide the TeX code so I could prepare this blueprint which crucially hinges on the result which his exposition \cite{butler} covers.
\chapter{Properties of the two dimensional $\SL_2(F)$}\label{Ch5_PropertiesOfSLOverAlgClosedField}


\section{General Notation}

Throughout this paper, $F$ will denote an arbitrary algebraically closed field. 
The letter $p$ will be used to denote the characteristic of $F$. 
Recall that the definition of the characteristic of a field is:

\begin{definition}[Characteristic of a field]
    Let $F$ be a field, the characteristic of a field, denoted by $\textrm{char}(F) \in \N$, is the smallest natural number $p \in \N_0$ such that

    \[
    \underbrace{1 + \ldots + 1}_{p} = 0
    \]

    where in the case there is no such number then $p = 0$.
\end{definition}

\begin{example}
    $\Z /p\Z$ is a field of characteristic $\textrm{char}(\Z/p\Z) = p$ as $p \cdot 1 = 0$.
\end{example}

\begin{example}
    The field $\Q$ is a field with $\textrm{char}(\Q) = 0$ as $n \cdot 1 \ne 0$ for all $n \in \N \subset \Q$.
\end{example}

\begin{remark}[The characteristic is either prime or zero]
    The characteristic of a field is either a prime number or zero.
\end{remark}

Unless otherwise stated, the letters $\alpha, \beta, \gamma, \delta$ and $\sigma$ will denote elements of $F$; 
whereas $\delta$ and $\rho$ will denote elements of $F^\times$, where $F^\times$ are the invertible, or equivalently, non-zero elements of $F$.

\section{Subsets of $\SL_2(F)$}

In this chapter we make some useful observations about specific elements and subgroups of $\SL_2(F)$. 

First, we define the following elements of $\SL_2(F)$.

\subsubsection{Special matrices of $\SL_2(F)$}

\begin{definition}[The diagonal matrix of $SL_2(F)$]
\label{SpecialMatrices.d}
\lean{SpecialMatrices.d}
\leanok
    Given an element $\delta \in F^\times$ we define the diagonal matrix:
    \[
    d_\delta = \begin{bmatrix}
        \delta & 0\\
        0 & \delta^{-1}
    \end{bmatrix}
    \]
\end{definition}


\begin{remark}[Constructing a term of $\SL_2(F)$]
    To construct a term of $\SL_2(F)$ one has to bear in mind that the special linear group is defined to
    be a subtype of matrices with determinant one, thus, in the \textit{anonymous constructor}
    one has to provide:

    \begin{itemize}
        \item The term of type \texttt{Matrix (Fin 2) (Fin 2) F}.
        \item The proof term that proves that the matrix term of type \texttt{Matrix (Fin 2) (Fin 2) F} has determinant one.
    \end{itemize}
\end{remark}

\begin{definition}[The shear matrix of $SL_2(F)$]
\label{SpecialMatrices.s}
\lean{SpecialMatrices.s}
\leanok
    Given an element $\delta \in F$ we define the shear matrix:
    \[
    s_\sigma  = \begin{bmatrix}
    1 & 0\\
    \sigma & 1
    \end{bmatrix}
    \]
\end{definition}


\begin{definition}[Rotation by $\pi / 2$ radians matrix]
\label{SpecialMatrices.w}
\lean{SpecialMatrices.w}
\leanok
 We denote the matrix which corresponds to a rotation by $\pi / 2$ radians to be:
 \[
 w = \begin{bmatrix}
    0 & -1\\
    1 & 0
 \end{bmatrix}
 \]
\end{definition}


The matrices $d$, $s$ and $w$ satisfy the following relations:


\begin{lemma}[Closure of $D$ under multiplication]
\label{SpecialMatrices.d_mul_d_eq_d_mul}
\uses{SpecialMatrices.d}
\lean{SpecialMatrices.d_mul_d_eq_d_mul}
\leanok
For any $\delta, \rho \in F^\times$ we have that
\[
d_\delta d_\rho = d_{\delta\rho}
\]
\end{lemma}
\begin{proof}
\leanok
    We verify by matrix multiplication that indeed:

    \begin{equation*}
        d_\delta d_\rho = \begin{bmatrix} \delta & 0 \\ 0 & \delta^{-1} \end{bmatrix} \begin{bmatrix} \rho & 0 \\ 0 & \rho^{-1} \end{bmatrix} = 
        \begin{bmatrix} \delta \rho & 0 \\ 0 & \delta^{-1} \rho^{-1} \end{bmatrix} = d_{\delta \rho}.
    \end{equation*}
\end{proof}


\begin{lemma}[Closure of $S$ under multiplication]
\label{SpecialMatrices.s_mul_s_eq_s_add}
\uses{SpecialMatrices.s}
\lean{SpecialMatrices.s_mul_s_eq_s_add}
\leanok
    For any $\sigma, \gamma \in F$ we have that
    \[
    s_\sigma s_\gamma = s_{\sigma + \gamma}.
    \]
\end{lemma}
\begin{proof}
\leanok
    We verify by matrix multiplication that indeed:
\begin{equation*}
    s_\sigma s_\gamma = \begin{bmatrix} 1 & 0 \\ \sigma & 1 \end{bmatrix} \begin{bmatrix} 1 & 0 \\ \gamma & 1 \end{bmatrix} = \begin{bmatrix} 1 & 0 \\ \sigma + \gamma & 1 \end{bmatrix} = s_{\sigma + \gamma}.
\end{equation*}
\end{proof}


\begin{lemma}
    \label{SpecialMatrices.s_pow_eq_s_mul}
    \uses{SpecialMatrices.s}
    \lean{SpecialMatrices.s_pow_eq_s_mul}
    \leanok
    For any $\sigma \in F$ and for any $n \in \N$, we have that $s_\sigma^n = s_{n \cdot \sigma}$
\end{lemma}
\begin{proof}
\uses{SpecialMatrices.s_mul_s_eq_s_add}
\leanok
    We prove this by induction, indeed for $n= 0$ the identity holds trivially.

    Suppose $s_\sigma^n = \begin{bmatrix}
        1 & 0\\
        n \cdot \sigma & 0\end{bmatrix}$ then consider $s_\sigma^{(n + 1)}$. Since 

        \[
        s_\sigma^{(n + 1)} = s_\sigma^n s_\sigma = s_{n \cdot \sigma} s_\sigma = s_{(n + 1)\sigma}
        \]
\end{proof}


\begin{lemma}[Order of nontrivial $s_\sigma$ ]
    \label{SpecialMatrices.order_s_eq_char}
    \uses{SpecialMatrices.s}
    \lean{SpecialMatrices.order_s_eq_char}
    \leanok
    The order of $s_\sigma$ for any $\sigma \ne 0$ is $\textrm{char}(F)$
    \end{lemma}
    
    \begin{proof}
    \uses{SpecialMatrices.s_pow_eq_s_mul}
    \leanok
    Let $p$ denote the characteristic of the field, and let $\sigma \in F$, by \ref{SpecialMatrices.s_pow_eq_s_mul} we know that for any $s_\sigma^p = s_{p \cdot \sigma}$. 
    Since $p$ is the characteristic of the field, we have that $p \cdot \sigma = 0$, and so $s_{p \cdot \sigma} = s_0 = I$
    \end{proof}


\begin{lemma}
\label{SpecialMatrices.d_mul_s_mul_d_inv_eq_s}
\uses{SpecialMatrices.d, SpecialMatrices.s}
\lean{SpecialMatrices.d_mul_s_mul_d_inv_eq_s}
\leanok
    We have that for all $\delta \in F^\times$ and $\sigma \in F$
    \[
    d_\delta s_\sigma d^{-1}_\delta = s_{\sigma \delta^{-2}}.
    \]
\end{lemma}
\begin{proof}
\leanok
    We verify by matrix multiplication that indeed:

    \begin{equation*}
        d_\delta s_\sigma d^{-1}_\delta = \! \begin{bmatrix} \delta & 0 \\ 0 & \delta^{-1} \end{bmatrix} \begin{bmatrix} 1 & 0 \\ \sigma & 1 \end{bmatrix} \begin{bmatrix} \delta^{-1} & 0 \\ 0 & \delta \end{bmatrix} = \begin{bmatrix} \delta & 0 \\ 0 & \delta^{-1} \end{bmatrix} \! \begin{bmatrix} \delta^{-1} & 0 \\ \sigma \delta^{-1} & \delta \end{bmatrix} \! = \! \begin{bmatrix} 1 & 0 \\ \sigma \delta^{-2} & 1 \end{bmatrix} \! = s_{\sigma \delta^{-2}}.
    \end{equation*}
\end{proof}




\begin{lemma}
\label{SpecialMatrices.w_mul_d_eq_d_inv_w}
\uses{SpecialMatrices.d, SpecialMatrices.w}
\lean{SpecialMatrices.w_mul_d_eq_d_inv_w}
\leanok
For any $\delta \in F^\times$ we have:
\[ 
w d_\delta w^{-1} = d^{-1}_\delta.
\]
\end{lemma}
\begin{proof} 
\leanok
We verify by matrix multiplication that indeed
\begin{align*}
w d_\delta w^{-1} &= \begin{bmatrix} 0 & 1 \\ - 1 & 0 \end{bmatrix} \begin{bmatrix} \delta & 0 \\ 0 & \delta^{-1} \end{bmatrix} \begin{bmatrix} 0 & - 1 \\ 1 & 0 \end{bmatrix}\\
&=  \begin{bmatrix} 0 & 1 \\ - 1 & 0 \end{bmatrix} \begin{bmatrix} 0 & - \delta \\ \delta^{-1} & 0 \end{bmatrix}\\
&= \! \begin{bmatrix} \delta^{-1} & 0 \\ 0 & \delta \end{bmatrix} \!= d^{-1}_\delta. 
\end{align*}
\end{proof}


We can now express familiar kinds of matrices of $\SL_2(F)$ in terms of these three matrices:

First we note the following observations:
\begin{corollary}
    \label{det_eq_mul_diag_of_lower_triangular}
    \lean{det_eq_mul_diag_of_lower_triangular}
    \leanok
    The determinant of a $2 \times 2$ lower triangular matrix, $M$, is the product of the diagonal entries $\det(M) = M_{11} M_{22}$.
\end{corollary}
\begin{proof}
\leanok
We use the $2 \times 2$ determinant formula.
\end{proof}


\begin{corollary}
    \label{SpecialLinearGroup.fin_two_diagonal_iff}
    \lean{SpecialLinearGroup.fin_two_diagonal_iff}
    \leanok
    A $2 \times 2$ matrix of $\SL_2(F)$, $x$ is a diagonal matrix if and only if $x = d_\delta$ for some $\delta \in F^\times$.
\end{corollary}
\begin{proof}
\uses{det_eq_mul_diag_of_lower_triangular}
\leanok
    Since $x$ is diagonal and belongs to the special linear group, the determinant is $x_{11} x_{22} = 1$ which shows $x_{11} = x_{22}^{-1}$, as required.
\end{proof}


\begin{corollary}
    \label{SpecialLinearGroup.fin_two_shear_iff}
    \uses{SpecialMatrices.s, det_eq_mul_diag_of_lower_triangular}
    \lean{SpecialLinearGroup.fin_two_shear_iff}
    \leanok
    A matrix of $\SL_2(F)$, $x$ is a shear matrix, that is of the form $\begin{bmatrix}
        \alpha & 0\\
        \sigma & \alpha
    \end{bmatrix}$ if and only if either $x = s_\sigma$ or $x = - s_\sigma$ for some $\sigma \in F$.
\end{corollary}
\begin{proof}
\leanok
Again using the formula for the determinant of a $2 \times 2$ matrix to show that indeed if $x$ is a shear matrix in the special linear group 
then $\alpha^2 = 1$ which shows $\alpha = \pm 1$, as required.
\end{proof}


\begin{corollary}
    \label{SpecialLinearGroup.fin_two_antidiagonal_iff}
    \lean{SpecialLinearGroup.fin_two_antidiagonal_iff}
    \leanok
    A matrix $A \in \SL_2(F)$ is anti-diagonal, that is of the form $\begin{bmatrix}
        0 & \beta\\
        \gamma & 0
    \end{bmatrix}$ if and only if $A = d_\delta w$
\end{corollary}
\begin{proof}
\leanok
This is shown by direct computation, we observe that $w$ flips the rows and changes the sign of one the flipped rows to account for the determinant needing to be equal to one.
\end{proof}


From these relations we can now single out the following subgroups of $\SL_2(F)$.

\subsubsection{Special subgroups of $\SL_2(F)$}

\begin{definition}[The subgroup of diagonal matrices]
\label{SpecialSubgroups.D}
\lean{SpecialSubgroups.D}
\leanok
    The set of diagonal matrices with matrix multiplication is a subgroup of $\SL_2(F)$: 
    \[
    D = \{d_\delta \; | \; \delta \in F^\times \} = \left\{ \begin{bmatrix}\delta & 0\\ 0 & \delta^{-1}\end{bmatrix} \; | \; \delta \in F^\times \right\}
    \]
\end{definition}


\begin{definition}[The subgroup of shear matrices]
\label{SpecialSubgroups.S}
\lean{SpecialSubgroups.S}
\leanok
    The set of shear matrices with matrix multiplication is a subgroup of $\SL_2(F)$:
    \[
    S = \{s_\sigma \; | \sigma \in F\} = \left\{\begin{bmatrix}1 & 0\\ \sigma & 1\end{bmatrix} \; | \; \sigma \in F \right\}
    \]
\end{definition}


\begin{definition}[The subgroup of lower triangular matrices]
\label{SpecialSubgroups.L}
\lean{SpecialSubgroups.L}
\leanok
    The set of lower triangular matrices (see below) with matrix multiplication is a subgroup of $\SL_2(F)$
    \[
    L = DS
    \]
    where $DS = \{d_\delta s_\sigma \; | \; \delta \in F^\times \text{ and } \sigma \in F \}$ is the pointwise product of $D$ and $S$.
\end{definition}


\begin{definition}[The subgroup of containing diagonal and antidigonal matrices]
    \label{SpecialSubgroups.DW}
    \lean{SpecialSubgroups.DW}
    \leanok
    The set of all diagonal and anti-diagonal matrices with matrix multiplication is a subgroup of $\SL_2(F)$

    \begin{equation}\label{antidiag} DW = \langle D, w\rangle  = \{d_\delta \} \cup \{ d_\delta w \} 
        % =  \left\{  \begin{bmatrix} \delta & 0 \\ 0 & \delta^{-1} \end{bmatrix} \begin{bmatrix} 0 & 1 \\ -1 & 0 \end{bmatrix} \right\} 
        % = \left\{ \begin{bmatrix} 0 & \delta \\ -\delta^{-1} & 0 \end{bmatrix}  \right\}. 
    \end{equation}
\end{definition}


\begin{remark}
    It is possible to have specified the subgroup $DW$ in \ref{SpecialSubgroups.DW} as the supremum $D \sqcup \langle w \rangle$
    but then it would require some additional work to show that the underlying set is indeed $D \cup Dw$.
\end{remark}


\begin{corollary}
    \label{mem_L_iff_lower_triangular}
    \uses{SpecialMatrices.d, SpecialMatrices.s}
    \lean{mem_L_iff_lower_triangular}
    \leanok
    The subgroup $L \le \SL_2(F)$ is the subgroup of $2 \times 2$ lower triangular matrices with determinant one, $L =\left\{\begin{bmatrix}
    \alpha & 0\\
    \gamma & \delta
    \end{bmatrix} \; | \; \alpha, \gamma, \delta \in F \text{ and } \alpha \delta = 1 \right\}$.
\end{corollary}
\begin{proof}
\leanok
    Observe that for every $l \in L$ there is some $\delta \in F^\times$ and $\sigma \in F$ such that $l  = d_\delta s_\sigma = \begin{bmatrix}
        \delta & 0\\
        \sigma * \delta^{-1} & \delta^{-1}
    \end{bmatrix}$ which is lower triangular. 
    
    Furthermore, for every lower triangular matrix $L = \begin{bmatrix}
        a & 0\\
        b & c
    \end{bmatrix}$ 
    
    Setting $\delta = a \in F^\times$ as $a d = 1$ and setting $\sigma = a c$

    indeed yields the equality

    \[
    d_\delta s_\sigma = \begin{bmatrix}
        a & 0\\
        c & d
    \end{bmatrix}
    \]

    Thus $L = D S$ is the set of lower triangular matrices.
\end{proof}


\begin{remark}
    To define the subgroups $D$, $S$ and $L$ in Lean. 
    
    One has to:
    
    \begin{enumerate}
        \item Specify what the underlying set is, what is called the \texttt{carrier}.
        \item Prove that the set is closed under multiplication, that is, provide a proof term to the field \texttt{mul\textunderscore mem'}.
        \item Prove that the set contains the identity element of the group, that is, provide a proof term to the field \texttt{one\textunderscore mem'}.
        \item Show that the group is closed under the inversion operator $(-)^{-1}$, \texttt{inv\textunderscore mem'}.
    \end{enumerate}

    Once these four fields have been filled in, one has succesfully defined a subgroup in Lean.
\end{remark}


\begin{remark}
    Despite the definition of $L$ as being $D S$, some work has to be shown that indeed $DS = D \sqcup S$.
    
    If either $D$ or $S$ were normal in $\SL_2(F)$, this fact would be immediate as we would be able to use \texttt{mul\_normal} or \texttt{normal\_mul}:
    
   
    

    However, given neither $D$ or $S$ are normal in $\SL_2(F)$ slightly more work is needed to show this.
    
    It is interesting how Lean really forces either increased understanding or increased frustration.
\end{remark}

These elements and subgroups are fundamental to this paper and the notation will be used throughout.

\begin{definition}[$(D, \cdot) \cong (F^\times, \cdot)$]
\label{SpecialSubgroups.D_iso_units}
\uses{SpecialSubgroups.D, SpecialMatrices.d_mul_d_eq_d_mul}
\lean{SpecialSubgroups.D_iso_units}
\leanok
The map $\psi : F^\times \overset{\sim}{\rightarrow} D$ defined by $\delta \mapsto d_\delta$ defines a group isomorphism.
\end{definition}

\begin{proof}
    \leanok
    The function $\psi: F^\times \rightarrow D$ defined by $\psi(\delta) = d_\delta$ is a homomorphism between the group $F^\times$ under normal multiplication and $D$ under normal matrix multiplication:
\begin{align*} 
  \psi(\delta \rho) = d_{\delta \rho} =  d_\delta d_\rho = \psi(\delta) \psi(\rho). 
\end{align*}
Observe that $\psi$ is trivially injective and surjective and thus an isomorphism. So $D\cong F^\times$ and $D$ is a subgroup of $L$.\\
\end{proof}




\begin{definition}[ $(S, \cdot) \cong (F, +)$ ]
\label{SpecialSubgroups.S_iso_F}
\uses{SpecialSubgroups.S}
\lean{SpecialSubgroups.S_iso_F}
\leanok
    The map $\phi : F \overset{\sim}{\rightarrow} S$ defined by $\sigma \mapsto s_\sigma$ defines a group isomorphism.
\end{definition}

\begin{proof}
\uses{SpecialMatrices.s_mul_s_eq_s_add}
\leanok
     The function $\phi: F \rightarrow T$ defined by $\phi(\sigma) = s_\sigma$ is a homomorphism between the group $F$ under addition and $S$ under normal matrix multiplication:
\begin{align*} \phi(\sigma + \gamma) = s_{\sigma + \gamma} = s_\sigma s_\gamma = \phi(\sigma) \phi(\gamma).
\end{align*}
It is clear that $\phi$ is injective and surjective and thus an isomorphism. So $ S \cong F$ and $S$ is a subgroup of $L$. \\
\end{proof}


\begin{remark}[Multiplicative]
    Putting the keyword \texttt{Multiplicative} in front a structure which carries an additive structure
    it creates a copy of the additive structure and carries it over to be defined as a multiplicative structure
    on the type. 
\end{remark}


\begin{lemma}
\label{SpecialSubgroups.normal_S_subgroupOf_L}
\lean{SpecialSubgroups.normal_S_subgroupOf_L}
\leanok
$S$ is a normal subgroup of $L$
\end{lemma}
\begin{proof}
    \leanok
    Let $s_\gamma$ and $d_\delta s_\sigma$ be arbitrary elements of $S$ and $L$ respectively. Conjugating $s_\gamma$ by $d_\delta s_\sigma$ gives,
\begin{align*} (d_\delta s_\sigma) s_\gamma (d_\delta s_\sigma)^{-1} &= (d_\delta s_\sigma) s_\gamma (s^{-1}_\sigma d^{-1}_\delta) \\[1.5ex]
&=
d_\delta (s_\sigma s_\gamma s_{-\sigma}) d^{-1}_\delta \qquad \tag{since $s^{-1}_\sigma=s_{-\sigma}$} \\[1.5ex] 
&=
d_\delta s_\gamma d^{-1}_\delta \\[1.5ex] 
&= s_{\gamma \delta^{-2}} \in S. 
\end{align*}
Since $s_\gamma$ was chosen arbitrarily from $\SL_2(F)$we have ($d_\delta s_\sigma) S (d_\delta s_\sigma)^{-1} = S$ and since $d_\delta s_\sigma$ was chosen arbitrarily from $L$, we have that $S \vartriangleleft L$. \\
\end{proof}


\begin{remark}[Subgroups of subgroups in Lean]
    \label{lattice}
    In Lean, $S$ is considered to be a subgroup of $\SL_2(F)$, yet it it is also a subgroup of $L$. 
    
    It is fairly easy to see that $S \not\lhd \SL_2(F)$, so when we say that $S \lhd L$, 
    we are implicitly restricting $S$ to be a subset of $L$ and thus we are actually thinking about the subgroup $S \cap L$,
    but in fact this does not change anything because $S = S \sqcap L$ as $S \le \SL_2(F)$.

    Informally we do not think twice about this, but when formalising this we do need to be clear which is the ambient group for $S$
    to be normal and for $S$ to be normal in an ambient group, it must be considered to be a subgroup of $L$, rather than $\SL_2(F)$.
    
    So this is why the informal statement corresponds to the formal statement:

    

    This example highlights how as useful as it is that Lean keeps track of what the ambient groups are, it can be tedious to change the perspective from which we view the object, where in this case we restricted
    a subgroup to be a subgroup of another subgroup that contains it. One of the challenges of Lean is becoming comfortable with these \textit{coercion}.
    
    On the positive side, the automation Lean offers, that is, the tactics and the unification algorithm (the algorithm which allows you to substitute equal terms when say you use the \texttt{rw} tactic) are continually being refined, 
    and it is increasingly able to do a lot of this bookkeeping on without human aid.
\end{remark}


\begin{lemma}
\label{SpecialSubgroups.D_join_S_quot_S_subgroupOf_D_join_S_mulEquiv_D_subgroupOf_D_join_S}
\uses{SpecialSubgroups.D, SpecialSubgroups.S}
\lean{SpecialSubgroups.D_join_S_quot_S_subgroupOf_D_join_S_mulEquiv_D_subgroupOf_D_join_S}
\leanok
    $L / S \cong D$.
\end{lemma}
\begin{proof} 
\uses{SpecialSubgroups.normal_S_subgroupOf_L}
\leanok
The function $\pi: L \rightarrow D$ defined by $\pi(d_\delta s_\sigma) = d_\delta$ is a homomorphism between $L$ under normal matrix multiplication and $D$ under normal matrix multiplication:
\begin{align*} \pi(d_\delta s_\sigma d_\rho s_\gamma) &= \pi(d_\delta d_\rho s_\sigma s_\gamma) \tag{where $\sigma = \sigma \rho^{2}$}
\\ &= d_\delta d_\rho
\\ &= \pi(d_\delta s_\sigma)\pi(d_\rho s_\gamma).
\end{align*}

We see that $\pi$ is trivially surjective and has kernel
\begin{align*}  \ker(\pi) &= \{ d_\delta s_\sigma \in L : \pi(d_\delta s_\sigma) = I_{\SL_2(F)}\} = S.
\end{align*}
Thus by the First Isomorphism Theorem,
\begin{align*} L / \ker(\pi) &\cong \text{Im}(\pi), \\
L / &\cong D.
\end{align*}
\end{proof}



\begin{remark}
Interestingly, this proof was quite hard to formalise for reasons I will expand on below, but first let me introduce some ideas.

There are two complete lattice structures at play here:
\begin{enumerate}
  \item One where the top element is $\top = \SL_2(F)$
  \item Another, where the top element is $\top = D \sqcup S$, this lattice is a sublattice of the first one.
\end{enumerate}  

The second sublattice is crucial because we need $S$ to be normal in an ambient group, and clearly $S \not\lhd \SL_2(F)$; therefore when restricting $S$ to begin a
subgroup of $D \sqcup S = L$. 

Given $S$ is a subgroup of $D \sqcup S = L$ since $S \le S\sqcup D = L$ and by \ref{SpecialSubgroups.normal_S_subgroupOf_L}
we know $S$ is normal in $L$.

We can then use then define the desired isomorphism by theorem
\texttt{QuotientGroup.quotientInfEquivProdNormalQuotient} which corresponds to the statement:



Which is in fact the second isomorphism theorem! Not the first isomorphism theorem! 

Which contrasts to how the statement was proved informally, where in for this particular theorem,
\texttt{QuotientGroup.quotientInfEquivProdNormalQuotient}, \texttt{H} is specialized to:




And \texttt{N} is specialized to:



Recall that within Lean, \texttt{F} denotes the base field for $\SL_2(F)$, $D$ and $S$.

Written informally, it then corresponds to the desired statement

\[
D \cong \frac{D}{\bot} = \frac{D}{S \sqcap D} \cong \frac{D \sqcup S}{S} = \frac{L}{S}
\]
\end{remark}

\section{The Center of $\SL_2(F)$}

\begin{definition}
% \lean{Subgroup.center}
% \leanok
The \textbf{center} $Z(G)$ of a group $G$ is the set of elements of  $G$ that commute with every element of $G$.
\begin {equation*} Z(G) = \{ z \in G : \forall g \in G, \hspace{6pt} gz=zg \}. \end{equation*}
It is an immediate observation that $Z(G)$ is a normal subgroup of $G$, 
since for each $z \in Z$, $gzg^{-1} = gg^{-1}z = z$, $\forall g \in G$. It's also clear that a group is abelian if and only if $Z(G)=G$.
\end{definition}

\begin{definition}
\label{SpecialSubgroups.Z}
\lean{SpecialSubgroups.Z}
\leanok
    Let $R$ be a commutative ring and define $Z$ to be the subgroup generated by $- I \in \SL_2(R)$
\end{definition}

\begin{remark}[Z as the subgroup closure of $\{-I\}$]
    Observe that the subgroup generated by an element $g \in G$, $\langle g \rangle$, 
    can be thought of more generally within any lattice (such as the lattice on modules) as the closure of a singleton set ${g}$. 
    
    Therefore, the subgroup generated by $-I$ is equal to
    
    \[\langle -I \rangle = \overline{\{-1\}} = \inf \{ K \le G \; | \; \{-1\} \subseteq K \}.\]

    When taking the closure of a singleton within the subgroup lattice; 
    the closure corresponds to taking the powers of the element in the singleton $\{g\}$,
    which is what is typically understood as the subgroup generated by $g$.

    The way $Z$ is defined in Lean is thus:

   
\end{remark}

\begin{corollary}
\label{SpecialSubgroups.closure_neg_one_eq}
\lean{SpecialSubgroups.closure_neg_one_eq}
\leanok
The subgroup closure of the singleton $\{-I\}$, or equivalently, the subgroup generated by $-I$ equals $\overline{\{-I\}} = \{I, -I\}$
\end{corollary}
\begin{proof}
\leanok
Since $-1^2 = 1$, we have that $-I^2 = I$ and thus the result follows.
\end{proof}



\begin{lemma}
\label{SpecialSubgroups.center_SL2_eq_Z}
\uses{SpecialSubgroups.Z}% Matrix.SpecialLinearGroup.mem_center_iff}
\lean{SpecialSubgroups.center_SL2_eq_Z}
\leanok
The center $Z(\SL_2(F)) = \langle - I_{\SL_2(F)}\rangle = Z$.
\end{lemma}
\begin{proof} 
\leanok
    Take an arbitrary element $x=\begin{bmatrix} \alpha & \beta \\ \gamma & \delta \end{bmatrix} \in \SL_2(F)$and  an arbitrary element $z = \begin{bmatrix} z_1 & z_2 \\ z_3 & z_4 \end{bmatrix} \in Z$ and consider their product:

\begin{align}\label{myeq1} zx = \begin{bmatrix} z_1 & z_2 \\ z_3 & z_4 \end{bmatrix} \begin{bmatrix} \alpha & \beta \\ \gamma & \delta \end{bmatrix} &= \begin{bmatrix} \alpha & \beta \\ \gamma & \delta \end{bmatrix} \begin{bmatrix} z_1 & z_2 \\ z_3 & z_4 \end{bmatrix} = xz, \nonumber \\[1.5ex]
\begin{bmatrix} z_1 \alpha + z_2 \gamma & z_1 \beta + z_2 \delta \\ z_3 \alpha + z_4 \gamma & z_3 \beta + z_4 \delta \end{bmatrix} &= \begin{bmatrix} z_1 \alpha + z_3 \beta & z_2 \alpha + z_4 \beta \\ z_1 \gamma + z_3 \delta & z_2 \gamma + z_4 \delta \end{bmatrix}.
\end{align}

\noindent Equating either the top left or bottom right entries, we see that $z_2 \gamma = z_3 \beta$. Since $\beta$ and $\gamma$ can take any values in $F$, for equality to always hold we must have $z_2 = 0 = z_3$. Hence equation (\ref{myeq1}) simplifies to

\begin{equation*} \begin{bmatrix} z_1 \alpha & z_1 \beta \\ z_4 \gamma & z_4 \delta \end{bmatrix} = \begin{bmatrix} z_1 \alpha & z_4 \beta \\ z_1 \gamma & z_4 \delta \end{bmatrix}. \end{equation*}

Thus 
\begin{equation*} 
    z_1 = z_4 \qquad  \text{and} \qquad z =  
    \begin{bmatrix} z_1 & 0 \\ 0 & z_1 \end{bmatrix}. 
\end{equation*}
Since we are working in the special linear group, det$(z)=1$, thus $z_1 = \pm 1$ and $Z = \langle - I_{\SL_2(F)}\rangle$ as required. Observe that this is a cyclic group of order 2 except in the case of $p = 2$ where $- I_{\SL_2(F)} = I_{\SL_2(F)}$. \\
\end{proof}


Following this result, for ease of notation, $Z(\SL_2(F))$ will be denoted simply by $Z$ throughout the rest of this blueprint.

\begin{lemma}
\label{SpecialSubgroups.exists_unique_orderOf_eq_two}
\lean{SpecialSubgroups.exists_unique_orderOf_eq_two}
\leanok
    If $p\neq 2$, then $\SL_2(F)$ contains a unique element of order 2. \\
\end{lemma}
\begin{proof}
\leanok
Consider an arbitrary element $x \in \SL_2(F)$with order 2. That is $x^2 = I_{\SL_2(F)}$, $x \neq I_{\SL_2(F)}$and thus $x=x^{-1}$.
\begin{equation*} 
    x = \begin{bmatrix} \alpha & \beta \\ \gamma & \delta \end{bmatrix} = \begin{bmatrix} \alpha & \beta \\ \gamma & \delta \end{bmatrix}^{-1} = \begin{bmatrix} \delta & - \beta \\ - \gamma & \alpha \end{bmatrix}.
\end{equation*}
\noindent Thus $\alpha = \delta$, $\beta = - \beta \Rightarrow 2\beta = 0$ and $\gamma = - \gamma \Rightarrow 2\gamma = 0$. In the case of $p \neq 2$ this gives $\beta = 0 = \gamma$. So
\begin{equation*} 
    x = \begin{bmatrix} \alpha & 0 \\ 0 & \alpha \end{bmatrix}.
\end{equation*}
\noindent Also $\alpha^2 = 1$ since $x \in$ $\SL_2(F)$, so $\alpha = \pm 1$. For $x$ to have order 2, we must have $\alpha = - 1$. Hence there is a unique element of order 2, namely $- I_{\SL_2(F)}$.
\end{proof}


\begin{lemma}
    \label{SpecialSubgroups.card_Z_eq_two_of_two_ne_zero}
    \lean{SpecialSubgroups.card_Z_eq_two_of_two_ne_zero}
    \leanok
    If $\textrm{char}(F) \ne 2$ then $|Z| = 2$.
\end{lemma}
\begin{proof}
\leanok
    If $\textrm{char}(F) \ne 2$ then $1 \ne -1$ as $2 \ne 0$ therefore, $I \ne -I$ which shows that $Z = \{I , -I\}$ contains two distinct elements.
\end{proof}


\begin{lemma}
    \label{SpecialSubgroups.card_Z_eq_one_of_two_eq_zero}
    \lean{SpecialSubgroups.card_Z_eq_one_of_two_eq_zero}
    \leanok
    If $\textrm{char}(F) = 2$ then $|Z| = 1$. 
\end{lemma}
\begin{proof}
\leanok
    If $\textrm{char}(F) = 2$ then $1 = -1$ as $2 = 0$ therefore, $I = -I$ which shows that $Z = \{I , -I\} = \{I\}$ only contains one element.
\end{proof}


\begin{lemma}[$Z$ is cyclic]
    \label{SpecialSubgroups.IsCyclic_Z}
    \lean{SpecialSubgroups.IsCyclic_Z}
    \leanok
\end{lemma}
\begin{proof}
\leanok
    By construction, $Z = \overline{\{-I\}} = \{-I^k \; | \; k \in \Z \} = \langle -I \rangle$, therefore $Z$ is generated by a single element and is thus cyclic.
\end{proof}


\begin{remark}[Typeclass instances]
    Observe that instead of telling Lean that \texttt{Is\_Cyclic\_Z} is a \texttt{theorem} we declare it to be
    an \texttt{instance} since we would like Lean to look for this fact whenever it requires it for a
    theorem that may require the assumption that $Z$ is a commutative subgroup.
\end{remark}

In the next chapter it will be useful to record the interactions between $S$ and $Z$. 
For instance we define the following subgroup

\begin{definition}
    \label{SpecialSubgroups.SZ}
    \uses{SpecialSubgroups.S, SpecialSubgroups.Z, SpecialMatrices.s_mul_s_eq_s_add}
    \lean{SpecialSubgroups.SZ}
    \leanok
    We define the subgroup $SZ$ to be the subgroup with the underlying set $S \cup -S$, or equivalently the pointwise product $SZ$.
\end{definition}



\begin{corollary}
\label{SpecialSubgroups.S_mul_Z_subset_SZ}
\uses{SpecialSubgroups.SZ}
\lean{SpecialSubgroups.S_mul_Z_subset_SZ}
$SZ = S \cup -S$
\leanok
\end{corollary}
\begin{proof}
\leanok
    By construction an element of $SZ$ is either of the form $s_\sigma  I = s_\sigma \in S$ or $s_\sigma -I = -s_\sigma\in -S$. The reverse subset inclusion is very similar.
\end{proof}


\begin{lemma}
    \label{SpecialSubgroups.S_join_Z_eq_SZ}
    \uses{SpecialSubgroups.Z, SpecialSubgroups.S, SpecialSubgroups.SZ}
    \lean{SpecialSubgroups.S_join_Z_eq_SZ}
    \leanok
    The join of subgroups satisfies $S \sqcup Z = SZ$
\end{lemma}
\begin{proof}
\uses{SpecialSubgroups.closure_neg_one_eq, SpecialSubgroups.S_mul_Z_subset_SZ}
\leanok
We show that $S \sqcup Z = SZ$ by antisymmetry, that is, we show both that
\begin{itemize}
    
    \item $S \sqcup Z \subseteq SZ$
    
    Let $x \in S \sqcup Z$, then if $x$ is in the subgroup closure then if $K$ is a subgroup whose underlying set contains $SZ$ then $x$ is in $K$,
    but since $SZ$ was shown to be a subgroup in \ref{SpecialSubgroups.SZ} we can conclude that $x \in SZ$ and thus $x = s_\sigma z$ for some $\sigma \in F$.
    
    \item $SZ \subseteq S \sqcup Z$
    
    Let $s_\sigma z \in SZ$ then we must show that $s_\sigma z$ is the subgroup closure of $S$ and $Z$ but since the subgroup closure 
    must at least contain the pointwise product whose underlying set is equal to the union $S \cup -S$, we are done.
\end{itemize}
\end{proof}


\section{Conjugacy of the Elements of $\SL_2(F)$}

\subsubsection{Classification of elements of $\SL_2(F)$ up to conjugation}

\begin{lemma}[Upper triangularizability criteria]
\label{isConj_upper_triangular_iff}
\lean{isConj_upper_triagnular_iff}
\leanok
    A matrix $M \in\textrm{Mat}(2; F)$ is triangularizable if and only if there exists an invertible matrix $C \in \GL_2(F)$ such that the bottom left entry
    $C M C^{-1}_{21} = 0$.
\end{lemma}

\begin{proof}
\leanok
    Given a matrix $U$ is in upper triangular form if and only if
    \[
    U = \begin{bmatrix}
    a & b\\
    0 & d
    \end{bmatrix}
    \]
    
    that is, the bottom left entry is zero. It then follows that $M$ is triangularizable if and only if
    there exists a $C \in \GL_2(F)$ such that $C M C^{-1}$ is in upper triangular form, that is, the bottom left entry of $C M C^{-1}$ is zero. 
\end{proof}


\begin{lemma}[Upper triangularizability of a $2 \times 2$ matrix over an algebraically closed field]
\label{isTriangularizable_of_algClosed}
\lean{isTriangularizable_of_algClosed}
\leanok
    When $F$ is an algebraically closed field, 
    for any $M \in \textrm{Mat}(2; F)$ there exists an invertible matrix $C \in \SL_2(F) \le \GL_2(F)$ such that $C M C^{-1} = U$ where
    \[
    U = \begin{bmatrix}
        a & b\\
        0 & d
    \end{bmatrix}\] for some $a, b, d \in F$.
\end{lemma}
\begin{proof}
    \uses{isConj_upper_triangular_iff}
    \leanok
We prove this by direct computation. 

Let 

\[
M = \begin{bmatrix}
\alpha & \beta\\
\gamma & \delta
\end{bmatrix} \in\textrm{Mat}(2; F)
\]

By lemma \ref{isConj_upper_triangular_iff}, we only need to show that we can find a matrix $C \in \SL_2(F)$ such that when it acts on $M$ by conjugation, the bottom left entry is annihilated.

\begin{itemize}
    \item Suppose on the one hand that $\beta \ne 0$
    
    Observe that 
    
    \begin{equation}\label{triang}
        s_\sigma M s_\sigma^{-1} = \left(\begin{bmatrix}
            -\beta \sigma + \alpha & \beta \\
            -\beta \sigma^{2} + \alpha \sigma - \delta \sigma + \gamma & \beta \sigma + \delta
            \end{bmatrix}\right)
    \end{equation}

    Given $F$ is algebraically closed we can set $\sigma \in F$ to be a root of the polynomial

    \[
    P(X) := -\beta X^{2} + \alpha X - \delta X + \gamma 
    \]

    setting $C := s_\sigma$ yields the desired element which triangularises $M$.
    

    \item Suppose on the other hand that $\beta = 0$
    
    Given the top right entry is zero, we only need find a matrix in $\SL_2(F)$ which flips the anti-diagonal entries (modulo modifying the signs)
    it is thus sufficient to use 
        \[
        w = \begin{bmatrix}
        0 & -1\\
        1 & 0
        \end{bmatrix} \quad \text{as indeed} \quad w M w^{-1} = \begin{bmatrix}
            \delta & -\gamma\\
            0 & \alpha
        \end{bmatrix} \text{ is in triangular form} 
        \]
\end{itemize}

\end{proof}


\begin{corollary}[Upper triangular matrices are conjugate to lower triangular matrices]
    \label{lower_triangular_isConj_upper_triangular}
    \uses{SpecialMatrices.w}
    \lean{lower_triangular_isConj_upper_triangular}
    \leanok
    For every $U \in\textrm{Mat}(2; F)$ that is upper triangular the matrix $w U w^{-1}$ is a lower triangular matrix
\end{corollary}
\begin{proof}
    \leanok
    Direct computation shows this result, see the Lean code!
\end{proof}


\begin{lemma}
    \label{upper_triangular_isConj_diagonal_of_nonzero_det}
    \lean{upper_triangular_isConj_diagonal_of_nonzero_det}
    \leanok
    An upper triangular matrix $U = \begin{bmatrix}
        \alpha & \beta\\
        0 & \delta
    \end{bmatrix}$ is conjugate to a diagonal matrix if $\alpha - \delta \ne 0$
\end{lemma}
\begin{proof}
    \leanok
We show this by direct computation.

Conjugation of $M$ by the matrix 

\[
C := \begin{bmatrix}
    1 & \frac{\beta}{\alpha - \delta}\\
    0 & 1
\end{bmatrix}
\]

yields a diagonal matrix (see the Lean code for the computation!).
\end{proof}


% \begin{remark}[Automation in Lean]
%     Observe that the proof of the theorem above primarily uses tactics! Tactics l 
% \end{remark}


\begin{proposition}
\label{SL2_IsConj_d_or_IsConj_s_or_IsConj_neg_s_of_AlgClosed}
\uses{SpecialMatrices.s, SpecialMatrices.d}
\lean{SL2_IsConj_d_or_IsConj_s_or_IsConj_neg_s_of_AlgClosed}
\leanok
    Each element of $\SL_2(F)$ is conjugate to either $d_\delta$ for some $\delta \in F^\times$, or to $\pm s_\sigma$ for some $\sigma \in F$.
\end{proposition}

\begin{proof}
\uses{isTriangularizable_of_algClosed, lower_triangular_isConj_upper_triangular, upper_triangular_isConj_diagonal_of_nonzero_det}
\leanok Since $F$ is algebraically closed, any element $x \in \SL_2(F)$can be regarded as a linear transformation in the 2 dimensional vector space over $F$, with the eigenvalues $\pi_1$ and $\pi_2$. \\
\\
\space If $\pi_1$ and $\pi_2$ are distinct, then $x$ is thus diagonalisable. That is, there exists an invertible matrix $a \in GL(2, F)$ such that $y = axa^{-1}$ is a diagonal matrix. Furthermore, we can multiply $a$ by a suitable scalar to find an element in $\SL_2(F)$which conjugates $x$ and $y$:

\begin{align*}
    \text{Set} \; b = \frac{a}{\sqrt {\text{det}(a)}}, \quad \text{thus } \; bxb^{-1} =\frac{a}{\sqrt {\text{det}(a)}} \; x \; (\sqrt{\text{det}(a)} \; )\,a^{-1} = axa^{-1} = y.
\end{align*}

Observe that det$(b)=1$, hence $x$ and $y$ are conjugate in $L$. Furthermore, since $y$ is a diagonal matrix it must belong to the set $D$, showing that $x$ is conjugate to $d_\delta$ for some $\delta \in F^\times$. \\
\\
\space If $\pi_1 = \pi_2$ then $x$ has just one repeated eigenvalue. Suppose that $x$ is diagonalisable. Then there exists an element $c \in GL(2, F)$ and a diagonal matrix $\pi_1 I_G$ such that $x = c(\pi_1 I_G)c^{-1} = \pi_1 I_G$. Thus $x = \pm I_G$, which trivially belongs to both $D$ and $Z$. \\
\\
Now assume that $x$ is not diagonalisable. Chapter 7 of \cite{matrix} shows that there exists an element $d \in GL(2, F)$, such that $x= djd^{-1}$, where, $$j = \begin{bmatrix} \pi_1 & 1 \\ 0 & \pi_1 \end{bmatrix}$$ is the Jordan Normal Form of $x$. By the method described above, we can multiply $d$ by a suitable scalar to show that $x$ is conjugate to $j$ in $L$. Now we conjugate $j$ by an element of $\SL_2(F)$whose top left entry is 0.

\begin{align*}
    \begin{bmatrix} 0 & -\gamma^{-1} \\ \gamma & \delta \end{bmatrix} \begin{bmatrix} \pi_1 & 1 \\ 0 & \pi_1 \end{bmatrix} \begin{bmatrix} \delta & \gamma^{-1} \\ -\gamma & 0 \end{bmatrix} = \begin{bmatrix} 0 & -\gamma^{-1} \\ \gamma & \delta \end{bmatrix} \begin{bmatrix} \pi_1 \delta - \gamma & \pi_1 \gamma^{-1} \\ -\pi_1 \gamma & 0 \end{bmatrix} = \begin{bmatrix} \pi_1 & 0 \\ -\gamma^{2} & \pi_1 \end{bmatrix}
\end{align*}
\\
Now clearly the determinant of $x$ is equal to the determinant of $j$, namely 1, which means that $\pi_1 = \pm 1$. This shows that $j$ is conjugate in $\SL_2(F)$to some element in $\times Z$ as well as $x$. Furthermore, since conjugation is transitive, $x$ is conjugate to $\pm s_\sigma$ for some $\sigma \in F$.

\end{proof}


\begin{remark}
formalising the classification of elements of $\SL_2(F)$ up to conjugation in Lean was surprisingly difficult because the informal proof of proposition \ref{SL2_IsConj_d_or_IsConj_s_or_IsConj_neg_s_of_AlgClosed} extracted from Christopher Butler's exposition 
uses the Jordan Normal Form theorem, and at the time of writing, the Jordan Normal form theorem is still not yet in \texttt{mathlib}.

The original approach to formalise the Jordan Normal form theorem for $2 \times 2$ matrices involved studying the eigenspace and generalized eigenspaces of the endomorphism associated to a $2 \times 2$ matrix.
This is one the standard approached taught in an undergraduate curriculum, yet surprisingly, to formalise the $2 \times 2$ case with this approach was rather untractable.

The reason this approach, and often other standard techniques might not integrate well with \texttt{mathlib}, often happens for the following reasons I will now outline.

The crux of formalising a mathematical result always lies at finding the right abstraction, as illustrated in \ref{lattice}, understanding the lattice structure on the set of subgroups becomes an indispensable tool for formalising
results regarding subgroups and their properties. For this particular formalisation, the right abstraction was not entirely.

Is it best to prove the theorem for matrices or for endomoprhisms? Which will be the easiest approach? Which approach is most general? Which approach yields the most amount of useful lemmas?

The reason why the Jordan Normal Form theorem is not yet in \texttt{mathlib} is because it hinges on the following two results which have not been formalised yet:

\begin{enumerate}
    \item The classification of nilpotent endomorphisms.
    \item The classification of semisimple endomorphisms.
\end{enumerate}

Such formalisation would be an amazing project to undertake. But bear in mind, the theorem formalized is the more general Jordan-Chevallier theorem.

To the authors understanding, the general theorem will be formalised by studying the eigenspace and general eigenspace. This approach turned out to be essentially equivalent in 
difficulty to formalising the special case of $2 \times 2$ matrices over an algebraically closed field with the same approach since the argument is inductive on the dimension.

Therefore, after discussions with Prof. Kevin Buzzard's it turned out to be much more effective approach to classify matrices of the special linear group up to 
conjugation by splitting on a few different cases of what a $2 \times 2$ matrix might look like and finding the suitable matrices by which to conjugate to put them in either the form of $d_\delta$ or $\pm s_\sigma$.
\end{remark}

\section{Centralizers \& Normalizers}

Both the centralizer and normalizer of a subset $H$ are subgroups of $G$. Note also that the centralizer is a stronger condition than the 
normalizer and any element in the centralizer of $H$ is also in its normalizer. If $H$ is a singleton then it's clear that its centralizer and normalizer are equal.\\

\subsubsection{Normalizers}

\begin{definition}
The \textbf{normalizer} $N_G(H)$ of a subset $H$ of a group $G$ is the set of elements of $G$ which stabilise $H$ under conjugation.
\begin{equation*} N_G(H) = \{ g \in G : gHg^{-1}=H\}. \end{equation*}
\end{definition}


\begin{corollary}
    \label{lower_triangular_iff_top_right_entry_eq_zero}
    \lean{lower_triangular_iff_top_right_entry_eq_zero}
    \leanok
    A matrix $M \in \textrm{Mat}(2; F)$ is lower triangular if and only if the $M_{12} = 0$.
\end{corollary}
\begin{proof}
    \leanok
It is easy to see the top right entry must be zero for a matrix to be lower triangular.
\end{proof}


\begin{proposition}[Normalizer of subgroups of $S$ are contained in $L$]
\label{normalizer_subgroup_S_le_L}
\uses{SpecialSubgroups.S, SpecialSubgroups.L}
\lean{normalizer_subgroup_S_le_L}
\leanok
 For any subgroup $S_0 \leq S$ with order greater than 1, we have that the normalizer $N_{\SL_2(F)}(S_0) \subset L$.
\end{proposition}
\begin{proof}
    \uses{mem_L_iff_lower_triangular, lower_triangular_iff_top_right_entry_eq_zero}
    \leanok
Let $s_\sigma$ be an arbitary element of $S_0$ with $\sigma \neq 0$. To determine the normalizer of $S_0$ in $\SL_2(F)$we consider which $x \in \SL_2(F)$ satisfy $x s_\sigma x^{-1} \in S_0$.
\begin{align*} x s_\sigma x^{-1} &= \begin{bmatrix} \alpha & \beta \\ \gamma & \delta \end{bmatrix} \begin{bmatrix} 1 & 0 \\ \sigma & 1 \end{bmatrix} \begin{bmatrix} \delta & - \beta \\ - \gamma & \alpha \end{bmatrix}
\\[1.5ex] &= \begin{bmatrix} \alpha & \beta \\ \gamma & \delta \end{bmatrix} \begin{bmatrix} \delta & - \beta \\ \delta \sigma - \gamma & \alpha - \beta \sigma \end{bmatrix}
\\[1.5ex] &= \begin{bmatrix} \alpha \delta - \beta \gamma + \beta \delta \sigma & - \beta^2  \sigma \\ \delta^2 \sigma & \alpha \delta - \beta \gamma - \beta \delta \sigma \end{bmatrix}.
\end{align*}
Since $x s_\sigma x^{-1} \in S_0$ we have $- \beta^2  \sigma = 0$ and since $\sigma \neq 0$, we have $\beta = 0$. Since $s_\sigma$ was chosen arbitrarily, 
any element which normalizes $S_0$ is a lower triangular matrix and is therefore in $L$ by \ref{mem_L_iff_lower_triangular}. Thus $N_{\SL_2(F)}(S_0) \subset L$ as required. \\
\end{proof}


\begin{lemma}
    \label{ex_of_card_D_gt_two}
    \lean{ex_of_card_D_gt_two}
    \leanok
    If the cardinality of finite subgroup of $D_0 \le D$ is greater than $2$ then there exists an element $x \in D_0$ which does not belong to the center $Z$, that is, $x \ne d_1 = I$ and $x \ne d_{-1} = -I$.
\end{lemma}
\begin{proof}
    \leanok
 Suppose for a contradiction that if $\delta \ne \pm 1$ then $d_\delta \notin D_0$. We show that $D_0 \le Z$ and therefore, $|D_0| \le 2$, a contradiction.
 
 Let $d_\delta \in D_0 \le D$ then given $d_\delta \notin D_0$ if $\delta \ne \pm 1$ and $Z = \langle -I\rangle = \{I, -I\}$. It immediately follows that $D_0 \le Z$.

\end{proof}


\begin{proposition}[Normalizers of subgroups of $D$ are contained in $L$]
\label{normalizer_subgroup_D_eq_DW}
\uses{SpecialSubgroups.D, SpecialSubgroups.DW}
\lean{normalizer_subgroup_D_eq_DW}
\leanok
    $N_{\SL_2(F)}(D_0) = \langle D , w \rangle$, where  $D_0$ is any subgroup of $D$ with order greater than 2. \\
    \end{proposition}

\begin{proof}
    \uses{SpecialLinearGroup.fin_two_diagonal_iff, SpecialLinearGroup.fin_two_antidiagonal_iff, ex_of_card_D_gt_two}
    \leanok
    Since $|D_0| > 3$, we can choose a $d_\delta \in D_0 \! \setminus \! Z$, that is where $\delta \neq 1$. To determine the normalizer of $D_0$ in $\SL_2(F)$we consider which $x \in \SL_2(F)$satisfy $x d_\delta x^{-1} \in D_0$.
    \begin{align}\label{6.3proof3} xd_\delta x^{-1} &= \begin{bmatrix} \alpha & \beta \\ \gamma & \delta \end{bmatrix} \begin{bmatrix} \delta & 0 \\ 0 & \delta^{-1} \end{bmatrix} \begin{bmatrix} \delta & - \beta \\ - \gamma & \alpha \end{bmatrix} \nonumber \\[1.5ex]
    &= \begin{bmatrix} \alpha & \beta \\ \gamma & \delta \end{bmatrix} \begin{bmatrix} \delta \delta & - \beta \delta \\ - \gamma \delta^{-1} & \alpha \delta^{-1} \end{bmatrix} \nonumber \\[1.5ex]
    &= \begin{bmatrix} \alpha \delta \delta - \beta \gamma \delta^{-1} & \alpha \beta (\delta^{-1} - \delta) \\ \gamma \delta (\delta - \delta^{-1}) & \alpha \delta \delta^{-1} - \beta \gamma \delta \end{bmatrix} \in D_0.
    \end{align}
    
    Since (\ref{6.3proof3}) is in $D_0$, the top right and bottom left entries must be 0. Since  $\delta \neq \pm 1$, we have $\delta \neq \delta^{-1}$ and so $\alpha \beta = 0 = \gamma \delta$. \\
    \\
     \space If $\alpha = 0$, then $\beta$ and $\gamma$ are non-zero since det$(x) = 1$, thus $\delta = 0$. So det$(x) = - \gamma \beta = 1$  and $- \gamma = \beta^{-1}$. (\ref{6.3proof3}) becomes $$\begin{bmatrix} \delta^{-1} & 0 \\ 0 & \delta \end{bmatrix} = d^{-1}_\delta.$$Since $D_0$ is a group, it contains the inverse of each of it's elements, so $d^{-1}_\delta \in D_0$ as required. In this case we have $x \in wD$. \\
    \\
     \space If $\alpha \neq 0$, then similarly $\beta = 0$, $\delta = \alpha^{-1}$ and $\gamma = 0$. (\ref{6.3proof3}) now becomes $$\begin{bmatrix} \delta & 0 \\ 0 & \delta^{-1} \end{bmatrix} = d_\delta \in D_0.$$This time we have $x \in D$. So $x \in D \cup wD = \langle D , w \rangle$ and any element which normalises $D_0$ is in $\langle D , w \rangle$, thus $N_{\SL_2(F)}(D_0) \subset \langle D , w \rangle$. \\
    \\
    Now take an arbitrary $y \in \langle D , w \rangle = D \cup wD$. If $y \in D$ then $y = d_{\rho 1}$, for some $\rho 1 \in F^\times$.
    \begin{align*} d_{\rho 1} d_\delta d^{-1}_{\rho 1} = d_\delta \in D_0.
    \end{align*}
    
    If $y \in wD$ then $y = w d_{\rho 2}$, for some $ d_{\rho 2} \in F^\times$.
    \begin{align*} (w d_{\rho 2}) d_\delta (w d_{\rho 2})^{-1} &= w d_{\rho 2} d_\delta d^{-1}_{\rho 2} w^{-1}
    \\ &= w d_\delta w^{-1}
    \\ &= d^{-1}_\delta \in D_0.
    \end{align*}
    
    Thus $y$ indeed the whole of $\langle D , w \rangle$ is contained in $N_{\SL_2(F)}(D_0)$. This inclusion gives the desired result, $N_{\SL_2(F)}(D_0) = \langle D , w \rangle$. \\
    
\end{proof}


%-----------------------------

\subsubsection{Centralisers}

\begin{definition}[Centralizer]
The \textbf{centralizer} $C_G(H)$ of a subset $H$ of a group $G$ is the set of elements of $G$ which commute with each element of $H$.
\begin{equation*} 
    C_G(H) = \{ g \in G  : gh=hg, \quad \forall h\in L \}. \end{equation*} 
\end{definition}

\begin{corollary}
    \label{centralizer_neg_eq_centralizer}
    \lean{centralizer_neg_eq_centralizer}
    \leanok
    Let $x \in \SL_2(F)$ then the centralizer of the negative equals $C_{SL_2(F)}(x) = C_{SL_2(F)}(-x)$.
\end{corollary}
\begin{proof}
    \leanok
 An element $y \in \SL_2(F)$ belongs to $C_{SL_2(F)}$ if and only if
  $1 = x y x^{-1} y^{-1} = (-x) y (-x^{-1}) y^{-1}$  if and only if $y$ belongs to $C_{\SL_2(F)}(-x)$.
\end{proof}

    

\begin{proposition}[Centralizer of noncenter $s_\sigma$]
\label{centralizer_s_eq_SZ}
\uses{SpecialSubgroups.S, SpecialSubgroups.Z, SpecialMatrices.s}
\lean{centralizer_s_eq_SZ}
\leanok
The centralizer $C_{\SL_2(F)}(\pm s_\sigma) =  S \times Z $ where $\sigma \neq 0$.
\end{proposition}

\begin{proof}
    \uses{SpecialLinearGroup.fin_two_shear_iff, centralizer_neg_eq_centralizer}
    \leanok
To determine the centralizer of $s_\sigma$ in $L$, we consider which $y \in \SL_2(F)$satisfy $y s_\sigma = s_\sigma y$ for an arbitrarily chosen $s_\sigma$, with $\sigma \neq 0$. \\
\vspace{-0.5mm}
\begin{align}\label{6.3proof2} y s_\sigma &= s_\sigma y, \nonumber \\[1.5ex]
\begin{bmatrix} \alpha & \beta \\ \gamma & \delta \end{bmatrix} \begin{bmatrix} 1 & 0 \\ \sigma & 1 \end{bmatrix} &= \begin{bmatrix} 1 & 0 \\ \sigma & 1 \end{bmatrix} \begin{bmatrix} \alpha & \beta \\ \gamma & \delta \end{bmatrix}, \nonumber \\[1.5ex]
\begin{bmatrix} \alpha + \beta \sigma & \beta \\ \gamma + \delta \sigma & \delta \end{bmatrix} &= \begin{bmatrix} \alpha & \beta \\ \gamma +  \alpha \sigma & \delta + \beta \sigma \end{bmatrix}.
\end{align}
\vspace{.5mm}

Equating the top left entries of (\ref{6.3proof2}) gives $\alpha + \beta \sigma = \alpha$ which means $\beta = 0$ since $\sigma \neq 0$ by assumption. Equating the bottom left entries gives that $\alpha = \delta$. Finally, since det$(y) = 1$, we have $\alpha \delta = 1$ so $\alpha = \pm 1$. Thus a $y \in C_{\SL_2(F)}(s_\sigma)$ is

\begin{align*} y &= \begin{bmatrix} \alpha & 0 \\ \gamma & \alpha \end{bmatrix}. \tag{where $\alpha = \pm 1$}
\end{align*}

So $y = \pm s_\sigma$ for some $\sigma \in F$, and $SZ = \{ \pm s_\sigma \} \subset C_{\SL_2(F)}(s_\sigma)$. Now take an arbitrary $s_\gamma z \in SZ$.
\begin{align*} (s_\gamma z) s_\sigma &= s_\sigma (s_\gamma z),
\\ s_\gamma s_\sigma z &= s_\sigma s_\gamma z, \tag{since $z \in Z$}
\\ s_{\gamma + \sigma} &= s_{\gamma + \sigma}.
\end{align*}

Thus $s_\gamma z$ and indeed the whole of $SZ$ is contained in $ C_{\SL_2(F)}(s_\sigma)$, so $C_{\SL_2(F)}(s_\sigma) = SZ$. \\
\\
Since $S$ commutes elementwise with $Z$ and $S \cap Z = \{ I_G \}$, we can apply Corollary \ref{directproductZ} and assert that $C_{\SL_2(F)}(s_\sigma) = SZ \cong S \times Z$ as required. The centralizer of $- s_\sigma$ is also $ S\times Z$, 
since an element $x$ commutes with $- s_\sigma$ if and only if it commutes with $s_\sigma$:
\begin{align*} 
    xs_\sigma = s_\sigma x \iff -(x s_\sigma) = - (s_\sigma x) \iff x(- s_\sigma) = (- s_\sigma)x.
\end{align*}

Note that in case of $\sigma = 0$, $\pm s_\sigma \in Z$ and thus it's centralizer is the whole of $L$.

\end{proof}


    

\begin{proposition}[Centralizer of noncenter $d_\delta$]
    \label{centralizer_d_eq_D}
    \uses{SpecialMatrices.d, SpecialSubgroups.D}
    \lean{centralizer_d_eq_D}
    \leanok
    The centralizer $C_{\SL_2(F)}(d_\delta) = D$ for $\delta \neq \pm 1$.
    \end{proposition}        
\begin{proof}
    \uses{SpecialLinearGroup.fin_two_diagonal_iff}
    \leanok
Now we consider which $y \in \SL_2(F)$satisfy $y d_\delta = d_\delta y$ for an arbitrarily chosen $d_\delta$, with $\delta \neq \pm 1$.
\begin{align}\label{6.3proof4} y d_ \delta &= d_\sigma y, \nonumber \\[1.5ex]
\begin{bmatrix} \alpha & \beta \\ \gamma & \delta \end{bmatrix} \begin{bmatrix} \delta & 0 \\ 0 & \delta^{-1} \end{bmatrix} &= \begin{bmatrix} \delta & 0 \\ 0 & \delta^{-1} \end{bmatrix} \begin{bmatrix} \alpha & \beta \\ \gamma & \delta \end{bmatrix}, \nonumber \\[1.5ex]
\begin{bmatrix} \alpha \delta & \beta \delta^{-1} \\ \gamma \delta & \delta \delta^{-1} \end{bmatrix} &= \begin{bmatrix} \alpha \delta & \beta \delta \\ \gamma \delta^{-1} & \delta \delta^{-1} \end{bmatrix}.
\end{align}

Equating the top right and bottom left entries of (\ref{6.3proof4}) gives that $\beta = 0 = \gamma$ since Since $\delta \neq \delta^{-1}$. Thus $\delta = \alpha^{-1}$ and 
\begin{align*} x = \begin{bmatrix} \alpha & 0 \\ 0 & \alpha^{-1} \end{bmatrix} \in D. 
\end{align*}

Thus $x$ and indeed the whole of $C_{\SL_2(F)}(d_\delta)$ is contained in $D$. Now take an arbitrary $d_\rho \in D$.
\begin{align*} d_\rho d_\delta = d_{\rho \delta} = d_\delta d_\rho.
\end{align*}
So clearly $D \subset C_{\SL_2(F)}(d_\delta)$ and thus $C_{\SL_2(F)}(d_\delta) = D$ as required.
\end{proof}


%---------------------------------------------------


\begin{proposition}[Centralizers of conjugate elements]
    \label{conjugate_centralizers_of_IsConj}
    \lean{conjugate_centralizers_of_IsConj}
    \leanok
    Let $a$ and $b$ be conjugate elements in a group $G$. Then $\exists \, x \in G$ such that $xC_G(a)x^{-1} = C_G(b)$. \vspace{3mm}
\end{proposition}
\begin{proof}
\leanok
This proposition essentially claims that conjugate elements have conjugate centralizers. Since $a$ and $b$ are conjugate there exists an $x \! \in \! G$ such that $b = xax^{-1}$. Let $g$ be an arbitrary element of $C_G(a)$. Then,

\begin{align*} (xgx^{-1})(xax^{-1}) &= xgax^{-1}\\
&= xagx^{-1} \tag{since $g \in C_G(a)$}\\
&= (xax^{-1})(xgx^{-1}). \end{align*}

Thus $xgx^{-1} \in C_G(xax^{-1})$. Since $g$ was chosen arbitrarily, $$xC_G(a)x^{-1} \subset C_G(xax^{-1}) = C_G(b).$$ 

Conversely, let $h$ be an arbitary element of $C_G(xax^{-1})$. Then,

\begin{align*} (x^{-1}hx)a &= x^{-1}h(xax^{-1})x \\
&= x^{-1}(xax^{-1})hx \tag{since $h \in C_G(xax^{-1})$} \\
&= a(x^{-1}hx). \end{align*}

So $x^{-1}hx \in C_G(a)$ and since $h$ was arbitrarily chosen from $C_G(xax^{-1})$, \linebreak $x^{-1}C_G(xax^{-1})x \subset C_G(a)$. Multiplication on the left by $x$ and on the right by $x^{-1}$ gives $C_G(b) =  C_G(xax^{-1}) \subset xC_G(a)x^{-1}$. Since we have shown that each set contains the other, $xC_G(a)x^{-1} = C_G(b)$ as required. \\
\end{proof}




\begin{corollary}[ Centralizer of non-central element is commutative]
    \label{IsCommutative_centralizer_of_not_mem_center}
    \lean{IsCommutative_centralizer_of_not_mem_center}
    \leanok
The centralizer of an element $x$ in $\SL_2(F)$ is abelian unless $x$ belongs to the centre of $L$. \vspace{3mm}
\end{corollary}

\begin{proof}
    \uses{SL2_IsConj_d_or_IsConj_s_or_IsConj_neg_s_of_AlgClosed, conjugate_centralizers_of_IsConj, centralizer_s_eq_SZ, centralizer_d_eq_D}
    \leanok
    This is almost an immediate consequence of the preceding results. Propositions \ref{centralizer_s_eq_SZ} and \ref{centralizer_d_eq_D} show that an element of the form $\pm s_\sigma$ which does not lie in the centre of $\SL_2(F)$ has centralizer $S \times Z$, whilst a non-central element of the form $d_\delta$ has centralizer $D$.
Both $S$ and $D$ are abelian since they are isomorphic to $F$ and $F^\times$ respectively. Let $s_\sigma z_1$ and $s_\gamma z_2$  be arbitrary elements of $S \times Z$.

\vspace{-.5mm}
\begin{align*} 
    (s_\sigma z_1)(s_\gamma z_2)  &= s_\sigma s_\gamma z_2 z_1  \tag{since  $z_1 \in Z$}
\\ &= s_\gamma s_\sigma z_2 z_1  \tag{since  $S$ is abelian}
\\ &= (s_\gamma z_2)(s_\sigma z_1).   \tag{since  $z_2 \in Z$}
\end{align*} 

Thus $S \times Z$ is also abelian. Since every element of $\SL_2(F)$ is conjugate to $d_\delta$ or $\pm s_\sigma$ by Proposition \ref{SL2_IsConj_d_or_IsConj_s_or_IsConj_neg_s_of_AlgClosed} and conjugate elements have conjugate centralizers by Proposition \ref{conjugate_centralizers_of_IsConj}, the centralizer of each $x \in \SL_2(F)\setminus Z$ is conjugate to either $\times Z$ or $D$. 
Since conjugate subgroups are isomorphic, they must have the same structure, and thus since both $S \times Z$ and $D$ are abelian, $C_{\SL_2(F)}(x)$ is also abelian. 
Note that in general this does hold for $x \in Z$, since its centralizer is the whole of $\SL_2(F)$ which is not abelian unless $\SL_2(F)= Z$.

\end{proof}




\section{The Projective Line \& Triple Transitivity}

It is convenient to sometimes take a geometric viewpoint and regard the elements of $\SL_2(F)$as pairs of vectors in the 2-dimensional vector space over $F$, which we will denote $V$. An element of $\SL_2(F)$is thus a linear transformation of $V$. 

\begin{definition} 
    Let $\mathscr{L}$ be the set of all 1-dimensional subspaces of $V$. A subset $\mathscr{S}$ of $\mathscr{L}$ is called a \textbf{subspace} of $\mathscr{L}$ if there is a subspace $U$ of $V$ such that $\mathscr{S}$ is the set of all 1-dimensional spaces of $U$. We have dim $U =$ dim $\mathscr{S} + 1$. The set $\mathscr{L}$ on which this concept of subspaces is defined is called the \textbf{projective line} on $V$ and an element of $\mathscr{L}$ is a 0-dimensional subspace of $\mathscr{L}$ and consequently called a \textbf{point}. The projective line can be considered as a straight line in the field, plus a point at infinity.
\end{definition}

Any 1-dimensional subspace of $V$ is a set of vectors of the form $\eta u$, where $u$ is a non-zero vector of $V$ and $\eta \in F^\times$. Thus the points of $\mathscr{L}$ are equivalence classes with the following relation defined on the set of vectors of $V$.
\begin{align*} u = \begin{bmatrix} u_1 \\ u_2 \end{bmatrix} \sim \begin{bmatrix} v_1 \\ v_2 \end{bmatrix} = v \iff u = \eta v, \qquad (\text{for $\eta \in F^\times$}).
\end{align*}

Notice that $u$ and $v$ are equivalent if and only if $u_1 v_2 = v_1 u_2$. Importantly each point $P_i$ of $\mathscr{L}$ can be represented by a corresponding equivalence class of vectors of $V$, that is, $P$ corresponds to $u$ if $P = u_1 / u_2$. In the case when $u_2 = 0$, this corresponds to the point at infinity.

\begin{definition} Let $S$ be a permutation group which acts on a set $X$ and $\{ x_1, x_2, x_3 \}$ and $\{ x_1', x_2', x_3' \}$ be two subsets of distinct elements of $X$. Then $S$ is said be \textbf{triply transitive} on $X$ if there is an element $\pi \in S$ such that,
\begin{align*} x^{\pi}_i = x'_i, \qquad(\text{$i$ = 1,2 or 3}).
\end{align*} 
\end{definition}

\begin{theorem} \label{6.6}
Let $\mathscr{L}$ be the projective line over the field $F$. Then $\SL_2(F)$is triply transitive on the set of the points of $\mathscr{L}$. \vspace{3mm}
\end{theorem}

\begin{proof} Let $P_1$, $P_2$ and $P_3$ be distinct points of $\mathscr{L}$ and $p_i$ be a vector in $V$ corresponding to $P_i$. Since each $P_i$ is distinct, $p_1$, $p_2$ and $p_3$ are thus pairwise linearly independent. Thus $p_1$ and $p_2$  form a basis for $V$ and it's clear that there exist $\alpha, \beta \in F^\times$ such that,
\begin{align*} p_3 = \alpha p_1 + \beta p_2.
\end{align*}

Now, let $Q_1$, $Q_2$ and $Q_3$ be three more distinct points of $\mathscr{L}$ and $q_i$ be a vector in $V$ corresponding to $Q_i$. Similarly, by the above argument, there exist $\gamma, \delta \in F^\times$ such that,
\begin{align*} q_3 = \gamma q_1 + \delta q_2.
\end{align*}

Let $\pi \in GL(2,F)$ be the linear transformation which sends $\alpha p_1$ to $\gamma q_1$  and $\beta p_2$ to $\delta q_2$. Thus,
\begin{align*} \pi(p_3) = \pi(\alpha p_1 + \beta p_2) = \pi(\alpha p_1) + \pi(\beta p_2) = \gamma q_1 + \delta q_2 = q_3 
\end{align*}

Hence we get $P^\pi_1 = Q_1$, $P^\pi_2 = Q_2$ and $P^\pi_3 = Q_3$ and $GL(2,F)$ is triply transitive. Now set,
\begin{align*} \eta = \sqrt{\frac{1}{\text{det }\pi}}.
\end{align*}

Consider the mapping $\theta$ which sends $\alpha p_1$ to $\eta \gamma q_1$ and $\beta p_2$ to $\eta \delta q_2$. Observe that,
\begin{align*} \text{det }\theta = \eta^2 \, \text{det } \pi = 1
\end{align*}

So $\theta \in SL(2,F) = \SL_2(F)$and since $P^\theta_1 = Q_1$, $P^\theta_2 = Q_2$ and $P^\theta_3 = Q_3$, we have that $\SL_2(F)$is also triply transitive. 

\end{proof}

The following proposition looks at what happens when the group $\SL_2(F)$acts on the projective line $\mathscr{L}$.

\begin{proposition} \label{6.7} (i) Each element of the form $d_\delta$ (with $\delta \neq \pm 1$), fixes the same two points on the projective line $\mathscr{L}$ and fix no other point. \vspace{3mm} \\
(ii) Each element of the form $\pm s_\sigma$ (with $\sigma \neq 0$), fixes the same point $P$ on $\mathscr{L}$ and fix no other point. Furthermore, \emph{Stab}$(P) = H$. \vspace{3mm} \\
(iii) All conjugate elements have the same number of fixed points on $\mathscr{L}$. \vspace{3mm} \\
(iv) Any noncentral element of $\SL_2(F)$has at most 2 fixed points on $\mathscr{L}$.
\end{proposition}

\begin{proof} 
(i) Let $P$ be a fixed a point of an arbitrary $d_\delta \in D$, with $\delta \neq \pm 1$ and let $u$ belong to the corresponding equivalence class of vectors of $V$ to $P$. \\
\begin{align*} d_\delta u = \begin{bmatrix} \delta & 0 \\ 0 & \delta^{-1} \end{bmatrix} \begin{bmatrix} u_1 \\ u_2 \end{bmatrix} &= \begin{bmatrix} u_1 \delta \\ u_2 \delta^{-1} \end{bmatrix} \sim \begin{bmatrix} u_1 \\ u_2 \end{bmatrix}, 
\\[1.5ex] u_1 u_2 \delta &= u_1 u_2 \delta^{-1}.
\end{align*}

Since $\delta \neq \pm 1$, $\delta$ does not equal $\delta^{-1}$, and so either $u_1 = 0$ or $u_2 = 0$. Thus $u$ is equivalent to either the vector $\begin{bmatrix} 0 \\ 1 \end{bmatrix}$ or $\begin{bmatrix} 1 \\ 0 \end{bmatrix}$ and these correspond to 2 distinct points of $\mathscr{L}$ which are fixed by $d_\delta$. \\
\\
(ii) Let $P$ be a fixed a point of an arbitrary $s_\sigma$, with $\sigma \neq 0$, and let $u$ be the corresponding element of $V$ to $P$. \\
\begin{align*} s_\sigma u = \begin{bmatrix} 1 & 0 \\ \sigma & 1 \end{bmatrix} \begin{bmatrix} u_1 \\ u_2 \end{bmatrix} &= \begin{bmatrix} u_1 \\ u_1 \sigma + u_2 \end{bmatrix} \sim \begin{bmatrix} u_1 \\ u_2 \end{bmatrix}, 
\\[1.5ex] u_1 u_2 &= {u_1}^2 \sigma + u_1 u_2.
\end{align*}

This gives ${u_1}^2 \sigma = 0$ and since $\sigma \neq 0$ we have $u_1 = 0$. Thus $s_\sigma$ has just one fixed point, $P$ which corresponds to the equivalence class of $\begin{bmatrix} 0 \\ 1 \end{bmatrix}$ in $V$. We show also that $P$ is also the only fixed point of $-s_\sigma$, with $\sigma \neq 0$.
\begin{align*} -s_\sigma u = \begin{bmatrix} -1 & 0 \\ \sigma & -1 \end{bmatrix} \begin{bmatrix} u_1 \\ u_2 \end{bmatrix} &= \begin{bmatrix} -u_1 \\ u_1 \sigma - u_2 \end{bmatrix} \sim \begin{bmatrix} u_1 \\ u_2 \end{bmatrix}, 
\\[1.5ex] -u_1 u_2 &= {u_1}^2 \sigma - u_1 u_2.
\end{align*}

So again $u_1 =0$ and $-s_\sigma$ fixes $P$ and no other point. We now calculate the stabiliser of $P$ in $L$, by considering which $x \in \SL_2(F)$fix $P$. \\
\begin{align*} x u = \begin{bmatrix} \alpha & \beta \\ \gamma & \delta \end{bmatrix} \begin{bmatrix} 0 \\ 1 \end{bmatrix} &= \begin{bmatrix} \beta \\ \delta \end{bmatrix} \sim \begin{bmatrix} 0 \\ 1 \end{bmatrix}.
\end{align*}

Thus $\beta = 0$ and $x \in H$. Since $x$ was chosen arbitrarily from Stab$(P)$, we have Stab$(P) \subset H$. Now let an arbitrarily chosen $y \in H$ act on $P$. \\
\begin{align*} y u = \begin{bmatrix} \alpha & 0 \\ \gamma & \alpha^{-1} \end{bmatrix} \begin{bmatrix} 0 \\ 1 \end{bmatrix} &= \begin{bmatrix} 0 \\ \alpha^{-1} \end{bmatrix} \sim \begin{bmatrix} 0 \\ 1 \end{bmatrix}.
\end{align*}

Thus $y$ and indeed $H$ is contained in Stab$(P)$, so Stab$(P) = H$ as desired. \\
\\
(iii) Let $P_i$ $(i = 1,2,...)$ be the fixed points of $x\in \SL_2(F)$and let $y$ be conjugate to $x$ in $L$. That is, there exists a $g \in \SL_2(F)$such that $x = gyg^{-1}$.
\begin{align*} x P_i &= P_i,
\\ gyg^{-1} P_i &= P_i,
\\ y(g^{-1} P_i) &= (g^{-1} P_i).
\end{align*}

This shows that $P_i$ is a fixed point of $x$ if and only if $g^{-1} P_i$ is a fixed point of $y$. Thus conjugate elements have the same number of fixed points. \\
\\
(iv) By Proposition \ref{ISL2_IsConj_d_or_IsConj_s_or_IsConj_neg_s_of_AlgClosed} every  element of $\SL_2(F)$is conjugate to either $d_\delta$ or $\pm s_\sigma$, so since conjugate elements have the same number of fixed points, every element of $\SL_2(F)\! \setminus \! Z$ has either the same number of fixed points as $d_\delta$ (with $\delta \neq \pm 1$), namely 2, or the same number as $\pm s_\sigma$, (with $\sigma \neq 0$), namely 1.

\end{proof}



\chapter[The Maximal Abelian Subgroup Class Equation]{The Maximal Abelian Subgroup Class Equation}\label{Ch6_MaximalAbelianSubgroupClassEquation}
% \chaptermark{The Class Equation}

\section[A finite subgroup of $\SL_2(F)$]{A Finite Subgroup of $\pmb{L}$}

We now return to the realm of finite groups and consider $G$ to be an arbitrary finite subgroup of $\SL_2(F)$. We will still continue to use $Z$ to denote the centre of $\SL_2(F)$, and will use $Z(G)$ whenever we refer to the centre of $G$. \\
\\
Observe that if $Z$ is not contained in $G$, then $Z$ must contain a non-identity element, thus $|Z| = 2$ and $p \neq 2$ by Lemma \ref{6.2}. Recall that $\SL_2(F)$ has a unique element of order 2 by Lemma \ref{6.2b}, $- I_L$, which is not in $G$, therefore $G$ has no element of order 2. \\
\\
By Cauchy's Theorem, which says that if a prime $p$ divides the order of a finite group, then the group contains an element of order $p$, we deduce that 2 does not divide the order of $G$. \\
\\
This means that $|G|$ and $|Z|$ are relatively prime, so $G \cap Z = \{ I_L \}$ and we can use Corollary \ref{directproductZ} to show that $GZ \cong G \times Z$. This shows that regardless of whether $G$ contains $Z$ or not, its structure is uniquely determined by $GZ$, so it suffices to only consider the case when $Z \subset G$. 

\section{Maximal Abelian Subgroups}

\begin{definition}[Maximal Abelian Subgroup]
\label{IsMaximalAbelian}
\lean{IsMaximalAbelian}
\leanok
Let $H$ and $J$ be subgroups of a group $G$ where $H$ is abelian. $H$ is called \textbf{maximal abelian} if $J$ is not abelian whenever $H \subsetneq J$.
\end{definition}

\begin{remark}
The definition was stated in positive form:

A subgroup $H$ is said to be a maximal abelian subgroup of $G$ if for every $J$ subgroup of $G$ satisfying $H \le J$ we have that $J \le H$. Which overall implies $H = J$ by antisymmetry of the preorder.

In Lean this statement looks like the following:

\begin{verbatim}
  def IsMaximalAbelian {L : Type*} [Group L] (G : Subgroup L) : Prop := Maximal (IsCommutative) G
\end{verbatim}

  where the definition of \texttt{Maximal} in mathlib implicitly recognises the existence of a $\le$ operator (a more primitive notion of a partial order) and is:

  \begin{verbatim}
    def Maximal (P : α → Prop) (x : α) : Prop := P x ∧ ∀ ⦃y⦄, P y → x ≤ y → y ≤ x
  \end{verbatim}

  Which informally means that an object $M$ that satisfies a property is maximal if any other object $K$ that also satisfies the property and is related to $M$ by $M \le K$ then in fact we must have the symmetric relation $K \le M$.

  When $\le := \subseteq$ then this is the natural notion of maximal.
\end{remark}


\begin{definition}[Elementary Abelian]
\label{IsElementaryAbelian}
\lean{IsElementaryAbelian}
\leanok
A group $G$ is said to be \textbf{elementary abelian} if it is abelian and every non-trivial element has order $p$, where $p$ is prime.
\end{definition}

\begin{remark}
  In Lean we define the notion of a subgroup of $H$ of $G$ being elementary abelian the following way:
  
  \begin{verbatim}
  def IsElementaryAbelian {G : Type*} [Group G] (p : ℕ) (H : Subgroup G) : Prop :=
  IsCommutative H ∧ ∀ h : H, h ≠ 1 → orderOf h = p
  \end{verbatim}
\end{remark}

\begin{definition}
\label{MaximalAbelianSubgroupsOf}
\uses{IsMaximalAbelian}
\lean{MaximalAbelianSubgroupsOf}
\leanok
Let $\mathfrak{M}$ denote the set of all maximal abelian subgroups of $G$.
\end{definition}

\begin{remark}
  When a set/object with some additional structure has been defined informally, when one wants to formalise results about the object it is often the case a decision has to be made
  about whether the set is defined in Lean as a set or whether it is defined as its own type. In this case, I have opted to define it as a set but later on when using quotients we shall see an example
  of how it is beneficial to define an object as a type/subtype in its own right.
\end{remark}

\vspace{3mm}

Maximal abelian subgroups play an important role in determining the structure of $G$. In particular, every element in $G$ must be contained in some maximal abelian subgroup, since every element commutes at least with itself and $Z$. This will allow us to decompose $G$ into the conjugacy classes of these maximal abelian subgroups. Note also that unless $G=Z$, $Z$ is not a maximal abelian subgroup, because for each $x \in G \! \setminus \! Z$, $\langle Z,x \rangle$ is clearly a larger abelian subgroup than $Z$. \\
\\
We will shortly prove an important theorem regarding the maximal abelian subgroups of $G$, but in order to do so we require the following two lemmas. \\

\begin{lemma}
  \label{IsElementaryAbelian.dvd_card}
  \lean{IsElementaryAbelian.dvd_card}
  \leanok
If $G$ is a finite group of order $p^m$ where $p$ is prime and $m > 0$, then $p$ divides $|Z(G)|$. 
\end{lemma}

\begin{proof}
Let $C(x)$ be the set of elements of $G$ which are conjugate in $G$ to $x$, we call this the conjugacy class of $x$. Bhattacharya shows that the set of all conjugacy classes form a partition of $G$ \cite[p.112]{bhattacharya}. Now consider the following rearranged class equation of $G$, where $S$ is a subset of $G$ containing exactly one element from each conjugacy class not contained in $Z(G)$. 
 
\begin{equation} \label{cen2}
|G| - \sum_{x \in S} [G:N_G(x)] = |Z(G)|.
\end{equation}

Since $|G| = p^m$, each subgroup of $G$ is of order $p^k$ for some $k \leq m$. In particular each $N_G(x)$ has order $p^k$ and is strictly contained in $G$ since $x \not \in Z(G)$ by assumption. Thus each $[G:N_G(x)] > 1$, and are therefore divisible by $p$. Since $p$ divides the left hand side of (\ref{cen2}), it must also divide the right, thus $p$ divides $|Z(G)|$. 

\end{proof}



\begin{lemma}
  \label{coprime_card_fin_subgroup_of_inj_hom_group_iso_units}
  \lean{coprime_card_fin_subgroup_of_inj_hom_group_iso_units}
  \leanok
Every finite subgroup of a multiplicative group of a field is cyclic.
\end{lemma}
%PROOF AND DEPENDENCIES

\begin{proof} 
  See \cite[p.41]{suzuki}.
\end{proof}

\begin{theorem}
  \label{MaximalAbelianSubgroup.centralizer_meet_G_in_MaximalAbelianSubgroups_of_noncentral}
  \uses{IsCommutative_centralizer_of_not_mem_center, MaximalAbelianSubgroupsOf}
  \lean{MaximalAbelianSubgroup.centralizer_meet_G_in_MaximalAbelianSubgroups_of_noncentral}
  \leanok 
  Let $G$ be an arbitrary finite subgroup of $\SL_2(F)$ containing $Z$. \\
If $x \in G \! \setminus \! Z$ then we have $C_G(x) \in \mathfrak{M}$. \vspace{3mm} \\
\end{theorem}
\begin{proof}
  Let $x$ be chosen arbitrarily from $G \! \setminus \! Z$. Then by Corollary \ref{6.5}, $C_{\SL_2(F)}(x)$ is abelian. By definition, $C_G(x) = C_{\SL_2(F)}(x) \cap G$, and using the elementary fact that the intersection of two subgroups is itself a subgroup, we have $C_G(x) < C_{\SL_2(F)}(x)$. Now since every subgroup of an abelian group is abelian, $C_G(x)$ is also abelian. \\
  \\
  Now let $J$ be a maximal abelian subgroup of $G$ containing $C_G(x)$. Since $J$ is abelian and $x \in C_G(x) \subset J$, we have $jx=xj$, $\forall j \in J$, thus $J \subset C_G(x)$. Therefore $J=C_G(x)$ and $C_G(x) \in \mathfrak{M}$. \\
\end{proof}

Before we continue proving properties about Maximal Abelian Subgroups we first need to understand how commutative subgroups interact with other subgroups. 
We now list a few results about commutative subgroups and their interaction with other subgroups:

\begin{remark}
\label{IsCommutative_of_IsCommutative_subgroupOf}
\lean{IsCommutative_of_IsCommutative_subgroupOf}
\leanok
Let $H, K$ be two subgroups of a group $G$ then $H \sqcap K = H \cap K$ is commutative if $H \sqcap K$ regarded as a subgroup of $K$ is commutative.
\end{remark}

\begin{remark}
  The remark above \ref{IsCommutative_of_IsCommutative_subgroupOf} probably seems ridiculous, but Lean genuinely understands both objects as belonging to completely different types and 
  this result is necessary to be able to jump between the corresponding contexts.
\end{remark}

\begin{definition}
  \label{center_mul}
  \lean{center_mul}
  \leanok
  Let $H$ be a subgroup of a group $G$ then the pointwise set product $Z(G) H$ is a subgroup of $G$
\end{definition}
\begin{proof}
\begin{enumerate}
  \item \texttt{one_mem'}: Both $Z(G)$ and $H$ are subgroups of $G$ so they contain the identity element, thus $1 \cdot 1 \in Z(G) H$.
  \item \texttt{mul_mem'}: Let $z_1 h_1, z_2 h_2 \in Z(G) H$ then $z_1h_1z_2h_2 = z_1z_2 h_1h_2 \in Z(G) H$ as $z_i$ is in the center.
  \item \texttt{inv_mem'}: Let $zh \in Z(G) H$ then $z^{-1} h^{-1} \in Z(G) H$ and $z h z^{-1} h^{-1} = zz^{-1}h h^{-1} = 1$.

\end{enumerate}
\end{proof}


\begin{lemma}
  \label{center_mul_subset_center_mul}
  \uses{center_mul}
  \lean{center_mul_subset_center_mul}
  \leanok
\end{lemma}

\begin{lemma}[ The join of a commutative subgroup with the center of a group is commutative]
  \label{IsComm_of_center_join_IsComm}
  \uses{center_mul_subset_center_mul, center_mul}
  \lean{IsComm_of_center_join_IsComm}
  \leanok

  Let $H$ be a commutative subgroup of $G$ then the subgroup $Z(G) \sqcup H$  is a commutative subgroup of $G$.
\end{lemma}
\begin{proof}
  Let $x, y \in Z(G) \sqcup H$ recalling that the supremum can be thought of taking the closure
  we know that if $x$ and $y$ belong to the closure then since $Z(G) H$ is a subgroup of $G$ and $Z(G) \sqcup H \subseteq Z(G) H$
  we know that $x, y \in Z(G) H$ and thus there exist $z_1 h_1 = x$ and $z_2 h_2 = y$. Therefore, we can now show that $x$ and $y$ commute:

  \begin{align*}
  x y &= z_1 h_1 z_2 h_2\\
  & = z_1 z_2 h_1 h_2 \tag{as $z_2$ is in the center}\\
  &= z_2 z_1 h_2 h_1 \tag{as $H$ is a commutative subgroup}\\
  &= z_2 h_2 z_1 h_1 \tag{as $z_1$ is in the center}
  \end{align*}
\end{proof}

\begin{lemma}[$Z$ is contained within any Maximal Abelian Subgroup of a subgroup containing $Z$]
  \label{MaximalAbelianSubgroup.center_le}
  \uses{IsCommutative_of_IsCommutative_subgroupOf, IsComm_of_center_join_IsComm}
  \lean{MaximalAbelianSubgroup.center_le}
  Let $H$ be a subgroup of $G$, if $Z(G) \le H$ then for any maximal abelian subgroup of $H$, $A$ we have that $Z(G) \le A$ 
  \leanok
\end{lemma}
\begin{proof}
  
\end{proof}

\begin{lemma}
\label{MaximalAbelianSubgroup.le_centralizer_of_mem}
\lean{MaximalAbelianSubgroup.le_centralizer_of_mem}
\leanok
Let $H$ be a subgroup of $G$ and let $A$ be a maximal abelian subgroup of $H$, and let $x \in A$ then $A \le C_G(x)$.
\end{lemma}

\begin{lemma}
  \label{MaximalAbelianSubgroup.not_le_of_ne}
  \lean{MaximalAbelianSubgroup.not_le_of_ne}
  \leanok
  Let $H$ be a subgroup of a group $G$ and let $A \ne B$ be maximal abelian subgroups of $H$ then $B \not\le A$.
\end{lemma}
\begin{proof}
Suppose for a contradiction that $B \le A$, then by the maximality of $B$ and because $A$ is commutative as it is maximal abelian we must have that $A \le B$.
But this shows $A = B$ by antisymmetry, a contradiction.
\end{proof}

\begin{lemma}
  \label{MaximalAbelianSubgroup.lt_cen_meet_G}
  \uses{MaximalAbelianSubgroup.not_le_of_ne, MaximalAbelianSubgroup.le_centralizer_of_mem}
  \lean{MaximalAbelianSubgroup.lt_cen_meet_G}
  \leanok
  Let $H$ be a subgroup of $G$, let $A \ne B$ be maximal abelian subgroups of $H$ and let $x \in A \cap B$ then $A < C_G(x) \sqcap H$.
\end{lemma}

\begin{theorem}
  \label{MaximalAbelianSubgroup.center_eq_meet_of_ne_MaximalAbelianSubgroups}
  \uses{MaximalAbelianSubgroup.centralizer_meet_G_in_MaximalAbelianSubgroups_of_noncentral, MaximalAbelianSubgroup.center_le, MaximalAbelianSubgroup.lt_cen_meet_G, MaximalAbelianSubgroup.le_centralizer_of_mem}
  \lean{MaximalAbelianSubgroup.center_eq_meet_of_ne_MaximalAbelianSubgroups}
For any two distinct subgroups $A$ and $B$ of $\mathfrak{M}$, we have
\begin{align*} A \cap B = Z. \end{align*}
\end{theorem}

\begin{proof}
  Consider $x \in A \cap B$. Since both $A$ and $B$ are abelian, $x$ commutes with each $a \in A$ and $b \in B$ and thus $C_G(x)$ contains both $A$ and $B$.  If $x \in G \setminus Z$, then $C_G(x) \in \mathfrak{M}$ by \ref{MaximalAbelianSubgroup.centralizer_meet_G_in_MaximalAbelianSubgroups_of_noncentral} and because $A$ and $B$ are distinct we have $A \subsetneq A \cup B \subset C_G(x)$. 
  This contradicts the fact that $A$ is maximum abelian and thus $x \in Z$. Finally, note that Z is contained in every maximal abelian subgroup, since otherwise we would have the contradiction that $\langle A, Z \rangle$ would generate a larger abelian subgroup than $A$. Hence $A \cap B = Z$. \\
\end{proof}

%---------------------------------------

\begin{lemma}
\label{MaximalAbelianSubgroup.singleton_of_cen_eq_G}
\uses{MaximalAbelianSubgroup.center_le, MaximalAbelianSubgroupsOf, IsMaximalAbelian}
\lean{MaximalAbelianSubgroup.singleton_of_cen_eq_G}
\leanok
Let $H$ be a subgroup of $G$ and suppose $H = Z(G)$ then the maximal abelian subgroups are $\mathfrak{M} = \{Z(G)\}$.
\end{lemma}
\begin{proof}
  We show that $A \in \mathfrak{M}$ if and only if $A = Z(G)$
  \begin{itemize}
    \item[$\Rightarrow$] Suppose $A$ is a maximal abelian subgroup of $H$, then by \ref{MaximalAbelianSubgroup.center_eq_meet_of_ne_MaximalAbelianSubgroup.center_le} $Z(G) \le A$. Furthermore, $A \le H = Z(G)$; which overall shows $A = Z(G)$ as required.
    \item[$\Leftarrow$] Suppose $A = Z(G)$ we now show that $A$ is a maximal abelian subgroup. 
      On the one hand, $A = Z(G)$ so it follows that $A$ is abelian.
      On the other hand, we need to show that $Z(G)$ is maximal. Let $B$ be a subgroup of $H$ that is commutative and such that $Z(G) \sqcap H \le B$, we show that it follows that $B \le Z(G) \sqcap H$. But this follows trivially as 
      $B \leq H = Z(G) \sqcap H = \top$.
  \end{itemize}
\end{proof}

\begin{lemma}
  \label{MaximalAbelianSubgroup.IsCyclic_and_card_Coprime_CharP_of_center_eq}
  \uses{MaximalAbelianSubgroup.singleton_of_cen_eq_G, SpecialSubgroups.card_Z_eq_two_of_two_ne_zero, SpecialSubgroups.IsCyclic_Z, SpecialSubgroups.card_Z_eq_one_of_two_eq_zero}
  \lean{MaximalAbelianSubgroup.IsCyclic_and_card_Coprime_CharP_of_center_eq}
  \leanok
  If $G = Z(G)$ then an element $A$ of $\mathfrak{M}$, the maximal abelian subgrups of $G$ is a cyclic group whose order is relatively prime to $p$.
\end{lemma}
\begin{proof}
  Here $G$ is the only element of $\mathfrak{M}$. If $p \neq 2$ then $|G|=2$ and $G$ is a cyclic group whose order is relatively prime to $p$. If $p=2$ then $G = I_G$ which is trivially a $S_p$-subgroup. \\
\end{proof}

\begin{remark}
  \label{mem_centralizer_self}
  \lean{mem_centralizer_self}
  \leanok
 Let $G$ be a group then centralizer of an element $x \in G$, $C_G(x)$ contains $x$. 
\end{remark}

\begin{lemma}
  \label{MaximalAbelianSubgroup.center_not_mem}
  \uses{mem_centralizer_self, MaximalAbelianSubgroup.centralizer_meet_G_in_MaximalAbelianSubgroups_of_noncentral}
  \lean{MaximalAbelianSubgroup.center_not_mem}
  \leanok
  Let $F$ be an algebraically closed field, let $G$ be a subgroup of $\SL_2(F)$ where $G \ne Z(\SL_2(F))$ then the center is not a maximal abelian subgroup of $G$, $Z(G) \notin \mathfrak{M}$.
\end{lemma}
% PROOF

\begin{lemma}
  \label{MaximalAbelianSubgroup.le_centralizer_meet}
  \uses{IsCommutative_of_IsCommutative_subgroupOf}
  \lean{MaximalAbelianSubgroup.le_centralizer_meet}
  \leanok

  Let $H$ be a subgroup of a group $G$, let $A$ be a maximal abelian subgroup of $H$, and suppose $x \in A \subseteq G$ then 
  $A \le C_{\SL_2(F)} \sqcap H$.
\end{lemma}
%PROOF

\begin{lemma}
  \label{MaximalAbelianSubgroup.eq_centralizer_meet_of_center_lt}
  \uses{MaximalAbelianSubgroup.centralizer_meet_G_in_MaximalAbelianSubgroups_of_noncentral, MaximalAbelianSubgroup.le_centralizer_meet}
  \lean{MaximalAbelianSubgroup.eq_centralizer_meet_of_center_lt}
  \leanok
  Let $F$ be an algebraically closed field, let $G$ and $A$ be a subgroup of $\SL_2(F)$ where $A$ is a maximal abelian subgroup of $G$ and $Z(\SL_2(F)) < A$ then there exists an element $x \in G \setminus Z(SL(2)) \subseteq \SL_2(F)$ such that
  $A = C_{\SL_2(F)}(x) \sqcap G = C_{G}(x)$.
\end{lemma}
%PROOF

\begin{theorem}
  \label{MaximalAbelianSubgroup.IsCyclic_and_card_coprime_CharP_of_IsConj_d}
  \uses{SpecialSubgroups.center_SL2_eq_Z, conjugate_centralizers_of_IsConj, centralizer_d_eq_D, SpecialMatrices.d, SpecialSubgroups.D_iso_units, 
    coprime_card_fin_subgroup_of_inj_hom_group_iso_units}
  \lean{MaximalAbelianSubgroup.IsCyclic_and_card_coprime_CharP_of_IsConj_d}
  \leanok
  Let $F$ be an algebraically closed field of characteristic $p$ and let $G$ be a finite subgroup of $\SL_2(F)$ containing $Z$, let $A$ be a subgroup of $\SL_2(F)$ which is a maximal abelian subgroup of $G$ and furthermore suppose 
  that $A = C_{\SL_2(F)}(x) \sqcap G$ where $x \in \SL_2(F) \setminus Z$ and that $x$ is conjugate to $d_\delta$ for some $\delta \in F^\times$ then $A$ is cyclic and the cardinality of $A$ is coprime to $p$.
\end{theorem}
%PROOF

To prove the statement when $x$ is conjugate to $s_\sigma$ for some $\sigma \in F$ we first need the following lemmas:

\begin{lemma}
  \label{MaximalAbelianSubgroup.centralizer_eq_conj_SZ_of_IsConj_s_or_IsConj_neg_s}
  \lean{MaximalAbelianSubgroup.centralizer_eq_conj_SZ_of_IsConj_s_or_IsConj_neg_s}
  \leanok
\end{lemma}

We need the following computations which essentially makes allowances which let us think of the complete lattice structure with the further property of
being a distributive lattice, that is, $(H \sqcup K) \sqcap L = (H \sqcap L) \sqcup (K \sqcap L)$.


\begin{lemma}
  \label{MaximalAbelianSubgroup.conj_T_join_Z_meet_G_eq_conj_T_meet_G_join_Z}
  \uses{SpecialSubgroups.center_SL2_eq_Z, SpecialSubgroups.S, SpecialSubgroups.Z}
  \lean{MaximalAbelianSubgroup.conj_T_join_Z_meet_G_eq_conj_T_meet_G_join_Z}
  \leanok

  Let $c \in \SL_2(F)$  and $G$ be a subgroup of $\SL_2(F)$ then $c(S \sqcup Z)c^{-1} \sqcap G = (cSc^{-1} \sqcap G) \sqcup Z$
\end{lemma}
% PROOF AND DEPENDENCIES

We also need the following computation:
\begin{lemma}
\label{MaximalAbelianSubgroup.conj_inv_conj_eq}
\uses{SpecialSubgroups.center_SL2_eq_Z}
\lean{MaximalAbelianSubgroup.conj_inv_conj_eq}
\leanok
Let $c \in \SL_2(F)$ and $G$ be a subgroup of $\SL_2(F)$ then 
\[
c^{-1}(c(S \sqcap G)c^{-1} \sqcup Z)c = (S \sqcap c^{-1}Gc) \sqcup Z
\]
\end{lemma}
% PROOF AND DEPENDENCIES

\begin{remark}
  \label{IsElementaryAbelian.subgroupOf}
  \lean{IsElementaryAbelian.subgroupOf}
  \leanok
  If a subgroup $H$ of a group $G$ is an elementary abelian subgroup then for any subgroup $K$ we have that $H \sqcap K$ is also an elementary abelian subgroup.
\end{remark}


\begin{lemma}
  \label{MaximalAbelianSubgroup.exists_noncenter_of_card_center_lt_card_center_Sylow}
  \uses{SpecialSubgroups.center_SL2_eq_Z, SpecialSubgroups.card_Z_eq_one_of_two_eq_zero, SpecialSubgroups.card_Z_eq_two_of_two_ne_zero, SpecialSubgroups.Z }
  \lean{MaximalAbelianSubgroup.exists_noncenter_of_card_center_lt_card_center_Sylow}
  \leanok
  Let $G$ be a finite subgroup of $\SL_2(F)$, let $S$ be a $p$-Sylow subgroup of $G$ where $p$ is the characteristic of the field $F$ and furthermore suppose $p \le |Z|$ then
  there exists a noncentral element in $S$, that is, $S \setminus Z \ne \varnothing$.
\end{lemma}
%PROOF AND DEPENDENCIES

To show the Sylowness of the subgroup we shall construct we need the following lemma:

\begin{lemma}
 \label{MaximalAbelianSubgroup.mul_center_inj}
 \uses{SpecialSubgroups.center_SL2_eq_Z}
 \lean{MaximalAbelianSubgroup.mul_center_inj}
 \leanok
 Let $S$ and $Q$ be subgroups of a group $\SL_2(F)$ where $S \le Q$ and furthermore, we have the added condition that either $I = -I$ or $-I \notin S$ and suppose $SZ = QZ$ then
 $S = Q$
\end{lemma}
%PROOF AND DEPENDENCIES

\begin{theorem}
\label{MaximalAbelianSubgroup.A_eq_Q_join_Z_of_IsConj_s_or_neg_s}
\uses{MaximalAbelianSubgroup.centralizer_eq_conj_SZ_of_IsConj_s_or_IsConj_neg_s, SpecialSubgroups.S_join_Z_eq_SZ, MaximalAbelianSubgroup.conj_T_join_Z_meet_G_eq_conj_T_meet_G_join_Z, MaximalAbelianSubgroup.conj_inv_conj_eq, SpecialSubgroups.center_SL2_eq_Z,
  SpecialMatrices.order_s_eq_char, orderOf_injective, MaximalAbelianSubgroup.center_le, 
  MaximalAbelianSubgroup.IsCyclic_and_card_coprime_CharP_of_IsConj_d, IsElementaryAbelian.subgroupOf, IsPGroup.exists_le_sylow,
  MaximalAbelianSubgroup.exists_noncenter_of_card_center_lt_card_center_Sylow, MaximalAbelianSubgroup.mul_center_inj}
\lean{MaximalAbelianSubgroup.A_eq_Q_join_Z_of_IsConj_s_or_neg_s}
\leanok
Let $F$ be an algebraically closed field of characteristic $p$ and let $G$ be a finite subgroup of $\SL_2(F)$ containing $Z$, let $A$ be a subgroup of $\SL_2(F)$ which is a maximal abelian subgroup of $G$ and furthermore suppose $Z < A$ and $A = C_{\SL_2(F)}(x) \sqcap G$ where $x \in G \setminus Z \subseteq \SL_2(F)$ and $x$ is conjugate to $s_\sigma$ for some $\sigma \in F$
then there exists a finite nontrivial elementary abelian Sylow $p$-subgroup of $G$ such that $A = Q \sqcup Z$.
\end{theorem}
%PROOF


\begin{lemma}
\label{MaximalAbelianSubgroup.IsCyclic_and_card_coprime_CharP_or_eq_Q_join_Z_of_center_ne}
\uses{MaximalAbelianSubgroup.center_not_mem, MaximalAbelianSubgroup.eq_centralizer_meet_of_center_lt, SL2_IsConj_d_or_IsConj_s_or_IsConj_neg_s_of_AlgClosed, MaximalAbelianSubgroup.A_eq_Q_join_Z_of_IsConj_s_or_neg_s}
\lean{MaximalAbelianSubgroup.IsCyclic_and_card_coprime_CharP_or_eq_Q_join_Z_of_center_ne}
\leanok
If $G \ne Z(G)$ then an element of $A$ of $\mathfrak{M}$, the maximal abelian subgroups of $G$, is either cyclic group whose order is relatively prime to $p$, the characteristic of the field $F$; or of the form $Q \times Z = Q \sqcup Z$ where $Q$ is an elementary abelian Sylow $p$-subgroup of $G$.
\end{lemma}
\begin{proof}
  Since $Z \not \in \mathfrak{M}$, each $A \in \mathfrak{M}$ contains at least one $x \not \in Z$. By Proposition  \ref{6.3} this $x$ is conjugate to either $d_\delta$ or $\pm s_\sigma$ in $\SL_2(F)$. It suffices to only consider these cases: \\
  \\
   \space $\pmb{x}$ \textbf{conjugate to} $\pmb{d_\delta}$ \textbf{in} $\pmb {L}$. There is a $y \in L$ such that $x = y d_\delta y^{-1}$. Since $x \not \in Z$, we have $d_\delta \not \in Z$, because otherwise we get the contradiction,
  \begin{align*} x =  y d_\delta y^{-1} = d_\delta \in Z.
  \end{align*}
  Thus $\omega \neq \pm 1$. Let $A = C_G(x)$, since $C_G(x) \in \mathfrak{M}$ by part (i). Observe that
  \begin{align*}  C_G(d_\delta) &<  C_{\SL_2(F)}(d_\delta)  \tag{see proof of (i)}
  \\ &= D  \tag{by Lemma \ref{6.4ii}}
  \\ &\cong F^*.  \tag{by Lemma \ref{6.1b}}
  \end{align*}
  
  Since $A$ is conjugate to $C_G(d_\delta)$ by Proposition \ref{conjcent}, we have that $A$ is isomorphic to a finite subgroup of $F^*$ and by Lemma \ref{finsubcyc}, $A$ is cyclic. By Lagrange's Theorem any finite subgroup of $F^*$ has an order which divides $p^m - 1$ for some $m \in \mathbb{Z}^+$, and since $p \nmid (p^m - 1)$, $|A|$ is relatively prime to $p$. \\
  \\
   \space $\pmb{x}$ \textbf{conjugate to} $\pmb{\pm s_\sigma}$ \textbf{in} $\pmb{L}$. Again let $A = C_G(x) \in \mathfrak{M}$. $A$ is conjugate to $C_G({\pm s_\sigma})$ in $\SL_2(F)$ by Proposition \ref{conjcent}. Since $x \notin Z$, we have $\lambda \neq 0$. Observe that
  \begin{align*}  C_G({\pm s_\sigma}) &<  C_{\SL_2(F)}({\pm s_\sigma})
  \\&= T \times Z  \tag{by Lemma \ref{6.4i}}
  \\&\cong F \times Z. \tag{by Lemma \ref{6.1b}}
  \end{align*}
  
  So $A$ is isomorphic to a finite subgroup of $F \times Z$, call it $Q \times Z$. Now $A = Q \times Z \cong QZ$ by Corollary \ref{directproductZ}, which means that an arbitrary element of $A$ is of the form $q_1z_1$, where $q_1 \in Q$, $z_1 \in Z$.
  \begin{align*} q_1z_1q_2z_2 &= q_2z_2 q_1z_1, \tag{$A \in \mathfrak{M}$}
  \\ q_1q_2z_1z_2 &= q_2q_1z_1z_2, \tag{$z_1$, $z_2 \in Z$}
  \\  q_1q_2z_1z_2(z_1z_2)^{-1} &= q_2q_1z_1z_2(z_1z_2)^{-1},
  \\ q_1q_2 &= q_2q_1.
  \end{align*}
  Thus $Q$ is also abelian. Recall from the proof of Proposition \ref{6.3}(ii) that all non-trivial elements of $S$ have order $p$, so each non-trivial element of $Q$ has order $p$ which means that $Q$ is elementary abelian. Thus $Q$ has order $p^m$, for some $m \in \mathbb{Z}^+$. \\
  \\
  Now let $S$ be a Sylow $p$-subgroup containing $Q$. We apply Lemma \ref{IsElementaryAbelian.dvd_card} to determine that $p$ divides $|Z(S)|$, moreover $|Z(S)| \geq p$. \\
  \\
  If $p=2$, then $Z=I_L$ by Lemma \ref{6.2}. So $|Z| = 1$ and hence $|Z(S)| \geq 2 > |Z|$.\\
  If $p > 2$, then  $Z = \langle - I_L \rangle$ also by Lemma \ref{6.2}. So $|Z| = 2$ and again we get $|Z(S)| > 2 = |Z|$. \\
  \\
  So $Z(S)$ must contain at least one element which is not in $Z$, let $y$ be one such element. Let $s_1z_1$ be an arbitrary element of $S \times Z$.
  \begin{align*}
  (s_1z_1)y(s_1z_1)^{-1} &= (s_1z_1)y(z_1^{-1}s_1^{-1})
  \\ &= s_1y(z_1z_1^{-1})s_1^{-1} \tag{since $y \in L$, $z_1 \in Z$}
  \\ &= y(s_1s_1^{-1}) \tag{since $s_1 \in S$, $y \in Z(S)$}
  \\ &= y
  \end{align*}
  
  Thus $s_1z_1 \in C_G(y)$ and since it was chosen arbitrarily, $S \times Z \subset C_G(y)$. Also since $y \in G \! \setminus \! Z$ we have $C_G(y) \in \mathfrak{M}$ by part (i).
  
  \begin{equation*}
  A = Q \times Z \subset S \times Z \subset C_G(y).
  \end{equation*}
  
  Since $A$ and $C_G(y)$ are both in $\mathfrak{M}$ it must be that $A = C_G(y)$. This means $Q = S$ and $Q$ is a Sylow $p$-subgroup of G.\\
\end{proof}

\begin{theorem}
\label{MaximalAbelianSubgroup.IsCyclic_and_card_coprime_CharP_or_eq_Q_join_Z}
\uses{MaximalAbelianSubgroup.IsCyclic_and_card_coprime_CharP_or_eq_Q_join_Z_of_center_ne, MaximalAbelianSubgroup.IsCyclic_and_card_Coprime_CharP_of_center_eq}
\lean{MaximalAbelianSubgroup.IsCyclic_and_card_coprime_CharP_or_eq_Q_join_Z}
\leanok
An element $A$ of $\mathfrak{M}$ is either a cyclic group whose order is relatively prime to $p$, or of the form $Q \times Z$ where $Q$ is an elementary abelian Sylow $p$-subgroup of $G$. \vspace{3mm}
\end{theorem}
\begin{proof}
  First consider the trivial case of $G=Z$.
  By \ref{MaximalAbelianSubgroup.IsCyclic_and_card_Coprime_CharP_of_center_eq} we yield that $A$ is cyclic and has cardinality coprime to $p$.
  \\
  Now assume $G \neq Z$.
  By \ref{MaximalAbelianSubgroup.IsCyclic_and_card_coprime_CharP_or_eq_Q_join_Z_of_center_ne} we yield that $A$ is either a cyclic group whose order is relatively prime to $p$, or of the form $Q \times Z$ where $Q$ is an elementary abelian Sylow $p$-subgroup of $G$.
\end{proof}

\begin{theorem}
  \label{MaximalAbelianSubgroup.index_normalizer_le_two}
  \uses{MaximalAbelianSubgroup.IsCyclic_and_card_coprime_CharP_or_eq_Q_join_Z, normalizer_subgroup_D_eq_DW}
  \lean{MaximalAbelianSubgroup.index_normalizer_le_two}
  \leanok
If $A \in \mathfrak{M}$ and $|A|$ is relatively prime to $p$, then we have $[N_G(A): A] \leq 2$. 

\end{theorem}
\begin{proof}
  (iv) If $|A| \leq 2$ then $A=Z=G$. So $A$ is trivially normal in $G$ and $[N_G(A): A] = 1$. \\
  \\
  Now assume that $|A| > 2$. Since $|A|$ is relatively prime to $p$, we have that $A$ is a cyclic group conjugate to a finite subgroup of $D$ in $\SL_2(F)$ by the proof of part \ref{MaximalAbelianSubgroup.IsCyclic_and_card_coprime_CharP_or_eq_Q_join_Z}, call this subgroup ${\widetilde{A}}$. Thus both ${\widetilde{A}}$ and $D$ have orders greater than 2. Applying Proposition \ref{normalizer_subgroup_D_eq_DW} we observe that
  \begin{align}\label{norm1}  N_{\SL_2(F)}({\widetilde{A}}) = \langle D , w \rangle = N_{\SL_2(F)}(D).
  \end{align}
  
  Since $A$ and ${\widetilde{A}}$ are conjugate in $\SL_2(F)$, there exists an element $z \in L$ such that $zAz^{-1} = {\widetilde{A}}$. This $z$ determines an inner automorphism of $\SL_2(F)$ defined by
  \begin{align*} 
      i_z: L \longrightarrow L,  \qquad \text{where} \quad  i_z(t) = z t z^{-1}  \quad \forall \; t \in L.
  \end{align*}
  
  Let $i_z(G) = {\widetilde{G}}$ denote the image of $G$ under $i_z$. Since $A$ is a maximal abelain subgroup of $G$ it's a simple task to show that ${\widetilde{A}}$ is a maximal abelian subgroup of ${\widetilde{G}}$ and I will leave this to the reader to verify. We now show that $i_z(N_G(A)) = N_{\widetilde{G}}({\widetilde{A}})$ . Take an arbitrary $g \in N_G(A)$.
  \begin{align*} (z g z^{-1}) {\widetilde{A}} (z g z^{-1})^{-1} &= z g (z^{-1} {\widetilde{A}} z) g^{-1} z^{-1}
  \\ &=  z (g A g^{-1}) z^{-1} \tag{since $zAz^{-1} = {\widetilde{A}}$ }
  \\ &= z A z^{-1} \tag{since $g \in N_G(A)$}
  \\ &= {\widetilde{A}}.
  \end{align*}
  
  So $z g z^{-1} = i_z(g) \in N_{\widetilde{G}}({\widetilde{A}})$ and since it was chosen arbitrarily, $i_z(N_G(A)) \subset N_{\widetilde{G}}({\widetilde{A}})$. Now take an arbitrary $z h z^{-1} \in N_{\widetilde{G}}({\widetilde{A}})$.
  \begin{align*} {\widetilde{A}} &= (z h z^{-1}) {\widetilde{A}} (z h z^{-1})^{-1}
  \\ &= z h (z^{-1} {\widetilde{A}} z) h^{-1} z^{-1}
  \\ &= z h A h^{-1} z^{-1}. \tag{since $A = z^{-1} {\widetilde{A}} z$}
  \end{align*}
  
  Now multiplication on the left by $z^{-1}$ and right by $z$ gives:
  \begin{align*} A = z^{-1} {\widetilde{A}} z = h A h^{-1},
  \end{align*}
  
  so $h \in N_G(A)$. Furthermore, $z h z^{-1}$ and indeed the whole of $N_{\widetilde{G}}({\widetilde{A}})$ is contained in $i_z(N_G(A))$. Thus $ i_z(N_G(A)) = N_{\widetilde{G}}({\widetilde{A}})$. In particular, we have,
  \begin{align}\label{6.8iv1} [N_G(A): A] = [N_{\widetilde{G}}({\widetilde{A}}): {\widetilde{A}}].
  \end{align}
  
  Since ${\widetilde{G}} < L$, the normaliser of ${\widetilde{A}}$ in ${\widetilde{G}}$ is simply the normaliser of ${\widetilde{A}}$ in $\SL_2(F)$ restricted to ${\widetilde{G}}$, thus $N_{\widetilde{G}}({\widetilde{A}}) < N_{\SL_2(F)}({\widetilde{A}}) = N_{\SL_2(F)}(D)$ by (\ref{norm1}). Now since $D \vartriangleleft N_{\SL_2(F)}(D)$, the Second Isomorphism Theorem shows that,
  \begin{align}\label{2iso} N_{\widetilde{G}}({\widetilde{A}})/( N_{\widetilde{G}}({\widetilde{A}}) \cap D) \; \cong \; DN_{\widetilde{G}}({\widetilde{A}}) / D.
  \end{align}
  \\
  Clearly ${\widetilde{A}} \subset {\widetilde{G}} \cap D$. We show that this inclusion is infact an equality. Assume that there exists some $d_\delta \in  {\widetilde{G}} \cap D$ which is not in ${\widetilde{A}}$. The group $\langle d_\delta , {\widetilde{A}} \rangle$ is thus an abelian subgroup of ${\widetilde{G}}$, strictly larger than ${\widetilde{A}}$ and contradicting the fact that ${\widetilde{A}}$ is maximal abelian in ${\widetilde{G}}$. Thus ${\widetilde{A}} =  {\widetilde{G}} \cap D$. It is trivial to see that ${\widetilde{A}} \subset N_{\widetilde{G}}({\widetilde{A}}) \cap D$. Also $N_{\widetilde{G}}({\widetilde{A}}) \cap D \subset {\widetilde{G}} \cap D = {\widetilde{A}}$. So,
  \begin{align}\label{parti} {\widetilde{A}} =  N_{\widetilde{G}}({\widetilde{A}}) \cap D.
  \end{align}
  
  Observe also that, 
  \begin{align}\label{index1or2} DN_{\widetilde{G}}({\widetilde{A}}) = \{ D, \langle D, w \rangle \} \subset \langle D, w \rangle = N_{\SL_2(F)}(D).
  \end{align}
  
  Now we piece the preceding results together to give the desired result.
  \begin{align*}  N_{\widetilde{G}}({\widetilde{A}}) / {\widetilde{A}} \; & \cong \;  N_{\widetilde{G}}({\widetilde{A}})/( N_{\widetilde{G}}({\widetilde{A}}) \cap D) \tag{by (\ref{parti})}
  \\ & \cong \; DN_{\widetilde{G}}({\widetilde{A}}) / D \tag{by (\ref{2iso})}
  \\ & \subset N_{\SL_2(F)}(D) / D \tag{by (\ref{index1or2})}
  \\ &= \langle D, w \rangle / D \; \cong \; \mathbb{Z}_2.
  \end{align*}
  
  We have shown that $N_{\widetilde{G}}({\widetilde{A}}) / {\widetilde{A}}$ is isomorphic to a subset of $\mathbb{Z}_2$. Thus by (\ref{6.8iv1}) we have established that, $$[N_G(A): A] = [N_{\widetilde{G}}({\widetilde{A}}): {\widetilde{A}}] \leq 2.$$
  \vspace{-2mm}
\end{proof}


\begin{theorem}
  \label{MaximalAbelianSubgroup.of_index_normalizer_eq_two}
  \uses{MaximalAbelianSubgroup.index_normalizer_le_two}
  \lean{MaximalAbelianSubgroup.of_index_normalizer_eq_two}
  If $A \in \mathfrak{M}$, $|A|$ is relatively prime to $p$, and if $[N_G(A): A] = 2$, then there is an element $y$ of $N_G(A) \! \setminus \! A$ such that, 
  \vspace{-1mm}
  \begin{align*} yxy^{-1} = x^{-1} \qquad \forall x \in A.\end{align*}
  \end{theorem}
\end{theorem}
\begin{proof}
  
  If $[N_G(A): A] = 2$, then the above argument at \ref{MaximalAbelianSubgroup.index_normalizer_le_two} shows that $N_{\widetilde{G}}({\widetilde{A}}) / {\widetilde{A}} \; \cong \; \mathbb{Z}_2$. Thus $DN_{\widetilde{G}}({\widetilde{A}}) = N_{\SL_2(F)}(D) = \langle D, w \rangle$. This means that $N_{\widetilde{G}}({\widetilde{A}})$ contains some element $wd_\omega$. In fact, since $w d_\delta \not \in D$, we have $w d_\delta \in N_{\widetilde{G}}({\widetilde{A}}) \! \setminus \! {\widetilde{A}}$. Take any element $x \in A$. Since ${\widetilde{A}} = zAz^{-1}$, $zxz^{-1} \in {\widetilde{A}}$, call it $d_\sigma$. Let $y = z^{-1}w d_\delta z$. Since $wd_\omega \in N_{\widetilde{G}}({\widetilde{A}}) \! \setminus \! {\widetilde{A}}$ it follows that $y \in N_G(A)\! \setminus \! A$. We show that this $y$ inverts $x$:
  \begin{align*} yxy^{-1} &= (z^{-1}w d_\delta z)(z^{-1} d_\sigma z)(z^{-1}d^{-1}_\omega w^{-1} z)
  \\ &= z^{-1} w d_\delta  d_\sigma d^{-1}_\omega w^{-1} z
  \\ &=  z^{-1} w  d_\sigma  w^{-1} z 
  \\ &=  z^{-1}  d^{-1}_\sigma z  \tag{by Lemma \ref{6.1}}
  \\ &= x^{-1}.
  \end{align*}
\end{proof}


\begin{theorem}
  \label{MaximalAbelianSubgroup.exists_IsCyclic_K_normalizer_eq_Q_join_K}
  \uses{normalizer_subgroup_S_le_L, MaximalAbelianSubgroup.IsCyclic_and_card_coprime_CharP_or_eq_Q_join_Z}
  \lean{MaximalAbelianSubgroup.exists_IsCyclic_K_normalizer_eq_Q_join_K}
  Let $Q$ be a Sylow $p$-subgroup of $G$. If $Q \neq \{I_G\}$, then there is a cyclic subgroup $K$ of $G$ such that $N_G(Q) = Q \sqcup K = QK$. \\
\end{theorem}
\begin{proof}
By part \ref{MaximalAbelianSubgroup.IsCyclic_and_card_coprime_CharP_or_eq_Q_join_Z}, $Q$ is conjugate to a finite subgroup of $S$ in $\SL_2(F)$. In fact, without loss of generality we can assume that $Q \subset S$, moreoever $Q \subset S \cap G$. We show that this is in fact an equality by showing that the reverse inclusion also holds. 
Let $s_\sigma$ be an arbitrary element of $S \cap G$. Then $\langle s_\sigma, Q \rangle$ is a $p$-group of $G$ which must be equal to $Q$ since it is a Sylow $p$-subgroup of $G$. Thus $s_\sigma \in Q$ and
\begin{align}\label{Q=TNG} Q = S \cap G.
\end{align}

Since $|Q| > 1$, Proposition \ref{normalizer_subgroup_S_le_L} gives that $N_G(Q) \subset N_{\SL_2(F)}(Q) \subset H$. So $N_G(Q) \subset H \cap G$. Now take an arbitrarily chosen $d_\delta s_\sigma \in H \cap G$ and $s_\gamma \in Q$.
\begin{align*} (d_\delta s_\sigma) s_\gamma (d_\delta s_\sigma)^{-1} &= d_\delta ( s_\sigma s_\gamma  s_{-\sigma}) d^{-1}_\delta
\\ &=  d_\delta s_\gamma d^{-1}_\delta \tag{by Lemma \ref{6.1}}
\\ &= t_\sigma. \tag{where $\sigma = \mu \omega^{-2}$, by Lemma \ref{6.1}}
\end{align*}

Since it is a product of elements of $G$, $s_\sigma \in S \cap G = Q$ by (\ref{Q=TNG}). Thus $d_\delta s_\sigma \in N_G(Q)$ and indeed the whole of $H \cap G$ is contained in $N_G(Q)$ and
\begin{align}\label{normQ=HNG} N_G(Q) = H \cap G.
\end{align}

We now define a map $\phi$ by,
\begin{align*} \phi : N_G(Q) \longrightarrow D, \qquad \text{where} \quad \! \phi(d_\delta s_\sigma) = d_\delta \quad \forall \; d_\delta s_\sigma \in N_G(Q).
\end{align*}

Next we determine the kernel of $\phi$.
\begin{align*} \ker(\phi) &= \{ d_\delta s_\sigma \in N_G(Q) : \phi(d_\delta s_\sigma) = I_G \}
\\ &= N_G(Q) \cap T
\\ &= H \cap G \cap T \tag{by (\ref{normQ=HNG})}
\\ &= T \cap G = Q. \tag{by (\ref{Q=TNG})}
\end{align*}

We show that $\phi$ is a group homomorphism. Take $d_\delta s_\sigma$, $d_\rho s_\gamma$ from $ N_G(Q)$.
\begin{align*} \phi(d_\delta s_\sigma d_\rho s_\gamma) &= \phi(d_\delta d_\rho t_\sigma s_\gamma) \tag{where $\sigma = \lambda \rho^2$, by Lemma \ref{6.1}}
\\ &= d_\delta d_\rho
\\ &= \phi(d_\delta s_\sigma) \phi(d_\rho s_\gamma).
\end{align*}

Thus by the First Isomorphism Theorem,
\begin{align}\label{6.8viso} N_G(Q) / Q &\cong \phi(N_G(Q)),
\end{align}

Since $N_G(Q)$ is a finite group, it's image under $\phi$ is thus a finite subgroup of $D$. Furthermore, since $D \cong F^*$ (by Lemma \ref{6.1b}), $\phi(N_G(Q))$ is a cyclic group whose order divides $p^m-1$ and is therefore relatively prime to $p$, and by \eqref{6.8viso}, so too is $N_G(Q) / Q$. \\
\\
Let $r$ be the order of $N_G(Q) / Q$. Since it is cyclic, $N_G(Q)/Q$ is generated by a single element, namely a coset of $Q$ in $N_G(Q)$, call it $kQ$. So $|kQ| = r$. Observe that,
\begin{align*} (kQ)^r &= Q,
\\ k^rQ &= Q,
\\ k^r &\in Q.
\end{align*}
Since $Q$ is elementary abelian, each of it's non-trivial elements has order $p$, so $k$ has order $r$ or $rp$. In either case, since gcd$(r,p)=1$, the order of $k^p$ is $r$. Let $K = \langle k^p \rangle$. Now $|K| = r$ and
\begin{align*} |N_G(Q)| &= r|Q|
\\ &= |K||Q|
\\ &= |QK|. \tag{since $Q \cap K = I_G$} 
\end{align*}
Thus,
\begin{align}\label{QK} N_G(Q) &= QK.
\end{align}
\end{proof}



\begin{theorem}
  \label{MaximalAbelianSubgroup.K_mem_MaximalAbelianSubgroups_of_center_lt_card_K}
  \uses{MaximalAbelianSubgroup.IsCyclic_and_card_coprime_CharP_or_eq_Q_join_Z}
  \lean{MaximalAbelianSubgroup.K_mem_MaximalAbelianSubgroups_of_center_lt_card_K}
  Let $Q$ be a Sylow $p$-subgroup of $G$. If $Q \neq \{I_G\}$, then there is a cyclic subgroup $K$ of $G$ such that $N_G(Q) = Q \sqcup K = QK$. Furthermore, If $|K| > |Z|$, then $K \in \mathfrak{M}$
\end{theorem}
\begin{proof} 
Assume $|K| > |Z|$. Since $K$ is abelian, it must be contained in some maximal abelian group $A \in \mathfrak{M}$. By part \ref{MaximalAbelianSubgroup.IsCyclic_and_card_coprime_CharP_or_eq_Q_join_Z}, $A$ must also be a cyclic group whose order is relatively prime to $p$. \\
\\
Since $A$ is conjugate in $\SL_2(F)$ to a subgroup of $D$, each non-central element of $A$ has exactly 2 fixed points on the projective line $\mathscr{L}$ by Proposition \ref{6.7}. Let $A = \langle x \rangle$ and let $P_1$ and $P_2$ be the points fixed by $x$. We show by induction on $n$ that $x^n$ also fixes $P_1$ and $P_2$, for all $n \in \mathbb{Z^+}$. We do this by assuming first that $x^{n-1}$ fixes $P_i$.
\begin{align*} x^n P_i = x(x^{n-1} P_i) = x (P_i) = P_i.
\end{align*}

The importance of this is that since each element of $A$ can be expressed as some power of $x$, they must have the same two fixed points, namely $P_1$ and $P_2$. In other words, 
\begin{align}\label{stab} A \subset S_L(P_i), \qquad (\text{$i$ = 1 or 2})
\end{align}

By Proposition \ref{6.7}(ii), each element of $S$ has a common fixed point $P$ and Stab$(P) = H$. Since $K \subset H$, each element in $K$ fixes $P$. Also, since $K \subset A$, this $P$ must be equal to either $P_1$ or $P_2$. Therefore by (\ref{stab}), $A \subset \text{Stab}(P) = H$. We arrive at the following result:
\begin{align*} A &\subset H \cap G 
\\ &= N_G(Q) \tag{by (\ref{normQ=HNG})}
\\ &= QK. \tag{by (\ref{QK})}
\end{align*}

Furthermore, we get,
\begin{align*} A &= QK \cap A
\\ &= QK \cap AK \tag{$K \subset A$ so $A = AK$}
\\ &= (Q \cap A)K
\\ &= K \tag{$Q \cap A = I_G$}
\end{align*}

Thus $K \in \mathfrak{M}$.\\
\\
\end{proof}

For the duration of this paper, unless otherwise stated, $Q$ will denote a Sylow $p$-subgroup of $G$ and $K$ will be as described above. 


\section{Conjugacy of Maximal Abelian Subgroups}

\begin{definition}
  \label{ConjClassOfSet}
  \uses{MaximalAbelianSubgroupsOf}
  \lean{ConjClassOfSet}
  \leanok
  Let $G$ be a subgroup of $\SL_2(F)$ and let $A \in \mathfrak{M}$ then define the conjugacy class of $A$ to be 
  \[
  \mathcal{C}(A) = \{ x A x^{-1} : x \in G \}.
  \]
\end{definition}

\begin{definition}[Noncenter of a subgroup]
  \label{Subgroup.noncenter}
  \lean{Subgroup.noncenter}
  \leanok
  Let $A$ be a subgroup of a group $G$ let $A^* = A \setminus Z(G)$ be the "noncenter" part of $A$.
\end{definition}

Now we define the noncenter version of \ref{ConjClassOfSet}

\begin{definition}
  \label{noncenter_ConjClassOfSet}
  \uses{MaximalAbelianSubgroupsOf, Subgroup.noncenter}
  \lean{noncenter_ConjClassOfSet}
  \leanok
  Let $G$ be a subgroup of $\SL_2(F)$ and let $A^* \in \mathfrak{M}^*$ then define the conjugacy class of $A^*$ to be
  \[
  \mathcal{C}(A^*) = \left\{x A^* x^{-1} \; | \; x \in G \right\}
   \]
\end{definition}

\begin{definition}
\label{noncenter_MaximalAbelianSubgroupsOf}
\uses{MaximalAbelianSubgroupsOf, Subgroup.noncenter}
\lean{noncenter_MaximalAbelianSubgroupsOf}
\leanok
Let $\mathfrak{M}^*$ be the set of all $A_i^*$ and let $\mathcal{C}_i^*$ be the conjugacy class of $A_i^*$. \\
\end{definition}

\begin{definition}[Conjugacy class of subgroup]
\label{C}
\lean{C}
\leanok
Let $A \in \mathfrak{M}$ and define the union of the conjugacy classes of $A$ to
\begin{align*} 
  C(A) = \bigcup_{x \in G} x A x^{-1}
\end{align*}
\end{definition}

Similarly, we define the analogous for the noncenter part of a maximal abelian subgroup:

\begin{definition}[Union of conjugacy class of noncenter part of a subgroup]
  \label{noncenter_C}
  \lean{noncenter_C}
  \leanok
  Let $A^* \in \mathfrak{M}^*$ then denote union of the conjugacy class of $A^*$ to be
\begin{align*}
  C(A^*) = \bigcup_{x \in G} x A^* x^{-1} = \bigcup_{B \in \mathcal{C}(A^*)} B.
\end{align*}
\end{definition}


In other words, $C_i$ denotes the set of elements of $G$ which belong to some element of $\mathcal{C}_i$. It's evident that $C_i^* = C_i \setminus Z$ and that there is a $C_i$ corresponding to each $\mathcal{C}_i$. Clearly we have the relation,

\begin{lemma}
\label{card_noncenter_C_eq_noncenter_MaximalAbelianSubgroup_mul_noncenter_ConjClassOfSet}
\uses{noncenter_MaximalAbelianSubgroupsOf, card_noncenter_C, card_noncenter_MaximalAbelianSubgroupsOf, card_noncenter_ConjClassOfSet}
\lean{card_noncenter_C_eq_noncenter_MaximalAbelianSubgroup_mul_noncenter_ConjClassOfSet}
\begin{align} |C_i^*| = |A_i^*||\mathcal{C}_i^*|.
\end{align}
\end{lemma}

Here the argument from Christopher Butler's exposition has been modified, it turns out to be significantly more
idiomatic to lean to first define the following equivalence relation and its corresponding quotient to eventually set up
the maximal abelian class equation.

\begin{lemma}[Equivalence relation on $\mathfrak{M}^*$]
\label{lift_noncenter_MaximalAbelianSubgroupsOf}
\uses{MaximalAbelianSubgroupsOf}
\lean{lift_noncenter_MaximalAbelianSubgroupsOf}
\leanok
 Let $G$ be a finite subgroup of $\SL_2(F)$, then the relation $\sim$ on the set of noncenter part of maximal abelian subgroups of $G$, $\mathfrak{M}^*$ defined by
 \[
 A \sim B \text{ if and only if } \exists x \in G \text{ such that } x A x^{-1} = B
 \]
 gives an equivalence relation.
\end{lemma}

\begin{proof}
 We show the relation $\sim$ defined above is in fact an equivalence relation on  $\mathfrak{M}^*$:

\begin{itemize}
\item $\sim$ is reflexive:

For any $x \in A$ as conjugation by an element in the subgroup defines an automorphism and so $A = x A x^{-1}$ as this automorphism fixes the subgroup.

Therefore, $A \sim A$ and $\sim$ is thus reflexive.

\item $\sim$ is symmetric:

If $A \sim B$, then $\exists \; x \in G$ such that,
\begin{align*} A = xBx^{-1} \iff x^{-1}Ax = B \iff B = yAy^{-1} \quad \text{for} \; y = x^{-1} \in G.
\end{align*}

Thus $B \sim A$ and $\sim$ is symmetric.\\

\item $\sim$ is transitive:

If $A \sim B$ and $B \sim C$, then $\exists \; x, y \in G$  such that,
\begin{align*} A = xBx^{-1} \; \text{and} \; B = yCy^{-1} \Rightarrow A = xyCy^{-1}x^{-1} = (xy)C(xy)^{-1}.
\end{align*}
Thus $A \sim C$ (since $xy \in G$), which shows that $\sim$ is transitive. \\
\end{itemize}

Therefore, we have shown that $\sim$ relation is in fact an equivalence relation on $\mathfrak{M}$
\end{proof}

\begin{remark}[Setoid type in Lean]
  TODO
\end{remark}

Now that we have set up the equivalence relation on maximal abelian subgroups we proceed to lift particular functions that will be of interest to set up the maximal abelian class equation and 
other suitable results.

\begin{lemma}[Equivalent noncenter subgroups of $\mathfrak{M}^*$ have the equal union of their conjugacy class]
  \label{noncenter_C_eq_of_related}
  \uses{noncenter_C, noncenter_MaximalAbelianSubgroupsOf}
  \lean{noncenter_C_eq_of_related}
  Let $G$ be a subgroup of $\SL_2(F)$ and let $A , B \in \mathfrak{M}*$ be a noncenter maximal abelian subgroups of $G$ where $A \sim B$
  then 
  \[
  \bigcup_{x \in G} x A x^{-1} = \bigcup_{x \in G} x B x^{-1}
  \]
\end{lemma}
% PROOF

\begin{definition}[Lift of the union of the conjugacy class of noncenter of a subgroup]
\label{lift_noncenter_C}
\uses{noncenter_C, noncenter_C_eq_of_related, noncenter_MaximalAbelianSubgroupsOf}
\lean{lift_noncenter_C}
\leanok
 Let $[A^*] \in \mathfrak{M}^* / \sim$, given for all $A^* \sim B^*$ we have that $C(A^*) = C(B^*)$ by \ref{noncenter_C_eq_of_related} we can define the lift of $C : \mathfrak{M}^* \rightarrow \mathcal{P}(\SL_2(F))$ to be 
  $\tilde{C}([A^*]) = \bigcup_{x \in G} x A^* x^{-1}$ where this map is well-defined for any choice of a representative of $[A^*]$.
\end{definition}

\begin{theorem}[The union of conjugacy classes of the set representatives of $\mathfrak{M}^* / \sim$ cover $G \setminus Z(\SL_2(F))$]
\label{union_lift_noncenter_C_eq_G_diff_center}
\uses{lift_noncenter_MaximalAbelianSubgroupsOf, lift_noncenter_C}
\lean{union_lift_noncenter_C_eq_G_diff_center}
  Let $G$ be a finite subgroup of $\SL_2(F)$ provided $\mathfrak{M}^* / \sim$ is a finite then we have the set equality
  \[
   G \setminus Z(\SL_2(F)) = \bigcup_{[A^*] \in \mathfrak{M}^* / \sim} C([A^*])
  \]
\end{theorem}
% PROOF

\begin{theorem}[Distinct elements of $\mathfrak{M}^* / \sim$ are mapped to disjoint sets through $\tilde{C}$]
  \label{disjoint_of_ne_lift_noncenter_MaximalAbelianSubgroupsOf}
  \uses{lift_noncenter_MaximalAbelianSubgroupsOf, lift_noncenter_C}
  \lean{disjoint_of_ne_lift_noncenter_MaximalAbelianSubgroupsOf}
  Let $[A^*], [B^*] \in \mathfrak{M}^* / \sim$ then
  \[
  \tilde{C}([A^*]) = \tilde{C}([B^*]) \iff [A^*] = [B^*]
  \]
  Or equivalently,
  \[ 
  C(A^*) \cap C(B^*) = \varnothing, \qquad \forall \;  A^* \not\sim B^* 
  \]
\end{theorem}
% PROOF

\begin{theorem}
  \label{card_noncenter_ConjClassOfSet_eq_card_ConjClassOfSet}
  \uses{MaximalAbelianSubgroupsOf, noncenter_MaximalAbelianSubgroupsOf, noncenter_ConjClassOfSet, ConjClassOfSet}
  \lean{card_noncenter_ConjClassOfSet_eq_card_ConjClassOfSet}
  For all maximal abelian subgroups $A \in \mathfrak{M}$ we have that 
  \[
  |\mathcal{C}(A)| = |\mathcal{C}(A^*)|
  \]
\end{theorem}
% PROOF


\begin{theorem}
\label{card_ConjClassOfSet_eq_index_normalizer}
\uses{MaximalAbelianSubgroupsOf, ConjClassOfSet}
\lean{card_ConjClassOfSet_eq_index_normalizer}

Let $G$ be a finite subgroup of $\SL_2(F)$ and let $A$ be a maximal abelian subgroup of $G$, 
$A \in \mathfrak{M}$ then $|\mathcal{C}(A)| = [G : N_G(A)]$.
\end{theorem}
%PROOF

\begin{theorem}[The maximal subgroup class equation]
  \label{card_noncenter_fin_subgroup_eq_sum_card_noncenter_mul_index_normalizer}
  \uses{lift_noncenter_MaximalAbelianSubgroupsOf, lift_card_noncenter, lift_card_noncenter_C}
  \lean{card_noncenter_fin_subgroup_eq_sum_card_noncenter_mul_index_normalizer}

Let $G$ be a finite subgroup of $\SL_2(F)$, define the equivalence relation on the maximal abelian subgroups of $G$, $\mathfrak{M}^*$ as above in \ref{lift_noncenter_MaximalAbelianSubgroupsOf}
then 
$|G \! \setminus  \! Z| = \sum_{[A^*] \in \mathfrak{M}^* / \sim} |A^*| [\tilde{C}([A^*])].$

\end{theorem}
% PROOF

\begin{proof}
(i)
\\
The equivalence class of $A_i^*$ in $\mathfrak{M}^*$ therefore coincides with the set $\mathcal{C}_i^* = \{ xA_i^*x^{-1} : x \in G \}$. Furthermore, this tells us that each $A_i^*$ belongs to exactly one conjugacy class. Thus the conjugacy classes $\mathcal{C}_i^*$ form a partition of $\mathfrak{M}^*$,
\begin{align*} \mathfrak{M}^* = \bigcup\limits_{A_i^* \in S} \mathcal{C}_i^*,  \qquad \text{and}  \qquad \mathcal{C}_i^* \cap \mathcal{C}_j^* = \varnothing, \qquad \forall \; i \neq j.
\end{align*}

Since the set of $\mathcal{C}_i^*$ are pairwise disjoint, it follows that the set of $C_i^*$ are also pairwise disjoint and we get the desired result,

\begin{align*} G \! \setminus \! Z = \bigcup\limits_{A_i^* \in S} C_i^*,  \qquad \text{and}  \qquad C_i^* \cap C_j^* = \varnothing, \qquad \forall \; i \neq j.
\end{align*}

(ii) Let $x A_i x^{-1} \in \mathcal{C}_i$ and $x A_i^* x^{-1} \in \mathcal{C}_i^*$. Since $x A_i x^{-1} \! \setminus \! Z = x A_i^* x^{-1}$, it is quite clear that,
\begin{align*} x A_i x^{-1} \in \mathcal{C}_i \iff x A_i^* x^{-1} \in \mathcal{C}_i^*.
\end{align*}
Thus $|\mathcal{C}_i^*| = |\mathcal{C}_i|$ as desired. \\
\\
(iii) Now we define a map $\phi$ by:
\begin{align*} \phi: \mathcal{C}_i &\longrightarrow G / N_G(A_i),
\\ \phi(xA_ix^{-1}) &= xN_G(A_i). \tag{$\forall \; x \in G, \; A_i \in \mathfrak{M}$}
\end{align*}

Clearly $\phi$ is trivially surjective. We now show that it is both well-defined and injective.
\begin{align*} xN_G(A_i) = yN_G(A_i) &\iff y^{-1}xN_G(A_i) = N_G(A_i) \\
&\iff y^{-1}x \in N_G(A_i) \\
&\iff (y^{-1}x)A_i(y^{-1}x)^{-1} = A_i \\
&\iff y^{-1}xA_ix^{-1}y = A_i \\
&\iff xA_ix^{-1} = yA_iy^{-1}.
\end{align*}

Hence $\phi$ is well-defined and injective. This shows that $\phi$ is a bijection proving that $|\mathcal{C}_i| = [G:N_G(A_i)]$. This is a crucial result which shows that the number of maximal abelian subgroups conjugate to $A_i$ is equal to the index of the normaliser of $A_i$ in $G$. \\
\\
(iv) This follows directly from parts (i), (ii) and (iii) and \eqref{orderorder}.
\begin{align*} G \! \setminus \! Z &= \bigcup\limits_{A_i^* \in S} C_i^*,  \qquad \text{and}  \qquad C_i^* \cap C_j^* = \varnothing, \qquad \forall \;  i \neq j, \\
 |G \! \setminus \! Z| &=  \sum_{A_i^* \in S} |C_i^*| = \sum_{A_i^* \in S} |A_i^*||\mathcal{C}_i^*| = \sum_{A_i^* \in S} |A_i^*||\mathcal{C}_i|
\\ &= \sum_{A_i^* \in S} |A_i^*| [G:N_G(A_i)].
\end{align*}

\end{proof}

This theorem proves that the non-central parts of the maximal abelian subgroups form a partition of the non-central part of $G$. This will serve as a powerful tool in decomposing $G$ and counting its elements.

\section{Constructing The Class Equation}

It is necessary to prove the following 2 short lemmas before we proceed further.
 
\begin{lemma}
\label{normalizer_noncentral_eq}
\lean{normalizer_noncentral_eq} $N_G(A) =N_G(A^*)$.
\end{lemma}

\begin{proof}
(iii) Let $x \in N_G(A^*)$. Take an arbitary $a \in A = A^* \cup Z$. If $a \in A^*$, then since  $x \in N_G(A^*)$, we have $xax^{-1} \in A^* \subset A$. If $a \in Z$, then $xzx^{-1} = zxx^{-1} = z \in A$. Therefore $x$ is in the normaliser of $A$ and $N_G(A^*) \subset N_G(A)$. \\
\\
Conversely, take $y \in N_G(A)$ and $a \in A^*$. $yay^{-1} \in A = A^* \cup Z$. If  $yay^{-1} \in Z$, then
\begin{align*} yay^{-1} &= z, \tag{some $z \in Z$}
\\ a &= y^{-1}zy = y^{-1}yz = z \not \in A^*.
\end{align*}
This contradicts the fact that $a \in A^*$. Therefore $yay^{-1} \in A^*$ and $y \in N_G(A^*)$. Since $y$ was chosen arbitrarily we get $N_G(A) \subset N_G(A^*)$ and hence $N_G(A) =N_G(A^*)$.

\end{proof}

\begin{lemma}
\label{normalizer_Sylow_join_center_eq_normalizer_Sylow}
\uses{MaximalAbelianSubgroup.IsCyclic_and_card_coprime_CharP_or_eq_Q_join_Z, MaximalAbelianSubgroup.index_normalizer_le_two}
\lean{normalizer_Sylow_join_center_eq_normalizer_Sylow}
$N_G(Q \times Z) = N_G(Q)$.
\end{lemma}

\begin{proof} 

If $p = 2$ then $Z = I_G$ and the result is trivial. Now assume $p \neq 2$. Thus $|Z| = 2$. Let $x$ and $q_1$ be arbitrarily chosen elements of $N_G(Q)$ and $Q$ respectively.
\begin{align*} xq_1x^{-1} &= q_2, \tag{for some $q_2 \in Q$}
\\ xq_1x^{-1}z_1 &= q_2z_1,
\\ xq_1z_1x^{-1} &= q_2z_1 \in Q \times Z.
\end{align*}
Thus any element $x$ which is in $N_G(Q)$ is also in $N_G(Q \times Z)$ so we have $N_G(Q) \subset N_G(Q \times Z)$. \\
\\
Let $q_1 z_1$ be an arbitrarily chosen element of $Q \times Z$ such that $q_1 \in Q$ and $z_1 \in Z$. Now let $y$ be an arbitrarily chosen element of $N_G(Q \times Z)$.
\begin{align*} y q_1 z_1 y^{-1} = q_2 z_2 \in Q \times Z. \qquad (\text{where $q_2 \in Q$ and $z_2 \in Z$}) 
\end{align*}

Consider now the order of $q_1z_1$ in $G$. Since $p \neq 2$, $Q \cap Z = I_G$ and $|q_1 z_1| = |q_1| |z_1|$. Note that $q_1 z_1$ and $q_2 z_2$ are conjugate in $G$, and thus their orders are equal. This means that $|z_1| = |z_2|$, because otherwise 2 would divide one of them and not the other. Thus $z_1 = z_2$ and,
\begin{align*} y q_1z_1 y^{-1} &=  q_2z_2 = q_2z_1
\\ y q_1 y^{-1} z_1 &= q_2z_1,
\\ y q_1 y^{-1} &= q_2 \in Q
\end{align*}
Hence $y \in N_G(Q)$. Furthermore, since $y$ was chosen arbitrarily, any element which is in $N_G(Q \times Z)$ is also in $N_G(Q)$, so $N_G(Q \times Z) = N_G(Q)$ as desired.

\end{proof}

We now start to count the elements of the seperate components of $G$ and use the preceeding 2 theorems to construct what will be an invaluable formula in determining the structure of $G$, something we will call the \textbf{Maximal Abelian Subgroup Class Equation} of $G$. \\
\\
First we split $\mathfrak{M}$ into the conjugacy classes of it's elements. Theorem \ref{MaximalAbelianSubgroup.IsCyclic_and_card_coprime_CharP_or_eq_Q_join_Z} tells us that every maximal abelian subgroup is either a cyclic subgroup whose order is relatively prime to $p$ or of the form $Q \times Z$ where $Q$ is a Sylow $p$-subgroup. Let $\mathcal{C}_1, \mathcal{C}_2,...,\mathcal{C}_s, \mathcal{C}_{s+1},..., \mathcal{C}_{s+t}$ (where $s, t \in \mathbb{Z}^+$) denote the conjugacy classes of the cyclic subgroups whose order is relatively prime to $p$. Recall that part (iv) of Theorem \ref{MaximalAbelianSubgroup.index_normalizer_le_two} tells us that $[N_G(A): A] = 1$ or 2. Let $A_i$ be a representative from each $\mathcal{C}_i$ such that,
\begin{align*} [N_G(A_i) : A_i] &= 1, \tag{for  $i \leq s$} \\[2mm]
[N_G(A_i) : A_i] &= 2. \tag{for  $s < i \leq s+t$}, \end{align*}

Now let $Q_1$ and $Q_2$ be any two Sylow $p$-subgroups of $G$. By the Second Sylow Theorem, $Q_1$ and $Q_2$ are conjugate to each other in $G$. That is, there exists a $g \in G$ such that $gQ_1g^{-1} = Q_2$.

\begin{align*} gQ_1g^{-1} = Q_2 &\iff gQ_1g^{-1}Z = Q_2Z 
\\ &\iff gQ_1Zg^{-1} = Q_2Z
\\ &\iff g(Q_1 \times Z)g^{-1} = (Q_2 \times Z). \tag{by Corollary \ref{directproductZ}}
\end{align*} 

So $Q_1 \times Z$ and $Q_2 \times Z$ belong to the same conjugacy class, furthermore there is thus only 1 conjugacy class of elements of this form in $\mathfrak{M}$. Let $\mathcal{C}_{Q \times Z}$ denote this conjugacy class and let $Q \times Z$ be a representative from it. The following diagram provides a visual representation of $G$ divided into it's maximal abelian subgroups.

% \begin{center}
% \begin{tikzpicture}[thick, scale=0.4]

% \draw (0,0) ellipse (22pt and 22pt); 

% \draw[dashed][rotate around={308:(0,0)},red] (3,0) ellipse (108pt and 41pt);  
% \draw[dashed][rotate around={318:(0,0)},red] (3,0) ellipse (108pt and 41pt);  
% \draw[rotate around={328:(0,0)},red] (3,0) ellipse (108pt and 41pt); 
% \draw[dashed][rotate around={338:(0,0)},red] (3,0) ellipse (108pt and 41pt);  

% \draw[dashed][rotate around={301:(0,0)},lightgray] (3,0) ellipse (94pt and 37pt); 
% \draw[dashed][rotate around={296:(0,0)},lightgray] (3,0) ellipse (94pt and 37pt); 
% \draw[dashed][rotate around={291:(0,0)},lightgray] (3,0) ellipse (94pt and 37pt);  

% \draw[dashed][rotate around={258:(0,0)},orange] (2,0) ellipse (79pt and 37pt);  
% \draw[rotate around={270:(0,0)},orange] (2,0) ellipse (79pt and 37pt);  
% \draw[dashed][rotate around={282:(0,0)},orange] (2,0) ellipse (79pt and 37pt); 

% \draw[dashed][rotate around={198:(0,0)},cyan] (3.4,0) ellipse (120pt and 35pt);  
% \draw[rotate around={203:(0,0)},cyan] (3.4,0) ellipse (120pt and 35pt);
% \draw[dashed][rotate around={208:(0,0)},cyan] (3.4,0) ellipse (120pt and 35pt);
% \draw[dashed][rotate around={213:(0,0)},cyan] (3.4,0) ellipse (120pt and 35pt);
% \draw[dashed][rotate around={218:(0,0)},cyan] (3.4,0) ellipse (120pt and 35pt);

% \draw[dashed][rotate around={128:(0,0)},blue] (2,0) ellipse (79pt and 37pt);  
% \draw[rotate around={148:(0,0)},blue] (2,0) ellipse (79pt and 37pt);
% \draw[dashed][rotate around={168:(0,0)},blue] (2,0) ellipse (79pt and 37pt);

% \draw[dashed][rotate around={108:(0,0)},lightgray] (3,0) ellipse (94pt and 37pt); 
% \draw[dashed][rotate around={113:(0,0)},lightgray] (3,0) ellipse (94pt and 37pt); 
% \draw[dashed][rotate around={118:(0,0)},lightgray] (3,0) ellipse (94pt and 37pt); 

% \draw[dashed][rotate around={82:(0,0)},teal] (3,0) ellipse (108pt and 41pt);  
% \draw[rotate around={86:(0,0)},teal] (3,0) ellipse (108pt and 41pt);  
% \draw[dashed][rotate around={90:(0,0)},teal] (3,0) ellipse (108pt and 41pt);  
% \draw[dashed][rotate around={94:(0,0)},teal] (3,0) ellipse (108pt and 41pt);  
% \draw[dashed][rotate around={98:(0,0)},teal] (3,0) ellipse (108pt and 41pt);  

% \draw[dashed][rotate around={18:(0,0)},green] (3.4,0) ellipse (120pt and 35pt);
% \draw[rotate around={26:(0,0)},green] (3.4,0) ellipse (120pt and 35pt);
% \draw[dashed][rotate around={34:(0,0)},green] (3.4,0) ellipse (120pt and 35pt); 

% \node[] at (0,-10) {\resizebox{8cm}{!}{Fig 1: $G$ arranged into it's maximal abelian subgroups}};
% \node[] at (0,0) {\resizebox{.3cm}{!}{$Z$}};

% \node[] at (6.1,-4.5) {\resizebox{.5cm}{!}{$A_1$}};
% \node[] at (-0.2,-5.6) {\resizebox{.5cm}{!}{$A_s$}};
% \node[] at (-7.8,-4.1) {\resizebox{.9cm}{!}{$A_{s+1}$}};
% \node[] at (-5.0,3.3) {\resizebox{.9cm}{!}{$A_{s+2}$}};
% \node[] at (0.2,7.6) {\resizebox{.9cm}{!}{$A_{s+t}$}};
% \node[] at (8.0,4.0) {\resizebox{1.1cm}{!}{$Q \times Z$}};

% \node[] at (7.9,-6.0) {\resizebox{.5cm}{!}{$\mathcal{C}_1$}};
% \node[] at (-0.2,-7.9) {\resizebox{.5cm}{!}{$\mathcal{C}_s$}};
% \node[] at (-10.9,-4.7) {\resizebox{1.0cm}{!}{$\mathcal{C}_{s+1}$}};
% \node[] at (-8.2,4.9) {\resizebox{1.0cm}{!}{$\mathcal{C}_{s+2}$}};
% \node[] at (-0.1,10.0) {\resizebox{1.0cm}{!}{$\mathcal{C}_{s+t}$}};
% \node[] at (11.6,5.1) {\resizebox{1.2cm}{!}{$\mathcal{C}_{Q \times Z}$}};

% \node[scale=1.6, rotate=143,gray] at (6.9,-5.1) { $\Bigg\{$ };
% \node[scale=1.1, rotate=90,gray] at (0,-6.6) { $\Bigg\{$ };
% \node[scale=1.3, rotate=28,gray] at (-8.9,-4.8) { $\Bigg\{$ };
% \node[scale=1.4, rotate=328,gray] at (-6.3,3.9) { $\Bigg\{$ };
% \node[scale=1.2, rotate=270,gray] at (0.0,8.7) { $\Bigg\{$ };
% \node[scale=1.2, rotate=206,gray] at (9.6,4.6) { $\Bigg\{$ };

% \end{tikzpicture}
% \end{center}

We can reformulate the counting formula in Theorem \ref{card_noncenter_fin_subgroup_eq_sum_card_noncenter_mul_index_normalizer} using the notation we have introduced to show that it agrees with the intuitive approach that Fig 1 suggests.
% MODIFY NOTATION

\begin{align*} 
  |G \! \setminus \! Z| = \sum_{A_i^* \in S} |A_i^*| [G:N_G(A_i)] = \sum_{A_i^* \in S} |C_i^*| = |C_{Q \times Z}^*| + \sum_{i=1}^{s+t} |C_i^*|.
\end{align*}

We are now able to begin to evaluate $G$. Firstly, let $|Z| = e$ and $|G| = eg$. We know well by now that $e = 1$ or 2 depending on whether $p$ equals 2 or not, and by Lagrange's Theorem, the order of a subgroup divides the order of the group, so $e$ divides $|G|$ since $Z < G$. \\
\\
We consider the cyclic case first. Again, by Lagrange's Theorem, since $Z$ is a subgroup of each $A_i$, $e$ divides $|A_i|$. So set $|A_i| = eg_i$. Since $Z \notin \mathfrak{M}$, each $A_i$ is therefore strictly larger than $Z$ and so each $g_i$ is an integer greater than or equal to 2. \\
\\
To determine the order of each $C_i$, we return to the set $\mathfrak{M}^*$. The size of one representative of each class is,
\begin{align*} |A_i^*| = |A_i \! \setminus \! Z| = eg_i-e = e(g_i-1). \end{align*}
The number of $A_i^*$ in each conjugacy class $\mathcal{C}_i$ for $i \leq s$ is thus,
\begin{align*} |\mathcal{C}_i^*| = |\mathcal{C}_i| = [G:N_G(A_i)] = \frac{|G|}{|A_i|} = \frac{eg}{eg_i} = \frac{g}{g_i}. \end{align*}
\\
Therefore the total number of elements of $G$ in the noncentral part of $C_i$ for $i \leq s$ is,
\begin{align} \label{classeq1of3} \sum_{i=1}^{s} |C_i^*| = \sum_{i=1}^{s} |A_i^*| |\mathcal{C}_i^*| = \sum_{i=1}^{s} \frac{eg(g_i-1)}{g_i}.
\end{align}
\\
The number of $A_i^*$ in each conjugacy class $\mathcal{C}_i$ for $s < i \leq s+t$ is thus,
\begin{align*} |\mathcal{C}_i^*| = |\mathcal{C}_i| = [G:N_G(A_i)] = \frac{|G|}{2|A_i|} = \frac{eg}{2eg_i} = \frac{g}{2g_i}. \end{align*}
\\
Therefore the total number of elements of $G$ in the noncentral part of $C_i$ for $s < i \leq s+t$ is,
\begin{align}\label{classeq2of3} \sum_{i=s+1}^{s+t} |C_i^*| = \sum_{i=s+1}^{s+t} |A_i^*| |\mathcal{C}_i^*| = \sum_{i=s+1}^{s+t} \frac{eg(g_i-1)}{2g_i}.
\end{align}
We next determine the order of $C_{Q \times Z}$. Let $|Q| = q$. If $p \nmid |G|$ then $q=1$ and if $p = 0$, then we consider a Sylow $p$-subgroup to simply be $I_G$. So $q$ is always at least 1. Since $Z < K$, we can let $|K| = ek$. Observe that if $K \in \mathfrak{M}$, then by Theorem \ref{6.8}(v), $K = A_i$ for some $0 < i \leq t$ and $k = g_i$. Recall that $N_G(Q) = QK$ and so,
\begin{align*} |N_G(Q \times Z)^*| &= |N_G(Q \times Z)|  \tag{by Lemma \ref{normalizer_noncentral_eq}}
\\ &= |N_G(Q)| \tag{by Lemma \ref{unsure}}
\\ &= |QK| = eqk.
\end{align*}

Again we count the size and number of these maximal abelian groups.
\begin{align*} |(Q \times Z)^*| = |QZ| - |Z| = e(q-1).
\end{align*}

Since there is only one conjugacy class of $Q \times Z$, the number of $(Q \times Z)^*$ in $\mathfrak{M}^*$ is thus,
\begin{align*} 
  |\mathcal{C}_{Q \times Z}^*| =  |\mathcal{C}_{Q \times Z}| =  [G: N_G(Q \times Z)] = \frac{|G|}{|N_G(Q \times Z)^*|} = \frac{eg}{eqk} = \frac{g}{qk}.
\end{align*}

Therefore the total number of elements of $G$ in the noncentral parts of each $Q \times Z$ is,
\begin{align} \label{classeq3of3} 
  |C_{Q \times Z}^*| = |(Q \times Z)^*| |\mathcal{C}_{Q \times Z}^*| = \frac{eg(q-1)}{qk}.
\end{align}

We now sum together (\ref{classeq1of3}), (\ref{classeq2of3}) and (\ref{classeq3of3}) to create the \textbf{Maximal Abelian Subgroup Class Equation} of $G$.

\begin{align}\label{classeq} |G \! \setminus \! Z| &= |C_{Q \times Z}^*| + \sum_{i=1}^{s+t} |C_i^*|, \nonumber \\
|G \! \setminus \! Z| &= |(Q \times Z)^*| |\mathcal{C}_{Q \times Z}^*| + \sum_{i=1}^{s} |A_i^*| |\mathcal{C}_i^*| + \sum_{i=s+1}^{s+t} |A_i^*| |\mathcal{C}_i^*|, \nonumber \\
eg - e &= \frac{eg(q-1)}{qk} + \sum_{i=1}^{s} \frac{eg(g_i-1)}{g_i} + \sum_{i=s+1}^{s+t} \frac{eg(g_i-1)}{2g_i}, \nonumber \\
1 &= \frac{1}{g} + \frac{q-1}{qk} + \sum_{i=1}^{s} \frac{g_i-1}{g_i} + \sum_{i=s+1}^{s+t} \frac{g_i-1}{2g_i}.
\end{align}

Since $g,k,q \in \mathbb{Z}^+$ this implies that,
\begin{align*} \frac{1}{g} > 0 \quad \text{and} \quad \frac{q-1}{qk} \geq 0.
\end{align*} 

Also, since $g_i \geq 2$ for $1 \leq i \leq s + t$, we have,
\begin{align*} 
  \frac{g_i-1}{g_i} \geq \frac{1}{2}, \quad \sum_{i=1}^{s} \frac{g_i-1}{g_i} \geq \frac{s}{2} \quad \text{and} \quad \sum_{i=s+1}^{s+t} \frac{g_i-1}{2g_i} \geq \frac{t}{4}.
\end{align*}

Thus we can find a lower bound for (\ref{classeq}) which limits the possible number of conjugacy classes somewhat,
\begin{align*} 1 > \frac{s}{2} + \frac{t}{4}.
\end{align*}

There are only 6 possible different pairs of values which $s$ and $S$ can take: \vspace{3mm}

\begin{center}
\centering
  \begin{tabular}{||c||c|c|c|c|c|c||}
\hline
Case & I & II & III & IV & V & VI \\ [1ex]
\hline\hline
 $s$ & 1 & 1 & 0 & 0 & 0 & 0 \\ [1ex]
\hline
$S$ & 0 & 1 & 0 & 1 & 2 & 3 \\ [1ex]
 \hline
\end{tabular}
\end{center}
\vspace{2mm}

Each case will be examined individually in the next chapter.
\chapter{Dickson's Classification Theorem for finite subgroups of $\SL_2(F)$}\label{Ch7_DicksonsClassificationTheorem}

\section{Five Lemmas}

Before we detemine the structure of $G$ in each of the 6 cases, it is necessary to prove a number of lemmas which will be used.

\begin{lemma}
    \label{IsPGroup.not_le_normalizer} 
    \lean{IsPGroup.not_le_normalizer}
    Let $H$ be a proper subgroup of a $p$-group $G$. Then $H \subsetneq N_G(H)$.
\end{lemma}
% DEPENDENCIES

\begin{proof} Let $S$ denote the set of left cosets of $H$ in $G$. That is,
\begin{align*} S = \{ x H : x \in G \}, \quad \text{and} \;\;\; |S| = [G : H] = p^k. \quad \text{ (for some $k \geq 1$)}
\end{align*}

Consider the action of $H$ on $S$ by left multiplication. We calculate the stabiliser of $xH \in S$ in $H$.
\begin{align*} \text{Stab}(xH) &= \{ y \in H : yxH = xH \}
\\ &= \{ y \in H : x^{-1}yx \in H \}.
\end{align*}

If $x \in H$ then $x^{-1}yx \in H$ for all $y \in H$. Thus the Stab$(xH) = H$ and by the Orbit-Stabiliser Theorem,
\begin{align*} |\text{Orb}(xH)| = [H : \text{Stab}(xH)] = 1.
\end{align*}

Observe that,
\begin{align*} S = \bigcup\limits_{xH \in S} \text{Orb}(xH),
\end{align*}

where the orbits are pairwise disjoint. Now since $p$ divides $|S|$, $p$ divides the sum of all the orbit sizes. Furthermore, since each orbit size is 1 or a multiple of $p$, there must be at least $p$ elements of $S$ which have an orbit of 1. In particular, there exists an $x_1 H \in S$ which has an orbit of 1 and $x_1 \not \in H$. That is,
\begin{align*} y x_1 H &= x_ 1 H, \tag{$\forall y \in H$}
\\ x_1^{-1} y x_1 &\in H,
\\ x_1^{-1} H x_1 &\subset H,
\\ x_1 &\in N_G(H) \! \setminus \! H. \qedhere
\end{align*} 

\end{proof}

\begin{lemma}
    \label{Sylow.not_normal_subgroup_of_G}
    \lean{Sylow.not_normal_subgroup_of_G}
Let $Q$ be a Sylow $p$-subgroup and $K$ a maximal abelian subgroup of $G$ such that $N_G(Q) = QK$ and $Q \cap K = \{ I_G \}$. If $[N_G(K) : K] = 2$, then $Q$ is not a normal subgroup of $G$.
\end{lemma}
% DEPENDENCIES

\begin{proof} The approach here is proof by contradiction, so we begin by assuming that $Q \vartriangleleft G$. Thus $N_G(Q) = G$ and $N_G(K) \subset N_G(Q)$. Consider the natural homomorphism of $N_G(Q)$ onto $N_G(Q)/Q$,
\begin{align*} \phi : N_G(Q) &\longrightarrow N_G(Q)/Q, \\
\phi(x) &= xQ, \\
ker(\phi) &= \{ x \in N_G(Q) : \phi(x) = I_G Q \} = Q.
\end{align*}

Let $\phi '$ be the restiction of $\phi$ to $N_G(K)$: 

\begin{equation*} \phi ' = \left.\phi\right|_{N_G(K)} : N_G(K) \longrightarrow N_G(Q)/Q.
\end{equation*}

Thus $ker(\phi ') = ker(\phi) \cap N_G(K) = Q \cap N_G(K)$. By the 1st Isomorphism Theorem,
\begin{align*} \text{Im}(\phi ') &\cong N_G(K) / ker(\phi '), \\
N_G(Q)/Q &\cong N_G(K) / (Q \cap N_G(K)), \\
K &\cong N_G(K) / (Q \cap N_G(K)) \tag{$N_G(Q) = QK$}, \\
|Q \cap N_G(K)| &= [N_G(K) : K] = 2. \tag{by assumption}
\end{align*}

So $2$ divides $|Q|$, which implies that $2 \nmid |K|$ since $Q \cap K = \{ I_G \}$. Moreover, $|Q \cap N_G(K)|$ and $|K|$ are relatively prime. \\
\\
Take $a \in ker(\phi') = Q \cap N_G(K)$ and $b \in N_G(K)$.
\begin{align*} \phi'(bab^{-1}) &= \phi'(b)\phi'(a)\phi'(b^{-1}) \\
&= \phi'(b)(I_G Q) \phi'(b^{-1}) \\
&=  \phi'(b)\phi'(b^{-1})(I_G Q) =  I_G Q. \end{align*}

Thus $bab^{-1} \in ker(\phi') = Q \cap N_G(K)$ and so $Q \cap N_G(K) \vartriangleleft N_G(K)$. \\
\\
Now let $x \in Q \cap N_G(K)$ and $y \in K$. Notice that both $x$ and $y$ are elements of $N_G(K)$,

\begin{align*} xyx^{-1}y^{-1} &=  (xyx^{-1})y^{-1} \in K, \tag{since $K \vartriangleleft N_G(K)$} \\
xyx^{-1}y^{-1} &= x(yx^{-1}y^{-1}) \in Q \cap N_G(K), \tag{since $Q \cap N_G(K) \vartriangleleft N_G(K)$} \\
xyx^{-1}y^{-1} &\in K \cap ( Q \cap N_G(K)) \\
&= I_G, \tag{since gcd$(|Q \cap N_G(K)|,|K|) = 1$} \\
xy &= yx. \\
\end{align*}

Therefore $(Q$ $\cap$ $N_G(K)) \times K$ is an abelian subgroup of which $K$ is a proper subgroup. This contradicts the fact that $K$ is a maximal abelian subgroup, thus $Q$ is not a normal subgroup of $G$.

\end{proof}

% MAY NOT BE NEEDED
\begin{lemma}
    \label{subfield} Let $p$ be the prime characteristic of $F$ and let $q= p^k$ for some $k>0$. Set,
\begin{align}\label{RRR} R = \{ \lambda \in F : \lambda^q -\lambda = 0 \}.
\end{align}
Then $R$ is a subfield of $F$.
\end{lemma}

\begin{proof} Since $R$ is a subset of $F$ it suffices to show that the following 3 criteria are met: \\
\\
(i) $0, 1 \in R$. \\
(ii) If $\lambda_1, \lambda_2 \in R$, then $\lambda_1 - \lambda_2 \in R$. \\
(iii) If $\lambda_1, \lambda_2 \in R$ and $\lambda_1 \neq 0 \neq \lambda_2$, then $\lambda_1 \lambda^{-1}_2 \in R$. \\
\\
We see immediately that (i) is satified. Since $p$ is the characteristic of $F$, any coeffiecients which are a multiple of $p$ vanish. We get,
\begin{align*} (\lambda_1 - \lambda_2)^q = (\lambda^p_1 - \lambda^p_2)^{p^{k-1}} = ... = \lambda^q_1 - \lambda^q_2 = \lambda_1 - \lambda_2.
\end{align*}

Thus $\lambda_1 - \lambda_2 \in R$ and (ii) is also satisifed. Finally observe that if $\lambda_2$ is a non-zero element of $R$, then $\lambda^{-1}_2 = \lambda^{-q}_2$ and,
\begin{align*} (\lambda_1 \lambda^{-1}_2)^q = \lambda^q_1 \lambda^{-q}_2 = \lambda_1 \lambda^{-1}_2.
\end{align*}

So $\lambda_1 \lambda^{-1}_2 \in R$ and $R$ is a subfield of $F$.

\end{proof}

Each finite field is uniquely determined up to isomorphism by the number of elements it contains \cite[p.227]{stewart}. Since the $R$ defined in \eqref{RRR} has $q$ elements, from now on when we use the notation $\mathbb{F}_q$ to denote a field of $q$ elements, we shall actually mean,
\begin{align}
    \label{subfield} \mathbb{F}_q = R \subset F.
\end{align}

\begin{lemma}
    \label{Matrix.card_GL_field}
    \lean{Matrix.card_GL_field}
    \leanok
    Let $\mathbb{F}_q$ be the field of $q$ elements, where $q$ is the power of a prime. The order of $GL(2,\mathbb{F}_q)$ is $(q^2-1)(q^2-q)$.
\end{lemma}
\begin{proof}
    In order to prove this, we again take a geometric viewpoint. Recall that $GL(2,\mathbb{F}_q)$ is the group of 2 x 2 invertible matrices over $\mathbb{F}_q$ under ordinary matrix multiplication. The order of $GL(2,\mathbb{F}_q)$ is thus equal to the number of ordered pairs $\{u,v\}$ of linearly independent vectors in a 2-dimensional vector space over $\mathbb{F}_q$. \\
    \\
    There are clearly $q^2$ different vectors in the 2-dimensional vector space over $\mathbb{F}_q$. The only restriction on the first vector $u$, is that it must be non-zero, so there are $(q^2 - 1)$ choices for $u$. To ensure the second vector $v$ is linearly independent of $u$, it must not be of the form $\alpha u$, where $\alpha \in \mathbb{F}_q$. Since there are $q$ choices for $\alpha$, there are $(q^2-q)$ choices for $v$. \\
    \\
    Thus the order of $GL(2,\mathbb{F}_q)$ is the product of the number of choices of $u$ and the number of choices of $v$, that is, $(q^2-1)(q^2-q)$ as required.
\end{proof}

\begin{lemma}
\label{card_SL_field}
\uses{Matrix.card_GL_field}
\lean{card_SL_field}

The order of $\SL_2(\mathbb{F}_q)$ is $q(q^2-1)$
\end{lemma}

\begin{proof} 
Consider the map $\phi$ defined as,
\begin{align*} \phi : GL(2,\mathbb{F}_q) \longrightarrow \mathbb{F}^*_q, \qquad \text{where} \quad \! \! \phi(x) = \text{det}(x), \quad \forall \; x \in GL(2,\mathbb{F}_q).
\end{align*}

Next we determine the kernel of $\phi$.
\begin{align*} ker(\phi) &= \{  GL(2,\mathbb{F}_q) : \text{det}(x) = 1 \} = \SL_2(\mathbb{F}_q).
\end{align*}

We show that $\phi$ is a group homomorphism. Take $x,y \in GL(2,\mathbb{F}_q)$,
\begin{align*} 
\phi(xy) = \text{det}(xy) = \text{det}(x) \text{det}(y) = \phi(x) \phi(y).
\end{align*}

Clearly $\phi$ is surjective, since $\alpha \in \mathbb{F}^*_q$ is the determinant of $\begin{bmatrix} \alpha & 0 \\ 0 & 1 \end{bmatrix} \in GL(2,\mathbb{F}_q)$. Therefore because $\SL_2(F) \lhd \GL_2(F)$, by the First Isomorphism Theorem,
\begin{align*} GL(2,\mathbb{F}_q) / \SL_2(\mathbb{F}_q) \cong \mathbb{F}^*_q.
\end{align*}
Thus,
\begin{align*} |\SL_2(\mathbb{F}_q)| =  \frac{|GL(2,\mathbb{F}_q)|}{|\mathbb{F}^*_q|} = \frac{(q^2-1)(q^2-q)}{q-1} = q(q^2-1).
\end{align*}

\end{proof}

\begin{lemma}
    \label{QuotientGroup.comapMk'OrderIso}
    \lean{QuotientGroup.comapMk'OrderIso}
Let $N$ be a normal subgroup of a group $G$ and let $H$ be a subgroup of $G$ which contains $N$.Then,
\begin{align*} H / N \vartriangleleft G / N \iff H \vartriangleleft G
\end{align*} 
\end{lemma}

\begin{proof} If $H \vartriangleleft G$, then it follows from the Third Isomorphism Theorem that $ H / N \vartriangleleft G / N$. Conversely, assume that $H / N$ is normal in $G / N$. Let $x$ be an arbitrary element of $G$ and $h$ be an arbitrary element of $H$. Since $H / N$ is normal in $G / N$ we have,
\begin{align*} x h x^{-1}N = (xN)(hN)(x^{-1}N) = (xN)(hN)(xN)^{-1} \in H / N.
\end{align*}
Thus $x h x^{-1} \in H$. Since $x$ and $h$ were chosen arbitrarily, we have that $H \vartriangleleft G$.

\end{proof}

\section {The Six Cases}

We now address individually the 6 possible combinations of $s$ and $t$ in \eqref{classeq} and determine the structure of $G$ in each case. \\
\\
\begin{theorem}[Case I]
\label{case_I}
\uses{card_noncenter_fin_subgroup_eq_sum_card_noncenter_mul_index_normalizer, MaximalAbelianSubgroup.K_mem_MaximalAbelianSubgroups_of_center_lt_card_K}
\lean{case_I}
Claim: \textit{In this case, the Sylow $p$-subgroup $Q$ is different from $G$ and is an elementary abelian normal subgroup of $G$. The factor group $G/Q$ is a cyclic group whose order is relatively prime to $p$.} \\
\\
\end{theorem}
% DEPENDENCIES AND STATEMENT
\begin{proof} Here, $s = 1$ and $t = 0$. Equation (\ref{classeq}) simplifies to:
\begin{align}\label{case1a} 1 &= \frac{1}{g} + \frac{q-1}{qk} + \frac{g_1-1}{g_1}, \nonumber
\\ 1 &= \frac{1}{g} + \frac{1}{k} - \frac{1}{qk}  + 1 - \frac{1}{g_1}, \nonumber
\\ \frac{1}{qk}  + \frac{1}{g_1} &= \frac{1}{g} + \frac{1}{k}.
\end{align}
 \space \textbf{Case Ia:} $\pmb{q = 1}$. Here we have $Q = I_G$ and is trivially an elementary abelian normal subgroup of $G$. Equation (\ref{case1a}) gives $g=g_1$, thus $G/Q = G = A_1$, which indeed is a cyclic group whose order is relatively prime to $p$. \\
\\
 \space \textbf{Case Ib:} $\pmb{q > 1}$. If $k=1$ then (\ref{case1a}) gives,
\begin{align*} \frac{1}{q}  + \frac{1}{g_1} &= \frac{1}{g} + 1 \; > \; 1.
\end{align*}
But since both $1/q$ and $1/g_i$ are at most $1/2$ each, this is a contradiction. Thus $k > 1$. This means that $|K| = ek > e = |Z|$, so $k = g_1$ by 
Theorem \ref{MaximalAbelianSubgroup.K_mem_MaximalAbelianSubgroups_of_center_lt_card_K}. Equation (\ref{case1a}) now gives $qk = g$.
\begin{align*} |G| = eg = eqk = |N_G(Q)|.
\end{align*}
Thus $G = N_G(Q)$ and so $Q \vartriangleleft G$. Therefore $Q \neq G$ and is an elementary abelian normal subgroup of $G$. Also,
\begin{align*} G/Q = N_G(Q)/Q \cong K = A_1.
\end{align*}
Thus $G/Q$ is a cyclic group whose order is relatively prime to $p$.

\end{proof}

\begin{theorem}[Case II]
\label{case_II}
\uses{card_noncenter_fin_subgroup_eq_sum_card_noncenter_mul_index_normalizer, MaximalAbelianSubgroup.of_index_normalizer_eq_two}
\lean{case_II}
Claim: \textit{The order of $G$ is relatively prime to $p$ and either $G \cong \SL_2(3)$ or $G$ is the group of order $4n$, where $n$ is odd, defined by the presentation:}
\begin{align*} \langle \, x,y \, | \, x^n = y^2, \, yxy^{-1} = x^{-1} \, \rangle. \\
\end{align*}
\end{theorem}
% DEPENDENCIES AND STATEMENT
\begin{proof} Here, $s = 1 = t$. Equation (\ref{classeq}) simplifies to:
\begin{align}\label{case2a} 1 &= \frac{1}{g} + \frac{q-1}{qk} + \frac{g_1-1}{g_1} +  \frac{g_2-1}{2g_2}, \nonumber
\\ 1 &= \frac{1}{g} + \frac{q-1}{qk} + 1 - \frac{1}{g_1} + \frac{1}{2} - \frac{1}{2g_2}, \nonumber
\\ \frac{1}{g_1}  + \frac{1}{2g_2} &= \frac{1}{2} + \frac{1}{g} + \frac{q-1}{qk}.
\end{align}

First assume that $q>1$. This means $(q-1)/qk \geq 1/2k$ and consequently we bound (\ref{case2a}) from below:
\begin{align*} \frac{1}{2g_2} &= \frac{1}{2} - \frac{1}{g_1} + \frac{1}{g} + \frac{q-1}{qk} \; > \; \frac{1}{2k}.
\end{align*}

Thus $k > g_2 \geq 2$. So $K \in \mathfrak{M}$ and $k=g_i$ for some $i$. Since it is strictly greater than $g_2$, we have $k=g_1$. Equation (\ref{case2a}) now becomes
\begin{align*} \frac{1}{g_1}  + \frac{1}{2g_2} \; &= \; \frac{1}{2} + \frac{1}{g} + \frac{q-1}{qg_1},
\\ \frac{1}{g_1}  + \frac{1}{2g_2} \; &> \; \frac{1}{2} + \frac{1}{2g_1},
\\ \frac{1}{4} + \frac{1}{4} \; \geq \; \frac{1}{2g_1}  + \frac{1}{2g_2} \; &> \; \frac{1}{2}.
\end{align*}

This contradiction disproves the assumption that $q > 1$, so we have that $q = 1$. This means that $Q$, a Sylow $p$-subgroup of $G$, is simply the identity element and so $|G|$ is relatively prime to $p$. Also, Equation (\ref{case2a}) now reduces to:
\begin{align}\label{case2b} \frac{1}{g_1}  + \frac{1}{2g_2} &= \frac{1}{2} + \frac{1}{g}.
\end{align}

If $g_1 \geq 4$ we get
\begin{align*} \frac{1}{2g_2} &= \frac{1}{2} + \frac{1}{g} - \frac{1}{g_1} \; > \; \frac{1}{4}.
\end{align*}

Since $g_2 > 1$  this gives a contradiction and thus $g_1 < 4$. We now have two seperate cases to consider.\\
\\
 \space \textbf{Case IIa:} $\pmb{g_1 = 2}$. Equation (\ref{case2b}) becomes
\begin{align*} \frac{1}{2g_2} &= \frac{1}{g}, \; \; \Longrightarrow \; \; g = 2g_2.
\end{align*}

If $e=1$, then $p=2$. Also since $q=1$, 2 does not divide $|G|$, but $|G| = eg = e2g_2$ which is a contradiction. So $e=2$ and $p \neq 2$. We now have:
\begin{align*} |N_G(A_2)| &= 2|A_2|  = 2eg_2 = eg = |G|,  \tag{since $s+t = 2$}
\\ |N_G(A_1)| &= |A_1| = eg_1 = 4. \tag{since $s=1$} 
\end{align*}
Thus $G = N_G(A_2)$, that is $A_2 \vartriangleleft G$.\\
\\
By Corollary \ref{5thsylow}, $A_1$ is contained in a Sylow 2-subgroup of $G$, call it $S$. If $S$ is strictly larger than $A_1$, then by Lemma \ref{case2q}, $A_1 \subsetneq N_S(A_1) \subset N_G(A_1)$. Since $A_1 = N_G(A_1)$ we conclude that $A_1$ is a Sylow 2-subgroup of $G$. This means that 8 does not divide $|G| = 4g_2$ and so $g_2 = n$, where $n$ is odd. \\
\\
Since $A_2$ is cyclic it is generated by a single element, so let $A_2 = \langle x \rangle$ and thus $x^{2n}= I_G$.  Recall that because $[N_G(A_2): A_2] = 2$, Theorem \ref{MaximalAbelianSubgroup.of_index_normalizer_eq_two} tells us that there exists a $y \in N_G(A_2) \! \setminus \! A_2$ such that $yxy^{-1} = x^{-1}$. \\
\\
Recall from Chapter 2 that the number of $A_i$ in each conjugacy class $\mathcal{C}_i$ is equal to $[G : N_G(A_i)]$ so,
\begin{align*}  |\mathcal{C}_2| = [G:N_G(A_2)] &= 1.
\end{align*}

Due to the fact that $y$ belongs to some maximal abelian subgroup of $G$, and since $y \not \in A_2$ and $|\mathcal{C}_2| = 1$, it must be that $y$ belongs to $A_1$ or one of its conjugate subgroups. Thus $y$ has an order which divides $|A_1| = 4$ and since the only elements of order 1 and 2 lie in $Z$, the order of $y$ is 4. Furthermore, both $x^n$ and $y^2$ have order 2. Recalling that $G$ has at most 1 element of order 2, this gives the relation $x^n = y^2$. \\ 
\\
Let $H$ be the group generated by $x$ and $y$ and the above relations:
\begin{align*} H = \langle \, x,y \, | \, x^n = y^2, \, yxy^{-1} = x^{-1} \rangle.
\end{align*}

Notice that the second relation gives that $y x^n y^{-1} = x^{-n}$, so
\begin{align*} x^{-n} = y x^n y^{-1} = y y^2 y^{-1} = y^2 = x^n.
\end{align*}

This shows that $y^4 = x^{2n} = I_G$ and that $H$ is finite. Moreoever,
\begin{align*} H = \{ x^k, x^ky :  0 < k \leq 2n \}.
\end{align*}

 Thus $|H| = 4n = |G|$ and $H = G$. \\
\\
 \space \textbf{Case IIb:} $\pmb{g_1 = 3}$.  Equation (\ref{case2b}) becomes
\begin{align*} \frac{1}{2g_2} &= \frac{1}{6} + \frac{1}{g} \; > \; \frac{1}{6}.
\end{align*}
Therefore $g_2 = 2$ and $g = 12$. Again, since $q=1$ and 2 divides $|G|$, we have $p \neq 2$ and so $e = 2$. Thus we have,
\begin{align*} |G| = eg = 24, \qquad |A_1| = eg_1 = 6, \qquad |A_2| = eg_2 = 4.
\end{align*}
Again we determine the number of maximal abelian subgroups in each conjugacy class.
\begin{align*}  |\mathcal{C}_1| = [G:N_G(A_1)] &= \frac{|G|}{|A_1|} = \frac{24}{6} = 4, 
\\[1.5ex] |\mathcal{C}_2| = [G:N_G(A_2)] &= \frac{|G|}{2|A_2|} = \frac{24}{8} = 3.
\end{align*}

% \newpage
The figure below shows $G$ divided into it's maximal abelian subgroups:


% \begin{center}
% \begin{tikzpicture}[thick, scale=0.4]

% \draw[dashed][rotate around={0:(0,0)},red] (3,0) ellipse (108pt and 41pt);  
% \draw[dashed][rotate around={20:(0,0)},red] (3,0) ellipse (108pt and 41pt);  
% \draw[rotate around={40:(0,0)},red] (3,0) ellipse (108pt and 41pt); 
% \draw[dashed][rotate around={60:(0,0)},red] (3,0) ellipse (108pt and 41pt);  

% \draw[dashed][rotate around={180:(0,0)},blue] (2,0) ellipse (79pt and 37pt);  
% \draw[rotate around={210:(0,0)},blue] (2,0) ellipse (79pt and 37pt);
% \draw[dashed][rotate around={240:(0,0)},blue] (2,0) ellipse (79pt and 37pt);

% \draw (0,0) ellipse (22pt and 22pt); 

% \node[] at (0,-8) {\resizebox{9cm}{!}{Fig 2: The elements of $G$ arranged into maximal abelian subgroups.}};
% \node[] at (0,0) {\resizebox{.3cm}{!}{$Z$}};
% \node[] at (5.7,4.9) {\resizebox{.5cm}{!}{$A_1$}};
% \node[] at (-4.6,-2.8) {\resizebox{.5cm}{!}{$A_2$}};
% \node[] at (8.6,5) {\resizebox{.5cm}{!}{$\mathcal{C}_1$}};
% \node[] at (-6.8,-3.6) {\resizebox{.5cm}{!}{$\mathcal{C}_2$}};

% \node[scale=1.8, rotate=30,gray] at (-5.4,-3.2) { $\Bigg\{$ };
% \node[scale=2, rotate=210,gray] at (7.3,4) { $\Bigg\{$ };

% \node[scale=2, black] at (-.45,0) {.};
% \node[scale=2, black] at (.45,0) {.};

% \node[scale=3, red] at (4,4) {.};
% \node[scale=3, red] at (4.7,4.2) {.};
% \node[scale=3, red] at (4.8,3.3) {.};
% \node[scale=3, red] at (3.9, 3.2) {.};
% \node[scale=2, red] at (4.8, 1.7) {.};
% \node[scale=2, red] at (5.2, 2.3) {.};
% \node[scale=2, red] at (5.9, 2.2) {.};
% \node[scale=2, red] at (5.6, 1.5) {.};
% \node[scale=2, red] at (6, 0.2) {.};
% \node[scale=2, red] at (5.5, -0.5) {.};
% \node[scale=2, red] at (4.6, -0.8) {.};
% \node[scale=2, red] at (3.7, -1) {.};
% \node[scale=2, red] at (3, 5.2) {.};
% \node[scale=2, red] at (2.2,4.5) {.};
% \node[scale=2, red] at (1.5, 4.0) {.};
% \node[scale=2, red] at (0.9, 3.3) {.};

% \node[scale=3, blue] at (-3.5,-1.6) {.};
% \node[scale=3, blue] at (-3.2,-2.4) {.};
% \node[scale=2, blue] at (-3.6,0.4) {.};
% \node[scale=2, blue] at (-2.4,0.6) {.};
% \node[scale=2, blue] at (-2,-3.3) {.};
% \node[scale=2, blue] at (-1.0,-2.9) {.};

% \end{tikzpicture}
% \end{center}

Let $A_2 = \langle x \rangle$. By Theorem \ref{MaximalAbelianSubgroup.of_index_normalizer_eq_two}, there is an element $y \in N_G(A_2) \! \setminus \! A_2$ such that $y x y^{-1} = x^{-1}$. Since $N_G(A_2)$ has order 8, the order of $y$ must divide 8. The order of $y$ cannot be 8 since $N_G(A_2)$ is not cyclic and the only elements with order 1 or 2  are found in $Z$, thus $y$ has order 4. By the uniqueness of the element of order 2, we have $x^2 = y^2$. So
\begin{align*} N_G(A_2) = \langle x, y \; | \; x^2 = y^2, y x y^{-1} = x^{-1} \rangle.
\end{align*}
For simplicity let $N = N_G(A_2)$ . Since $|A_1| = 6$, the only elements in $C_1$ with order $2^k$ are those in $Z$, so every element of $G$ with order $2^k$ must belong to $C_2$. Since $C_2$ has order 8 it is equal to $N$ because each element of $N$ has order $2^k$. Furthermore, $N$ is thus a unique Sylow $2$-subgroup of $G$ and by Corollary \ref{4thsylow}, we have $N \vartriangleleft G$. \\
\\
Now consider the quotient group $G / N$, that is the set of left (or right) cosets of $N$ in $G$.
\begin {align*} G / N = \{ N, rN, r^2N \} \cong \langle r \rangle \cong \mathbb{Z}_3,
\end{align*}
where $r$ is some element of $G\! \setminus \! N$ with order 3. Without loss of generality we may regard $r$ to be a generator of $H$, where $H$ is the cyclic subgroup of $A_1$ of order 3. \\
\\
Let $H$ act on $N$ by conjugation. Since $|H| = 3$ the orbit of $x \in N$ has size 1 or 3.
\begin{align*} \text{Orb}(x) =  \{ r^k x r^{-k} : r^k \in H \}.
\end{align*}

Since $H$ is not contained in the centraliser of $x$ we conclude that the orbit of $x$ has size 3. Let $A_2, A'_2$ and $A''_2$ be the 3 elements of $\mathcal{C}_2$. Without loss of generality we may assume $y \in A'_2$ and consequently $xy \in A''_2$. Using the two relations between $x$ and $y$ we observe that,
\begin{align*} (xy)^{-1} = y^{-1} x^{-1} = y^{-1} (y x y^{-1}) = x y^{-1} = x^{-1} x^2 y^{-1} = x^{-1} y = yx
\end{align*}

% \begin{center}
% \begin{tikzpicture}[thick, scale=0.8]

% \draw[rotate around={60:(0,0)},blue] (2,0) ellipse (79pt and 37pt);  
% \draw[rotate around={90:(0,0)},blue] (2,0) ellipse (79pt and 37pt);
% \draw[rotate around={120:(0,0)},blue] (2,0) ellipse (79pt and 37pt);

% \draw (0,0) ellipse (22pt and 22pt); 

% \node[] at (0,-2) {\resizebox{9cm}{!}{Fig 3: The elements of $N$ arranged into maximal abelian subgroups.}};
% \node[] at (0,0) {\resizebox{.3cm}{!}{$Z$}};
% \node[] at (-2.5,4.7) {\resizebox{.5cm}{!}{$A_2$}};
% \node[] at (0.0,5.4) {\resizebox{.5cm}{!}{$A'_2$}};
% \node[] at (2.3,4.8) {\resizebox{.5cm}{!}{$A''_2$}};

% \node[scale=3, black] at (-.45,0) {.};
% \node[scale=3, black] at (.45,0) {.};

% \node[scale=3, blue] at (-1.7, 3.3) {.};
% \node[] at (-1.7,3.6) {\resizebox{.22cm}{!}{$x$}};
% \node[scale=3, blue] at (-2.2, 2.5) {.};
% \node[] at (-2.2,2.9) {\resizebox{.6cm}{!}{$x^{-1}$}};
% \node[scale=3, blue] at (-0.5,3.8) {.};
% \node[] at (0.5,4.2) {\resizebox{.21cm}{!}{$y$}};
% \node[scale=3, blue] at (0.5,3.8) {.};
% \node[] at (-0.25,4.3) {\resizebox{.6cm}{!}{$y^{-1}$}};
% \node[scale=3, blue] at (1.7,3.3) {.};
% \node[] at (1.7,3.6) {\resizebox{.4cm}{!}{$xy$}};
% \node[scale=3, blue] at (2.2,2.5) {.};
% \node[] at (2.2,2.8) {\resizebox{.4cm}{!}{$yx$}};

% \end{tikzpicture}
% \end{center}

The elements of $Z$ are fixed points under this group action and the remaining 6 elements of $N$ form 2 orbit cycles of order 3, with each cycle containing exactly one element from the noncentral parts of $A_2, A'_2$ and $A''_2$ in some order. If $y$ inverts $x$, then $y$ inverts all powers of $x$ including $x^{-1}$. Also, if $y$ inverts $x$, then $y^{-1}$ inverts $x^{-1}$ and thus inverts $x$ also. So the 2 relations we have established between $x$ and $y$ actually hold for any pair of elements of $N \! \setminus \! Z$ which belong to different elements of $\mathfrak{M}$. Therefore without loss of generality, we may assume that $x$ and $y$ are in the same orbit cycle and that $r x r^{-1} = y$. Fig 3 shows that there are only 2 elements which could complete this cycle, $xy$ and $yx$. If $r y r^{-1} = xy$, then we have the following 3 relations on $G$.
\begin{align}\label{3rel} r x r^{-1} = y, \qquad r y r^{-1} = xy, \qquad r xy x^{-1} = x.
\end{align}

Otherwise $r y r^{-1} = yx$. In this case, consider the orbit of $x$ under conjugation by $r^2$ instead. This gives the same orbit cycle but in the opposite direction:
\begin{align*} r^2 x r^{-2} = yx, \qquad r^2 yx r^{-2} = y, \qquad r^2 y r^{-2} = x.
\end{align*}
Observe that $x(yx) = x (x^{-1} y) = y$. Thus without loss of generality we can rename $r^2$ as $r$, $yx$ as $y$ and $y$ as $xy$. Notice that this now gives the same relations as in \eqref{3rel}. Since $x$ and $y$ generate a group of order 8 and $r$ has order 3, the group given by the following presentation has order at most 24 and is thus a presentation of $G$. 
\begin{align*} \langle x, y, r \, |  \, x^2= y^2, \, y x y^{-1} = x^{-1}, \, r^3 = I, \, r x r^{-1} = y, \, r y r^{-1} = xy, \, r xy r^{-1} = x \rangle,
\end{align*}

By Lemma \ref{ordersl2q}, we observe that the order of $\SL_2(3)$ is $3(3^2-1) = 24$. Now consider the following the elements of $\SL_2(3)$:
\begin{align*} a = \begin{bmatrix} 1 & 1 \\ 1 & 2 \end{bmatrix}, \qquad b = \begin{bmatrix} 0 & 2 \\ 1 & 0 \end{bmatrix}, \qquad c = \begin{bmatrix} 2 & 1 \\ 2 & 0 \end{bmatrix}.
\end{align*}

One can verify easily that each of the following relations hold:
\begin{align*} a^2 &= b^2, \qquad b a b^{-1} = a^{-1}, \qquad \quad \; c^3 = I, 
\\ c a c^{-1} &= b,  \qquad \; \: c b c^{-1} = ab, \qquad \! c ab c^{-1} = a.
\end{align*}

Since $G$ and $\SL_2(3)$ have the same order and since their respective generators satisfy the corresponding relations, there is an isomorphism mapping $x \mapsto a$, $y \mapsto b$ and $r \mapsto c$. Thus,
\begin{align*} G = \langle x, y, r \rangle \cong \langle a, b, c \rangle = \SL_2(3). 
\end{align*} 
\end{proof}


\begin{theorem}[Case III]
\label{case_III}
\uses{card_noncenter_fin_subgroup_eq_sum_card_noncenter_mul_index_normalizer, MaximalAbelianSubgroup.K_mem_MaximalAbelianSubgroups_of_center_lt_card_K}
\lean{case_III}
Claim: \textit{We have $G = Q \times Z$.}
\end{theorem}
% DEPENDENCIES AND STATEMENT
\begin{proof} Here, $s = 0 = t$. Equation (\ref{classeq}) simplifies to:
\begin{align}\label{case3a} 1 &= \frac{1}{g} + \frac{q-1}{qk}, \nonumber
\\ 1 &= \frac{1}{g} + \frac{1}{k} - \frac{1}{qk}, \nonumber
\\ 1 + \frac{1}{qk} &= \frac{1}{g} + \frac{1}{k}.
\end{align}

Since $s = 0 = t$, there are no cyclic maximal abelian subgroups whose order is relatively prime to $p$, so $K \not \in \mathfrak{M}$. Then by Theorem \ref{MaximalAbelianSubgroup.K_mem_MaximalAbelianSubgroups_of_center_lt_card_K} we have,
\begin{align*} ek = |K| \leq |Z| = e.
\end{align*} 
Thus $k = 1$ and equation (\ref{case3a}) reduces to $1/q = 1/g$, that is $g=q$.
\begin{align*} |G| =  eg &= eq = |Q \times Z|,
\\ G &= Q \times Z.
\end{align*}
\qedhere
\end{proof}


\begin{theorem}[Case IV]
\label{case_IV}
\lean{card_noncenter_fin_subgroup_eq_sum_card_noncenter_mul_index_normalizer}
\lean{case_IV}
Claim: \textit{Either $p=2$ and $G$ is isomorphic to the dihedral group of order $2n$, where $n$ is odd, or $p=3$ and $G \cong \SL_2(3)$.}
\end{theorem}
\begin{proof} Here, $s = 0$ and $t = 1$. Equation (\ref{classeq}) simplifies to:
\begin{align}\label{case4a} 1 &= \frac{1}{g} + \frac{q-1}{qk} +  \frac{g_1-1}{2g_1}, \nonumber
\\ 1 &= \frac{1}{g} + \frac{q-1}{qk} + \frac{1}{2} - \frac{1}{2g_1}, \nonumber
\\ \frac{1}{2} + \frac{1}{2g_1} &= \frac{1}{g} + \frac{q-1}{qk}.
\end{align}

Recall that $|A_1|=eg_1$ and $[N_G(A_1): A_1] = 2$ and so,
\begin{align*} eg = |G| \geq |N_G(A_1)| = 2eg_1.
\end{align*}

So $g \geq 2g_1$ and $1/2g_1 \geq 1/g$ and hence we can bound Equation (\ref{case4a}):
\begin{align*} \frac{1}{2} \; \leq \; \frac{1}{2} + \frac{1}{2g_1} - \frac{1}{g} &= \frac{q-1}{qk}.
\end{align*}

Clearly this forces $k = 1$ and also $q > 1$. We can now simplify and bound Equation (\ref{case4a}) as follows:
\begin{align*} \frac{1}{q} + \frac{1}{4} \; \geq \; \frac{1}{q} + \frac{1}{2g_1} &= \frac{1}{g} + \frac{1}{2} \; > \; \frac{1}{2}. 
\end{align*}

This gives $1/q > 1/4$ and so $q$ is equal to either 2 or 3. We examine each case individually. \\
\\
 \space \textbf{Case IVa:} $\pmb{q = 2}$. Equation (\ref{case4a}) becomes
\begin{align*} \frac{1}{2g_1} &= \frac{1}{g}, \; \; \Longrightarrow \; \; g = 2g_1,
\end{align*}

and we show that $A_1$ is a normal subgroup of $G$:
\begin{align*} |G| = eg = e2g_1 = 2|A_1| = |N_G(A_1)|. 
\end{align*}
In this case, a Sylow $p$-subgroup has order 2 so we have $p=2$ and also $e=1$. By it's definition, the order of $A_1$ is relatively prime to $p=2$, so we have that $|A_1|= g_1 = n$, where $n$ is odd, and consequently $G$ has order $2n$. \\  
\\
We now know enough about the structure of $G$ to establish some relations on it. Let $A_1 = \langle x \rangle$, so $x^n = I_G$. By Theorem \ref{MaximalAbelianSubgroup.of_index_normalizer_eq_two} there exists a $y \in N_G(A_1) \! \setminus \! A_1$ such that $y x y^{-1} = x^{-1}$.
\begin{align*} |\mathcal{C}_1| &= [G : N_G(A_1)] = 1.
\\ |\mathcal{C}_{Q \times Z}| &= [G : N_G(Q \times Z)] = \frac{|G|}{eqk} = \frac{2n}{2} = n.
\end{align*}
The only maximal abelian subgroups of $G$ are thus $A_1$ and the $n$ conjugate subgroups of $\mathcal{C}_{Q \times Z}$.

% \begin{center}
% \begin{tikzpicture}[thick, scale=0.4]

% \draw[rotate around={0:(0,0)},green] (3,0) ellipse (108pt and 41pt);  
% \draw[dashed][rotate around={20:(0,0)},green] (3,0) ellipse (108pt and 41pt);  
% \draw[dashed][rotate around={40:(0,0)},green] (3,0) ellipse (108pt and 41pt); 
% \draw[dashed][rotate around={60:(0,0)},lightgray] (3,0) ellipse (108pt and 41pt);  
% \draw[dashed][rotate around={80:(0,0)},lightgray] (3,0) ellipse (108pt and 41pt);  
% \draw[dashed][rotate around={100:(0,0)},lightgray] (3,0) ellipse (108pt and 41pt);  
% \draw[dashed][rotate around={120:(0,0)},green] (3,0) ellipse (108pt and 41pt);  

% \draw[rotate around={210:(0,0)},blue] (3,0) ellipse (108pt and 41pt);

% \draw (0,0) ellipse (22pt and 22pt); 

% \node[] at (0,-6) {\resizebox{9cm}{!}{Fig 4: The elements of $G$ arranged into maximal abelian subgroups.}};
% \node[] at (0,0) {\resizebox{.3cm}{!}{$Z$}};
% \node[] at (8.4,0.2) {\resizebox{1cm}{!}{$Q \times Z$}};
% \node[] at (-6.8,-3.1) {\resizebox{.5cm}{!}{$A_1$}};
% \node[] at (4.7,8.3) {\resizebox{1.1cm}{!}{$\mathcal{C}_{Q \times Z}$}};

% \node[scale=2.5, rotate=240,gray] at (4.0,6.7) { $\Bigg\{$ };

% \node[scale=2, black] at (-.45,0) {.};

% \node[scale=3, green] at (5.5, -0.1) {.};
% \node[scale=2, green] at (5.5, 2.0) {.};
% \node[scale=2, green] at (4.5,3.7) {.};
% \node[scale=1.3, gray] at (2.8,5.0) {.};
% \node[scale=1.3, gray] at (1.1,5.7) {.};
% \node[scale=1.3, gray] at (-0.9,5.7) {.};
% \node[scale=2, green] at (-2.8, 4.8) {.};

% \node[scale=1.6, blue] at (-5.1,-3.0) {.};
% \node[scale=3, blue] at (-3.4,-2.4) {.};
% \node[scale=1.6, blue] at (-3.6,-1.4) {.};
% \node[scale=1.6, blue] at (-2.4,-0.7) {.};
% \node[scale=1.6, blue] at (-2,-1.9) {.};
% \node[scale=1.6, blue] at (-0.8,-1.2) {.};

% \end{tikzpicture}
% \end{center}

Since $y$ belongs to some maximal abelian subgroup and $y \not \in A_1$, $y$ must belong to some element of $\mathcal{C}_{Q \times Z}$. Since $|Q \times Z|$ = 2, the order of $y$ is 2 and $y^2 = I_G$. We have established the following presentation of G.
\begin{align*} G = \langle x, y \; | \; x^n = I_G = y^2, \; y x y^{-1} = x^{-1} \rangle.
\end{align*}

Let $D_n$ denote the dihedral group of order $2n$, that is the group of symmetries of a regular polygon wih $n$ vertices. Let $r$ denote a clockwise rotation by $2\theta /n$ radians and $s$ denote a reflection. For $n$ odd, it can easily be verified that $D_n$ has the following presentation.
\begin{align*} D_n = \langle r, s \; | \; r^n = I = s^2, \; s r s^{-1} = r^{-1} \rangle.
\end{align*}

Since $G$ and $D_n$ have the same order and since their respective generators satisfy the corresponding relations, there is an isomorphism mapping $x \mapsto r$ and $y \mapsto s$. Thus,
\begin{align*} G = \langle x, y \rangle \cong \langle r, s \rangle = D_n.
\end{align*}

 \space \textbf{Case IVb:} $\pmb{q = 3}$. Now equation (\ref{case4a}) becomes
\begin{align*} \frac{1}{2g_1} &= \frac{1}{g} + \frac{1}{6} \; > \; \frac{1}{6}.
\end{align*}
This means that $g_1 = 2$ and $g = 12$. Since $q=3$ we have $p=3$ and $e=2$. Furthermore we have,
\begin{align*} |G| = 24, \quad |A_1| &= 4,  \quad |N_G(A_1)| = 8, \quad |Q \times Z| = 6 \quad |N_G(Q \times Z)| = 6
\end{align*}
\begin{align*} |\mathcal{C}_1| &= [G : N_G(A_1)] = \frac{24}{8} = 3
\\ |\mathcal{C}_{Q \times Z}| &= [G : N_G(Q \times Z)] = \frac{24}{6} = 4
\end{align*}
% \begin{center}
% \begin{tikzpicture}[thick, scale=0.4]

% \draw[dashed][rotate around={0:(0,0)},green] (3,0) ellipse (108pt and 41pt);  
% \draw[dashed][rotate around={20:(0,0)},green] (3,0) ellipse (108pt and 41pt);  
% \draw[rotate around={40:(0,0)},green] (3,0) ellipse (108pt and 41pt); 
% \draw[dashed][rotate around={60:(0,0)},green] (3,0) ellipse (108pt and 41pt);  

% \draw[dashed][rotate around={180:(0,0)},blue] (2,0) ellipse (79pt and 37pt);  
% \draw[rotate around={210:(0,0)},blue] (2,0) ellipse (79pt and 37pt);
% \draw[dashed][rotate around={240:(0,0)},blue] (2,0) ellipse (79pt and 37pt);

% \draw (0,0) ellipse (22pt and 22pt); 

% \node[] at (0,-7) {\resizebox{9cm}{!}{Fig 5: The elements of $G$ arranged into maximal abelian subgroups.}};
% \node[] at (0,0) {\resizebox{.3cm}{!}{$Z$}};
% \node[] at (5.4,5.2) {\resizebox{1cm}{!}{$Q \times Z$}};
% \node[] at (-4.6,-2.8) {\resizebox{.5cm}{!}{$A_1$}};
% \node[] at (9.3,5.3) {\resizebox{1.1cm}{!}{$\mathcal{C}_{Q \times Z}$}};
% \node[] at (-6.8,-3.6) {\resizebox{.5cm}{!}{$\mathcal{C}_1$}};

% \node[scale=1.8, rotate=30,gray] at (-5.4,-3.2) { $\Bigg\{$ };
% \node[scale=2, rotate=210,gray] at (7.7,4.4) { $\Bigg\{$ };

% \node[scale=2, black] at (-.45,0) {.};
% \node[scale=2, black] at (.45,0) {.};

% \node[scale=3, green] at (4,4) {.};
% \node[scale=3, green] at (4.7,4.2) {.};
% \node[scale=3, green] at (4.8,3.3) {.};
% \node[scale=3, green] at (3.9, 3.2) {.};
% \node[scale=2, green] at (4.8, 1.7) {.};
% \node[scale=2, green] at (5.2, 2.3) {.};
% \node[scale=2, green] at (5.9, 2.2) {.};
% \node[scale=2, green] at (5.6, 1.5) {.};
% \node[scale=2, green] at (6, 0.2) {.};
% \node[scale=2, green] at (5.5, -0.5) {.};
% \node[scale=2, green] at (4.6, -0.8) {.};
% \node[scale=2, green] at (3.7, -1) {.};
% \node[scale=2, green] at (3, 5.2) {.};
% \node[scale=2, green] at (2.2,4.5) {.};
% \node[scale=2, green] at (1.5, 4.0) {.};
% \node[scale=2, green] at (0.9, 3.3) {.};

% \node[scale=3, blue] at (-3.5,-1.6) {.};
% \node[scale=3, blue] at (-3.2,-2.4) {.};
% \node[scale=2, blue] at (-3.6,0.4) {.};
% \node[scale=2, blue] at (-2.4,0.6) {.};
% \node[scale=2, blue] at (-2,-3.3) {.};
% \node[scale=2, blue] at (-1.0,-2.9) {.};

% \end{tikzpicture}
% \end{center}

Notice that Fig 5 is almost identical to Fig 2 in the study of Case IIb. This is a strong indication that these 2 cases are isomorphic to each other and hence also to $\SL_2(3)$, albeit not a proof. However, an argument analogous to the one outlined in the proof of Case IIb can be directly applied here with a simple renaming of the conjugacy classes and representatives. It would be to repeat this argument again and I will leave it to the reader to verify.

\end{proof}
\begin{theorem}[Case V]
\label{case_V}
\uses{card_noncenter_fin_subgroup_eq_sum_card_noncenter_mul_index_normalizer, MaximalAbelianSubgroup.K_mem_MaximalAbelianSubgroups_of_center_lt_card_K, MaximalAbelianSubgroup.IsCyclic_and_card_coprime_CharP_or_eq_Q_join_Z}
\lean{case_V}
Claim: \textit{We have one of the following three cases: \\
\\
(i) $G \cong \SL_2(\mathbb{F}_q)$. \\
\\
(ii) $G \cong \langle \SL_2(\mathbb{F}_q), d_\pi \rangle$, where $\pi \in \mathbb{F}_{q^2} \setminus \mathbb{F}_q$, $\pi^2 \in \mathbb{F}_q$ and $\SL_2(\mathbb{F}_q) \vartriangleleft G$. \\
\\
(iii) $G \cong \SL_2(5)$ and $p=3=q$.}
\end{theorem}

\begin{proof} Here, $s = 0$ and $t = 2$. Equation (\ref{classeq}) simplifies to:
\begin{align} \label{case5a} 1 &= \frac{1}{g} + \frac{q-1}{qk} + \frac{g_1 -1}{2g_1} + \frac{g_2 -1}{2g_2}, \nonumber
\\ 
\frac{1}{2g_1} + \frac{1}{2g_2} &= \frac{1}{g} + \frac{q-1}{qk}. \end{align}

Recall that,
\begin{align*} eg = |G| \geq  |N_G(A_i)| \geq 2eg_i, \qquad \text{thus} \quad \! \frac{1}{g} \leq \frac{1}{2g_i}.
\end{align*}
Equation (\ref{case5a}) is therefore bounded from below:
\begin{align*}  \frac{2}{g} \leq \frac{1}{2g_1} + \frac{1}{2g_2} = \frac{1}{g} + \frac{q-1}{qk}. 
\end{align*}
Therefore $q>1$, since if $q=1$ we arrive at the contradiction $2/g \leq 1/g$. With this in mind we have $(q-1)/q \geq 1/2$ and since $g_i \geq 2$ this allows us to bound (\ref{case5a}) on either side.

\begin{align*} \frac{1}{2} &\geq \frac{1}{2g_1} + \frac{1}{2g_2} = \frac{1}{g} + \frac{q-1}{qk} > \frac{q-1}{qk} \geq \frac{1}{2k}.
\end{align*}

This gives $k > 1$ and so by Theorem \ref{MaximalAbelianSubgroup.K_mem_MaximalAbelianSubgroups_of_center_lt_card_K}, $k$ must equal $g_1$ or $g_2$ since the inequality $ek = |K| > |Z| = e$ holds. Without loss of generality we let $k=g_1$ and (\ref{case5a}) becomes,

\begin{align} \label{case5b} \frac{1}{2g_1} + \frac{1}{2g_2} &= \frac{1}{g} + \frac{q-1}{qg_1} = \frac{1}{g} + \frac{1}{g_1} - \frac{1}{qg_1}, \nonumber \\[1.5ex]
 \frac{1}{2g_2} &= \frac{1}{g} + \frac{1}{2g_1} - \frac{1}{qg_1}.
\end{align}
\\
Let $N_G(Q)$ act on $Q \! \setminus \! I_G$ by conjugation and consider the stabiliser in $N_G(Q)$ of an arbitrarily chosen $x \in Q \! \setminus \! I_G$.
\begin{align*} \text{Stab}(x) &= \{ g \in N_G(Q) : g x g^{-1} = x \}
\\ &= C_G(x) \cap N_G(Q)
\\ &= (Q \times Z) \cap N_G(Q) \tag{by Theorem \ref{MaximalAbelianSubgroup.IsCyclic_and_card_coprime_CharP_or_eq_Q_join_Z}}
\\ &= Q \times Z. \tag{since $Q \times Z \subset N_G(Q)$}
\end{align*}

Thus by the Orbit-Stabiliser Theorem,
\begin{align*} |\text{Orb}(x)| = [N_G(Q) : Q \times Z] = \frac{eqk}{eq} = k
\end{align*}

Since $x$ was chosen arbitrarily from $Q \! \setminus \! I_G$, each element of $Q \! \setminus \! I_G$ has an orbit in $N_G(Q)$ of size $k$. Considering also the fact that $Q \! \setminus \! I_G$ is equal to the union of the pairwise disjoint orbits of its elements, we conclude that $k = g_1$ divides $|Q \! \setminus \! I_G|$. Thus there exists some $d \in \mathbb{Z^+}$ such that,
\begin{align}\label{6.14} q-1 = d g_1.
\end{align}

Now set,
\begin{align} \label{6.14a} i = \frac{2 g_1 g_2 q}{g} > 0,
\end{align}
and multiply \eqref{case5b} by $ig$ to give,
\begin{align}\label{6.15} g_1 q &= i + (q-2) g_2.
\end{align}
Thus $i$ is an integer and since it is greater than zero by definition, \eqref{6.15} gives,
\begin{align}\label{6.16b} g_1 > \frac{(q-2) g_2}{q}.
\end{align}
Also, using \eqref{6.14} and \eqref{6.15} we get,
\begin{align}\label{6.16a} g_1 q &= i + (q-1) g_2 - g_2 \nonumber
\\ &= i + d g_1 g_2 - g_2, \nonumber
\\ g_2 &= i + (d g_2 - q) g_1.
\end{align}

Applying Lemma \ref{caseVlemma} we observe that $Q$ is not normal in $G$, and so 
\begin{align*} eg = |G| &> |N_G(Q)| = eqk = eqg_1, \\[1.5ex]
\frac{1}{qg_1} &> \frac{1}{g}.
\end{align*}
And (\ref{case5b}) gives us,
\begin{align}\label{6.13}  \frac{1}{2g_2} &= \frac{1}{g} - \frac{1}{qg_1} + \frac{1}{2g_1} < \frac{1}{2g_1}, \nonumber
\\[1.5ex] g_1 &< g_2.
\end{align}

Consider now,
\begin{align*} [G : N_G(Q)] = \frac{eg}{e q k} = \frac{g}{q g_1} = \frac{2 g_2}{i} \in \mathbb{Z}. \tag{by \eqref{6.14a}}
\end{align*}
Thus $i$ divides $2 g_2$. Recall that the order of $A_2$ is relatively prime to $p$ by Theorem \ref{MaximalAbelianSubgroup.IsCyclic_and_card_coprime_CharP_or_eq_Q_join_Z}, so $g_2$ is also relatively prime to $p$. Therefore if $p \neq 2$, $i$ is relatively prime to $p$ and if $p=2$ then $p$ divides $i$ but $p^2$ does not. Now since $Q$ is a Sylow $p$-subgroup of $G$, this means that greatest common denominator of $i$ and $q$ is either 1 or 2.
Now consider,
\begin{align*} [G : N_G(A_2)] = \frac{eg}{2 e g_2} = \frac{g_1 q}{i} \in \mathbb{Z}. \tag{by \eqref{6.14a}}
\end{align*}
Thus $i$ divides $g_1 q$ and since gcd$(i, q) = 1$ or 2, i must divide $2 g_1$. So there exists some $m \in \mathbb{Z^+}$ such that,
\begin{align}\label{6.17} i = \frac{2 g_1}{m}.
\end{align}

We consider now the separate cases which arise for different values of $q$. \\
\\
 \space \textbf{Cases Va and Vb:} $\pmb{q \geq 4}$. This condition gives us a lower bound for the inequality in \eqref{6.16b},
\begin{align*} g_1 > \frac{(q-2) g_2}{q} > \frac{g_2}{2}.
\end{align*}
Combining this with \eqref{6.13} we have,
\begin{align}\label{6.18} g_1 < g_2 < 2 g_1.
\end{align}

Substituting \eqref{6.17} into \eqref{6.16a} gives,
\begin{align*} g_2 = \left( \frac{2}{m} + d g_2 - q \right) g_1
\end{align*}
Thus \eqref{6.18} gives that,
\begin{align*} 1 < \frac{2}{m} + d g_2 - q < 2.
\end{align*}

This means that $2/m$ is some fraction between 0 and 1 and $d g_2 - q = 1$. So \eqref{6.16a} becomes,
\begin{align}\label{6.19} g_2 = g_1 + i.
\end{align}

Substituting this into \eqref{case5b} we find that,
\begin{align*} g_1 q &= i + (q - 2)(g_1 + i),
\\ 2 g_1 &= i(q - 1) = i d g_1, \tag{by \eqref{6.14}}
\\ 2 &= i d.
\end{align*}

We remark that since both $i$ and $d$ are positive integers, $i$ (and indeed $d$) must equal 1 or 2. Thus by \eqref{6.19} and \eqref{6.14a},
\begin{align*} g_1 &= \frac{i(q-1)}{2}, \qquad g_2 = \frac{i(q + 1)}{2}, \qquad g = \frac{2 g_1 g_2 q}{i} = \frac{iq(q^2 - 1)}{2}.
\end{align*}

Thus we have the following expressions for the orders of $K$ and $G$:
\begin{align}\label{orderGK} |K| = \frac{ei(q-1)}{2}, \qquad |G| = \frac{eiq(q^2-1)}{2}.
\end{align}

By Proposition \ref{6.7}, each noncentral element of $Q$ has a unique common fixed point on the projective line $\mathscr{L}$, call it $P_1$. Furthermore, we saw in the proof of Theorem \ref{MaximalAbelianSubgroup.K_mem_MaximalAbelianSubgroups_of_center_lt_card_K} that each noncentral element of $K$ also fixes $P_1$ as well as one other point, call it $P_2$. Let $u$ be a noncentral element of $Q$ and set $P_3 = P_2^u$. Clearly $P_3$ is different from $P_1$ and $P_2$ because otherwise a contradiction is reached. By Theorem \ref{6.6}, $PSL(\mathscr{L})$ is triply transitive, so there exists a $v \in L$ such that,
\begin{align*} P_1^v = R_1 = \begin{bmatrix} 0 \\ 1 \end{bmatrix}, \qquad P_2^v = R_2 = \begin{bmatrix} 1 \\ 0 \end{bmatrix}, \qquad P_3^v = R_3 = \begin{bmatrix} 1 \\ 1 \end{bmatrix}.
\end{align*} 

Observe that,
\begin{align*} vQv^{-1}R_1 &= vQP_1 = vP_1 = R_1,
\\ vKv^{-1}R_i &= vKP_i = vP_i = R_i. \qquad (i=1,2)
\end{align*} 

Thus $vQv^{-1}$ fixes $R_1$ whilst $vKv^{-1}$ fixes both $R_1$ and $R_2$. The only elements of $L$ that fix $R_1$ are the lower triangular matrices, thus  $vQv^{-1} \subset H$, whilst the only elements that fix $R_2$ are the upper triangular matrices, thus $vKv^{-1} \subset D$. Furthermore, each noncentral element of $vQv^{-1}$ has order $p$. The only elements of $H$ with order $p$ are those in $T$, thus $vQv^{-1} \subset T$. Since $u \in Q \setminus I_G$, we have that $v u v^{-1} = t_\gamma$ for some $\gamma \in F$.
\begin{align*} v u v^{-1}R_2 &= v u P_2 = v P_3 = R_3,
\\[1.5ex] \begin{bmatrix} 1 & 0\\ \gamma & 1 \end{bmatrix} \begin{bmatrix} 1 \\ 0 \end{bmatrix} &= \begin{bmatrix} 1 \\ \gamma  \end{bmatrix} \sim \begin{bmatrix} 1 \\ 1 \end{bmatrix}. \Longrightarrow \gamma = 1.
\end{align*}

So $v u v^{-1} = t_1$. If we now consider $\widetilde{G} = vGv^{-1}$ instead of $G$, we can assume without loss of generality that,
\begin{align*} Q \subset T, \qquad K \subset D, \qquad u = t_1.
\end{align*}

Let $x$ be a generator of $K$. By Theorem \ref{MaximalAbelianSubgroup.of_index_normalizer_eq_two} there exists a $y \in N_{\widetilde{G}}(K) \! \setminus \! K$ such that $y x = x^{-1} y$. Since $R_1$ is fixed by both $x$ and $x^{-1}$ we have,
\begin{align*} x^{-1} y R_1 =  y x R_1 = y R_1.
\end{align*}
Thus $x^{-1}$ fixes $y R_1$, that is $y R_1 \in \{ R_1, R_2 \}$. Similarly, $y R_2 \in \{ R_1, R_2 \}$. Assume $y R_1 = R_1$. Since $R_1$ and $R_2$ are distinct points in $\mathscr{L}$ this implies that $y R_2 = R_2$.

\begin{align*} y R_1 = \begin{bmatrix} \alpha & \beta \\ \gamma & \delta \end{bmatrix} \begin{bmatrix} 0 \\ 1 \end{bmatrix} &= \begin{bmatrix} \beta \\ \delta \end{bmatrix} \sim \begin{bmatrix} 0 \\ 1 \end{bmatrix} \Longrightarrow \beta = 0.
\\[1.5ex] y R_2 = \begin{bmatrix} \alpha & \beta \\ \gamma & \delta \end{bmatrix} \begin{bmatrix} 1 \\ 0 \end{bmatrix} &= \begin{bmatrix} \alpha \\ \gamma \end{bmatrix} \sim \begin{bmatrix} 1 \\ 0 \end{bmatrix} \Longrightarrow \gamma = 0.
\end{align*}

Thus $y \in D$, which is a contradiction since elements in $D$ do not invert $x \in D$, hence,
\begin{align}\label{yinterchange} y R_1 = R_2, \qquad \text{and} \quad y R_2 = R_1.
\end{align}
 
This allows us to determine more about $y$,
\begin{align*} y R_1 = \begin{bmatrix} \alpha & \beta \\ \gamma & \delta \end{bmatrix} \begin{bmatrix} 0 \\ 1 \end{bmatrix} &= \begin{bmatrix} \beta \\ \delta \end{bmatrix} \sim \begin{bmatrix} 1 \\ 0 \end{bmatrix} \Longrightarrow \delta = 0.
\\[1.5ex] y R_2 = \begin{bmatrix} \alpha & \beta \\ \gamma & \delta \end{bmatrix} \begin{bmatrix} 1 \\ 0 \end{bmatrix} &= \begin{bmatrix} \alpha \\ \gamma \end{bmatrix} \sim \begin{bmatrix} 0 \\ 1 \end{bmatrix} \Longrightarrow \alpha = 0.
\end{align*}

Thus $y$ is an anti-diagonal matrix. Recalling \eqref{antidiag}, for some $\rho \in F^*$ we have,
\begin{align*} y = d_\rho w = \begin{bmatrix} 0 & \rho \\ -\rho^{-1} & 0 \end{bmatrix}.
\end{align*}

Consider now the set of right cosets of $N_{\widetilde{G}}(Q)$ of the form $N_{\widetilde{G}}(Q) y q$, (where $q \in Q$) in $N_{\widetilde{G}}(Q) y Q$. For $q_1, q_2 \in Q$ we have,
\vspace{2mm}
\begin{align*} N_{\widetilde{G}}(Q) y q_1 = N_{\widetilde{G}}(Q) y q_2 &\iff y q_2 {q_1}^{-1} y^{-1} \in N_{\widetilde{G}}(Q)
\\ &\iff q_2 {q_1}^{-1} \in y^{-1} N_{\widetilde{G}}(Q) y
\\ &\iff (Q \cap y^{-1} N_{\widetilde{G}}(Q) y) q_2 = (Q \cap y^{-1} N_{\widetilde{G}}(Q) y) q_1. \\
\end{align*}

So the number of right cosets of $N_{\widetilde{G}}(Q)$ in $N_{\widetilde{G}}(Q) y Q$ is equal to the number of right cosets of $Q \cap y^{-1} N_{\widetilde{G}}(Q) y$ in $Q$. That is,
\vspace{2mm}
\begin{align}\label{doublecoset} [N_{\widetilde{G}}(Q) y Q : N_{\widetilde{G}}(Q)] = [Q : Q \cap y^{-1} N_{\widetilde{G}}(Q) y]. \\ \nonumber
\end{align}

Let $g$ be an arbitrary element of $N_{\widetilde{G}}(Q)$. By Theorems \ref{6.4i}(i) and \ref{6.7}(ii) we have $N_{\widetilde{G}}(Q) \subset H = \text{Stab}(R_1)$, thus $g$ fixes $R_1$. Using \eqref{yinterchange} we see that,
\vspace{2mm}
\begin{align*} y^{-1} g y R_2 = y^{-1} g R_1 = y^{-1} R_1 = R_2. \\
\end{align*}

Hence $R_2$ is a fixed point of $y^{-1} g y$. Since $g$ was chosen arbitrarily, we assert that each element of $y^{-1} N_{\widetilde{G}}(Q) y$ fixes $R_2$. On the contrary, the only element of $Q$ which fixes $R_2$ is $I_{\widetilde{G}}$, thus $Q \cap y N_{\widetilde{G}}(Q) y^{-1} = I_{\widetilde{G}}$.
\vspace{2mm}
\begin{align}\label{qwed} [N_{\widetilde{G}}(Q) y Q : N_{\widetilde{G}}(Q)] &= [Q : Q \cap y^{-1} N_{\widetilde{G}}(Q) y] = q, \nonumber
\\[1ex] |N_{\widetilde{G}}(Q) y Q| &= q|N_{\widetilde{G}}(Q)|. \\ \nonumber
\end{align}

We show next that $N_{\widetilde{G}}(Q) y Q \cap N_{\widetilde{G}}(Q) = \varnothing$. Let $t_\lambda d_\omega$ and $t_\mu$ be arbitrarily chosen from $N_{\widetilde{G}}(Q)$ and $Q$ respectively so that $t_\lambda d_\omega y t_\mu$ is an arbitrary element of $N_{\widetilde{G}}(Q) y Q$.
\begin{align}\label{onemore} t_\lambda d_\omega y t_\mu &= \begin{bmatrix} 1 & 0 \\ \lambda & 1 \end{bmatrix} \begin{bmatrix} \omega & 0 \\ 0 & \omega^{-1} \end{bmatrix} \begin{bmatrix} 0 & \rho \\ -\rho^{-1} & 0 \end{bmatrix}  \begin{bmatrix} 1 & 0 \\ \mu & 1 \end{bmatrix} \nonumber
\\[1.5ex] &= \begin{bmatrix} \omega & 0 \\ \omega \lambda & \omega^{-1} \end{bmatrix} \begin{bmatrix} \rho \mu & \rho \\ -\rho^{-1} & 0 \end{bmatrix} \nonumber
\\[1.5ex] &= \begin{bmatrix} \omega \rho \mu & \omega \rho  \\ \omega \lambda \rho \mu - \omega^{-1} \rho^{-1} & \omega \rho \lambda \end{bmatrix}.
\end{align}

Since $\omega$, $\rho \in F^*$, the top right entry of \eqref{onemore} is non-zero. Recall also that $N_{\widetilde{G}}(Q) \subset H$ by Theorem \ref{6.4i}(i) and that $H$ is the set of all lower triangular matrices of $L$. Since $t_\lambda d_\omega d_\rho w t_\mu$ was chosen arbitraily, no element of $N_{\widetilde{G}}(Q) y Q$ is in $H$ whilst the whole of $N_{\widetilde{G}}(Q)$ is contained in $H$, thus they are disjoint. Using \eqref{qwed} and \eqref{orderGK} we also observe that,
\begin{align*} |N_{\widetilde{G}}(Q) y Q| + |N_{\widetilde{G}}(Q)| = (q+1)|N_{\widetilde{G}}(Q)| = (q+1)eqg_1 = \frac{eiq(q^2-1)}{2} = |{\widetilde{G}}|.
\end{align*}
Since $N_{\widetilde{G}}(Q) y Q$ and $N_{\widetilde{G}}(Q)$ are disjoint and the sum of their orders is equal to the order of ${\widetilde{G}}$, they partition ${\widetilde{G}}$ into the set of elements that belong to $H$ and the set that don't.
\begin{align}\label{gsplit} {\widetilde{G}} = N_{\widetilde{G}}(Q) y Q \cup N_{\widetilde{G}}(Q).
\end{align}

Let $\mathbb{N} = \{ \lambda : t_\lambda \in Q \}$. We will show that $\mathbb{N} =\mathbb{F}_q$. For each $t_\lambda \in Q \setminus Z$, the element $y t_\lambda y^{-1} \notin H$, so by $\eqref{gsplit}$, $y t_\lambda y^{-1} \in N_{\widetilde{G}}(Q) y Q$. Thus there exists $t_\mu, t_\upsilon \in Q$ and $d_\omega \in K$ such that,
\begin{align*} y t_\lambda y^{-1} &= t_\mu d_\omega y t_\upsilon,
\\[1.5ex] \begin{bmatrix} 0 & \rho \\ -\rho^{-1} & 0 \end{bmatrix}\begin{bmatrix} 1 & 0 \\ \lambda & 1 \end{bmatrix}\begin{bmatrix} 0 & -\rho \\ \rho^{-1} & 0 \end{bmatrix} &= \begin{bmatrix} 1 & 0 \\ \mu & 1 \end{bmatrix}\begin{bmatrix} \omega & 0 \\ 0 & \omega^{-1} \end{bmatrix}\begin{bmatrix} 0 & \rho \\ -\rho^{-1} & 0 \end{bmatrix}\begin{bmatrix} 1 & 0 \\ \upsilon & 1 \end{bmatrix},
\\[1.5ex] \begin{bmatrix} 0 & \rho \\ -\rho^{-1} & 0 \end{bmatrix}\begin{bmatrix} 0 & -\rho \\ \rho^{-1} & -\rho \lambda \end{bmatrix} &= \begin{bmatrix} \omega & 0 \\ \omega \mu & \omega^{-1} \end{bmatrix}\begin{bmatrix} \rho \upsilon & \rho \\ -\rho^{-1} & 0 \end{bmatrix},
\\[1.5ex] \begin{bmatrix} 1 & -\rho^2 \lambda \\ 0 & 1 \end{bmatrix} &= \begin{bmatrix} \omega \rho \upsilon & \omega \rho \\ \omega \rho \mu \upsilon - \omega^{-1} \rho^{-1} & \omega \rho \mu \end{bmatrix}.
\end{align*}

Equating the top right entries gives,
\begin{align}\label{mattr} \omega = -\rho \lambda.
\end{align}

Since $t_1 \in Q$, so is it's inverse, thus $-1 \in \mathbb{N}$. Letting $\lambda = -1$ in \eqref{mattr} gives $\omega = \rho$, which means that $d_\rho \in K$. Consequently, this shows that $w = d_\rho^{-1} y \in {\widetilde{G}}$ and we may replace $y$ by $w$ in \eqref{gsplit} without it affecting the partition of ${\widetilde{G}}$. This is equivalent to letting $\rho = 1$, and \eqref{mattr} simplifies to,
\begin{align}\label{mattr2} \omega = -\lambda.
\end{align}

Let $\mathbb{M} = \{ \omega : d_\omega \in K \}$. Recall from \eqref{orderGK} that $|K| = i(q-1)$. We consider the different cases which arise depending on the values of $i$ and $e$. \\
\\
Let \textbf{Case Va} be the case when $e=1$ or $i = 1$. Observe that $i$ and $e$ cannot both equal 1, since this would imply that 2 divides $q-1$ (by \eqref{orderGK}), but if $e=1$ it follows that $q-1$ is even. Hence $ei = 2$ and $K$ has order $q-1$. Furthermore, the order of each element of $K$ divides $q-1$, so for each $\omega \in \mathbb{M}$,
\begin{align}
    \label{roots} \omega^{q-1} = 1.
\end{align}
Also, the following polynomial has at most $q-1$ roots in $F$.
\begin{align}\label{rootsx} x^{q-1} = 1.
\end{align}
By \eqref{subfield}, $\mathbb{F}_q \subset F$ and each element of $\mathbb{F}^*_q$ is a root of \eqref{rootsx}. Thus each $\omega$ of $\mathbb{M}$ is in $\mathbb{F}^*_q$ and since they have the same cardinality, $\mathbb{M} = \mathbb{F}^*_q$. By \eqref{mattr2}, $\lambda$ also ranges over $\mathbb{F}^*_q$ and considering also that $\lambda$ can be 0, we have $\mathbb{N} =\mathbb{F}_q$. \\
\\
Observe that each element of ${\widetilde{G}}$ is either of the form $t_\lambda d_\omega$ or $t_\lambda d_\omega w t_\mu$ (where $\lambda, \mu \in \mathbb{F}_q$, $\omega \in \mathbb{F}^*_q$), so ${\widetilde{G}} \subset \SL_2(\mathbb{F}_q)$. Also, Propostion \ref{ordersl2q} gives that, $|\SL_2(\mathbb{F}_q)| = q(q^2-1) = |{\widetilde{G}}|$, so ${\widetilde{G}} = \SL_2(\mathbb{F}_q)$. Since ${\widetilde{G}}$ is conjugate in $L$ to $G$, we have $G \cong \SL_2(\mathbb{F}_q)$  as desired. \\
\\
Let \textbf{Case Vb} be the case when $i = 2 = e$. This time the order of each element of $K$ divides $2(q-1)$, so for each $\omega \in \mathbb{M}$,
\begin{align}
    \label{roots} \omega^{2(q-1)} = 1.
\end{align}
As in the case of $i=1$, each element of $\mathbb{F}^*_q$ is a root of the polynomial in \eqref{rootsx}, as are each $\omega^2$. Thus $\omega^2$ ranges over $\mathbb{F}^*_q$ and by \eqref{subfield}, $\omega \in \mathbb{F}_{q^2} \setminus \mathbb{F}_q$. Simple matrix multiplication shows that, \\
\begin{align*} d_\omega^{-1} t_\lambda d_\omega = t_{\omega^2 \lambda}.
\end{align*}
Hence since $t_0, t_1 \in Q$, it follows that $t_{\omega^2} \in Q$ for each $\omega^2 \in \mathbb{F}^*_q$, thus $\mathbb{N} = \mathbb{F}_q$. Since $K$ is a cyclic group of order $2(q-1)$, so too is $\mathbb{M}$. Let $\pi$ be a generator of $\mathbb{M}$. It follows that $\pi^2$ has order $q-1$ and is therefore a generator of $\mathbb{F}^*_q$. Since $K = \langle d_\pi \rangle$, we have:
\begin{align*} {\widetilde{G}} = \langle t_\lambda, d_\pi, w : \lambda \in \mathbb{F}_q \rangle = \langle \SL_2(\mathbb{F}_q), d_\pi \rangle.
\end{align*}
Again, since ${\widetilde{G}}$ is conjugate in $L$ to $G$, we have $G \cong \langle \SL_2(\mathbb{F}_q), d_\pi \rangle$ as desired. Now we take an arbitrary $x$ from $\SL_2(\mathbb{F}_q)$ and conjugate it by $d_\pi$.
\begin{align*} d_\pi x d_\pi^{-1} &= \begin{bmatrix} \pi & 0 \\ 0 & \pi^{-1} \end{bmatrix} \begin{bmatrix} \alpha & \beta \\ \gamma & \delta \end{bmatrix}\begin{bmatrix} \pi^{-1} & 0 \\ 0 & \pi \end{bmatrix}
\\[1.5ex] &=  \begin{bmatrix} \pi & 0 \\ 0 & \pi^{-1} \end{bmatrix}  \begin{bmatrix} \alpha \pi^{-1} & \beta \pi \\ \gamma \pi^{-1} & \delta \pi \end{bmatrix}
\\[1.5ex] &= \begin{bmatrix} \alpha & \beta \pi^{-2} \\ \gamma \pi^{2} & \delta \end{bmatrix}. 
\end{align*}
Since $\pi^2 \in \mathbb{F}_q$, we have that $d_\pi x d_\pi^{-1} \in \SL_2(\mathbb{F}_q)$ and since $x$ was chosen arbitrarily, $d_\pi$ belongs to the normaliser of $\SL_2(\mathbb{F}_q)$ in $\langle \SL_2(\mathbb{F}_q), d_\pi \rangle$. This shows that $\SL_2(\mathbb{F}_q) \vartriangleleft \langle \SL_2(\mathbb{F}_q), d_\pi \rangle$ as desired. \\
\\
 \space \textbf{Cases Vc and Vd:} $\pmb{q \leq 3}$. Since $q - 1 = d g_1 \geq 2$ by \eqref{6.14}, $q$ cannot equal 2. So $q = 3 = p$, $e = 2$ and thus $g_1 = 2$. The inequalities in \eqref{6.13} and \eqref{6.16b} give,
\begin{align*} 2 < g_2 < 6.
\end{align*}
Also, since $g_2$ is relatively prime to $p=3$, we have $g_2 = 4$ or 5. Let \textbf{Case Vc} be the case when $g_2 = 4$. \eqref{case5b} becomes,
\begin{align*} \frac{1}{8} = \frac{1}{g} + \frac{1}{4} - \frac{1}{6},
\end{align*}

which gives $g = 24$. Observe that,
\begin{align*} |K| = 4 = i(q-1), \qquad |G| = 48 = iq(q^2-1),
\end{align*}
where $i=2$, thus we have the situation as described in Case Vb. That is, $G \cong \langle \SL_2(\mathbb{F}_q), d_\pi \rangle$ with $q=3$.\\
\\
Alternatively, \textbf{Case Vd} occurs when $g_2 = 5$. \eqref{case5b} becomes,
\begin{align*} \frac{1}{10} = \frac{1}{g} + \frac{1}{4} - \frac{1}{6}.
\end{align*}

Thus $g = 60 $ and $|G| = 120$. We verify, using Proposition \ref{ordersl2q}, that $\SL_2(5)$ has the same order as $G$, that is $|\SL_2(5)| = 5(5^2-1) =120$. Observe that,
\begin{align*} |\mathcal{C}_1| &= [G : N_G(A_1)] = \frac{eg}{2eg_1} = 15,
\\[1ex] |\mathcal{C}_2| &= [G : N_G(A_2)] = \frac{eg}{2eg_2} = 6,
\\[1ex] |\mathcal{C}_{Q \times Z}| &= [G : N_G(Q \times Z)] = \frac{eg}{ekq} = 10.
\end{align*}

Now consider the quotient group $G / Z$ of order 60. It's trivial that for all $A_i, A_j \in \mathfrak{M}$, $A_i / Z$ belongs to the same conjugacy class as $A_j / Z$ if and only $A_i$ and $A_j$ belong to the same conjugacy class. So the number of subgroups conjugate to $A_i / Z$ is $|\mathcal{C}_i|$. Similarly, the number of subgroups conjugate to $(Q\times Z) / Z$ is $|\mathcal{C}_{Q \times Z}|$. \\
\\
We now calculate the order of each maximal abelian subgroup of $G$ when we quotient out $Z$.
\begin{align*} |A_1 / Z| = 2, \qquad |A_2 / Z| = 5, \qquad |(Q \times Z) / Z| = 3.
\end{align*}

We now know enough about $G / Z$ to determine the order of each of it's elements: \\
\\
 \space The identity has order 1.gives \\
 \space The non-central element of $A_1 / Z$ has order 2, as does the non-central element in each of the $|\mathcal{C}_1| = 15$ subgroups conjugate to $A_1 / Z$. So there are $15$ elements of order 2. \\
 \space The 4 non-central elements of $A_2 / Z$ have order 5, as do the non-central elements in each of the $|\mathcal{C}_2| = 6$ subgroups conjugate to $A_2 / Z$. Thus there are $24$ elements of order 5. \\
 \space  The 2 non-central elements of $(Q \times Z) / Z$ have order 3, as do the non-central elements in each of the $|\mathcal{C}_{Q \times Z}| = 10$ subgroups conjugate to $(Q \times Z) / Z$. Thus there are $20$ elements of order 3. \\
\\
Since $1+15+24+20=60$, all elements of $G / Z$ are accounted for. \\
\\
Let $N$ be a normal subgroup of $G / Z$. Observe that each non-central element of $A_2 / Z$ is a generator of it, so if $N$ contains one non-central element of $A_2 / Z$, then it contains the whole of it, due to the closure of the group under multiplication and the fact that each element of $A_2 / Z$ is a power of any non-central element. Also, it can easily be seen that normal subgroups are composed of whole conjugacy classes, so since $N$ is normal in $G$, if it contains $A_2 / Z$, it must contain all subgroups conjugate to $A_2 / Z$. The consequence of this is that if $N$ has an element of order 5, then it contains all 24 elements of $G / Z$ of order 5. Similarly, if it contains an element of order 2, it contains all 15 of them and if it contains an element of order 3, it contains all 20 of them. This means that $|N|$ is partitioned by some or all of the elements in $\{ 1, 15, 20, 24 \}$. Bearing in mind that the order of $N$ divides 60 and that $N$ contains the identity element, this means that $N$ is equal to either the identity element or it is the whole of $G / Z$, since it's easy to see that no other partition of those numbers divides 60. Thus $G / Z$ has no non-trivial normal subgroups and is simple. \\
\\
By \cite[p.145]{dummit}, the only simple groups of order 60 are those isomorphic to the alternating group $A_5$ (not to be confused with an element of $\mathfrak{M}$), thus $G / Z \cong A_5$. Since $Z \cong \mathbb{Z}_2$, we have that $G$ is isomorphic to a central extension of $A_5$ which, according to Schur \cite{schur}, is unique and isomorphic to $\SL_2(5)$ as desired. The proofs of these 2 claims are beyond the scope of this thesis. \qedhere

\end{proof}

\begin{theorem}[Case VI]
\label{case_VI}
\uses{card_noncenter_fin_subgroup_eq_sum_card_noncenter_mul_index_normalizer, MaximalAbelianSubgroup.of_index_normalizer_eq_two}
\lean{case_VI}
Claim: \textit{We have one of the following three cases: \\
\\
(i) $G = \langle \, x,y \, | \, x^n = y^2, \, yxy^{-1} = x^{-1} \, \rangle$, where $n$ is even. \\
\\
(ii) $G = \widehat{S}_4$. \\
\\
(iii) $G \cong \SL_2(5)$ and $p$ does not divide $|G|$. \\
\\
Where $\widehat{S}_4$ is one of the representation groups of the symmetric group $S_4$ in which the transpositions correspond to the elements of order 4.} \\
\end{theorem}
% DEPENDENCIES AND STATEMENT

\begin{proof} Here, $s = 0$ and $t = 3$. Equation \eqref{classeq} simplifies to:
\begin{align} \label{case6a} 1 &= \frac{1}{g} + \frac{q-1}{qk} + \frac{g_1 -1}{2g_1} + \frac{g_2 -1}{2g_2} + \frac{g_3 -1}{2g_3}, \nonumber
\\[1ex] \frac{1}{2g_1} + \frac{1}{2g_2} + \frac{1}{2g_3} &= \frac{1}{g} + \frac{q-1}{qk} + \frac{1}{2}.
\end{align}

First assume that $q > 1$ and $k=1$. \eqref{case6a} is thus bounded as follows,
\begin{align*} \frac{3}{4} > \frac{1}{2g_1} + \frac{1}{2g_2} + \frac{1}{2g_3} &= \frac{1}{g} + \frac{q-1}{qk} + \frac{1}{2} > 1,
\end{align*}
which is a contradiction. Now assume that $q > 1$ and $k > 1$. This means that $k=g_i$ for some $i$. Without loss of generality we can assume that $k=g_1$. Now \eqref{case6a} becomes,
\begin{align*} \frac{1}{2} \geq \frac{1}{2g_2} + \frac{1}{2g_3} &\geq \frac{1}{g} + \frac{1}{2} > \frac{1}{2},
\end{align*}
which again is a contradiction, thus we conclude that $q=1$. \eqref{case6a} simplifies and we can now determine the possible values of each $g_i$.
 \begin{align} \label{case6b} \frac{1}{2g_1} + \frac{1}{2g_2} + \frac{1}{2g_3} &= \frac{1}{g} + \frac{1}{2}.
\end{align}

Without loss of generality we may assume that $2 \leq g_1 \leq g_2 \leq g_3$. If $g_1 \neq 2$ we arrive at the following contradiction
\begin{align*} \frac{1}{6} + \frac{1}{6} + \frac{1}{6} \geq \frac{1}{2g_1} + \frac{1}{2g_2} + \frac{1}{2g_3} &= \frac{1}{g} + \frac{1}{2}.
\end{align*}
Thus $g_1 = 2$ and we have,
\begin{align}\label{case6c} \frac{1}{2g_2} + \frac{1}{2g_3} > \frac{1}{4}.
\end{align}
\newpage
Clearly $g_2$ must equal either 2 or 3. If $g_2 = 2$ it is easily shown that $g=2 g_3$. If $g_2 = 3$ we see that $g_3 \in \{ 3,4,5 \}$. Assume that $g_2$ and $g_3 = 3$. Notice that since  $g_1 = 2$, 2 must divide the order of $G$. Recall also that a Sylow $p$-subgroup of $G$ has order 1, so we assert that $p \neq 2$ and $e=2$. We see from \eqref{case6b} that $|G| = 24$ and thus a Sylow $3$-subgroup has order 3. The maximal abelian subgroups conjugate to $A_2$ or $A_3$ have order 6 and therefore each contains a Sylow $3$-subgroup of $G$. Let $B_2$ and $B_3$ be the Sylow $3$-subgroups contained in $A_2$ and $A_3$ respectively. Observe that for $i = 2$ or 3,
\begin{align}\label{case6d} A_i \cong \mathbb{Z}_6 \cong \mathbb{Z}_3 \times \mathbb{Z}_2 \cong B_i \times Z \cong B_i Z. 
\end{align}
Let $b_2 \in B_2$, $b_3 \in B_3$ and $z \in Z$. Recall that $B_2$ and $B_3$ are conjugate in $G$ by Sylow's Second Theorem, so there exists an $x \in G$ such that,
\begin{align*} x b_2 x^{-1} &= b_3,
\\ x b_2 x^{-1} z &= b_3 z,
\\ x b_2 z x^{-1} &= b_3 z.
\end{align*} 
Since $b_2$, $b_3$ and $z$ were chosen arbitrarily, we observe that $B_2 Z$ is conjuagate to $B_3 Z$ and thus by \eqref{case6d}, $A_2 \cong A_3$. This contradicts the fact that $A_2$ and $A_3$ are representatives of different conjugacy classes of maximal abelian subgroups of $G$, which means that $g_2$ and $g_3$ cannot both equal 3. Thus we are left with the following three cases:
\begin{align*} g_1 = 2, \qquad g_2&=2, \qquad g=2 g_3.
\\[1ex] g_1 = 2, \qquad g_2&=3, \qquad g_3 = 4.
\\[1ex] g_1 = 2, \qquad g_2&=3, \qquad g_3 = 5.
\end{align*}
\\
 \space \textbf{Case VIa:} $\pmb{g_1 = 2, g_2 = 2, g=2 g_3}$. First observe that,
\begin{align*} [G : N_G(A_1)] = \frac{eg}{2eg_1} = \frac{g_3}{2}.
\end{align*}
Thus $g_3/2$ is an integer which means that $g_3$ must be even, call it $n$. Now let $A_3 = \langle x \rangle$. Since $|A_3| = eg_3$, the order of $x$ is $2n$ and $x^n$ has order 2. By Theorem \eqref{6.8}(iv) there exists a $y \in N_G(A_3) \! \setminus \! A_3$ such that $y x y^{-1} = x^{-1}$. Also,
\begin{align*} |\mathcal{C}_3| = [G : N_G(A_3)] = 1.
\end{align*}
Since $y \not \in A_3$ and $A_3$ has no conjugate subgroups (aside from itself), $y$ must lie in a maximal abelian subgroup conjugate to either $A_1$ or $A_2$. This means that since $|A_1| = 4 = |A_2|$ and $y \not \in Z$, the order of $y$ must be 4. By the uniqueness of the element of order 2, we have the relation $x^n = y^2$ and $G$ is given by the presentation,
\begin{align*} G = \langle \, x,y \, | \, x^n = y^2, \, yxy^{-1} = x^{-1} \, \rangle. \qquad \text{(where $n$ is even)}
\end{align*}

 \space \textbf{Case VIb:} $\pmb{g_1 = 2, g_2 = 3, g_3 = 4}$. In this case \eqref{case6b} becomes,
\begin{align*} \frac{1}{4} + \frac{1}{6} + \frac{1}{8} &= \frac{1}{g} + \frac{1}{2}.
\end{align*}
Thus $g = 24$ and $|G| = 48$. Consider the quotient group $G / Z$ of order 24 and the quotient group $N_G(A_2) / Z$ which, for convenience, we will call $H$.
\begin{align*} |H| = \frac{2eg_2}{e} = 6.
\end{align*}

Let $x$ be an element of order 6 from $A_2$. By Theorem \ref{MaximalAbelianSubgroup.of_index_normalizer_eq_two} there exists a $y \in N_G(A_2) \! \setminus \! A_2$ such that $y x = x^{-1} y$. Thus for $xZ, yZ, x^{-1}Z \in H$ we have,
\begin{align*} yZ xZ = yxZ =  x^{-1}yZ = x^{-1}Z yZ.
\end{align*}
If $H$ is abelian, then $xZ = x^{-1}Z$ and thus $x^2 \in Z$. Also, since $x$ has order 6, $x^2$ has order 3. This is contradiction since there is no element of order 3 in $Z$. Thus $H$ is non-abelian and is therefore isomorphic to the symmetric group $S_3$. \\
\\
Now we determine the normal subgroups of $H$. The identity and $H$ itself are trivially normal. Furthermore, the elementary result that any subgroup of index 2 is normal implies that $A_2 / Z$, the subgroup of $H$ of order 3, is normal. It remains to check the subgroups of order 2. Let r be a generator of one of the subgroups of order 2 and let $x$ be an arbitrary element of $H$. If $\langle r \rangle$ is normal in $H$, then $x r x^{-1} \in \{ I , r \}$. Since $r \neq I$ it follows that $x r x^{-1} \neq I$. Alternatively if $x r x^{-1} = r$, then $r \in Z(H)$. By the elementary result that $Z(S_n) = \{ I \}$ for $n > 2$, we have that $Z(H) = \{ I \}$ and the contradiction $r=I$. Thus $x r x^{-1} \not \in \langle r \rangle$ and $H$ has no normal subgroup of order 2. We conclude that the only normal subgroups of $H$ are those of order 1, 3 or 6. \\
\\
Note that the index of $H$ in $G / Z$ is 4. Let $G / Z$ act by left multiplication on the set of left cosets of $H$. By Theorem \ref{symhomoker}, this action induces a homomorphism $\phi : G / Z \longrightarrow S_4$ with kernel,
\begin{align*} ker(\phi) = \bigcap\limits_{x \in G / Z} x H x^{-1}  \subset H.
\end{align*}

Recall the elementary result that the kernel of a homomorphism is a normal subgroup of it's domain. Thus the kernel of $\phi$ is normal in $G / Z$ and consequently in $H$ as well, that is $ker(\phi) \in\{ I , A_2 / Z, H \}$. \\
\\
If $ker(\phi) = A_2 / Z$, then $A_2 / Z \vartriangleleft G / Z$ and by Lemma \ref{QuotientGroup.comapMk'OrderIso} $A_2 \vartriangleleft G$. This is a contradiction since the normaliser in $G$ of $A_2$ is a proper subgroup of $G$, thus $ker(\phi) \neq A_2 / Z$. \\
\\
If $ker(\phi) = H$, then $H \vartriangleleft G / Z$. Take an arbitrary $x \in G / Z$. Since $A_2 / Z$ is a subgroup of $H$ we get,
\begin{align*} x (A_2 / Z) x^{-1} \subset H.
\end{align*}
Furthermore, since $A_2 / Z$ has order 3, any subgroup conjugate to it has order 3. Since the only subgroup of $H$ of order 3 is $A_2 / Z$, and since $x$ was chosen arbitrarily, $A_2 / Z \vartriangleleft G / Z$. We have already shown that this leads to a contradiction, thus $ker(\phi) \neq H$. \\
\\
We conclude that $ker(\phi) = \{ I \}$ and so $\phi$ is injective. Since $G / Z$ has 24 elements, it's image under $\phi$ is the whole of $S_4$, that is $G / Z \cong S_4$. Thus $G$ is a \textit{representation group} of $S_4$, denoted by $\widehat{S}_4$ (for a full defintion of this, see \cite{suzuki}). Suzuki proves that $S_4$ has 2 distinct representation groups up to isomorphism \cite[p.301]{suzuki}, which are distinguished by the property that the elements corresponding to transpositions have either order 2 or order 4. Since $G$ has a unique element of order 2, it must be isomorphic to the representation group of $S_4$ in which the transpositions correspond to the elements of order 4, as desired.\\
\\
 \space \textbf{Case VIc:} $\pmb{g_1 = 2, g_2 = 3, g_3 = 5}$.  In this case \eqref{case6b} becomes,
\begin{align*} \frac{1}{4} + \frac{1}{6} + \frac{1}{10} &= \frac{1}{g} + \frac{1}{2}.
\end{align*}
Thus $|g| = 60$ and $|G| = 120$. Observe that a simple relabelling of the maximal abelian subgroups gives the same situation as described in \textbf{Case Vd:}. Thus $G \cong \SL_2(5)$, however in this case $p$ does not divide $|G|$.

\end{proof}

\section{Dickson's Classification Theorem}

We now state the main result of this paper, Dickson's classification of finite subgroups of $\SL_2(F)$. Observe that it is not the focus of this paper to determine whether the following groups actually exist, rather that this theorem can be regarded as an \textit{upper bound}, so to speak, of the only possible subgroups of $\SL_2(F)$.\\

\begin{theorem}[Class I]
    \label{dicksons_classification_theorem_class_I}
    \uses{case_I, case_II, case_III, case_VI}
    \lean{dicksons_classification_theorem_class_I} Let $F$ be an arbitary algebraically closed field of characteristic $p$. Any finite subgroup $G$ of $\SL_2(F)$ is isomorphic to one of the following groups. \vspace{3mm} \\
: When $p=0$ or $|G|$ is relatively prime to $p$: \vspace{1mm} \\
(i) A cyclic group. \vspace{3mm} \\
(ii) The group defined by the presentation:
\begin{equation*} \langle \, x,y \, | \, x^n = y^2, \, yxy^{-1} = x^{-1} \, \rangle.
\end{equation*}
(iii) The Special Linear Group $\SL_2(3)$. \vspace{3mm} \\
(iv) The Special Linear Group $\SL_2(5)$. \vspace{3mm} \\
(v) $\widehat{S}_4$, the representation group of $S_4$ in which the transpositions correspond to the elements of order $4$. \\
\\
\end{theorem}
% DEPENDENCIES
\begin{proof}
    Case Ia: This leads to Class I (i). \\
    Case IIa: This leads to Class I (ii) where $n$ is odd. \\
    Case IIb: This leads to Class I (iii). \\
    Case III where $G=Z$: This leads to Class I (i).\\
    Case VIa: This leads to Class I (ii) where $n$ is even. \\
    Case VIb: This leads to Class I (v). \\
    Case VIc: This leads to Class I (iv). \\
    
\end{proof}

\begin{theorem}[Class II]
    \label{dicksons_classification_theorem_class_II}
    \uses{case_I, case_III, case_IV, case_V}
    \lean{dicksons_classification_theorem_class_II}
    When $|G|$ is divisible by $p$: \vspace{1mm} \\
(vi) $Q$ is elementary abelian, $Q \vartriangleleft G$ and $G/Q$ is a cyclic group whose order is relatively prime to $p$. \vspace{3mm} \\
(vii) $p=2$ and $G$ is a dihedral group of order $2n$, where $n$ is odd. \vspace{3mm} \\
(viii) The Special Linear Group $\SL_2(5)$, where $p=3=q$. \vspace{3mm} \\
(ix) The Special Linear Group $\SL_2(\mathbb{F}_q)$. \vspace{3mm} \\
(x) The group $\langle \SL_2(\mathbb{F}_q), d_\pi \rangle$, where $\SL_2(\mathbb{F}_q) \vartriangleleft \langle \SL_2(\mathbb{F}_q), d_\pi \rangle$. \vspace{3mm} \\

Here, $Q$ is a Sylow $p$-subgroup of $G$ of order $q$, $\mathbb{F}_q$ is a field of $q$ elements, $\mathbb{F}_{q^2}$ is a field of $q^2$ elements, $\pi \in \mathbb{F}_{q^2} \setminus \mathbb{F}_q$ and $\pi^2 \in \mathbb{F}_q$. \vspace{3mm}
\end{theorem}

\begin{proof}% \\
% If $Z \subset G$, then $G$ has the same structure as one of the 6 cases previously discussed. We match the separate cases to the above classes. \\
% \\

Case Ib: This leads to Class II (vi). \\
Case III where $G \ne Z$: This leads to Class II (vi). \\
Case IVa: This leads to Class II (vii). \\
Case IVb: This leads to Class II (ix) with $q=3$. \\
Case Va: This leads to Class II (ix). \\
Case Vb: This leads to Class II (x). \\
Case Vc: This leads to Class II (x) with $q=3$. \\
Case Vd: This leads to Class II (viii). \\
\end{proof}

\begin{lemma}
    If $Z \not \subset G$, then $G$ has no element of order 2 and $|G|$ is therefore odd. Observe that in Cases II, IV, V and VI, $|G|$ is always even, thus we have either Case I or III. These correspond to Class I (i) or Class II (vi). \\
\end{lemma}

\section{Classification of finite subgroups of $\PGL_2(\Fbar_p)$}

\begin{theorem}[Classification of finite subgroup of $\PGL_2(\Fbar)$]
    \label{FLT_classification_fin_subgroups_of_PGL2_over_AlgClosure_ZMod}
    \uses{dicksons_classification_theorem_class_II, dicksons_classification_theorem_class_I, PGL_iso_PSL, SpecialSubgroups.center_SL2_eq_Z}
    \lean{FLT_classification_fin_subgroups_of_PGL2_over_AlgClosure_ZMod}
    Let $G$ be a finite subgroup of $\PGL_2(\Fbar_p)$ then $G$ is isomorphic to either a cyclic group, a dihedral group, $A_4$, $S_5$, $A_5$, or is isomorphic to $\PSL_2(k)$ or $\PGL_2(k)$ for some finite field $k$ of characteristic $p$.
\end{theorem}
% DEPENDENCY AND PROOF

\chapter{Bibliography}

% \bibliographystyle{plain} % We choose the "plain" reference style
% \bibliography{bibliography}

\begin{thebibliography}{3}

    \bibitem{butler}
    Butler, C. (2019). 
    \textit{Dickson's Classification of Finite Subgroups of the Two-dimensional Special Linear Group over an Algebraically Closed Field.}
    Master's Theses in Mathematical Sciences 2019: E63.
    
    \bibitem{sangwin}
    Sangwin, C. (2023). 
    \textit{Sums of the first n odd integers.}
    The Mathematical Gazette, 107(568), 10-24.
    
    \bibitem{dtt}
    Henri Darmon, Fred Diamond, and Richard Taylor. 
    \textit{Fermat’s last theorem}. 
    In Current developments in mathematics, 
    1995 (Cambridge, MA), pages 1–154. Int. Press, Cambridge, MA, 1994.
    \bibitem{alperin} 
    Alperin, J.L., Bell, R.B. 
    \textit{Groups and Representations}. 
    Springer,
    (1995).
    
    \bibitem{bhattacharya} 
    Bhattacharya, P.B., Jain, S.K., Nagpaul, S.R. 
    \textit{Basic Abstract Algebra, Second Edition}. 
    Cambridge University Press,
    (1994).
    
    \bibitem{dickson} 
    Dickson, L.E. 
    \textit{Linear Groups, with an Exposition of the Galois Field Theory}. 
    B.G.Teubner, Leipzig,
    (1901).
    
    \bibitem{dummit} 
    Dummit, D.S., Foote, R.M. 
    \textit{Abstract Algebra}. 
    Wiley,
    (2004).
    
    \bibitem{matrix} 
    Holst, A., Ufnarovski, V. 
    \textit{Matrix Theory}. 
    Studentlitteratur,
    (2014).
    
    \bibitem{hungerford} 
    Hungerford, T.W. 
    \textit{Abstract Algebra: An Introduction, Third Edition}. 
    Brooks/Cole, Cengage Learning,
    (2014).
    
    \bibitem{schur} 
    Schur, I. 
    \textit{Über die Darstellung der symmetrischen und der alternierenden Gruppe durch gebrochene lineare Substitutionen.} Journal für die reine und angewandte Mathematik (Crelles Journal) (139), p.155-250. 
    De Gruyter,
    (1911).
    
    \bibitem{stewart} 
    Stewart, I. 
    \textit{Galois Theory, Third Edition}. 
    Chapman \& Hall/CRC,
    (2003).
    
    \bibitem{suzuki}
    Suzuki, M. 
    \textit{Group Theory I}. 
    Spinger-Verlag, Berlin, Heidelberg, New York, 
    (1982).
    
\end{thebibliography}
\end{document}